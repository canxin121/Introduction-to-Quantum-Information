\documentclass[12pt]{article}

\usepackage[a4paper,margin=2.2cm]{geometry}
\usepackage{amsmath,amssymb}
\usepackage{booktabs}
\usepackage{enumitem}
\usepackage{xeCJK}
\usepackage{hyperref}
\usepackage{graphicx}

\setCJKmainfont{Source Han Serif CN}
\setCJKsansfont{Source Han Sans CN}
\setCJKmonofont{Source Han Sans CN}
\setlist[itemize]{leftmargin=*,nosep}
\setlist[enumerate]{leftmargin=*,nosep}

\title{量子信息导论:作业解答}
\author{}
\date{}

\begin{document}

\begin{enumerate}
\def\labelenumi{\arabic{enumi}.}
\item
  一个粒子的波函数为:
\end{enumerate}

\begin{quote}
\[\psi\left( {x,t} \right) = Aexp\left( {- \frac{1}{2}ax^{2} - i\omega t} \right)\]

将该波函数归一化
\end{quote}

\textbf{相关概念与定义:}
\begin{itemize}
\item \textbf{波函数与概率密度:}$\psi(x,t)=\langle x|\psi(t)\rangle$ 为位置表象波函数,
      概率密度 $\rho(x,t)=|\psi(x,t)|^2$。
\item \textbf{归一化条件:}$\int_{-\infty}^{\infty}|\psi(x,t)|^2\,dx=1$。
\item \textbf{高斯积分:}$\int_{-\infty}^{\infty}e^{-a x^2}dx=\sqrt{\pi/a}$($a>0$)。
\end{itemize}

\textbf{解答:}
\[\begin{aligned}
{\int\left| {\psi\left( {x,t} \right)} \right|}^{2} &= A^{2}{\int{exp\left( {- ax^{2}} \right)dx}} = A^{2}I = 1 \\
I^{2} &= {\int{exp\left( {- ax^{2}} \right)dx}}{\int{exp\left( {- ay^{2}} \right)dy}} \\
&= {\iint\limits_{}{exp\left( {- a\left( {x^{2} + y^{2}} \right)} \right)dxdy}} \\
&= {\iint\limits_{}{rexp\left( {- ar^{2}} \right)drd\theta}} = \frac{\pi}{a} \\
I &= \sqrt{\frac{\pi}{a}} \\
A &= \left( \frac{a}{\pi} \right)^{\frac{1}{4}}
\end{aligned}\]

\begin{enumerate}
\def\labelenumi{\arabic{enumi}.}
\setcounter{enumi}{1}
\item
  证明\(\frac{d}{dt}{\int\limits_{- \infty}^{+ \infty}{\left| {\psi\left( {x,t} \right)} \right|^{2}dx}} = 0\)
\end{enumerate}

\begin{quote}
\textbf{知识点:}
\begin{itemize}
\item \textbf{概率守恒:}总概率不随时间变化,等价于
      $\frac{d}{dt}\int_{-\infty}^{\infty}|\psi|^2\,dx=0$。
\item \textbf{连续性方程:}一维形式 $\partial_t\rho+\partial_x j=0$,
      连接概率密度与概率流,给出守恒的局域表达。
\item \textbf{含时薛定谔方程:}用其与共轭式消去势能项并形成全导数结构。
\item \textbf{分部积分与边界条件:}可归一化态满足 $\psi(\pm\infty)=0$,
      使边界项消失,从而得到守恒结论。
\end{itemize}
\textbf{相关概念与定义:}
\begin{itemize}
\item \textbf{概率密度与归一化:}$\rho(x,t)=|\psi(x,t)|^2$,
      概率守恒等价于 $\frac{d}{dt}\int_{-\infty}^{\infty}\rho\,dx=0$。
\item \textbf{乘积法则与共轭求导:}$\partial_t(\psi^\ast\psi)=\psi^\ast\partial_t\psi+(\partial_t\psi^\ast)\psi$,
      且 $\partial_t\psi^\ast=(\partial_t\psi)^\ast$。
\item \textbf{含时薛定谔方程(1D):}$i\hslash\partial_t\psi=
      -\frac{\hslash^2}{2m}\partial_x^2\psi+V(x)\psi$。
\item \textbf{边界条件:}可归一化态满足 $\psi(\pm\infty)=0$,积分分部的边界项消失。
\end{itemize}

\textbf{详细步骤:}
\begin{align*}
\frac{d}{dt}\int_{-\infty}^{+\infty}|\psi(x,t)|^2\,dx
&=\int_{-\infty}^{+\infty}\frac{\partial}{\partial t}|\psi(x,t)|^2\,dx
\quad(\text{交换求导与积分})\\
\frac{\partial}{\partial t}|\psi|^2
&=\frac{\partial}{\partial t}(\psi^\ast\psi)
=\psi^\ast\frac{\partial\psi}{\partial t}
+\frac{\partial\psi^\ast}{\partial t}\psi
\quad(\text{乘积法则})
\end{align*}

由含时薛定谔方程
\[
i\hslash\frac{\partial\psi}{\partial t}
=-\frac{\hslash^2}{2m}\frac{\partial^2\psi}{\partial x^2}+V(x)\psi
\]
得
\[
\frac{\partial\psi}{\partial t}
=\frac{i\hslash}{2m}\frac{\partial^2\psi}{\partial x^2}
-\frac{i}{\hslash}V(x)\psi,
\]
对其取共轭可得
\[
\frac{\partial\psi^\ast}{\partial t}
=-\frac{i\hslash}{2m}\frac{\partial^2\psi^\ast}{\partial x^2}
+\frac{i}{\hslash}V(x)\psi^\ast.
\]
代回得到
\[
\frac{\partial}{\partial t}|\psi|^2
=\frac{i\hslash}{2m}\left(\psi^\ast\frac{\partial^2\psi}{\partial x^2}
-\psi\frac{\partial^2\psi^\ast}{\partial x^2}\right)
=\frac{i\hslash}{2m}\frac{\partial}{\partial x}
\left(\psi^\ast\frac{\partial\psi}{\partial x}
-\psi\frac{\partial\psi^\ast}{\partial x}\right).
\]
因此
\begin{align*}
\frac{d}{dt}\int_{-\infty}^{+\infty}|\psi|^2\,dx
&=\frac{i\hslash}{2m}\int_{-\infty}^{+\infty}\frac{\partial}{\partial x}
\left(\psi^\ast\frac{\partial\psi}{\partial x}
-\psi\frac{\partial\psi^\ast}{\partial x}\right)\,dx\\
&=\frac{i\hslash}{2m}\left.\left(\psi^\ast\frac{\partial\psi}{\partial x}
-\psi\frac{\partial\psi^\ast}{\partial x}\right)\right|_{-\infty}^{+\infty}=0,
\end{align*}
其中最后一步使用边界条件 $\psi(\pm\infty)=0$。
\end{quote}

\begin{enumerate}
\def\labelenumi{\arabic{enumi}.}
\setcounter{enumi}{2}
\item
  已知波函数\(\psi = \frac{1}{r}e^{ikr}\),计算其概率流密度。
\end{enumerate}

\begin{quote}
\textbf{知识点:}
\begin{itemize}
\item \textbf{概率流密度:}定义
      $\vec{j}=\frac{\hslash}{2mi}\left(\psi^\ast\nabla\psi-\psi\nabla\psi^\ast\right)$,
      表示概率在空间中的“流动方向与强度”。
\item \textbf{梯度与链式法则:}对 $\psi=(1/r)e^{ikr}$ 需用乘积法则、
      $\nabla(e^{g})=e^{g}\nabla g$ 等规则。
\item \textbf{球对称径向关系:}$r=\sqrt{x^2+y^2+z^2}$,
      $\nabla r=\hat r$,从而 $\nabla(1/r)=-\hat r/r^2$。
\end{itemize}
\textbf{相关概念与定义:}
\begin{itemize}
\item \textbf{概率流密度:}$\vec{j}=\frac{\hslash}{2mi}\left(\psi^\ast\nabla\psi-\psi\nabla\psi^\ast\right)$。
\item \textbf{梯度算符:}$\nabla=\left(\partial_x,\partial_y,\partial_z\right)$。
\item \textbf{径向变量与单位矢量:}$r=\sqrt{x^2+y^2+z^2}$,
      $\nabla r=\left(\frac{x}{r},\frac{y}{r},\frac{z}{r}\right)=\hat{r}$。
\end{itemize}

\textbf{详细步骤:}
\[
r=\sqrt{x^2+y^2+z^2},\qquad \nabla r=\left(\frac{x}{r},\frac{y}{r},\frac{z}{r}\right)=\hat{r}.
\]
这里
\[
\nabla=\left(\frac{\partial}{\partial x},\frac{\partial}{\partial y},\frac{\partial}{\partial z}\right),
\quad
\nabla f=\left(\frac{\partial f}{\partial x},\frac{\partial f}{\partial y},\frac{\partial f}{\partial z}\right).
\]
对
\(\psi=\frac{1}{r}e^{ikr}\) 有
\[
\nabla\psi=\nabla\!\left(\frac{1}{r}\right)e^{ikr}
+\frac{1}{r}\nabla\!\left(e^{ikr}\right).
\]
其中
\[
\nabla\!\left(\frac{1}{r}\right)
=-\frac{1}{r^2}\nabla r
=-\frac{1}{r^2}\hat{r},
\]
以及
\[
\nabla\!\left(e^{ikr}\right)
=e^{ikr}\nabla(ikr)
=ik\,e^{ikr}\nabla r
=ik\,e^{ikr}\hat{r}.
\]
这里用到多元链式法则:若 $g(\vec{r})$ 为标量函数,则
\[
\nabla\!\left(e^{g}\right)=e^{g}\nabla g.
\]
代回得
\[
\nabla\psi
=-\frac{\hat{r}}{r^2}e^{ikr}+\frac{1}{r}\left(ik\,e^{ikr}\hat{r}\right)
=\left(\frac{ik}{r}-\frac{1}{r^2}\right)e^{ikr}\hat{r}.
\]
同理
\[
\nabla\psi^\ast=\left(-\frac{ik}{r}-\frac{1}{r^2}\right)e^{-ikr}\hat{r}.
\]
因此
\[
\psi^\ast=\frac{1}{r}e^{-ikr},\quad \psi=\frac{1}{r}e^{ikr}.
\]
先算
\[
\psi^\ast\nabla\psi
=\left(\frac{1}{r}e^{-ikr}\right)\left(\frac{ik}{r}-\frac{1}{r^2}\right)e^{ikr}\hat{r}
=\left(\frac{ik}{r^2}-\frac{1}{r^3}\right)\hat{r},
\]
\[
\psi\nabla\psi^\ast
=\left(\frac{1}{r}e^{ikr}\right)\left(-\frac{ik}{r}-\frac{1}{r^2}\right)e^{-ikr}\hat{r}
=\left(-\frac{ik}{r^2}-\frac{1}{r^3}\right)\hat{r}.
\]
代入
\[
J=\frac{i\hslash}{2m}\left(\psi\nabla\psi^{\ast}-\psi^{\ast}\nabla\psi\right)
\]
得
\[
J=\frac{\hslash k}{mr^2}\hat{r}
=\frac{\hslash k}{mr^3}\vec{r}.
\]
\end{quote}

\begin{enumerate}
\def\labelenumi{\arabic{enumi}.}
\setcounter{enumi}{3}
\item
  氢原子处于基态,\(\psi\left( {r,\theta,\varphi} \right) = \frac{1}{\sqrt{\pi a_{0}^{3}}}e^{- {r/a_{0}}}\),求
\end{enumerate}

\begin{quote}
\textbf{知识点:}
\begin{itemize}
\item \textbf{势能期望值:}对库仑势 $V(r)=-e^2/r$,
      $\langle V\rangle=\int \psi^\ast V(r)\psi\,dV$,只需计算 $\langle 1/r\rangle$。
\item \textbf{动能期望值:}$\hat T=-\frac{\hslash^2}{2m}\nabla^2$,
      因而 $\langle T\rangle=-\frac{\hslash^2}{2m}\int \psi^\ast(\nabla^2\psi)\,dV$。
\item \textbf{球对称拉普拉斯:}若 $\psi=\psi(r)$,
      $\nabla^2\psi=\frac{1}{r^2}\frac{d}{dr}\left(r^2\frac{d\psi}{dr}\right)$。
\item \textbf{角向积分:}球对称态有 $\int_0^\pi\!\int_0^{2\pi}\sin\theta\,d\theta d\varphi=4\pi$,
      将三维积分化为径向积分。
\end{itemize}
\textbf{相关概念与定义:}
\begin{itemize}
\item \textbf{期望值:}$\langle A\rangle=\int \psi^\ast A\psi\,dV$,
      特别地 $\langle r\rangle=\int r|\psi|^2\,dV$。
\item \textbf{球坐标体积元:}$dV=r^2\sin\theta\,dr\,d\theta\,d\varphi$。
\item \textbf{球对称拉普拉斯:}$\nabla^2 f(r)=\frac{1}{r^2}\frac{d}{dr}\left(r^2\frac{df}{dr}\right)$。
\item \textbf{常用积分:}$\int_0^\infty r^n e^{-\alpha r}dr=\frac{n!}{\alpha^{n+1}}$($\alpha>0$)。
\end{itemize}

\textbf{已知:}
\[
\psi(r,\theta,\varphi)=\frac{1}{\sqrt{\pi a_0^3}}e^{-r/a_0},\quad
|\psi|^2=\frac{1}{\pi a_0^3}e^{-2r/a_0},
\quad dV=r^2\sin\theta\,dr\,d\theta\,d\varphi.
\]
归一化核对($\int|\psi|^2dV=1$):
\[
\int |\psi|^2\,dV
=\int_0^\infty\!\!\int_0^\pi\!\!\int_0^{2\pi}
\frac{1}{\pi a_0^3}e^{-2r/a_0}\,r^2\sin\theta\,d\varphi\,d\theta\,dr
=\frac{4}{a_0^3}\int_0^\infty r^2 e^{-2r/a_0}\,dr.
\]
用公式 $\int_0^\infty r^2 e^{-\alpha r}dr=\frac{2}{\alpha^3}$($\alpha>0$),取 $\alpha=2/a_0$:
\[
\int |\psi|^2\,dV=\frac{4}{a_0^3}\cdot\frac{2}{(2/a_0)^3}=1.
\]
因此归一化成立。
这里 $\vec r$ 表示位置\textbf{矢量},$r=|\vec r|$ 表示\textbf{模长}。
因此 $\psi(\vec r)$ 与 $\psi(r,\theta,\varphi)$ 是同一波函数在不同坐标下的写法。
角向积分给出 $\int_0^\pi\!\int_0^{2\pi}\sin\theta\,d\theta d\varphi=4\pi$,因此只剩径向积分。

\textbf{(1)$\langle r\rangle$:}
\[
\langle r\rangle=\int r|\psi|^2\,dV
=\int_0^\infty\!\!\int_0^\pi\!\!\int_0^{2\pi}
r\,\frac{1}{\pi a_0^3}e^{-2r/a_0}\,r^2\sin\theta\,d\varphi\,d\theta\,dr.
\]
先做角向积分:
\[
\int_0^\pi\!\!\int_0^{2\pi}\sin\theta\,d\theta\,d\varphi=4\pi,
\]
因此
\[
\langle r\rangle=\frac{4}{a_0^3}\int_0^\infty r^3 e^{-2r/a_0}\,dr.
\]
用公式 $\int_0^\infty x^n e^{-ax}dx=\frac{n!}{a^{n+1}}$($a>0$),得
\[
\langle r\rangle=\frac{4}{a_0^3}\cdot\frac{3!}{(2/a_0)^4}
=\frac{3}{2}a_0.
\]

\textbf{(2)$\langle V\rangle$:}
\[
\left\langle -\frac{e^2}{r}\right\rangle
=-\int \frac{e^2}{r}|\psi|^2\,dV
=-\int_0^\infty\!\!\int_0^\pi\!\!\int_0^{2\pi}
\frac{e^2}{r}\frac{1}{\pi a_0^3}e^{-2r/a_0}\,r^2\sin\theta\,d\varphi\,d\theta\,dr.
\]
角向积分给出 $4\pi$,因此
\[
\left\langle -\frac{e^2}{r}\right\rangle
=-\frac{4e^2}{a_0^3}\int_0^\infty r e^{-2r/a_0}\,dr
=-\frac{4e^2}{a_0^3}\cdot\frac{1!}{(2/a_0)^2}
=-\frac{e^2}{a_0}.
\]

\textbf{(3)$\langle T\rangle$:}
球对称情形
\[
\nabla^2\psi=\frac{1}{r^2}\frac{d}{dr}\left(r^2\frac{d\psi}{dr}\right).
\]
\textbf{等价快公式:}$\nabla^2\psi=\frac{1}{r}\frac{d^2}{dr^2}(r\psi)$,
可直接作用于 $r\psi$ 以减少计算量。
\textbf{用快公式再算一遍:}令 $u(r)=r\psi$,则
\[
u(r)=\frac{r}{\sqrt{\pi a_0^3}}e^{-r/a_0},
\quad
u'(r)=\frac{1}{\sqrt{\pi a_0^3}}e^{-r/a_0}\left(1-\frac{r}{a_0}\right),
\]
\[
u''(r)=\frac{1}{\sqrt{\pi a_0^3}}e^{-r/a_0}
\left(-\frac{2}{a_0}+\frac{r}{a_0^2}\right).
\]
于是
\[
\nabla^2\psi=\frac{1}{r}u''(r)
=\left(\frac{1}{a_0^2}-\frac{2}{a_0 r}\right)\psi,
\]
与下式一致。
先求导:
\[
\frac{d\psi}{dr}=-\frac{1}{a_0}\psi,
\]
\[
r^2\frac{d\psi}{dr}=-\frac{r^2}{a_0}\psi,
\]
\[
\frac{d}{dr}\left(r^2\frac{d\psi}{dr}\right)
=-\frac{2r}{a_0}\psi+\frac{r^2}{a_0^2}\psi.
\]
因此
\[
\nabla^2\psi=\left(\frac{1}{a_0^2}-\frac{2}{a_0 r}\right)\psi.
\]
动能期望值
\[
\langle T\rangle=-\frac{\hslash^2}{2m}\int\psi^\ast(\nabla^2\psi)\,dV
=-\frac{\hslash^2}{2m}\int\psi^\ast\left(\frac{1}{a_0^2}-\frac{2}{a_0 r}\right)\psi\,dV
\]
展开积分为
\[
\langle T\rangle
=-\frac{\hslash^2}{2m}\left[\frac{1}{a_0^2}\int|\psi|^2\,dV
-\frac{2}{a_0}\int\frac{1}{r}|\psi|^2\,dV\right].
\]
利用归一化 $\int|\psi|^2\,dV=1$,并记
\[
\left\langle\frac{1}{r}\right\rangle=\int\frac{1}{r}|\psi|^2\,dV,
\]
得到
\[
\langle T\rangle=-\frac{\hslash^2}{2m}\left(\frac{1}{a_0^2}-\frac{2}{a_0}\left\langle\frac{1}{r}\right\rangle\right).
\]
而
\[
\left\langle\frac{1}{r}\right\rangle
=\int \frac{1}{r}|\psi|^2\,dV
=\frac{4}{a_0^3}\int_0^\infty r e^{-2r/a_0}dr
=\frac{4}{a_0^3}\cdot\frac{1!}{(2/a_0)^2}
=\frac{1}{a_0}.
\]
于是
\[
\langle T\rangle=-\frac{\hslash^2}{2m}\left(\frac{1}{a_0^2}-\frac{2}{a_0^2}\right)
=\frac{\hslash^2}{2m a_0^2}.
\]
\end{quote}

\begin{enumerate}
\def\labelenumi{\arabic{enumi}.}
\setcounter{enumi}{4}
\item
  写出角动量算符的表达式,并且计算\(L_{x}\)和\(L_{y}\)的对易关系
\end{enumerate}

\begin{quote}
\textbf{相关概念与定义:}
\begin{itemize}
\item \textbf{角动量算符:}$\vec{L}=\vec{r}\times\vec{p}$,位置表象中 $\vec{p}=-i\hslash\nabla$。
\item \textbf{基本对易关系:}$[x_i,x_j]=0$,$[p_i,p_j]=0$,$[x_i,p_j]=i\hslash\delta_{ij}$。
\end{itemize}

\textbf{角动量算符:}
\[
\vec{L}=\vec{r}\times\vec{p},\quad
L_x=y p_z-z p_y,\quad
L_y=z p_x-x p_z,\quad
L_z=x p_y-y p_x.
\]
\textbf{分量展开的详细计算:}写成分量
\[
\vec r=(x,y,z),\quad \vec p=(p_x,p_y,p_z).
\]
叉乘用行列式记忆:
\[
\vec L=\vec r\times\vec p=
\begin{vmatrix}
\vec e_x & \vec e_y & \vec e_z\\
x & y & z\\
p_x & p_y & p_z
\end{vmatrix}.
\]
按第一行展开:
\[
\vec L=\vec e_x(yp_z-zp_y)-\vec e_y(xp_z-zp_x)+\vec e_z(xp_y-yp_x).
\]
因此得到
\[
L_x=yp_z-zp_y,\quad
L_y=zp_x-xp_z,\quad
L_z=xp_y-yp_x.
\]
\textbf{这里 $p_x,p_y,p_z$ 的含义:}它们是动量算符的三个分量。
在位置表象中
\[
p_x=-i\hslash\frac{\partial}{\partial x},\quad
p_y=-i\hslash\frac{\partial}{\partial y},\quad
p_z=-i\hslash\frac{\partial}{\partial z}.
\]
代入即可得到 $L_x,L_y,L_z$ 的微分算符形式:
\[
L_x=-i\hslash\left(y\frac{\partial}{\partial z}-z\frac{\partial}{\partial y}\right),\quad
L_y=-i\hslash\left(z\frac{\partial}{\partial x}-x\frac{\partial}{\partial z}\right),\quad
L_z=-i\hslash\left(x\frac{\partial}{\partial y}-y\frac{\partial}{\partial x}\right).
\]
在位置表象中 $p_i=-i\hslash\,\partial_i$,于是
\[
L_x=-i\hslash\left(y\frac{\partial}{\partial z}-z\frac{\partial}{\partial y}\right),\quad
L_y=-i\hslash\left(z\frac{\partial}{\partial x}-x\frac{\partial}{\partial z}\right).
\]

\textbf{对易关系计算:}使用基本对易关系
\[
[x_i,x_j]=0,\quad [p_i,p_j]=0,\quad [x_i,p_j]=i\hslash\,\delta_{ij}.
\]
因此有
\[
[p_i,x_j]=-[x_j,p_i]=-i\hslash\,\delta_{ij}.
\]
并用到对易子的乘积法则:
\[
[AB,C]=A[B,C]+[A,C]B,\qquad [A,BC]=[A,B]C+B[A,C].
\]
例如
\[
[y p_z, z p_x]
=y[p_z,zp_x]+[y,zp_x]p_z
=y([p_z,z]p_x+z[p_z,p_x])+([y,z]p_x+z[y,p_x])p_z.
\]
由于 $[p_z,p_x]=[y,z]=[y,p_x]=0$,仅剩
\[
[y p_z, z p_x]=y[p_z,z]p_x.
\]
再代入基本对易关系 $[p_z,z]=-i\hslash$:
\[
[y p_z, z p_x]=y(-i\hslash)p_x=-i\hslash\,y p_x.
\]
先展开
\[
[L_x,L_y]=[y p_z-z p_y,\ z p_x-x p_z].
\]
用线性与 $[AB,C]=A[B,C]+[A,C]B$ 得
\[
[L_x,L_y]=[y p_z, z p_x]-[y p_z, x p_z]-[z p_y, z p_x]+[z p_y, x p_z].
\]
逐项计算:
\[
[y p_z, z p_x]=y[p_z,z]p_x=-i\hslash\,y p_x,
\]
因为 $[p_z,z]=-i\hslash$,其余对易子为 0。
\[
[y p_z, x p_z]=0,\qquad [z p_y, z p_x]=0
\]
(不同分量互对易)。
最后
\[
[z p_y, x p_z]=x[z,p_z]p_y=+i\hslash\,x p_y.
\]
合并得到
\[
[L_x,L_y]=i\hslash(x p_y-y p_x)=i\hslash L_z.
\]
\textbf{结论:}\(\,[L_x,L_y]=i\hslash L_z\)。
\end{quote}

\begin{enumerate}
\def\labelenumi{\arabic{enumi}.}
\setcounter{enumi}{5}
\item
  对于一个动量为\(p\),势能为\(V(x)\)的基本微观粒子,证明牛顿力学的基本方程
\end{enumerate}

\begin{quote}
\textbf{相关概念与定义:}
\begin{itemize}
\item \textbf{动量算符与期望值:}$\hat{p}=-i\hslash\partial_x$,
      $\langle p\rangle=\int \psi^\ast\hat{p}\psi\,dx$。
\item \textbf{含时薛定谔方程(1D):}$i\hslash\partial_t\psi=
      -\frac{\hslash^2}{2m}\partial_x^2\psi+V(x)\psi$。
\item \textbf{分部积分与边界条件:}$\psi(\pm\infty)=0$ 使边界项消失。
\end{itemize}

\textbf{整体思路:}
\begin{itemize}
\item 写出动量期望值 $\langle p\rangle$ 的积分表达式;
\item 对时间求导并用乘积法则展开;
\item 用含时薛定谔方程(及其共轭)替换 $\partial_t\psi,\partial_t\psi^\ast$;
\item 对含空间导数项分部积分,并利用边界条件消去边界项;
\item 最终得到 $\frac{d\langle p\rangle}{dt}=\left\langle-\frac{\partial V}{\partial x}\right\rangle$。
\end{itemize}

\textbf{详细推导:}
\[
\langle p\rangle=\int_{-\infty}^{+\infty}\psi^\ast\left(-i\hslash\frac{\partial}{\partial x}\right)\psi\,dx.
\]
对时间求导:
\begin{align*}
\frac{d\langle p\rangle}{dt}
&=\frac{d}{dt}\int_{-\infty}^{+\infty}\psi^\ast\left(-i\hslash\frac{\partial\psi}{\partial x}\right)\,dx\\
&=-i\hslash\int_{-\infty}^{+\infty}\left(
\frac{\partial\psi^\ast}{\partial t}\frac{\partial\psi}{\partial x}
+\psi^\ast\frac{\partial}{\partial x}\frac{\partial\psi}{\partial t}\right)dx
\quad(\text{交换求导与积分 + 乘积法则})
\end{align*}
对第二项做分部积分:
\begin{align*}
\int \psi^\ast\frac{\partial}{\partial x}\frac{\partial\psi}{\partial t}\,dx
&=\text{取 }u=\psi^\ast,\; dv=\frac{\partial}{\partial x}\frac{\partial\psi}{\partial t}\,dx,\;
 du=\frac{\partial\psi^\ast}{\partial x}\,dx,\; v=\frac{\partial\psi}{\partial t}\\
&\quad\Rightarrow \int u\,dv=uv-\int v\,du\\
&=\left.\psi^\ast\frac{\partial\psi}{\partial t}\right|_{-\infty}^{+\infty}
-\int \frac{\partial\psi^\ast}{\partial x}\frac{\partial\psi}{\partial t}\,dx\\
&=-\int \frac{\partial\psi^\ast}{\partial x}\frac{\partial\psi}{\partial t}\,dx
\quad(\psi(\pm\infty)=0),
\end{align*}
代回得到
\begin{align*}
\frac{d\langle p\rangle}{dt}
&=-i\hslash\int\left(\frac{\partial\psi^\ast}{\partial t}\frac{\partial\psi}{\partial x}
-\frac{\partial\psi}{\partial t}\frac{\partial\psi^\ast}{\partial x}\right)dx
\end{align*}

含时薛定谔方程与其共轭:
\[
\frac{\partial\psi}{\partial t}
=\frac{i\hslash}{2m}\frac{\partial^2\psi}{\partial x^2}
-\frac{i}{\hslash}V\psi,\quad
\frac{\partial\psi^\ast}{\partial t}
=-\frac{i\hslash}{2m}\frac{\partial^2\psi^\ast}{\partial x^2}
+\frac{i}{\hslash}V\psi^\ast.
\]
代入得
\[
\frac{d\langle p\rangle}{dt}
=-\frac{\hslash^2}{2m}\int\left(\frac{\partial^2\psi^\ast}{\partial x^2}\frac{\partial\psi}{\partial x}
+\frac{\partial^2\psi}{\partial x^2}\frac{\partial\psi^\ast}{\partial x}\right)dx
+\int V\frac{\partial}{\partial x}(|\psi|^2)\,dx.
\]
第一项是全导数并在边界消失,第二项分部积分:
\[
\int\left(\frac{\partial^2\psi^\ast}{\partial x^2}\frac{\partial\psi}{\partial x}
+\frac{\partial^2\psi}{\partial x^2}\frac{\partial\psi^\ast}{\partial x}\right)dx
=\int \frac{\partial}{\partial x}\left(\frac{\partial\psi^\ast}{\partial x}\frac{\partial\psi}{\partial x}\right)dx
=\left.\frac{\partial\psi^\ast}{\partial x}\frac{\partial\psi}{\partial x}\right|_{-\infty}^{+\infty}=0,
\]
其中利用可归一化态在无穷远处 $\psi,\partial_x\psi\to 0$。
\[
\int V\frac{\partial}{\partial x}(|\psi|^2)\,dx
=\left.V|\psi|^2\right|_{-\infty}^{+\infty}
-\int \frac{\partial V}{\partial x}|\psi|^2\,dx.
\]
边界项为 0,因此
\[
\frac{d\langle p\rangle}{dt}
=\left\langle -\frac{\partial V}{\partial x}\right\rangle.
\]
\textbf{为什么这就是牛顿方程:}经典力学中
\[
\frac{dp}{dt}=F(x)=-\frac{\partial V}{\partial x}.
\]
\textbf{经典力学回顾(1D):}
\begin{itemize}
\item 牛顿第二定律:$F=ma$,而 $p=mv$,因此 $\frac{dp}{dt}=F$。
\item 势能与力:对保守力有 $dW=F\,dx=-dV$,故
      $F(x)=-\frac{dV}{dx}$。
\end{itemize}
量子力学得到的是“期望值版本”:
\[
\frac{d\langle p\rangle}{dt}
=\left\langle-\frac{\partial V}{\partial x}\right\rangle,
\]
说明动量平均值的变化率等于力的平均值(Ehrenfest 定理)。
当波包足够窄、势能变化平缓时,
\[
\left\langle-\frac{\partial V}{\partial x}\right\rangle
\approx -\frac{\partial V}{\partial x}\Big|_{x=\langle x\rangle},
\]
于是回到经典牛顿方程形式。
\end{quote}

\begin{enumerate}
\def\labelenumi{\arabic{enumi}.}
\setcounter{enumi}{6}
\item
  证明时间和频率的测不准关系:\(\Delta t\Delta\omega \geq \frac{1}{2}\)
\end{enumerate}

\begin{quote}
\textbf{相关概念与定义:}
\begin{itemize}
\item \textbf{傅里叶变换(时间-频率):}$F(\omega)=\int f(t)e^{i\omega t}dt$,
      逆变换 $f(t)=\frac{1}{2\pi}\int F(\omega)e^{-i\omega t}d\omega$。
\item \textbf{Parseval定理:}$\int |f(t)|^2dt=\frac{1}{2\pi}\int |F(\omega)|^2d\omega$。
\item \textbf{方差定义:}若均值取 0,则
      $\sigma_t^2=\int t^2|f(t)|^2dt$,
      $\sigma_\omega^2=\frac{1}{2\pi}\int \omega^2|F(\omega)|^2d\omega$。
\item \textbf{柯西-许瓦兹不等式:}$\left|\int f g^\ast\right|^2\le
      \left(\int |f|^2\right)\left(\int |g|^2\right)$。
\end{itemize}

一个信号的时域和频域表达式可以通过傅里叶变换来表示:

\[F(\omega) = {\int{f(t)e^{i\omega t}dt}}\]

此时\(\left| {f(t)} \right|^{2}\)可以表示瞬时的能量,信号总的能量可以表示为:\(\left\| f \right\|^{2} = {\int{\left| {f(t)} \right|^{2}dt}}\)

则\(\frac{\left| {f(t)} \right|^{2}}{\left\| f \right\|^{2}}\)可以看成概率密度函数,为了简单起见,我们假设\(\left\| f \right\|^{2} = 1\)由此可以得到时间的平均值和时间的方差,我们假设均值为0,时间的方差可以表示为:

\[\sigma_{t}^{2} = {\int t^{2}}\left| {f(t)} \right|^{2}dt\]

同理,利用傅里叶变换关系,能量也可以在频域中表示,由此可以得到频率的方差为:

\[\sigma_{\omega}^{2} = {\int\omega^{2}}\frac{\left| {F(\omega)} \right|^{2}}{2\pi}d\omega\]

由此可得:

\(\sigma_{t}^{2}\sigma_{\omega}^{2} = \frac{1}{2\pi}{\int t^{2}}\left| {f(t)} \right|^{2}dt{\int\omega^{2}}\left| {F(\omega)} \right|^{2}d\omega\)、

由傅里叶变换的微分性质和帕斯瓦尔定理:

\[f^{'}(t) = i\omega F(\omega)\]

\[{\int\left| {f^{'}(t)} \right|^{2}}dt = \frac{1}{2\pi}{\int{\omega^{2}\left| {F(\omega)} \right|^{2}}}d\omega\]
\end{quote}

\(\sigma_{t}^{2}\sigma_{\omega}^{2} = {\int t^{2}}\left| {f(t)} \right|^{2}dt{\int\left| {f^{'}(t)} \right|^{2}}dt\)

利用柯西-许瓦兹不等式:

\[\left| {\int{f(x)g^{*}(x)dx}} \right|^{2} \leq \left( {\int{\left| {f(x)} \right|^{2}dt}} \right)\left( {\int{\left| {g(x)} \right|^{2}dx}} \right)\]

因此有:

\[\sigma_{t}^{2}\sigma_{\omega}^{2} = {\int t^{2}}\left| {f(t)} \right|^{2}dt{\int\left| {f^{’}(t)} \right|^{2}}dt \geq \left| {{\int t}f^{*}(t)\frac{d}{dt}f(t)dt} \right|^{2}\]

注意到:
\[\begin{split}
{\int t}f(t)\frac{d}{dt}f^{*}(t)dt &= t\left. \left| {f(t)} \right|^{2} \right|_{- \infty}^{+ \infty} - {\int{f^{*}(t)}}d\left( {tf(t)} \right) \\
&= - {\int{f^{*}(t)}}d\left( {tf(t)} \right) \\
&= - {\int{f^{*}(t)\left( {f(t) + tf^{'}(t)} \right)}}dt \\
&= - 1 - {\int{f^{*}(t)tf^{'}(t)dt}}
\end{split}\]

因此有:

\[\left. {\int{f^{*}(t)tf^{'}(t)dt}} + {\int t}f(t)\frac{d}{dt}f^{*}(t)dt = - 1 \right.\]

\[\Rightarrow 2Re\left( {\int{f^{*}(t)tf^{'}(t)dt}} \right) = - 1\]

由于\(\int{f^{*}(t)tf^{'}(t)dt}\)是一个复数,利用\(\left. \left| {Re(z)} \right| \leq |z|\Rightarrow\left| {Re(z)} \right|^{2} \leq |z|^{2} \right.\)

因此有:
\[\frac{1}{4} = \left( {Re\left( {\int{f^{*}(t)tf^{'}(t)dt}} \right)} \right)^{2} \leq \left| {\int{f^{*}(t)tf^{'}(t)dt}} \right|^{2}\]

因此有:
\[\sigma_{t}^{2}\sigma_{\omega}^{2} = {\int t^{2}}\left| {f(t)} \right|^{2}dt{\int\left| {f^{'}(t)} \right|^{2}}dt \geq \left| {{\int t}f^{*}(t)\frac{d}{dt}f(t)dt} \right|^{2} \geq \frac{1}{4}\]

因此有:\(\sigma_{t}^{}\sigma_{\omega}^{} \geq \frac{1}{2}\)

\begin{enumerate}
\def\labelenumi{\arabic{enumi}.}
\setcounter{enumi}{7}
\item
  粒子处于状态:
\end{enumerate}

\begin{quote}
\[\psi(x) = \left( \frac{1}{2\pi\xi^{2}} \right)^{\frac{1}{2}}exp\left( {\frac{i}{\hslash}p_{0}x - \frac{x^{2}}{4\xi^{2}}} \right)\]

求粒子的动量平均值,并且计算测不准关系\[\overline{\Delta x^{2}}\overline{\Delta p^{2}} = ?\]

解:由归一化已求得 $\xi=\frac{1}{\sqrt{2\pi}}$,因此可直接写成
\[
\psi(x)=\exp\left(\frac{i}{\hslash}p_0 x-\frac{\pi x^2}{2}\right),\quad
|\psi|^2=e^{-\pi x^2},\quad
\int_{-\infty}^{+\infty}|\psi|^2dx=1.
\]

动量的平均值:
\[
\overline{p} = \int\psi^{*}\left( {- i\hslash\frac{\partial}{\partial x}} \right)\psi\, dx.
\]
先求导
\[
\frac{\partial}{\partial x}\psi(x)
=\left(\frac{i p_0}{\hslash}-\pi x\right)\psi(x).
\]
因此
\[
\psi^\ast\left(-i\hslash\frac{\partial}{\partial x}\right)\psi
=\left[p_0+i\hslash\pi x\right]|\psi|^2.
\]
积分得
\[
\overline{p}=p_0\int|\psi|^2dx+i\hslash\pi\int x|\psi|^2dx=p_0.
\]

\[\overline{\Delta x^{2}}\overline{\Delta p^{2}} = ?\]

\[\begin{aligned}
\overline{x} &= \int x|\psi|^2dx = \int_{-\infty}^{+\infty}x e^{-\pi x^2}dx =0,\\
\overline{x^{2}} &= \int x^{2}|\psi|^{2}dx = \int_{-\infty}^{+\infty}x^{2}e^{-\pi x^{2}}dx.
\end{aligned}\]
分部积分:取 $u=x$,$dv=x e^{-\pi x^2}dx$,则 $du=dx$,$v=-\frac{1}{2\pi}e^{-\pi x^2}$。
\[
\int_{-\infty}^{+\infty}x^{2}e^{-\pi x^{2}}dx
=\left.-\frac{x}{2\pi}e^{-\pi x^2}\right|_{-\infty}^{+\infty}
+\frac{1}{2\pi}\int_{-\infty}^{+\infty}e^{-\pi x^2}dx
=\frac{1}{2\pi},
\]
因此
\[
\overline{x^{2}}=\frac{1}{2\pi},\quad \overline{\Delta x^{2}}=\frac{1}{2\pi}.
\]

\textbf{动量方差所需的 $\overline{p^2}$(按二阶导数展开):}
\[
\overline{p^{2}} = -\hslash^{2}\int\psi^{*}\frac{d^{2}\psi}{dx^{2}}\, dx.
\]
先求二阶导数:
\[
\psi''=\left(-\pi-\frac{p_0^2}{\hslash^2}-\frac{2 i\pi p_0}{\hslash}x+\pi^2 x^2\right)\psi.
\]
因此
\[
-\hslash^{2}\psi^*\psi''
=\left(p_0^2+\hslash^2\pi+ 2 i\hslash\pi p_0 x-\hslash^2\pi^2 x^2\right)|\psi|^2.
\]
积分得
\[
\overline{p^2}
=p_0^2+\hslash^2\pi+2 i\hslash\pi p_0\int x|\psi|^2dx
-\hslash^2\pi^2\int x^2|\psi|^2dx.
\]
其中 $\int x|\psi|^2dx=0$(奇函数),且 $\int x^2|\psi|^2dx=\frac{1}{2\pi}$,因此
\[
\overline{p^2}=p_0^2+\hslash^2\pi-\hslash^2\pi^2\cdot\frac{1}{2\pi}
=p_0^2+\frac{\pi}{2}\hslash^2.
\]
于是
\[
\overline{\Delta p^{2}}=\overline{p^{2}}-\overline{p}^{2}=\frac{\pi}{2}\hslash^2.
\]

因此
\[
\overline{\Delta x^{2}}\overline{\Delta p^{2}} = \frac{1}{2\pi}\cdot\frac{\pi}{2}\hslash^{2} = \frac{\hslash^{2}}{4}.
\]
\end{quote}

\begin{enumerate}
\def\labelenumi{\arabic{enumi}.}
\setcounter{enumi}{8}
\item
  运用概率叠加原理解释电子的双缝干涉现象:

  \begin{enumerate}
  \def\labelenumii{\arabic{enumii}.}
  \item
    电子双缝干涉的解释
  \item
    为什么加入光源探测电子是从那一条缝通过时,干涉条纹会消失
  \end{enumerate}
\end{enumerate}

\begin{quote}
\includegraphics[width=1.93333in,height=3.1in]{assets/figs_used/lzlxzy_pic1.png}

解答:
\end{quote}

\begin{quote}
\textbf{简化背诵版(按答案推导法):}

\textbf{(1) 双缝干涉:幅度相加}
\[
\alpha_1=\langle x|1\rangle\langle 1|S\rangle=|\alpha_1|e^{i\varphi_1},\quad
\alpha_2=\langle x|2\rangle\langle 2|S\rangle=|\alpha_2|e^{i\varphi_2}.
\]
\[
I_1=|\alpha_1|^2,\quad I_2=|\alpha_2|^2,\quad
I=|\alpha_1+\alpha_2|^2
=I_1+I_2+2\sqrt{I_1I_2}\cos(\varphi_1-\varphi_2).
\]
最后一项是干涉项,随位置变化 $\Rightarrow$ 明暗条纹。

\textbf{(2) 加探测光源:路径可区分}
用探测器态标记“哪条缝”。设正确探测幅度为 $\beta_1$,误探为 $\beta_2$,
则
\[
\langle xD_1|SP\rangle=\alpha_1\beta_1+\alpha_2\beta_2,\quad
\langle xD_2|SP\rangle=\alpha_1\beta_2+\alpha_2\beta_1.
\]
总概率
\[
I=|\langle xD_1|SP\rangle|^2+|\langle xD_2|SP\rangle|^2.
\]
若能确定哪条缝(\(\beta_2\to 0\)),则
\[
I=|\beta_1|^2\left(|\alpha_1|^2+|\alpha_2|^2\right),
\]
干涉项消失。

\textbf{记忆句:}不可区分 $\Rightarrow$ 幅度相加有干涉;可区分 $\Rightarrow$ 概率相加无干涉。
\end{quote}

\begin{enumerate}
\def\labelenumi{\arabic{enumi}.}
\setcounter{enumi}{9}
\item
  假设体系的势场与时间无关,波函数可以分解为各个本征态的叠加
\end{enumerate}

\begin{quote}
\textbf{简化背诵版:}
\begin{itemize}
\item \textbf{展开:}$\psi(x,t)=\sum_n c_n u_n(x)e^{-iE_n t/\hslash}$,$\hat H u_n=E_n u_n$。
\item \textbf{正交:}$\int u_m^\ast(x)u_n(x)\,dx=\delta_{mn}$。
\item \textbf{结论:}$\langle \hat H\rangle=\langle \hat E\rangle=\sum_n |c_n|^2 E_n$。
\end{itemize}

证明思路(记三行即可):
\[
\langle \hat H\rangle=\int \psi^\ast \hat H\psi\,dx
=\sum_{mn} c_m^\ast c_n e^{i(E_m-E_n)t/\hslash} E_n \int u_m^\ast u_n\,dx
=\sum_n |c_n|^2 E_n.
\]
\[
\langle \hat E\rangle=\int \psi^\ast (i\hslash\partial_t)\psi\,dx
=\sum_{mn} c_m^\ast c_n e^{i(E_m-E_n)t/\hslash} E_n \int u_m^\ast u_n\,dx
=\sum_n |c_n|^2 E_n.
\]

\textbf{记忆口诀:}定态展开 + 正交归一 $\Rightarrow$ 只留下 $m=n$ 项。
\end{quote}

\begin{enumerate}
\def\labelenumi{\arabic{enumi}.}
\setcounter{enumi}{10}
\item
  在一维无限深势阱中,势阱内\(0 \leq x \leq a\),\(V(x) = 0\),其余位置势能函数为无穷大,如果已知粒子的波函数在初始时刻\(\psi\left( {x,0} \right) = Ax\left( {a - x} \right)\),求
\end{enumerate}

\begin{enumerate}
\def\labelenumi{\arabic{enumi}.}
\item
  归一化常数\(A\)
\item
  任意时刻\(t\)的波函数\(\psi\left( {x,t} \right)\)。
\end{enumerate}

\begin{quote}
\textbf{知识点与方法:}
\begin{itemize}
\item \textbf{无限深势阱边界条件:}势阱外 $V\to\infty$,故波函数在边界满足
      $\psi(0,t)=\psi(a,t)=0$,且阱外 $\psi=0$。
\item \textbf{定态本征解:}阱内 $V=0$,定态薛定谔方程
      $-\frac{\hslash^2}{2m}\frac{d^2u}{dx^2}=E u$,本征函数
      $u_n(x)=\sqrt{\frac{2}{a}}\sin\!\left(\frac{n\pi x}{a}\right)$,
      本征值 $E_n=\frac{n^2\pi^2\hslash^2}{2ma^2}$。
\item \textbf{展开与系数:}任意初态可展开为
      $\psi(x,0)=\sum_{n=1}^\infty c_n u_n(x)$,
      系数由正交归一给出
      $c_n=\int_0^a u_n^\ast(x)\psi(x,0)\,dx$。
\item \textbf{时间演化:}势能与时间无关时,
      $\psi(x,t)=\sum_{n=1}^\infty c_n u_n(x)e^{-iE_n t/\hslash}$。
\item \textbf{归一化:}由 $\int_0^a|\psi(x,0)|^2dx=1$ 决定常数 $A$。
\end{itemize}

\textbf{解答:}

\[\left. {\int\limits_{0}^{a}{\left( {Ax\left( {a - x} \right)} \right)^{2}dx}} = A^{2}\frac{a^{5}}{30} = 1\Rightarrow A = \sqrt{\frac{30}{a^{5}}} \right.\]

由势阱内的一维定态薛定谔方程:

\[\left( {- \frac{\hslash^{2}}{2m}\frac{d^{2}}{dx^{2}}} \right)\psi(x) = E\psi(x)\]

令\(k = \sqrt{\frac{2mE}{\hslash^{2}}}\),方程的通解为:\(\psi(x) = A\sin\left( {kx} \right) + B\cos\left( {kx} \right)\)

由边界条件\(\psi(0) = 0\),可知\(B = 0\),\(\psi(a) = 0\),可知,\(k = \frac{n\pi}{a}\)。

由于波函数是归一化的,由此可得:

\[u_{n}(x) = \sqrt{\frac{2}{a}}\sin\left( {\frac{n\pi}{a}x} \right)\]

\[E_{n} = \frac{n^{2}\pi^{2}\hslash^{2}}{2ma^{2}}\]

\[\psi\left( {x,t} \right) = {\sum\limits_{n}^{}{c_{n}u_{n}(x)\exp\left( {- i\frac{E_{n}}{\hslash}t} \right)}}\]

\[c_{n} = {\int_{0}^{a}{u_{n}^{*}(x)\psi\left( {x,0} \right)}}dx\]

因此:\(\psi\left( {x,t} \right) = \sqrt{\frac{30}{a}}\left( \frac{2}{\pi} \right)^{3}{\sum\limits_{n = 1,3,5,7...}^{}{\frac{1}{n^{3}}\sin\left( \frac{n\pi x}{a} \right)\exp\left( {- i\frac{n^{2}\pi^{2}\hslash}{2ma^{2}}t} \right)}}\)
\end{quote}

\begin{enumerate}
\def\labelenumi{\arabic{enumi}.}
\setcounter{enumi}{11}
\item
  1)写出位置算符\(x\)的本征值和本征函数
\end{enumerate}

\begin{quote}
2)写出动量算符\(- i\hslash\frac{d}{dx}\)的本征值和本征函数
\end{quote}

解答:

由于:\(x\delta\left( {x - x_{0}} \right) = x_{0}\delta\left( {x - x_{0}} \right)\)

位置算符\(x\)的本征值是\(x_{0}\)处的坐标,其本征函数是一个冲击函数:

\[\delta\left( {x - x_{0}} \right)\]

设动量算符的本征函数是\(f(x)\),根据本征函数的定义为:

\[- i\hslash\frac{df(x)}{dx} = p_{x}f(x)\]

因此有:

\[\frac{df(x)}{f(x)} = i\frac{p_{x}}{\hslash}dx\]

两边积分:
\[\int \frac{df}{f}=\int i\frac{p_x}{\hslash}\,dx\]
得到
\[\ln f=i\frac{p_x}{\hslash}x+C_0\]
两边取指数:
\[f(x)=C\exp\left(i\frac{p_x}{\hslash}x\right),\quad C=e^{C_0}.\]

因此:

\[f(x) = Cexp\left( {i\frac{p_{x}}{\hslash}x} \right)\]

本征函数具有归一化的特点:
\[\begin{split}
1 &= |C|^{2}{\int{exp\left( {i\frac{p_{x} - p_{x}^{'}}{\hslash}x} \right)dx}} \\
&= |C|^{2}\hslash{\int{exp\left( {i\frac{p_{x} - p_{x}^{'}}{\hslash}x} \right)d\left( \frac{x}{\hslash} \right)}}
\end{split}\]

做变量代换:\(x_{1} = \frac{x}{\hslash}\),因此有

\[\begin{aligned}
1 &= |C|^{2}\hslash{\int{exp\left( {i\left( {p_{x} - p_{x}^{'}} \right)x_{1}} \right)dx_{1}}} \\
&= 2\pi|C|^{2}\hslash
\end{aligned}\]

因此

\[C = \frac{1}{\sqrt{2\pi\hslash}}\]

动量算符的本征值是粒子的动量,本征函数是一个复指数函数:

\[\frac{1}{\sqrt{2\pi\hslash}}exp\left( {i\frac{p_{x}}{\hslash}x} \right)\]
\textbf{说明:}这是\textbf{一维}动量本征函数。三维情形下本征函数为
\[
\psi_{\vec p}(\vec r)=\frac{1}{(2\pi\hslash)^{3/2}}
\exp\left(\frac{i}{\hslash}\vec p\cdot\vec r\right),
\]
可理解为三个一维平面波的直积,因此归一化因子不同。

\begin{enumerate}
\def\labelenumi{\arabic{enumi}.}
\setcounter{enumi}{12}
\item
  一维无限深势阱中坐标算符和动量算符在能量表象中的矩阵元
\end{enumerate}

\begin{quote}
\textbf{解答1:}

一维无限深势阱中,本征函数为\(u_{n}(x) = \sqrt{\frac{2}{a}}sin\left( {\frac{n\pi}{a}x} \right)\)

坐标算符的对角元:
\textbf{由定义:}能量表象矩阵元
\[
x_{mn}=\langle m|\hat x|n\rangle
=\int_0^a \langle m|x\rangle\,x\,\langle x|n\rangle\,dx
=\int_0^a u_m^\ast(x)\,x\,u_n(x)\,dx.
\]
因此对角元 \(x_{nn}\) 就是取 \(m=n\) 后的积分。
\[\begin{split}
x_{nn} &= {\int\limits_{0}^{a}\frac{2}{a}}sin\left( {\frac{n\pi}{a}x} \right)xsin\left( {\frac{n\pi}{a}x} \right)dx \\
&= \frac{2}{a}{\int\limits_{0}^{a}{xsin^{2}\left( {\frac{n\pi}{a}x} \right)dx}} = \frac{a}{2}
\end{split}\]

当\(m \neq n\)时

\[\begin{aligned}
x_{mn} &= {\int\limits_{0}^{a}\frac{2}{a}}sin\left( {\frac{m\pi}{a}x} \right)xsin\left( {\frac{n\pi}{a}x} \right)dx \\
&= \frac{1}{a}{\int\limits_{0}^{a}{x\left\lbrack {cos\left( {\frac{m - n}{a}\pi x} \right) - cos\left( {\frac{m + n}{a}\pi x} \right)} \right\rbrack}}dx \\
&= \frac{1}{a}\Bigg[ \left. \left( {\frac{a^{2}}{\left( {m - n} \right)^{2}\pi^{2}}cos\left( {\frac{m - n}{a}\pi x} \right) + \frac{ax}{\left( {m - n} \right)\pi}sin\left( {\frac{m - n}{a}\pi x} \right)} \right) \right|_{0}^{a} \\
&\quad - \left. \left( {\frac{a^{2}}{\left( {m + n} \right)^{2}\pi^{2}}cos\left( {\frac{m + n}{a}\pi x} \right) + \frac{ax}{\left( {m + n} \right)\pi}sin\left( {\frac{m + n}{a}\pi x} \right)} \right) \right|_{0}^{a}\Bigg] \\
&= \frac{a}{\pi^{2}}\left\lbrack {\left( {- 1} \right)^{m - n} - 1} \right\rbrack\left\lbrack {\frac{1}{\left( {m - n} \right)^{2}} - \frac{1}{\left( {m + n} \right)^{2}}} \right\rbrack \\
&= \frac{a}{\pi^{2}}\frac{4mn}{\left( {m^{2} - n^{2}} \right)^{2}}\left\lbrack {\left( {- 1} \right)^{m - n} - 1} \right\rbrack
\end{aligned}\]

动量算符的对角元为:

\[\begin{aligned}
p_{nn} &= {\int{u_{n}^{*}(x)\left( {- i\hslash\frac{d}{dx}} \right)u_{n}(x)dx}} \\
&= - i\hslash{\int{\frac{2}{a}sin\left( {\frac{n\pi}{a}x} \right)\left( \frac{d}{dx} \right)sin\left( {\frac{n\pi}{a}x} \right)dx}} \\
&= - i\hslash\frac{2n\pi}{a^{2}}{\int\limits_{0}^{a}{sin\left( {\frac{n\pi}{a}x} \right)}}cos\left( {\frac{n\pi}{a}x} \right)dx \\
&= - i\hslash\frac{n\pi}{a^{2}}{\int\limits_{0}^{a}{sin\left( {\frac{2n\pi}{a}x} \right)}}dx \\
&= i\hslash\frac{n\pi}{a^{2}}\frac{a}{2n\pi}{\int\limits_{0}^{a}{dcos}}\left( {\frac{2n\pi}{a}x} \right) \\
&= i\hslash\frac{1}{2a}\left. {cos\left( {\frac{2n\pi}{a}x} \right)} \right|_{0}^{a} = 0
\end{aligned}\]

动量算符的非对角元为:

\[\begin{aligned}
p_{mn} &= {\int{u_{m}^{*}(x)\left( {- i\hslash\frac{d}{dx}} \right)u_{n}(x)dx}} \\
&= - i\hslash{\int{\frac{2}{a}sin\left( {\frac{m\pi}{a}x} \right)\left( \frac{d}{dx} \right)sin\left( {\frac{n\pi}{a}x} \right)dx}} \\
&= - i\hslash\frac{2n\pi}{a^{2}}{\int\limits_{0}^{a}{sin\left( {\frac{m\pi}{a}x} \right)}}cos\left( {\frac{n\pi}{a}x} \right)dx \\
&= - i\hslash\frac{n\pi}{a^{2}}\left\lbrack {{\int\limits_{0}^{a}{sin\left( {\frac{m + n}{a}\pi x} \right)}} + sin\left( {\frac{m - n}{a}\pi x} \right)} \right\rbrack dx \\
&= i\hslash\frac{n\pi}{a^{2}}\left. \left\lbrack {\frac{a}{\left( {m + n} \right)\pi}cos\left( {\frac{m + n}{a}\pi x} \right) + \frac{a}{\left( {m - n} \right)\pi}cos\left( {\frac{m - n}{a}\pi x} \right)} \right\rbrack \right|_{0}^{a} \\
&= i\hslash\frac{n}{a}\left\lbrack {\frac{1}{\left( {m + n} \right)} + \frac{1}{\left( {m - n} \right)}} \right\rbrack\left\lbrack {\left( {- 1} \right)^{m - n} - 1} \right\rbrack \\
&= \frac{i2mn\hslash}{a\left( {m^{2} - n^{2}} \right)}\left\lbrack {\left( {- 1} \right)^{m - n} - 1} \right\rbrack
\end{aligned}\]

\end{quote}

\begin{enumerate}
\def\labelenumi{\arabic{enumi}.}
\setcounter{enumi}{13}
\item
  在动量表象中,角动量算符\(L_{x}\)的矩阵元。
\end{enumerate}

\begin{quote}
解答:

\textbf{前置知识(表象/基底 vs 空间,以本题为例):}
\begin{itemize}
\item \textbf{表象/基底}是希尔伯特空间中的一组基向量;\textbf{空间变量}(如 \(x\)、\(\vec p\))只是用来给基矢贴标签。
\item 位置表象用 \(\{|x\rangle\}\) 作基底,动量表象用 \(\{|p\rangle\}\) 作基底;
      它们都是同一希尔伯特空间中的不同基,不是“不同物理空间里的态”。
\item 本题是\textbf{动量表象}:矩阵元写成 \(\langle p'|\hat L_x|p\rangle\),
      规则为 \(\hat{\vec p}\to \vec p\)(乘法),\(\hat{\vec r}\to i\hbar\nabla_p\)(微分)。
\end{itemize}

\textbf{更详细的解读(避免“自变量=表象”的误解):}
这里的“表象”是\textbf{选哪组基矢}来表示态与算符。
本题的\(\langle p'|\hat L_x|p\rangle\)已经固定了基是\(\{|p\rangle\}\),
所以最终结果必须写成对\(\vec p\)的乘法或对\(\vec p\)的微分。
中间步骤出现\(x,y,z\)只是因为插入了位置完备性
\(\int d^3r\,|r\rangle\langle r|=\mathbb{I}\),
把矩阵元写成空间积分以便使用已知的\(\hat L_x\)在位置表象的形式。
这并不意味着“改用了位置表象”,而是\textbf{在动量表象问题中暂时借用位置基做积分}。

\textbf{关键思路:}
\begin{itemize}
\item 先在位置基下让\(\hat L_x=-i\hbar(y\partial_z-z\partial_y)\)作用到
      \(\langle r|p\rangle\propto e^{\frac{i}{\hbar}\vec p\cdot\vec r}\);
\item 再用恒等式
      \(y\,e^{\frac{i}{\hbar}\vec p\cdot\vec r}
      =-i\hbar\,\frac{\partial}{\partial p_y}e^{\frac{i}{\hbar}\vec p\cdot\vec r}\)
      (同理对\(z\))把\(r\)换成对\(p\)的导数;
\item 最后把导数移到动量变量上,得到
      \(\hat L_x=-i\hbar\left(p_z\frac{\partial}{\partial p_y}-p_y\frac{\partial}{\partial p_z}\right)\),
      这就是\textbf{动量表象}中的角动量算符形式。
\end{itemize}

动量算符的本征函数为:
\[\psi_{p}(r) = \frac{1}{\left( \sqrt{2\pi\hslash} \right)^{3}}exp\left( {\frac{i}{\hslash}\overset{\rightarrow}{p} \cdot \overset{\rightarrow}{r}} \right)\]

其中\(\overset{\rightarrow}{p} \cdot \overset{\rightarrow}{r} = p_{x}x + p_{y}y + p_{z}z\)

角动量算符\(\hat{L_{x}}\)的表达式为\(\hat{L_{x}} = - i\hslash\left( {y\frac{\partial}{\partial z} - z\frac{\partial}{\partial y}} \right)\)

因此其矩阵元可以表示为如下的积分:

\[\begin{aligned}
\left( \hat{L_{x}} \right)_{p_{1}p} &= {\int{\psi_{p_{1}}^{*}\hat{L_{x}}}}\psi_{p}dV \\
&= {\int{\frac{1}{\left( \sqrt{2\pi\hslash} \right)^{3}}exp\left( {- \frac{i}{\hslash}\overset{\rightarrow}{p_{1}} \cdot \overset{\rightarrow}{r}} \right)}} \\
&\quad \times \left( {- i\hslash\left( {y\frac{\partial}{\partial z} - z\frac{\partial}{\partial y}} \right)} \right)\frac{1}{\left( \sqrt{2\pi\hslash} \right)^{3}}exp\left( {\frac{i}{\hslash}\overset{\rightarrow}{p} \cdot \overset{\rightarrow}{r}} \right)dV \\
&= \frac{1}{\left( {2\pi\hslash} \right)^{3}}{\int{exp\left( {- \frac{i}{\hslash}\overset{\rightarrow}{p_{1}} \cdot \overset{\rightarrow}{r}} \right)}} \\
&\quad \times \left( {- i\hslash\left( {y\frac{\partial}{\partial z}exp\left( {\frac{i}{\hslash}\overset{\rightarrow}{p} \cdot \overset{\rightarrow}{r}} \right) - z\frac{\partial}{\partial y}exp\left( {\frac{i}{\hslash}\overset{\rightarrow}{p} \cdot \overset{\rightarrow}{r}} \right)} \right)} \right)dV \\
&= \frac{1}{\left( {2\pi\hslash} \right)^{3}}{\int{exp\left( {- \frac{i}{\hslash}\overset{\rightarrow}{p_{1}} \cdot \overset{\rightarrow}{r}} \right)}} \\
&\quad \times \left( {- i\hslash\left( {y\left( \frac{i}{\hslash} \right)p_{z}exp\left( {\frac{i}{\hslash}\overset{\rightarrow}{p} \cdot \overset{\rightarrow}{r}} \right) - z\left( \frac{i}{\hslash} \right)p_{y}exp\left( {\frac{i}{\hslash}\overset{\rightarrow}{p} \cdot \overset{\rightarrow}{r}} \right)} \right)} \right)dV \\
&= \frac{1}{\left( {2\pi\hslash} \right)^{3}}{\int{exp\left( {- \frac{i}{\hslash}\overset{\rightarrow}{p_{1}} \cdot \overset{\rightarrow}{r}} \right)\left( {yp_{z} - zp_{y}} \right)exp\left( {\frac{i}{\hslash}\overset{\rightarrow}{p} \cdot \overset{\rightarrow}{r}} \right)}}dV
\end{aligned}\]

注意到:
\[\left. \frac{\partial}{\partial p_{y}}exp\left( {\frac{i}{\hslash}\overset{\rightarrow}{p} \cdot \overset{\rightarrow}{r}} \right) = \frac{i}{\hslash}yexp\left( {\frac{i}{\hslash}\overset{\rightarrow}{p} \cdot \overset{\rightarrow}{r}} \right) \right.\]

\[\Rightarrow yexp\left( {\frac{i}{\hslash}\overset{\rightarrow}{p} \cdot \overset{\rightarrow}{r}} \right) = - i\hslash\frac{\partial}{\partial p_{y}}exp\left( {\frac{i}{\hslash}\overset{\rightarrow}{p} \cdot \overset{\rightarrow}{r}} \right)\]

同理:\(zexp\left( {\frac{i}{\hslash}\overset{\rightarrow}{p} \cdot \overset{\rightarrow}{r}} \right) = - i\hslash\frac{\partial}{\partial p_{z}}exp\left( {\frac{i}{\hslash}\overset{\rightarrow}{p} \cdot \overset{\rightarrow}{r}} \right)\)

因此有:

\[\begin{aligned}
\left( \hat{L_{x}} \right)_{p_{1}p} &= \frac{1}{\left( {2\pi\hslash} \right)^{3}}{\int{exp\left( {- \frac{i}{\hslash}\overset{\rightarrow}{p_{1}} \cdot \overset{\rightarrow}{r}} \right)\left( {yp_{z} - zp_{y}} \right)}} \\
&\quad \times exp\left( {\frac{i}{\hslash}\overset{\rightarrow}{p} \cdot \overset{\rightarrow}{r}} \right)dV \\
&= \frac{\left( {- i\hslash} \right)}{\left( {2\pi\hslash} \right)^{3}}{\int{exp\left( {- \frac{i}{\hslash}\overset{\rightarrow}{p_{1}} \cdot \overset{\rightarrow}{r}} \right)}} \\
&\quad \times \left( {\frac{\partial}{\partial p_{y}}p_{z} - \frac{\partial}{\partial p_{z}}p_{y}} \right)exp\left( {\frac{i}{\hslash}\overset{\rightarrow}{p} \cdot \overset{\rightarrow}{r}} \right)dV \\
&= \left( {- i\hslash} \right)\left( {p_{z}\frac{\partial}{\partial p_{y}} - p_{y}\frac{\partial}{\partial p_{z}}} \right) \\
&\quad \times \left( {\frac{1}{\left( {2\pi\hslash} \right)^{3}}{\int{exp\left( {- \frac{i}{\hslash}\overset{\rightarrow}{p_{1}} \cdot \overset{\rightarrow}{r}} \right)exp\left( {\frac{i}{\hslash}\overset{\rightarrow}{p} \cdot \overset{\rightarrow}{r}} \right)}}dV} \right)
\end{aligned}\]

由正交归一性可以知道:
\[\begin{split}
&\frac{1}{\left( {2\pi\hslash} \right)^{3}}{\int{exp\left( {- \frac{i}{\hslash}\overset{\rightarrow}{p_{1}} \cdot \overset{\rightarrow}{r}} \right)exp\left( {\frac{i}{\hslash}\overset{\rightarrow}{p} \cdot \overset{\rightarrow}{r}} \right)}}dV \\
&= \delta\left( {\overset{\rightarrow}{p} - \overset{\rightarrow}{p_{1}}} \right)
\end{split}\]

因此:
\[\left( \hat{L_{x}} \right)_{p_{1}p} = \left( {- i\hslash} \right)\left( {p_{z}\frac{\partial}{\partial p_{y}} - p_{y}\frac{\partial}{\partial p_{z}}} \right)\delta\left( {\overset{\rightarrow}{p} - \overset{\rightarrow}{p_{1}}} \right)\]
\textbf{等价写法:}这就是动量空间中的角动量算符
\[
\hat L_x=-i\hslash(\vec p\times\nabla_p)_x
= i\hslash\left(p_y\frac{\partial}{\partial p_z}-p_z\frac{\partial}{\partial p_y}\right),
\]
\textbf{行列式推导:}
\[
\vec p\times\nabla_p=
\begin{vmatrix}
\hat e_x & \hat e_y & \hat e_z\\
p_x & p_y & p_z\\
\frac{\partial}{\partial p_x} & \frac{\partial}{\partial p_y} & \frac{\partial}{\partial p_z}
\end{vmatrix},
\]
因此
\[
(\vec p\times\nabla_p)_x
=p_y\frac{\partial}{\partial p_z}-p_z\frac{\partial}{\partial p_y},
\]
从而
\[
\hat L_x=-i\hslash(\vec p\times\nabla_p)_x
=-i\hslash\left(p_y\frac{\partial}{\partial p_z}-p_z\frac{\partial}{\partial p_y}\right).
\]
因此矩阵元也可写成
\[
\left( \hat{L_{x}} \right)_{p_{1}p}=\hat L_x\,\delta\left( {\vec p - \vec p_{1}} \right).
\]

\textbf{解答2(不借用位置基,直接在动量表象计算):}

\textbf{(1) 先写出动量表象中的算符形式:}
动量表象中用替换规则
\(\hat{\vec p}\to \vec p\)、\(\hat{\vec r}\to i\hslash\nabla_p\),
所以
\[
\hat{\vec L}=\hat{\vec r}\times\hat{\vec p}
\;\Rightarrow\;
\hat{\vec L}\to i\hslash(\vec p\times\nabla_p),
\]
从而
\[
\hat L_x
= i\hslash\left(p_y\frac{\partial}{\partial p_z}-p_z\frac{\partial}{\partial p_y}\right).
\]

\textbf{(2) 先把“本征函数/特征值/自变量”说清楚:}
动量算符的本征矢与特征值由本征方程给出:
\[
\hat{\vec p}\,|\vec p_1\rangle=\vec p_1\,|\vec p_1\rangle.
\]
动量表象中 \(\hat{\vec p}\to \vec p\),因此对应的本征函数满足
\[
\vec p\,\phi_{\vec p_1}(\vec p)=\vec p_1\,\phi_{\vec p_1}(\vec p).
\]
动量算符的本征矢是 \(|\vec p\rangle\),其特征值就是标签 \(\vec p\)。
在动量表象中,对应的本征函数写成
\[
\phi_{\vec p_1}(\vec p)=\langle \vec p|\vec p_1\rangle=\delta(\vec p-\vec p_1),
\]
这里\(\vec p\) 是\textbf{自变量},\(\vec p_1\) 是\textbf{特征值标签}。

\textbf{(3) 用这些基函数算矩阵元(类比第13题的写法):}
为避免把“自变量”与“本征值标签”混在一起,记
\(|\vec p_2\rangle\) 为右侧本征矢(题目中的 \(|\vec p\rangle\))。
则
\[
\langle \vec p_1|\hat L_x|\vec p_2\rangle
=\int d^3p\,\phi_{\vec p_1}^{*}(\vec p)\,
\hat L_x^{(\vec p)}\,\phi_{\vec p_2}(\vec p).
\]
代入 \(\phi_{\vec p_2}(\vec p)=\delta(\vec p-\vec p_2)\) 得
\[
\langle \vec p_1|\hat L_x|\vec p_2\rangle
=\int d^3p\,\delta(\vec p-\vec p_1)\,
\hat L_x^{(\vec p)}\,\delta(\vec p-\vec p_2).
\]
这里用到 \(\delta\) 的取值性质:
\[
\int d^3p\,\delta(\vec p-\vec p_1)F(\vec p)=F(\vec p_1),
\]
令 \(F(\vec p)=\hat L_x^{(\vec p)}\delta(\vec p-\vec p_2)\),即可。
对 \(\vec p\) 积分后
\[
\langle \vec p_1|\hat L_x|\vec p_2\rangle
=\hat L_x^{(\vec p_1)}\delta(\vec p_1-\vec p_2)
= i\hslash\left(p_{1y}\frac{\partial}{\partial p_{1z}}-p_{1z}\frac{\partial}{\partial p_{1y}}\right)
\delta\!\left(\vec p_1-\vec p_2\right).
\]

\textbf{补充说明:}若从“角动量是旋转生成元”出发,
对动量表象波函数有
\((U_R\phi)(\vec p)=\phi(R^{-1}\vec p)\)。
对小旋转作一阶展开同样得到
\(\hat{\vec L}=i\hslash(\vec p\times\nabla_p)\),与上式一致。
\end{quote}

\begin{enumerate}
\def\labelenumi{\arabic{enumi}.}
\setcounter{enumi}{14}
\item
  已知在\(Q\)表象中,态矢为\(\psi\),一个力学量\(F\)在\(Q\)表象中矩阵为\(F_{Q}\),求该力学量\(F\)在自身表象中,处于各个本征态的概率。
\end{enumerate}

\begin{quote}
解答:

\textbf{已知:}在 \(Q\) 表象中,态矢列向量 \(\psi\),算符矩阵 \(F_Q\)。

\textbf{目标:}在 \(F\) 的本征态基底(自身表象)中,各本征态的概率 \(P_n\)。

\textbf{前置知识点:}同一算符在不同基下的矩阵表示相互之间是酉相似变换,
因此\textbf{特征值(谱)与表象无关}。

\begin{enumerate}
\item 已知 \(\psi_Q\) 与 \(F_Q\)。先求 \(F_Q\) 的特征值 \(\lambda_n\);
这些就是算符 \(F\) 的特征值。
\item 用这些特征值组成对角矩阵
\(
F_F=\Lambda=\mathrm{diag}(\lambda_1,\lambda_2,\dots)
\)。
\item 基变换矩阵 \(U\) 把 \(F_Q\) 变为 \(F_F\):
\(
F_F=U^\dagger F_Q U
\)。
因此 \(U\) 就是把 \(F_Q\) 对角化的矩阵(本征向量按列组成)。
\item 用同一个基变换矩阵把态矢变到 \(F\) 表象:
\(
\psi_F=U^\dagger\psi_Q
\)。
\item \(\psi_F\) 是 \(|\psi\rangle\) 在 \(F\) 的本征态基底下的系数列向量,
因此第 \(n\) 个分量的模平方就是概率:
\(
P_n=|(\psi_F)_n|^2
\)。
\end{enumerate}

\textbf{最终概率:}
\[
P_n=|(\psi_F)_n|^2=|(U^\dagger\psi)_n|^2.
\]
若本征值退化,对应多个本征态的概率需相加。

\end{quote}

\begin{enumerate}
\def\labelenumi{\arabic{enumi}.}
\setcounter{enumi}{15}
\item
  已知角动量算符的矩阵为:
\end{enumerate}

\begin{quote}
\[L_{x} = \frac{\hslash\sqrt{2}}{2}\begin{pmatrix}
0 & 1 & 0 \\
1 & 0 & 1 \\
0 & 1 & 0
\end{pmatrix}\]

求角动量算符在自身表象下的矩阵表示:。

算符的本征值方程为:

\[\frac{\hslash\sqrt{2}}{2}\begin{pmatrix}
0 & 1 & 0 \\
1 & 0 & 1 \\
0 & 1 & 0
\end{pmatrix}\psi = \lambda\psi\]

\[\lambda = \frac{\hslash\sqrt{2}}{2}\lambda_{1}\]

\[\left| \begin{matrix}
{- \lambda_{1}} & 1 & 0 \\
1 & {- \lambda_{1}} & 1 \\
0 & 1 & {- \lambda_{1}}
\end{matrix} \right| = \left( {- \lambda_{1}} \right)^{3} + 2\lambda_{1} = 0\]

\[\lambda_{1,1} = 0,\lambda_{1,2} = \sqrt{2},\lambda_{1,3} = - \sqrt{2}\]

因此:

\[\lambda = 0, \quad \lambda = \hslash, \quad \lambda = - \hslash\]

因此测量 $L_x$ 只有\textbf{三种}结果:$\hslash,\,0,\,-\hslash$(不是 1,而是 $\hslash$ 的倍数)。
对应的\textbf{归一化本征向量}(在给定基 $\{|m=+1\rangle,|m=0\rangle,|m=-1\rangle\}$ 下)可取为:
\[
|\!+\rangle_x=\frac{1}{2}\begin{pmatrix}1\\ \sqrt{2}\\ 1\end{pmatrix},\quad
|0\rangle_x=\frac{1}{\sqrt{2}}\begin{pmatrix}1\\ 0\\ -1\end{pmatrix},\quad
|\!-\rangle_x=\frac{1}{2}\begin{pmatrix}1\\ -\sqrt{2}\\ 1\end{pmatrix}.
\]
把它们按列排成矩阵
\[
U=\begin{pmatrix}
\frac{1}{2} & \frac{1}{\sqrt{2}} & \frac{1}{2}\\
\frac{\sqrt{2}}{2} & 0 & -\frac{\sqrt{2}}{2}\\
\frac{1}{2} & -\frac{1}{\sqrt{2}} & \frac{1}{2}
\end{pmatrix},
\]
则
\[
U^\dagger L_x U=\mathrm{diag}(\hslash,\,0,\,-\hslash).
\]
对角元的排列顺序可以互换,只要与对应本征向量的列顺序一致。

\end{quote}

\begin{enumerate}
\def\labelenumi{\arabic{enumi}.}
\setcounter{enumi}{16}
\item
  已知态矢\(\left| \psi \right\rangle = {\sum\limits_{n}^{}{exp\left( {- \frac{1}{2}|\alpha|^{2}} \right)\frac{\alpha^{n}}{\sqrt{n!}}\left| n \right\rangle}}\),求投影算符\(\left| k \right\rangle\left\langle k \right|\)的平均值,其中\(\alpha\)是复数。
\end{enumerate}

\begin{quote}
\[\left\langle k \right.\left| \psi \right\rangle = {\sum\limits_{n}^{}{exp\left( {- \frac{1}{2}|\alpha|^{2}} \right)\frac{\alpha^{n}}{\sqrt{n!}}\left\langle k \right.\left| n \right\rangle}}\]

\textbf{说明:}$|n\rangle$ 与 $|k\rangle$ 属于\textbf{同一组}正交归一基底;
$n$ 是遍历指标,$k$ 是选定的某一个基矢编号(来自投影算符 $|k\rangle\langle k|$)。
因此它们是同一基底上的不同基向量。

\[= {\sum\limits_{n}^{}{exp\left( {- \frac{1}{2}|\alpha|^{2}} \right)\frac{\alpha^{n}}{\sqrt{n!}}\delta_{kn}}} = exp\left( {- \frac{1}{2}|\alpha|^{2}} \right)\frac{\alpha^{k}}{\sqrt{k!}}\]

\[\begin{split}
\left\langle \psi \right.\left| k \right\rangle\left\langle k \right|\left. \psi \right\rangle &= exp\left( {- \frac{1}{2}|\alpha|^{2}} \right)\frac{\left( \alpha^{*} \right)^{k}}{\sqrt{k!}}exp\left( {- \frac{1}{2}|\alpha|^{2}} \right)\frac{\alpha^{k}}{\sqrt{k!}} \\
&= exp\left( {- |\alpha|^{2}} \right)\frac{|\alpha|^{2k}}{k!}
\end{split}\]
\end{quote}

\begin{enumerate}
\def\labelenumi{\arabic{enumi}.}
\setcounter{enumi}{17}
\item
  将下列公式用狄拉克符号表示
\end{enumerate}

\begin{quote}
1) \(F\left( {x,i\hslash\frac{\partial}{\partial x}} \right)\psi\left( {x,t} \right) = \Phi\left( {x,t} \right)\)

2)\(i\hslash\frac{\partial}{\partial t}\psi\left( {x,t} \right) = H\left( {x, - i\hslash\frac{\partial}{\partial x}} \right)\psi\left( {x,t} \right)\)

3)\(H\left( {x, - i\hslash\frac{\partial}{\partial x}} \right)u_{n}(x) = E_{n}u_{n}(x)\)

4) \({\int{u_{m}^{*}(x)u_{n}(x)dx}} = \delta_{mn}\)

5)\(\psi\left( {x,t} \right) = {\sum\limits_{n}^{}{a_{n}(t)u_{n}(x)}}\)

解答

\textbf{题意说明(先看这一段再做转换):}
\begin{itemize}
\item \textbf{狄拉克符号}就是 bra-ket 记号:态用 \(|\psi\rangle\),对偶用 \(\langle\phi|\),内积为 \(\langle\phi|\psi\rangle\),算符作用为 \(A|\psi\rangle\)。
\item \textbf{位置表象}与\textbf{抽象态}的关系:\(\psi(x,t)=\langle x|\psi(t)\rangle\),
      同理 \(\Phi(x,t)=\langle x|\Phi(t)\rangle\)。
\item \textbf{“用狄拉克符号表示”}就是把“在位置表象写成函数方程”的式子,
      翻译成“抽象态/算符”的表达式,必要时可写成
      \(\langle x|A|\psi\rangle\) 的形式(这等价于在位置表象取分量)。
\item \textbf{为什么只写 \(x\):}原题公式都在位置表象,因此自然用 \(|x\rangle\);
      若换到动量表象,则写成 \(\langle p|A|\psi\rangle\) 并用 \(p\) 的表示。
\item \textbf{\(\langle x|A|\psi\rangle\) 与“乘法”的关系:}
      它表示“先让 \(A\) 作用在 \(|\psi\rangle\),再取 \(x\) 表象分量”,即
      \(\langle x|A|\psi\rangle=(A\psi)(x)\)。
      当 \(A=A(x)\) 时,这是乘法:\(\langle x|A|\psi\rangle=A(x)\psi(x)\);
      当 \(A=\hat p\) 时,这是微分:\(\langle x|\hat p|\psi\rangle=-i\hbar\,\partial_x\psi(x)\)。
\end{itemize}

(1)\(F\left( {x,i\hslash\frac{\partial}{\partial x}} \right)\psi\left( {x,t} \right) = \Phi\left( {x,t} \right)\)

\[\langle x|F|\psi\rangle=\langle x|\Phi\rangle\quad\text{或}\quad F|\psi\rangle=|\Phi\rangle\]
\textbf{说明:}$F$ 是算符,$\Phi$ 是算符作用后的新态;
写成 \(\langle x|F|\psi\rangle\) 表示“在位置表象取分量”。 

(2)\(i\hslash\frac{\partial}{\partial t}\psi\left( {x,t} \right) = H\left( {x, - i\hslash\frac{\partial}{\partial x}} \right)\psi\left( {x,t} \right)\)

\[i\hbar\frac{\partial}{\partial t}\left\langle x|\psi\right\rangle=\left\langle x|H|\psi\right\rangle\quad\text{或}\quad i\hbar\frac{\partial}{\partial t}|\psi\rangle=H|\psi\rangle\]
\textbf{说明:}这是含时薛定谔方程,\(H\) 为哈密顿算符;
右式是不带表象的抽象表达。 

(3)\(H\left( {x, - i\hslash\frac{\partial}{\partial x}} \right)u_{n}(x) = E_{n}u_{n}(x)\)

\textbf{原始过程:}
\[
H|n\rangle=E_{n}|n\rangle
\]
左乘 \(\langle x|\) 得
\[
\langle x|H|n\rangle=E_n\langle x|n\rangle.
\]
定义 \(u_n(x)\equiv\langle x|n\rangle\),并在位置表象中写出算符作用:
\[
\langle x|H|n\rangle=\big(H(x,-i\hslash\partial_x)u_n\big)(x),
\]
于是
\[
H\left(x,-i\hslash\frac{\partial}{\partial x}\right)u_n(x)=E_nu_n(x).
\]
\textbf{说明:}这是能量本征方程;\(|n\rangle\) 为能量本征态,\(E_n\) 为本征值。

(4)\({\int{u_{m}^{*}(x)u_{n}(x)dx}} = \delta_{mn}\)

\textbf{原始过程:}
设 \(|u_n\rangle\) 表示这组正交归一基(也常写作 \(|n\rangle\)),
其位置表象分量为 \(u_n(x)=\langle x|u_n\rangle\),则
\[
\int u_m^*(x)u_n(x)\,dx
=\int \langle u_m|x\rangle\langle x|u_n\rangle\,dx
=\langle u_m|\left(\int |x\rangle\langle x|\,dx\right)|u_n\rangle
=\langle u_m|u_n\rangle.
\]
因此
\[
\langle u_m|u_n\rangle=\delta_{mn}.
\]
\textbf{说明:}正交归一在抽象态中就是
\(\langle u_m|u_n\rangle=\delta_{mn}\)(或写作 \(\langle m|n\rangle=\delta_{mn}\)),
取位置表象就变成上面的积分形式。

(5)\(\psi\left( {x,t} \right) = {\sum\limits_{n}^{}{a_{n}(t)u_{n}(x)}}\)

\textbf{原始过程:}
在正交归一基 \(\{|n\rangle\}\) 下插入完备关系
\[
\sum_n |n\rangle\langle n|=I,
\]
于是
\[
|\psi\rangle=\left(\sum_n |n\rangle\langle n|\right)|\psi\rangle
=\sum_n |n\rangle\langle n|\psi\rangle.
\]
定义系数
\[
a_n(t)\equiv \langle n|\psi(t)\rangle,
\]
则
\[
|\psi(t)\rangle=\sum_n a_n(t)|n\rangle.
\]
取位置表象分量:
\[
\psi(x,t)=\langle x|\psi(t)\rangle
=\sum_n \langle x|n\rangle\,\langle n|\psi(t)\rangle
=\sum_n a_n(t)\,u_n(x).
\]
\end{quote}

\begin{enumerate}
\def\labelenumi{\arabic{enumi}.}
\setcounter{enumi}{18}
\item
  已知一个算符\(\hat{F}\)在A表象中的矩阵表示为\(F^{a}\),在B表象中的矩阵表示为\(F^{b}\),求A表象和B表象的转换矩阵\(S\)。
\end{enumerate}

\begin{quote}

解答:
\textbf{按第15题同样思路(矩阵形式结论):}
\begin{enumerate}
\item 已知 \(F^{a}\) 与 \(F^{b}\),它们表示同一算符在两种表象下的矩阵,
满足
\[
F^{b}=S^\dagger F^{a} S.
\]
\item 对角化得到
\[
F^{a}=A\Lambda A^\dagger,\qquad
F^{b}=B\Lambda B^\dagger,
\]
其中 \(\Lambda\) 为同一特征值对角阵,\(A,B\) 由各自的本征向量组成。
\item 写出基变换关系并代入:
\[
F^{b}=S^\dagger F^{a} S.
\]
将 \(F^{a}=A\Lambda A^\dagger\) 和 \(F^{b}=B\Lambda B^\dagger\) 代入得
\[
B\Lambda B^\dagger=S^\dagger A\Lambda A^\dagger S.
\]
左乘 \(B^\dagger\)、右乘 \(B\):
\[
\Lambda=(B^\dagger S^\dagger A)\,\Lambda\,(A^\dagger S B).
\]
令
\[
X\equiv A^\dagger S B,
\]
则有
\[
\Lambda=X^\dagger\Lambda X.
\]
由于 \(\Lambda\) 为对角阵,最简单的满足方式是取
\(
X=I
\)(更一般地,若有简并,\(X\) 可在简并子空间内取任意酉矩阵)。
因此
\[
A^\dagger S B=I\quad\Rightarrow\quad S=AB^\dagger.
\]
\end{enumerate}

\end{quote}

\end{document}
