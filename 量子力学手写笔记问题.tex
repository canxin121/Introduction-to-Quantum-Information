\documentclass[12pt]{article}

\usepackage[a4paper,margin=2.2cm]{geometry}
\usepackage{amsmath,amssymb}
\usepackage{physics}
\usepackage{enumitem}
\usepackage{xeCJK}
\usepackage{xcolor}
\usepackage{hyperref}

\setCJKmainfont{Source Han Serif CN}
\setCJKsansfont{Source Han Sans CN}
\setCJKmonofont{Source Han Sans CN}
\setlist[itemize]{leftmargin=*,nosep}
\setlist[enumerate]{leftmargin=*,nosep}
\hypersetup{
  colorlinks=true,
  linkcolor=blue!55!black,
  urlcolor=blue!55!black,
  citecolor=blue!55!black
}

\title{量子力学手写笔记问题(默写版)}
\author{}
\date{}

\begin{document}
\maketitle
\tableofcontents
\newpage

\section{常用积分与体积元}

\subsection{高斯积分与指数积分}
\begin{enumerate}
  \item 计算高斯积分:$\displaystyle \int_{-\infty}^{\infty} e^{-a x^2}\,dx$($a>0$)。
  \item 计算高斯积分:$\displaystyle \int_{-\infty}^{\infty} x e^{-a x^2}\,dx$($a>0$)。
  \item 计算高斯积分:$\displaystyle \int_{-\infty}^{\infty} x^2 e^{-a x^2}\,dx$($a>0$)。
  \item 写出指数型积分:$\displaystyle \int_{0}^{\infty} x^n e^{-a x}\,dx$($a>0,\ n=0,1,2,\dots$)。
\end{enumerate}

\subsection{分部积分}
\begin{enumerate}
  \item 写出分部积分公式。
  \item 写出常见写法:$\displaystyle \int_a^b g(x)\,df(x)$ 的等价形式。
  \item 写出 $f,g$ 可微时的“把导数项借位”公式。
  \item 写出多变量情形下对某一变量分部积分的公式。
\end{enumerate}

\subsection{球坐标体积元与径向拉普拉斯}
\begin{enumerate}
  \item 写出球坐标体积元 $dV$。
  \item 写出径向函数 $f(r)$ 的拉普拉斯 $\nabla^2 f(r)$。
\end{enumerate}

\subsection{常见边界与归一化}
\begin{enumerate}
  \item 写出束缚态(无限域)的边界条件与归一化条件。
  \item 写出一维无限深势阱 $0<x<a$ 的边界条件与归一化条件。
\end{enumerate}

\section{$\delta$ 函数与向量分析}

\subsection{$\delta$ 函数的换元公式}
\begin{enumerate}
  \item 若 $g(x_i)=0$ 且 $g'(x_i)\neq 0$,写出 $\delta(g(x))$ 的换元公式。
\end{enumerate}

\subsection{梯度、散度、旋度与叉乘}
\begin{enumerate}
  \item 写出 $\nabla f$ 与算符 $\nabla$ 的分量表达式。
  \item 写出散度与旋度的符号表示。
  \item 写出向量叉乘 $\vec a \times \vec b$ 的行列式形式。
\end{enumerate}

\section{对易关系与表象}

\subsection{基本对易关系}
\begin{enumerate}
  \item 写出 $x_i,p_j$ 的三条基本对易关系。
  \item 写出角动量算符 $L_x,L_y,L_z$ 的 6 个对易关系。
\end{enumerate}

\subsection{对易子的性质}
\begin{enumerate}
  \item 写出对易子的定义与两条常用性质。
\end{enumerate}

\subsection{位置与动量表象}
\begin{enumerate}
  \item 写出位置表象中 $\hat x,\hat p$ 的表示。
  \item 写出动量表象中 $\hat p,\hat x$ 的表示。
\end{enumerate}

\subsection{矩阵元(常见例子)}
\begin{enumerate}
  \item 一维无限深势阱能量表象中,求 $x_{mn}=\langle m|\hat x|n\rangle$ 与 $p_{mn}=\langle m|\hat p|n\rangle$,分别给出 $m=n$ 与 $m\ne n$ 的结果。
  \item 在动量表象中,写出 $\hat L_x^{(p)}$,并给出 $\langle \vec p_1|\hat L_x|\vec p_2\rangle$ 的表达式(写成算符作用于 $\delta$ 函数的形式)。
\end{enumerate}

\subsection{概率密度与概率流}
\begin{enumerate}
  \item 写出概率密度 $\rho$ 与概率流 $\vec j$ 的表达式。
  \item 写出连续性方程。
\end{enumerate}

\section{薛定谔方程与定态解}

\subsection{四大算符}
\begin{enumerate}
  \item 写出动能、势能、哈密顿、能量四大算符。
\end{enumerate}

\subsection{薛定谔方程}
\begin{enumerate}
  \item 写出薛定谔方程 $\hat E\,\psi=\hat H\psi$。
  \item 写出时间无关势下的定态解形式与展开式。
  \item 写出能量期望值 $\langle H\rangle$(或 $\langle E\rangle$)的表达式。
\end{enumerate}

\subsection{一维无限深势阱($0<x<a$)}
\begin{enumerate}
  \item 写出势阱与边界条件。
  \item 写出定态本征函数 $u_n(x)$ 与本征能量 $E_n$。
  \item 写出展开系数 $C_n$ 的表达式。
\end{enumerate}

\section{氢原子基态}

\subsection{库仑势与基态波函数}
\begin{enumerate}
  \item 写出库仑势 $V(r)$。
  \item 写出基态($1s$)波函数 $\psi_{100}(r)$。
\end{enumerate}

\subsection{基态期望值}
\begin{enumerate}
  \item 写出 $\langle r\rangle$ 与 $\left\langle\frac{1}{r}\right\rangle$。
  \item 写出 $\langle V\rangle$ 与 $\langle T\rangle$ 的表达式,并给出 $\langle T\rangle$ 与 $E_1$ 的关系。
\end{enumerate}

\section{期望值随时间的演化(Ehrenfest)}
\begin{enumerate}
  \item 写出期望值定义 $\langle A\rangle$ 与薛定谔方程。
  \item 写出 Ehrenfest 定理:$d\langle A\rangle/dt$ 的一般表达式。
  \item 写出常见结果:$d\langle x\rangle/dt$ 与 $d\langle p\rangle/dt$。
\end{enumerate}

\section{本征值与本征函数}

\subsection{位置算符}
\begin{enumerate}
  \item 写出位置算符的本征方程,并指出对应本征函数与本征值。
\end{enumerate}

\subsection{动量算符}
\begin{enumerate}
  \item 写出动量算符的本征方程。
  \item 写出本征函数 $f_p(x)$ 的形式与归一化常数。
\end{enumerate}

\section{不确定关系}
\begin{enumerate}
  \item 写出一般不确定关系 $\Delta A\,\Delta B$。
  \item 写出位置与动量的不确定关系。
  \item 写出方差定义与 $\Delta x$ 的表达式。
  \item 写出时间与频率(能量)的不确定关系。
  \item 写出达到等号时的物理例子。
\end{enumerate}

\end{document}
