\documentclass[12pt]{article}

\usepackage[a4paper,margin=2.2cm]{geometry}
\usepackage{amsmath,amssymb}
\usepackage{booktabs}
\usepackage{enumitem}
\usepackage{xcolor}
\usepackage{graphicx}
\usepackage{hyperref}
\usepackage{xeCJK}

\graphicspath{{assets/figs_used/}}

\setCJKmainfont{Source Han Serif CN}
\setCJKsansfont{Source Han Sans CN}
\setCJKmonofont{Source Han Sans CN}
\setlist[itemize]{leftmargin=*,nosep}
\setlist[enumerate]{leftmargin=*,nosep}
\hypersetup{
  colorlinks=true,
  linkcolor=blue!55!black,
  urlcolor=blue!55!black,
  citecolor=blue!55!black
}

% 定义示例环境
\usepackage{tcolorbox}
\tcbuselibrary{breakable}
\newtcolorbox{example}[1][]{
  colback=blue!5!white,
  colframe=blue!75!black,
  title=\textbf{例题},
  breakable,
  #1
}

\newtcolorbox{derivation}[1][]{
  colback=green!5!white,
  colframe=green!75!black,
  title=\textbf{推导},
  breakable,
  #1
}

\title{量子信息导论:模块1——数学基础(精简重构版)}
\author{}
\date{}

\begin{document}
\maketitle

\section*{【了解】希腊字母读音表}
\begin{center}
\begin{tabular}{llll}
\toprule
\textbf{大写} & \textbf{小写} & \textbf{英文名} & \textbf{读音(汉语拼音)} \\
\midrule
$\Psi$ & $\psi$ & psi & sai(赛)\\
$\Phi$ & $\phi, \varphi$ & phi & fai(fei)\\
$\Theta$ & $\theta$ & theta & sita(西塔)\\
$\Lambda$ & $\lambda$ & lambda & lamuda(兰姆达)\\
$\Sigma$ & $\sigma$ & sigma & xigema(西格玛)\\
$\Omega$ & $\omega$ & omega & omiga(欧米伽)\\
$\Delta$ & $\delta$ & delta & delta(德尔塔)\\
$\Gamma$ & $\gamma$ & gamma & gama(伽玛)\\
$\Pi$ & $\pi$ & pi & pai(派)\\
-- & $\alpha$ & alpha & aerfǎ(阿尔法)\\
-- & $\beta$ & beta & beita(贝塔)\\
-- & $\gamma$ & gamma & gama(伽玛)\\
-- & $\epsilon, \varepsilon$ & epsilon & yipuxilóng(伊普西龙)\\
-- & $\eta$ & eta & yita(伊塔)\\
-- & $\kappa$ & kappa & kapa(卡帕)\\
-- & $\mu$ & mu & miù(谬)\\
-- & $\nu$ & nu & niǔ(纽)\\
-- & $\rho$ & rho & ròu(肉)\\
-- & $\tau$ & tau & tao(涛)\\
-- & $\chi$ & chi & kǎi(凯)\\
\bottomrule
\end{tabular}
\end{center}

\paragraph{常见符号}
\begin{itemize}
  \item $|\psi\rangle$:量子态(ket),读作"ket-sai"或"态-赛"
  \item $\langle\phi|$:对偶态(bra),读作"bra-fei"或"对偶-fei"
  \item $\langle\phi|\psi\rangle$:内积,读作"bra-fei-ket-sai"
  \item $\lambda$:本征值(eigenvalue),读作"兰姆达"
  \item $\sigma$:Pauli矩阵,读作"西格玛"
  \item $\dagger$:厄米共轭(Hermitian conjugate),读作"dagger"或"匕首"
\end{itemize}

\section*{【了解】学习目标}
本模块专注于量子信息的数学语言。完成学习后,你应能:
\begin{itemize}
  \item 熟练进行复数运算(模、共轭、指数形式)
  \item 掌握向量/矩阵的基本运算和Dirac记号
  \item 理解线性组合、线性无关、基与维数
  \item 理解内积、正交归一基、完备关系
  \item 理解基变换与酉矩阵的关系
  \item 熟悉矩阵常用操作(转置、迹、行列式、逆)
  \item 计算厄米/酉矩阵的本征值与本征向量
  \item 使用张量积描述复合量子系统
  \item 判断算符是否对易或反对易并理解其物理意义
  \item 计算基本测量概率并理解 Born 规则
  \item 使用密度算子描述混态并进行偏迹
\end{itemize}

\tableofcontents

\section{【背】复数(Complex Numbers)}

\subsection{【背】定义与基本运算}

\paragraph{定义} 复数 $z=x+iy$,其中 $x,y\in\mathbb{R}$,$i^2=-1$。

\paragraph{核心运算}
\begin{itemize}
  \item \textbf{复共轭}:$z^\ast=x-iy$
  \item \textbf{模}:$|z|=\sqrt{x^2+y^2}=\sqrt{zz^\ast}$
  \item \textbf{逆}:$\displaystyle z^{-1}=\frac{z^\ast}{|z|^2}=\frac{x-iy}{x^2+y^2}$
  \item \textbf{指数形式}:$z=|z|e^{i\theta}$,其中 $\theta=\arg(z)=\arctan(y/x)$
\end{itemize}

\paragraph{重要性质}
\begin{align}
  (z^\ast)^\ast &= z \\
  (z+w)^\ast &= z^\ast+w^\ast \\
  (z-w)^\ast &= z^\ast-w^\ast \\
  (zw)^\ast &= z^\ast w^\ast \\
  \left(\frac{z}{w}\right)^\ast &= \frac{z^\ast}{w^\ast}\quad (w\neq 0) \\
  |zw| &= |z|\,|w| \\
  |e^{i\theta}| &= 1 \\
  (e^{i\theta})^\ast &= e^{-i\theta}
\end{align}

\paragraph{几何表示与欧拉公式}
复数 $z=x+iy$ 对应复平面上一点 $(x,y)$,模长 $|z|$ 为到原点距离,幅角为 $\theta=\arg z$。
欧拉公式给出
\[
e^{i\theta}=\cos\theta+i\sin\theta
\]
从而
\[
z=|z|(\cos\theta+i\sin\theta)=|z|e^{i\theta}
\]
由此得到 De Moivre 公式:
\[
(re^{i\theta})^n=r^n e^{in\theta}
\]

\begin{example}
\textbf{计算 $(1+i)^8$}

\textbf{解:}
\begin{enumerate}
  \item 写成指数形式:$1+i = \sqrt{2}\,e^{i\pi/4}$(因为 $|1+i|=\sqrt{2}$,$\arg(1+i)=\pi/4$)
  \item 计算幂次:
  \[
  (1+i)^8 = (\sqrt{2})^8\,e^{i8\pi/4} = 2^4\,e^{i2\pi} = 16\cdot 1 = 16
  \]
\end{enumerate}
\end{example}

\begin{example}
\textbf{求 $z=3+4i$ 的逆}

\textbf{解:}
\begin{enumerate}
  \item $|z|^2=3^2+4^2=25$
  \item $z^{-1}=\dfrac{3-4i}{25}=\dfrac{3}{25}-\dfrac{4}{25}i$
\end{enumerate}
\end{example}

\section{【背】向量空间与Dirac记号}

\paragraph{Dirac(狄拉克)记号}
也称 bra-ket 记号,用 $\langle\cdot|$ 与 $|\cdot\rangle$ 统一书写向量与对偶向量,便于表达内积与算符作用。

\paragraph{希尔伯特空间}
量子态所在的数学空间是\textbf{复内积空间},通常称为希尔伯特空间。有限维情形下可以等同于 $\mathbb{C}^n$,其向量、内积与线性算符的运算规则与线性代数一致。

\subsection{【背】向量与矩阵}

\paragraph{列向量(ket)}
\[
|\psi\rangle = \begin{pmatrix} a_1 \\ a_2 \\ \vdots \\ a_n \end{pmatrix}, \quad a_i\in\mathbb{C}
\]

\paragraph{行向量(bra)}
\[
\langle\psi| = (|\psi\rangle)^\dagger = \begin{pmatrix} a_1^\ast & a_2^\ast & \cdots & a_n^\ast \end{pmatrix}
\]

注:$\dagger$ 表示厄米共轭(conjugate transpose),即先复共轭再转置。

\paragraph{常用 ket(计算基与相关基)}
单比特计算基:
\[
|0\rangle=\begin{pmatrix}1\\0\end{pmatrix},\qquad
|1\rangle=\begin{pmatrix}0\\1\end{pmatrix}
\]
Hadamard(哈达玛)基:
\[
|+\rangle=\frac{|0\rangle+|1\rangle}{\sqrt{2}},\qquad
|-\rangle=\frac{|0\rangle-|1\rangle}{\sqrt{2}}
\]
$Y$ 基:
\[
|+i\rangle=\frac{|0\rangle+i|1\rangle}{\sqrt{2}},\qquad
|-i\rangle=\frac{|0\rangle-i|1\rangle}{\sqrt{2}}
\]
常用两比特计算基(张量积):
\[
|00\rangle=|0\rangle\otimes|0\rangle,\ 
|01\rangle=|0\rangle\otimes|1\rangle,\ 
|10\rangle=|1\rangle\otimes|0\rangle,\ 
|11\rangle=|1\rangle\otimes|1\rangle
\]
\paragraph{常用纠缠态(Bell 态)}
\[
|\Phi^+\rangle=\frac{|00\rangle+|11\rangle}{\sqrt{2}},\quad
|\Phi^-\rangle=\frac{|00\rangle-|11\rangle}{\sqrt{2}}
\]
\[
|\Psi^+\rangle=\frac{|01\rangle+|10\rangle}{\sqrt{2}},\quad
|\Psi^-\rangle=\frac{|01\rangle-|10\rangle}{\sqrt{2}}
\]
\paragraph{常用三比特计算基}
\[
|000\rangle,\ |001\rangle,\ |010\rangle,\ |011\rangle,\ |100\rangle,\ |101\rangle,\ |110\rangle,\ |111\rangle
\]

\paragraph{矩阵乘法} 若 $A$ 为 $m\times n$ 矩阵,$B$ 为 $n\times p$ 矩阵,则
\[
(AB)_{ij} = \sum_{k=1}^n A_{ik}B_{kj}
\]

\paragraph{矩阵的基本操作}
\begin{itemize}
\item \textbf{转置}:$A^T$,满足 $(A^T)_{ij}=A_{ji}$
\item \textbf{共轭}:$A^\ast$,满足 $(A^\ast)_{ij}=(A_{ij})^\ast$
\item \textbf{迹}:$\mathrm{tr}(A)=\sum_i A_{ii}$,满足 $\mathrm{tr}(AB)=\mathrm{tr}(BA)$
\item \textbf{行列式}:$\det(A)$,若 $\det(A)\neq 0$ 则 $A$ 可逆
\item \textbf{不可交换性}:一般有 $AB\neq BA$
\end{itemize}

\paragraph{行列式与可逆性(补充)}\label{para:det-invertibility}
齐次方程 $M x = 0$ 有非零解 $\Leftrightarrow$ $M$ 不可逆。
线性代数结论:$M$ 可逆 $\Leftrightarrow$ $\det(M)\neq 0$,因此
\[
Mx=0 \text{ 有非零解 }\Leftrightarrow \det(M)=0
\]
行列式可理解为矩阵对体积的缩放因子:若 $\det\neq 0$,体积被缩放且矩阵可逆;若 $\det=0$,体积被压扁到更低维,矩阵不可逆,因此才可能有非零解。

\paragraph{不可交换性的几何直觉}
二维空间中“先旋转后剪切”与“先剪切后旋转”通常给出不同结果,
反映了矩阵乘法的顺序依赖性。

\begin{figure}[htbp]
  \centering
  \includegraphics[width=0.45\linewidth]{fig_vector_rotation.png}
  \caption{二维向量旋转/线性变换的几何直观示意。}
\end{figure}

\paragraph{常见矩阵类型}
\begin{itemize}
  \item \textbf{对角矩阵}:非对角元为 $0$,对角矩阵之间可交换
  \item \textbf{单位矩阵}:$I=\mathrm{diag}(1,1,\ldots,1)$
  \item \textbf{对称矩阵}:$A=A^T$(实矩阵情形)
  \item \textbf{逆矩阵}:$AA^{-1}=A^{-1}A=I$(并非所有矩阵都有逆)
\end{itemize}

\paragraph{对角矩阵的投影分解(补充)}\label{para:diag-projector}
设标准基 $\{|e_j\rangle\}$ 为"第 $j$ 个分量为 1,其余为 0"的向量,则任意对角矩阵
\[
D=\mathrm{diag}(d_1,\ldots,d_n)
\]
可写成
\[
D=\sum_{j=1}^n d_j\,|e_j\rangle\langle e_j|
\]
\textbf{理由:}对任意 $k$,有 $D|e_k\rangle=d_k|e_k\rangle$,而
\[
\left(\sum_{j} d_j|e_j\rangle\langle e_j|\right)|e_k\rangle
=\sum_j d_j|e_j\rangle\langle e_j|e_k\rangle
=d_k|e_k\rangle
\]
因此两算符对标准基作用相同,故相等。

\textbf{数值例子:}取
\[
D=\mathrm{diag}(2,3)=\begin{pmatrix}2&0\\0&3\end{pmatrix},\quad
|e_1\rangle=\begin{pmatrix}1\\0\end{pmatrix},\ |e_2\rangle=\begin{pmatrix}0\\1\end{pmatrix}
\]
则
\[
|e_1\rangle\langle e_1|=\begin{pmatrix}1&0\\0&0\end{pmatrix},\quad
|e_2\rangle\langle e_2|=\begin{pmatrix}0&0\\0&1\end{pmatrix}
\]
所以
\[
2|e_1\rangle\langle e_1|+3|e_2\rangle\langle e_2|
=\begin{pmatrix}2&0\\0&3\end{pmatrix}=D.
\]

\subsection{【背】线性空间的基本概念}

\paragraph{线性组合与张成}\label{para:span}
给定向量集合 $\{|v_1\rangle,\ldots,|v_k\rangle\}$,其线性组合为
\[
\sum_{j=1}^k c_j |v_j\rangle, \quad c_j\in\mathbb{C}
\]
所有线性组合的集合称为该集合的\textbf{张成}(span)。

\paragraph{索引集(补充)}\label{para:index-set}
索引集是\textbf{用来标记一族对象}的集合,例如 $\{v_k\}_{k\in\mathcal{I}}$ 表示“用 $k\in\mathcal{I}$ 来编号”的向量族。
这里的 $\mathcal{I}$(有时也写作 $I$)就是\textbf{索引集合本身},它可以是任意集合,常见如有限情形 $\{1,\ldots,n\}$ 或无限情形 $\mathbb{N}$。
为避免与本文中的单位算符 $I$ 混淆,这里通常写作 $\mathcal{I}$。
例如取 $\mathcal{I}=\{1,2,3\}$,则 $\{v_k\}_{k\in\mathcal{I}}=\{v_1,v_2,v_3\}$,表示用 1、2、3 给这三个向量编号。

\paragraph{线性无关与线性相关}
若
\[
\sum_{j=1}^k c_j |v_j\rangle = 0 \Rightarrow c_1=\cdots=c_k=0
\]
则 $\{|v_j\rangle\}$ 线性无关;否则线性相关。

\paragraph{基与维数}
一组向量若既线性无关又能张成整个空间,则称为\textbf{基}。基向量个数称为\textbf{维数}。

\paragraph{坐标表示}\label{para:coordinate-representation}
在基 $\{|e_i\rangle\}$ 下,任意向量可表示为
\[
|\psi\rangle=\sum_i c_i|e_i\rangle
\]
若基为正交归一,则 $c_i=\langle e_i|\psi\rangle$。

\subsection{【背】厄米共轭(Hermitian Conjugate)}

\paragraph{定义}
对于任意矩阵或算符 $A$,其\textbf{厄米共轭}(或称共轭转置)记作 $A^\dagger$。
若 $A$ 表示 $A$ 在某一基下的矩阵表示,则矩阵元满足
\[
(A^\dagger)_{ij} = (A_{ji})^\ast
\]

注意:对于向量,$\langle\psi| = (|\psi\rangle)^\dagger$ 就是将列向量做厄米共轭得到行向量。

\paragraph{重要运算性质}
\begin{enumerate}
  \item \textbf{双重共轭}:$(A^\dagger)^\dagger = A$
  \item \textbf{线性性质}:
  \begin{itemize}
    \item $(cA)^\dagger = c^\ast A^\dagger$(其中 $c$ 为复数)
    \item $(A+B)^\dagger = A^\dagger + B^\dagger$
  \end{itemize}
  \item \textbf{乘积的共轭}:$(AB)^\dagger = B^\dagger A^\dagger$(注意顺序反转)
  \item \textbf{内积关系}:$\langle\phi|A|\psi\rangle = \langle A^\dagger\phi|\psi\rangle = \langle\psi|A^\dagger|\phi\rangle^\ast$
\end{enumerate}

\begin{derivation}
\textbf{内积关系的详细推导:}
把态写成列向量,则 $\langle\phi|=(|\phi\rangle)^\dagger=\phi^\dagger$,
\[
\langle\phi|A|\psi\rangle = \phi^\dagger A \psi
\]
而伴随算符满足 $(A^\dagger \phi)^\dagger = \phi^\dagger A$,因此
\[
\phi^\dagger A \psi = (A^\dagger \phi)^\dagger \psi = \langle A^\dagger \phi|\psi\rangle
\]
\textbf{关于共轭:}对任意内积都有 $\langle u|v\rangle^\ast=\langle v|u\rangle$,于是
\[
(\langle A^\dagger \phi|\psi\rangle)^\ast
= \langle \psi|A^\dagger \phi\rangle
= \langle \psi|A^\dagger|\phi\rangle
\]
从而得到
\[
\langle\phi|A|\psi\rangle = \langle A^\dagger\phi|\psi\rangle = \langle\psi|A^\dagger|\phi\rangle^\ast
\]
\end{derivation}

\begin{example}
\textbf{验证乘积的共轭性质}

设 $A=\begin{pmatrix}1&i\\0&1\end{pmatrix}$,$B=\begin{pmatrix}2&0\\1&3\end{pmatrix}$,验证 $(AB)^\dagger = B^\dagger A^\dagger$。

\textbf{解:}
\begin{align*}
AB &= \begin{pmatrix}1&i\\0&1\end{pmatrix}\begin{pmatrix}2&0\\1&3\end{pmatrix} = \begin{pmatrix}2+i&3i\\1&3\end{pmatrix} \\
(AB)^\dagger &= \begin{pmatrix}2-i&1\\-3i&3\end{pmatrix}
\end{align*}

另一方面:
\begin{align*}
A^\dagger &= \begin{pmatrix}1&0\\-i&1\end{pmatrix}, \quad B^\dagger = \begin{pmatrix}2&1\\0&3\end{pmatrix} \\
B^\dagger A^\dagger &= \begin{pmatrix}2&1\\0&3\end{pmatrix}\begin{pmatrix}1&0\\-i&1\end{pmatrix} = \begin{pmatrix}2-i&1\\-3i&3\end{pmatrix}
\end{align*}

因此 $(AB)^\dagger = B^\dagger A^\dagger$ \checkmark
\end{example}

\subsection{【背】内积(Inner Product)}

\paragraph{定义}
\[
\langle\phi|\psi\rangle = \sum_{k=1}^n a_k^\ast b_k \in \mathbb{C}
\]
其中 $|\phi\rangle=\begin{pmatrix}a_1\\\vdots\\a_n\end{pmatrix}$,$|\psi\rangle=\begin{pmatrix}b_1\\\vdots\\b_n\end{pmatrix}$。

\paragraph{基本性质}
\begin{enumerate}
  \item \textbf{共轭对称}:$\langle\phi|\psi\rangle = \langle\psi|\phi\rangle^\ast$
  \item \textbf{对第二变量线性}:$\langle\phi|(c_1|\psi_1\rangle+c_2|\psi_2\rangle) = c_1\langle\phi|\psi_1\rangle + c_2\langle\phi|\psi_2\rangle$
  \item \textbf{对第一变量共轭线性}:$\langle c_1\phi_1+c_2\phi_2|\psi\rangle = c_1^\ast\langle\phi_1|\psi\rangle + c_2^\ast\langle\phi_2|\psi\rangle$
  \item \textbf{正定性}:$\langle\psi|\psi\rangle \ge 0$,且 $\langle\psi|\psi\rangle=0 \Leftrightarrow |\psi\rangle=0$
\end{enumerate}

\paragraph{夹心表达式(矩阵元)性质}
给定算符 $A$,夹心表达式(矩阵元)
\[
\langle\phi|A|\psi\rangle
\]
满足:
\begin{itemize}
  \item \textbf{对 ket 线性}:$\langle\phi|A(c_1|\psi_1\rangle+c_2|\psi_2\rangle)=c_1\langle\phi|A|\psi_1\rangle+c_2\langle\phi|A|\psi_2\rangle$
  \item \textbf{对 bra 共轭线性}:$\langle c_1\phi_1+c_2\phi_2|A|\psi\rangle=c_1^\ast\langle\phi_1|A|\psi\rangle+c_2^\ast\langle\phi_2|A|\psi\rangle$
  \item \textbf{共轭关系}:
  \[
  \big(\langle\phi|A|\psi\rangle\big)^\ast=\langle\psi|A^\dagger|\phi\rangle
  \]
  特别地,若 $A$ 为厄米算符($A^\dagger=A$),则
  \[
  \langle\psi|A|\psi\rangle\in\mathbb{R}
  \]
  \textbf{推导:}
  \[
  \big(\langle\psi|A|\psi\rangle\big)^\ast
  =\langle\psi|A^\dagger|\psi\rangle
  =\langle\psi|A|\psi\rangle
  \]
  因此它等于自身的复共轭,必为实数。
\end{itemize}

\paragraph{范数(norm)与归一化}
\begin{itemize}
  \item 范数:$\||\psi\rangle\| = \sqrt{\langle\psi|\psi\rangle}$
  \item 归一化:若 $\langle\psi|\psi\rangle=1$,称 $|\psi\rangle$ 已归一化
  \item 归一化方法:$\displaystyle |\tilde{\psi}\rangle = \frac{|\psi\rangle}{\||\psi\rangle\|}$
\end{itemize}

\begin{example}
\textbf{归一化向量 $|\psi\rangle=\begin{pmatrix}1\\i\\1\end{pmatrix}$}

\textbf{解:}
\begin{enumerate}
  \item 计算内积:
  \[
  \langle\psi|\psi\rangle = 1^\ast\cdot 1 + i^\ast\cdot i + 1^\ast\cdot 1 = 1 + (-i)(i) + 1 = 1+1+1 = 3
  \]
  \item 归一化:
  \[
  |\tilde{\psi}\rangle = \frac{1}{\sqrt{3}}\begin{pmatrix}1\\i\\1\end{pmatrix}
  \]
  \item 验证:$\langle\tilde{\psi}|\tilde{\psi}\rangle = \frac{1}{3}\cdot 3 = 1$ \checkmark
\end{enumerate}
\end{example}

\subsection{【背】外积(Outer Product)与投影}

\paragraph{定义}
\[
|u\rangle\langle v| \quad \text{是一个矩阵(算符)}
\]

\paragraph{作用规则}
\[
(|u\rangle\langle v|)\,|w\rangle = |u\rangle\,\langle v|w\rangle
\]
注意:$\langle v|w\rangle$ 是一个复数,所以结果是 $|u\rangle$ 乘以这个复数。

\paragraph{矩阵元形式}
若 $|u\rangle=(u_1,\ldots,u_n)^T$,$|v\rangle=(v_1,\ldots,v_n)^T$,则
\[
[\,|u\rangle\langle v|\,]_{ij}=u_i v_j^\ast
\]
这里的共轭来自 bra:$\langle v|=(|v\rangle)^\dagger=\sum_j v_j^\ast\langle j|$,
所以矩阵元中对 $v_j$ 取共轭,而不是对 $u_i$ 取共轭。

\paragraph{投影算符} 若 $|\psi\rangle$ 已归一化,则
\[
P_\psi = |\psi\rangle\langle\psi|
\]
满足 $P_\psi^2=P_\psi$(幂等性)和 $P_\psi^\dagger=P_\psi$(厄米性)。

\subsection{【背】正交归一基(Orthonormal Basis)}\label{subsec:onb}

\paragraph{Kronecker delta符号}
定义 Kronecker delta(克罗内克 delta)符号:
\[
\delta_{ij} = \begin{cases} 1, & i=j \\ 0, & i\neq j \end{cases}
\]
这是一个非常常用的记号,用来简洁地表示"相等时为1,不等时为0"。

\paragraph{正交与归一化}
\begin{itemize}
  \item \textbf{正交}:若 $\langle\phi|\psi\rangle=0$,则称 $|\phi\rangle$ 与 $|\psi\rangle$ 正交
  \item \textbf{归一化}:若 $\langle\psi|\psi\rangle=1$,则称 $|\psi\rangle$ 已归一化
\end{itemize}

\paragraph{正交归一基(ONB)的定义}
一组向量 $\{|e_1\rangle,\ldots,|e_n\rangle\}$ 称为\textbf{正交归一基},如果它们满足:
\[
\boxed{\langle e_i|e_j\rangle = \delta_{ij}}
\]
即:
\begin{itemize}
  \item 当 $i\neq j$ 时,$\langle e_i|e_j\rangle=0$(互相正交)
  \item 当 $i=j$ 时,$\langle e_i|e_i\rangle=1$(各自归一化)
\end{itemize}

\paragraph{单位算符(恒等算符)}
单位算符 $I$(Identity operator)满足:对任意向量 $|\psi\rangle$,都有
\[
I|\psi\rangle = |\psi\rangle
\]
在有限维空间 $\mathbb{C}^n$ 中,$I$ 就是 $n\times n$ 单位矩阵:
\[
I = \begin{pmatrix}
1 & 0 & \cdots & 0 \\
0 & 1 & \cdots & 0 \\
\vdots & \vdots & \ddots & \vdots \\
0 & 0 & \cdots & 1
\end{pmatrix}
\]

\paragraph{完备关系(单位分解)}
若 $\{|e_1\rangle,\ldots,|e_n\rangle\}$ 是正交归一基,则单位算符可以表示为:
\[
\boxed{I = \sum_{j=1}^n |e_j\rangle\langle e_j|}
\]
这个公式称为\textbf{完备关系}或\textbf{单位分解}。

\begin{derivation}
\textbf{证明:}对任意 $|\psi\rangle$,设 $|\psi\rangle=\sum_k c_k|e_k\rangle$,则
\begin{align*}
\left(\sum_j |e_j\rangle\langle e_j|\right)|\psi\rangle
&= \sum_j |e_j\rangle\langle e_j|\left(\sum_k c_k|e_k\rangle\right) \\
&= \sum_{j,k} c_k|e_j\rangle\langle e_j|e_k\rangle \\
&= \sum_{j,k} c_k|e_j\rangle\delta_{jk} \quad \text{(利用 $\langle e_j|e_k\rangle=\delta_{jk}$)} \\
&= \sum_j c_j|e_j\rangle = |\psi\rangle
\end{align*}
因此 $\sum_j |e_j\rangle\langle e_j| = I$。
\end{derivation}

\paragraph{应用:展开系数计算}
若 $|\psi\rangle=\sum_j c_j|e_j\rangle$,对等式两边左乘 $\langle e_k|$:
\[
\langle e_k|\psi\rangle = \sum_j c_j\langle e_k|e_j\rangle = \sum_j c_j\delta_{kj} = c_k
\]
因此 $\boxed{c_k = \langle e_k|\psi\rangle}$。

\paragraph{测量概率与展开系数(补充)}
在基 $\{|e_j\rangle\}$ 上测量,结果 $e_j$ 的概率为
\[
p(j)=|c_j|^2=|\langle e_j|\psi\rangle|^2
\]
测量后态塌缩为对应本征态 $|e_j\rangle$。

\begin{example}
\textbf{计算基展开}

设计算基 $|0\rangle=\begin{pmatrix}1\\0\end{pmatrix}$,$|1\rangle=\begin{pmatrix}0\\1\end{pmatrix}$,求 $|\psi\rangle=\begin{pmatrix}3\\4i\end{pmatrix}$ 在该基下的展开系数。

\textbf{解:}
\begin{align*}
c_0 &= \langle 0|\psi\rangle = (1\quad 0)\begin{pmatrix}3\\4i\end{pmatrix} = 3 \\
c_1 &= \langle 1|\psi\rangle = (0\quad 1)\begin{pmatrix}3\\4i\end{pmatrix} = 4i
\end{align*}
因此 $|\psi\rangle = 3|0\rangle + 4i|1\rangle$。
\end{example}

\paragraph{正交归一本征基(补充)}
若一组正交归一向量同时是某算符 $A$ 的本征向量,则称为 $A$ 的\textbf{正交归一本征基}。
若本征值\textbf{不简并},对应本征向量天然正交;若\textbf{简并},可在该本征子空间内做 Gram--Schmidt 正交化得到正交归一基。
本征值/本征向量的定义见\hyperref[sec:eigen]{本征值与本征向量}。

\paragraph{正交归一基与酉算符}\label{para:orthonormal-unitary}
若将正交归一基按列排成矩阵
\[
U=\big(|\phi_1\rangle\ |\phi_2\rangle\ \cdots\ |\phi_n\rangle\big)
\]
则 $U^\dagger U=I$,因此 $U$ 为酉算符。
反之,任一酉算符的列向量构成正交归一基。
同一事实也可写成完备关系:
\[
UU^\dagger=\sum_{j=1}^n |\phi_j\rangle\langle\phi_j|=I
\]

\subsection{【背】基变换与酉矩阵}

\paragraph{从展开系数到基变换}
我们已经知道,给定正交归一基 $\{|e_i\rangle\}$,任意向量 $|\psi\rangle$ 可展开为
\[
|\psi\rangle=\sum_i c_i|e_i\rangle
\]
其中展开系数由内积给出:$c_i=\langle e_i|\psi\rangle$。

现在考虑:如果我们有\textbf{另一组}正交归一基 $\{|f_j\rangle\}$,同一个向量 $|\psi\rangle$ 也可在新基下展开:
\[
|\psi\rangle=\sum_j c'_j|f_j\rangle,\quad c'_j=\langle f_j|\psi\rangle
\]
这引出一个核心问题:\textbf{新基 $|f_j\rangle$ 和旧基 $|e_i\rangle$ 之间是什么关系?}

\paragraph{新基在旧基下的展开}
由于 $\{|e_i\rangle\}$ 是完备基,新基向量 $|f_j\rangle$ 本身也可在旧基下展开。利用完备关系 $I=\sum_i |e_i\rangle\langle e_i|$:
\[
|f_j\rangle = I|f_j\rangle
= \sum_i |e_i\rangle\langle e_i|f_j\rangle
= \sum_i \langle e_i|f_j\rangle\,|e_i\rangle
\]
这告诉我们:$|f_j\rangle$ 在旧基下的第 $i$ 个系数就是 $\langle e_i|f_j\rangle$。

\paragraph{基变换矩阵的定义}
将所有新基向量在旧基下的系数收集起来,定义\textbf{基变换矩阵}
\[
\boxed{U_{ij}=\langle e_i|f_j\rangle}
\]
则上述展开公式可写成
\[
\boxed{|f_j\rangle=\sum_i U_{ij}|e_i\rangle}
\]
\textbf{直观意义:}$U$ 的第 $j$ 列就是新基向量 $|f_j\rangle$ 在旧基下的展开系数。

\paragraph{列矩阵写法(常用)}
把基向量\textbf{并成列矩阵},上式可写成
\[
\big(|f_1\rangle\ |f_2\rangle\ \cdots\ |f_n\rangle\big)
=\big(|e_1\rangle\ |e_2\rangle\ \cdots\ |e_n\rangle\big)U
\]
这只是“按列展开”的记号:第 $j$ 列等式就是 $|f_j\rangle=\sum_i U_{ij}|e_i\rangle$。

\paragraph{基变换矩阵必为酉矩阵}
由于新旧基都是正交归一的,基变换必须\textbf{保持内积与长度}。利用完备关系验证:
\begin{align*}
(U^\dagger U)_{jk}
&=\sum_i U_{ij}^\ast U_{ik}
=\sum_i \langle f_j|e_i\rangle\langle e_i|f_k\rangle \\
&=\langle f_j|\left(\sum_i |e_i\rangle\langle e_i|\right)|f_k\rangle
=\langle f_j|I|f_k\rangle
=\langle f_j|f_k\rangle
=\delta_{jk}
\end{align*}
因此 $U^\dagger U=I$,即 $U$ 为\textbf{酉矩阵}。

\paragraph{向量展开系数的变换}
若向量 $|\psi\rangle$ 在旧基下展开系数为 $\vec{c}=(c_1,\dots,c_n)^T$,在新基下展开系数为 ${\vec{c}\,}'=(c'_1,\dots,c'_n)^T$,则利用 $c_i=\langle e_i|\psi\rangle$、$c'_j=\langle f_j|\psi\rangle$,以及旧基的完备关系 $I=\sum_i |e_i\rangle\langle e_i|$:
\begin{align*}
c'_j
&= \langle f_j|\psi\rangle \\
&= \langle f_j|I|\psi\rangle
\quad \text{(插入单位算符)} \\
&= \langle f_j|\left(\sum_i |e_i\rangle\langle e_i|\right)|\psi\rangle
\quad \text{(用完备关系展开 $I$)} \\
&= \sum_i \langle f_j|e_i\rangle\langle e_i|\psi\rangle
\quad \text{(内积的线性性)} \\
&= \sum_i U_{ij}^\ast c_i
\quad \text{(定义:$U_{ij}=\langle e_i|f_j\rangle$,故 $\langle f_j|e_i\rangle=U_{ij}^\ast$)} \\
&= \sum_i U^\dagger_{ji} c_i
\quad \text{(厄米共轭定义:$U^\dagger_{ji}=U_{ij}^\ast$)}
\end{align*}
矩阵形式为
\[
\boxed{{\vec{c}\,}' = U^\dagger \vec{c}}
\]

\begin{derivation}
\textbf{完整推导总结:}

我们依次用到了前面的所有工具:

\textbf{1) 完备关系 $\Rightarrow$ 展开公式}
\[
|f_j\rangle = I|f_j\rangle
= \sum_i |e_i\rangle\langle e_i|f_j\rangle
= \sum_i U_{ij}|e_i\rangle
\]

\textbf{2) 正交归一性 + 完备关系 $\Rightarrow$ 酉性}
\begin{align*}
(U^\dagger U)_{jk}
&= \sum_i \langle f_j|e_i\rangle\langle e_i|f_k\rangle
= \langle f_j|f_k\rangle = \delta_{jk}
\end{align*}

\textbf{3) 系数计算公式 $\Rightarrow$ 系数变换}
\[
c'_j = \langle f_j|\psi\rangle
= \sum_i \langle f_j|e_i\rangle\langle e_i|\psi\rangle
= \sum_i U^\dagger_{ji} c_i
\]

\textbf{物理含义:}正交归一基之间的变换必保持内积与范数,这正是酉矩阵的几何本质。
\end{derivation}

\paragraph{基变换算符 $U$}
基变换 $U$ 既可以看作描述"新基在旧基下展开系数"的数值矩阵,也可以看作一个\textbf{酉算符}。
作为算符,它的作用是把旧基映射到新基:
\[
U|e_j\rangle = |f_j\rangle
\]
矩阵元 $U_{ij}$ 可以通过算符作用或内积两种等价方式计算:
\[
U_{ij}=\langle e_i|U|e_j\rangle=\langle e_i|f_j\rangle
\]
无论从哪个角度看,$U$ 都是酉的:$U^\dagger U=I$。
算符在不同基下的变换将在\hyperref[sec:operator-basis-change]{算符的基变换}一节详细讨论。

\paragraph{例:Hadamard(哈达玛)基}
定义
\[
|+\rangle=\frac{|0\rangle+|1\rangle}{\sqrt{2}},\quad
|-\rangle=\frac{|0\rangle-|1\rangle}{\sqrt{2}}
\]
这里的\textbf{计算基}指标准基 $\{|0\rangle,|1\rangle\}$,其中
\[
|0\rangle=\begin{pmatrix}1\\0\end{pmatrix},\qquad |1\rangle=\begin{pmatrix}0\\1\end{pmatrix}
\]
\textbf{变换矩阵}指“把新基向量在旧基下展开”的系数矩阵。若旧基为 $\{|e_i\rangle\}$、新基为 $\{|f_j\rangle\}$,
则 $U_{ij}=\langle e_i|f_j\rangle$,并有
\[
|f_j\rangle=\sum_i U_{ij}|e_i\rangle
\]
因此矩阵 $U$ 的\textbf{第 $j$ 列}就是 $|f_j\rangle$ 在旧基下的展开系数。

对 Hadamard 基,令旧基 $|e_1\rangle=|0\rangle,|e_2\rangle=|1\rangle$,新基 $|f_1\rangle=|+\rangle,|f_2\rangle=|-\rangle$。
计算矩阵元:
\[
U_{11}=\langle 0|+\rangle=\frac{1}{\sqrt{2}},\quad
U_{21}=\langle 1|+\rangle=\frac{1}{\sqrt{2}}
\]
\[
U_{12}=\langle 0|-\rangle=\frac{1}{\sqrt{2}},\quad
U_{22}=\langle 1|-\rangle=-\frac{1}{\sqrt{2}}
\]
所以 $\{|+\rangle,|-\rangle\}$ 为一组正交归一基,与计算基的变换矩阵为
\[
U=\frac{1}{\sqrt{2}}\begin{pmatrix}1&1\\1&-1\end{pmatrix}
\]

\paragraph{Hadamard(哈达玛)门的用途与用法}
Hadamard 门(又称哈达玛门,记为 $H$)\textbf{既是厄米的也是酉的},即
\[
H^\dagger=H,\qquad H^\dagger H=I
\]
因此 $H^2=I$。它最常见的作用是\textbf{在计算基与 Hadamard 基之间切换},以及\textbf{生成叠加态}:
\[
H|0\rangle=|+\rangle=\frac{|0\rangle+|1\rangle}{\sqrt{2}},\qquad
H|1\rangle=|-\rangle=\frac{|0\rangle-|1\rangle}{\sqrt{2}}
\]
因此 $H$ 把确定态变成等幅叠加态;反过来再施加一次 $H$ 会回到原态($H^2=I$)。

\paragraph{如何用:改变测量基}
若想在 $X$ 基($|+\rangle,|-\rangle$)测量,可先施加 $H$,再在计算基做 $Z$ 测量(“测量”的定义与规则见后文\hyperref[sec:measurement]{测量}一节)。
因为
\[
HZH = X
\]
这表示“先变基再测量”等价于直接测量另一方向。

\begin{example}
\textbf{示例:用 $H$ 将 $Z$ 基测量变为 $X$ 基测量}

设量子态为 $|\psi\rangle=\alpha|0\rangle+\beta|1\rangle$。若要测量 $X$ 基的概率:
\begin{enumerate}
  \item 先施加 $H$ 得到 $H|\psi\rangle$。
  \item 再在计算基测量($Z$ 基)。
\end{enumerate}
这样测得 $|0\rangle$ 的概率等于原态在 $|+\rangle$ 的概率,
测得 $|1\rangle$ 的概率等于原态在 $|-\rangle$ 的概率。
\end{example}

\begin{figure}[htbp]
  \centering
  \includegraphics[width=0.45\linewidth]{fig_basis_change.png}
  \caption{不同正交基之间的基变换示意($\{|e_i\rangle\}$ 与 $\{|\tilde e_i\rangle\}$)。}
\end{figure}

\subsection{【了解】离散谱与连续谱(Dirac $\delta$ 归一化)}\label{subsec:discrete-continuous}

\paragraph{离散基 vs 连续基}
离散正交归一基满足 $\langle e_i|e_j\rangle=\delta_{ij}$,而连续谱的本征态满足
\[
\langle x|x'\rangle=\delta(x-x')
\]
其中 $\delta(x-x')$ 为狄拉克 $\delta$ 函数。两者都是“正交归一”的广义形式。

\paragraph{连续完备性}
连续基的完备关系写成积分形式:
\[
\int |x\rangle\langle x|\,dx = I,\qquad
\int |p\rangle\langle p|\,dp = I
\]

\paragraph{波函数与归一化}
\[
\psi(x)=\langle x|\psi\rangle,\qquad
\int |\psi(x)|^2\,dx = 1
\]

因此,“离散谱 $\leftrightarrow$ 求和”,“连续谱 $\leftrightarrow$ 积分”是同一数学结构在不同谱类型下的体现。
\paragraph{离散与连续的统一表述(补充)}
常将求和与积分统一写作
\[
I=\int |a\rangle\langle a|\,da
\]
其中对离散谱该符号代表求和,对连续谱代表积分。

\section{【背】线性算符(Linear Operators)}

\subsection{【了解】基本概念}

\paragraph{算符作用}
算符(operator)$A$ 是将一个向量映射到另一个向量的"函数":
\[
A: |\psi\rangle \mapsto A|\psi\rangle
\]
在有限维空间中,\textbf{选定一组基后},算符可用矩阵表示(即 $A^{(e)}$),
不同基会给出不同矩阵,但它们对应同一个算符 $A$。

\paragraph{线性性}
线性算符满足
\[
A\left(c_1|\psi_1\rangle + c_2|\psi_2\rangle\right)
= c_1A|\psi_1\rangle + c_2A|\psi_2\rangle
\]

\paragraph{单位算符与逆算符}
单位算符 $I$ 满足 $I|\psi\rangle=|\psi\rangle$。
若存在 $A^{-1}$ 使得 $AA^{-1}=A^{-1}A=I$,则 $A$ 可逆。
算符乘积一般不交换($AB\neq BA$),其物理含义将在对易子部分进一步讨论。

\paragraph{算符相等(补充)}
若对任意态 $|\psi\rangle$ 都有 $A|\psi\rangle=B|\psi\rangle$,则 $A=B$。

\paragraph{期望值}
给定一个算符 $A$ 和一个归一化的态 $|\psi\rangle$($\langle\psi|\psi\rangle=1$),定义 $A$ 在态 $|\psi\rangle$ 下的\textbf{期望值}(expectation value)为:
\[
\langle A \rangle_\psi := \langle\psi|A|\psi\rangle
\]
若在 $A$ 的本征基下(本征值/本征向量定义见后文\hyperref[sec:eigen]{本征值与本征向量})$A|a_j\rangle=a_j|a_j\rangle$ 且 $|\psi\rangle=\sum_j c_j|a_j\rangle$,则
\[
\langle A\rangle_\psi=\sum_j a_j |c_j|^2
\]
体现"\textbf{本征值按概率加权平均}"。
这是一个复数。在量子力学中,期望值对应测量的平均值。

\begin{example}
\textbf{计算期望值}

设 $|\psi\rangle = \frac{1}{\sqrt{2}}\begin{pmatrix}1\\1\end{pmatrix}$,$A=\begin{pmatrix}1&0\\0&-1\end{pmatrix}$,求 $\langle A\rangle_\psi$。

\textbf{解:}
\begin{align*}
\langle A\rangle_\psi &= \langle\psi|A|\psi\rangle \\
&= \frac{1}{\sqrt{2}}(1\quad 1)A\frac{1}{\sqrt{2}}\begin{pmatrix}1\\1\end{pmatrix} \\
&= \frac{1}{2}(1\quad 1)\begin{pmatrix}1\\-1\end{pmatrix} \\
&= \frac{1}{2}(1-1) = 0
\end{align*}
\end{example}

\paragraph{对角化与酉对角化(定义)}\label{para:diagonalization}
\textbf{对角化:}若存在可逆矩阵 $P$ 使
\[
P^{-1}AP=D
\]
其中 $D$ 为对角矩阵,则称 $A$ \textbf{可对角化},也可写成
\[
A=PDP^{-1}
\]
\textbf{含义:}$P$ 的列向量是一组线性无关本征向量,$D$ 的对角元就是对应本征值。
换句话说,在"以本征向量为基"的表象中,算符矩阵变为对角矩阵。

\textbf{酉对角化:}若存在\textbf{酉矩阵} $U$ 使
\[
U^\dagger A U = D
\]
则称 $A$ \textbf{酉对角化}。这等价于:$A$ 存在一组\textbf{正交归一}本征向量作为基。
对厄米算符(以及更一般的正规算符)一定可以酉对角化。

\subsection{【背】厄米算符(Hermitian Operators)}

\paragraph{定义}
若算符 $H$ 满足 $H=H^\dagger$,则称 $H$ 为\textbf{厄米算符}(或自伴算符)。

\paragraph{物理意义}
在量子力学中,所有可观测量(如能量、动量、位置)都用厄米算符表示,因为测量结果必须是实数。

\paragraph{2×2厄米矩阵的通用形式}
\[
H = \begin{pmatrix} a & b \\ b^\ast & d \end{pmatrix}, \quad a,d\in\mathbb{R}, b\in\mathbb{C}
\]
即:对角元素必须是实数,非对角元素互为共轭。

\paragraph{关键性质}
\begin{enumerate}
  \item \textbf{期望值为实数}:对任意归一化态 $|\psi\rangle$,$\langle\psi|H|\psi\rangle\in\mathbb{R}$
  \item \textbf{本征值为实数}(定义见后文\hyperref[sec:eigen]{本征值与本征向量})
  \item \textbf{不同本征值的本征向量正交}(见后文\hyperref[sec:eigen]{本征值与本征向量})
\end{enumerate}

\paragraph{对角化形式(补充)}
任意厄米算符都可被酉对角化:
\[
H=U\,\mathrm{diag}(\lambda_1,\ldots,\lambda_n)\,U^\dagger
\]
其中 $U$ 的列向量为正交归一的本征态。

\begin{derivation}
\textbf{证明期望值为实数:}
\begin{align*}
\langle\psi|H|\psi\rangle^\ast
&= \langle\psi|H^\dagger|\psi\rangle \quad \text{(共轭转置的性质)} \\
&= \langle\psi|H|\psi\rangle \quad \text{(因为 $H=H^\dagger$)}
\end{align*}
因此 $\langle\psi|H|\psi\rangle$ 等于其自身的复共轭,故为实数。
\end{derivation}

\begin{example}
\textbf{判断是否为厄米算符}
判断 $\sigma_z = \begin{pmatrix}1&0\\0&-1\end{pmatrix}$ 是否为厄米算符。

\textbf{解:}
\[
\sigma_z^\dagger = (\sigma_z^T)^\ast = \begin{pmatrix}1&0\\0&-1\end{pmatrix}^\ast = \begin{pmatrix}1&0\\0&-1\end{pmatrix} = \sigma_z
\]
因此 $\sigma_z$ 是厄米算符。
\end{example}

\subsection{【背】酉算符(Unitary / 幺正 Operators)}

\paragraph{定义} 若 $U^\dagger U = UU^\dagger = I$,则称 $U$ 为酉算符(或幺正算符)。

\paragraph{关键性质}
\begin{enumerate}
  \item \textbf{保持内积}:$\langle U\phi|U\psi\rangle = \langle\phi|\psi\rangle$
  \item \textbf{保持范数}:$\|U|\psi\rangle\| = \||\psi\rangle\|$
  \item \textbf{逆等于厄米共轭}:$U^{-1}=U^\dagger$
  \item \textbf{列向量构成正交归一基}:设 $U$ 为 $n\times n$ 酉矩阵,记 $|u_j\rangle$ 为 $U$ 的第 $j$ 列(作为列向量),则这 $n$ 个列向量 $\{|u_1\rangle,|u_2\rangle,\ldots,|u_n\rangle\}$ 构成正交归一基,即满足 $\langle u_i|u_j\rangle = \delta_{ij}$(反之亦成立,见\hyperref[para:orthonormal-unitary]{正交归一基与酉矩阵})
\end{enumerate}

\paragraph{谱性质(补充)}
酉算符的本征值满足 $|\lambda|=1$(位于单位圆上),因此 $\det(U)$ 也满足 $|\det(U)|=1$。
时间演化算符可写成
\[
U=\exp\!\left(-\frac{i}{\hbar}H\Delta t\right)
\]
其中 $H$ 为厄米算符。

\begin{derivation}
\textbf{证明保持内积:}
\begin{align*}
\langle U\phi|U\psi\rangle
&= (U|\phi\rangle)^\dagger(U|\psi\rangle) \\
&= \langle\phi|U^\dagger U|\psi\rangle \\
&= \langle\phi|I|\psi\rangle = \langle\phi|\psi\rangle
\end{align*}
\end{derivation}

\begin{example}
\textbf{验证Hadamard门是酉算符}

Hadamard门:$\displaystyle H = \frac{1}{\sqrt{2}}\begin{pmatrix}1&1\\1&-1\end{pmatrix}$

\textbf{解:}因元素全为实数,$H^\dagger=H^T=H$。计算:
\begin{align*}
 H^\dagger H &= H^2 = \frac{1}{2}\begin{pmatrix}1&1\\1&-1\end{pmatrix}\begin{pmatrix}1&1\\1&-1\end{pmatrix} \\
&= \frac{1}{2}\begin{pmatrix}2&0\\0&2\end{pmatrix} = I
\end{align*}
因此 $H$ 是酉算符(同时也是厄米算符)。
\end{example}

\subsection{【背】算符的矩阵表示(Matrix Elements)}

\paragraph{从向量展开到算符作用}
我们已经知道,给定正交归一基 $\{|e_i\rangle\}$,任意向量 $|\psi\rangle$ 可展开为
\[
|\psi\rangle=\sum_i c_i|e_i\rangle
\]
其中展开系数由内积给出:$c_i=\langle e_i|\psi\rangle$。

现在考虑:算符 $A$ 作用在基向量 $|e_j\rangle$ 上,得到一个新向量 $A|e_j\rangle$。这个新向量也可以在基 $\{|e_i\rangle\}$ 下展开:
\[
A|e_j\rangle=\sum_i (\text{某个系数})\,|e_i\rangle
\]
这引出一个核心问题:\textbf{这些系数是什么?}

\paragraph{算符作用后向量的展开系数}
由于 $\{|e_i\rangle\}$ 是完备基,我们可以利用完备关系 $I=\sum_i |e_i\rangle\langle e_i|$ 展开 $A|e_j\rangle$:
\[
A|e_j\rangle = I\cdot A|e_j\rangle
= \sum_i |e_i\rangle\langle e_i|A|e_j\rangle
\]
这告诉我们:$A|e_j\rangle$ 在基下的第 $i$ 个系数就是 $\langle e_i|A|e_j\rangle$。

\paragraph{矩阵元的定义}
将算符 $A$ 作用在所有基向量上得到的展开系数收集起来,定义\textbf{矩阵元}
\[
\boxed{A^{(e)}_{ij}=\langle e_i|A|e_j\rangle}
\]
则上述展开公式可写成
\[
\boxed{A|e_j\rangle=\sum_i A^{(e)}_{ij}|e_i\rangle}
\]

\paragraph{记号约定:算符与矩阵表示}
为了明确区分抽象的算符和依赖于基的矩阵表示,我们采用如下记号:
\begin{itemize}
\item \textbf{算符}(独立于基):$A$
\item \textbf{算符在基 $\{|e_i\rangle\}$ 下的矩阵表示}:$A^{(e)}$
\item \textbf{矩阵元}(算符在基下的矩阵元素):$A^{(e)}_{ij}=\langle e_i|A|e_j\rangle$
\end{itemize}

\textbf{直观意义:}矩阵 $A^{(e)}$ 的第 $j$ 列就是向量 $A|e_j\rangle$ 在基 $\{|e_i\rangle\}$ 下的展开系数。

将所有基向量的作用结果按列排列,得到算符 $A$ 在基 $\{|e_i\rangle\}$ 下的矩阵表示:
\[
A^{(e)} = \begin{pmatrix}
A^{(e)}_{11} & A^{(e)}_{12} & \cdots & A^{(e)}_{1n} \\
A^{(e)}_{21} & A^{(e)}_{22} & \cdots & A^{(e)}_{2n} \\
\vdots & \vdots & \ddots & \vdots \\
A^{(e)}_{n1} & A^{(e)}_{n2} & \cdots & A^{(e)}_{nn}
\end{pmatrix}
= \begin{pmatrix}
| & | & & | \\
A|e_1\rangle & A|e_2\rangle & \cdots & A|e_n\rangle \\
| & | & & |
\end{pmatrix}_{\text{展开系数}}
\]

\begin{example}
\textbf{例:二维算符的矩阵表示}

设 $\{|e_1\rangle,|e_2\rangle\}$ 为二维正交归一基,算符 $A$ 作用为
\[
A|e_1\rangle=2|e_1\rangle+3|e_2\rangle,\qquad
A|e_2\rangle=-i|e_1\rangle+5|e_2\rangle
\]

\textbf{步骤 1:}求第 1 列,即 $A|e_1\rangle$ 的展开系数:
\[
A^{(e)}_{11}=\langle e_1|A|e_1\rangle=2,\quad A^{(e)}_{21}=\langle e_2|A|e_1\rangle=3
\]

\textbf{步骤 2:}求第 2 列,即 $A|e_2\rangle$ 的展开系数:
\[
A^{(e)}_{12}=\langle e_1|A|e_2\rangle=-i,\quad A^{(e)}_{22}=\langle e_2|A|e_2\rangle=5
\]

因此矩阵表示为
\[
A^{(e)} = \begin{pmatrix}2&-i\\3&5\end{pmatrix}
\]

\textbf{验证:}第 1 列 $(2,3)^T$ 确实是 $A|e_1\rangle$ 的展开系数,第 2 列 $(-i,5)^T$ 是 $A|e_2\rangle$ 的展开系数。
\end{example}

\paragraph{算符对任意向量的作用}
现在我们知道算符 $A$ 作用在基向量上的结果:$A|e_j\rangle=\sum_i A^{(e)}_{ij}|e_i\rangle$。

那么对任意向量 $|\psi\rangle=\sum_j c_j|e_j\rangle$,算符 $A$ 的作用是什么?利用算符的线性性:
\begin{align*}
A|\psi\rangle
&= A\left(\sum_j c_j|e_j\rangle\right) \\
&= \sum_j c_j\,A|e_j\rangle
\quad \text{(线性性)} \\
&= \sum_j c_j\sum_i A^{(e)}_{ij}|e_i\rangle
\quad \text{(代入 $A|e_j\rangle=\sum_i A^{(e)}_{ij}|e_i\rangle$)} \\
&= \sum_i\left(\sum_j A^{(e)}_{ij}c_j\right)|e_i\rangle
\quad \text{(交换求和顺序)}
\end{align*}

因此 $A|\psi\rangle$ 的第 $i$ 个展开系数为
\[
c'_i = \sum_j A^{(e)}_{ij}c_j
\]
用列向量记号,若 $|\psi\rangle$ 的展开系数为 $\vec{c}=(c_1,\dots,c_n)^T$,$A|\psi\rangle$ 的展开系数为 ${\vec{c}\,}'=(c'_1,\dots,c'_n)^T$,则
\[
\boxed{{\vec{c}\,}' = A^{(e)}\vec{c}}
\]
这就是矩阵乘法!\textbf{算符作用对应矩阵乘法。}

\begin{derivation}
\textbf{完整推导总结:}

我们依次用到了前面的所有工具:

\textbf{1) 完备关系 $\Rightarrow$ 展开公式}
\[
A|e_j\rangle = I\cdot A|e_j\rangle
= \sum_i |e_i\rangle\langle e_i|A|e_j\rangle
= \sum_i A^{(e)}_{ij}|e_i\rangle
\]

\textbf{2) 线性性 $\Rightarrow$ 对任意向量的作用}
\[
A|\psi\rangle = \sum_j c_j A|e_j\rangle
= \sum_i\left(\sum_j A^{(e)}_{ij}c_j\right)|e_i\rangle
\]

\textbf{3) 矩阵元定义 $\Rightarrow$ 矩阵乘法}
\[
c'_i = \sum_j A^{(e)}_{ij}c_j
\quad\Longleftrightarrow\quad
{\vec{c}\,}' = A^{(e)}\vec{c}
\]

\textbf{物理含义:}算符 $A$ 在基 $\{|e_i\rangle\}$ 下由矩阵 $A^{(e)}$ 及其矩阵元 $A^{(e)}_{ij}=\langle e_i|A|e_j\rangle$ 完全确定,其作用通过矩阵乘法实现。
\end{derivation}

\begin{example}
\textbf{例:算符作用的矩阵计算}

沿用上例,设 $|\psi\rangle=|e_1\rangle+2|e_2\rangle$,展开系数 $\vec{c}=(1,2)^T$,则
\[
A|\psi\rangle \text{ 的展开系数 } = \begin{pmatrix}2&-i\\3&5\end{pmatrix}\begin{pmatrix}1\\2\end{pmatrix}
= \begin{pmatrix}2-2i\\13\end{pmatrix}
\]
即 $A|\psi\rangle=(2-2i)|e_1\rangle+13|e_2\rangle$。

验证:
\begin{align*}
A|\psi\rangle &= A|e_1\rangle+2A|e_2\rangle \\
&= (2|e_1\rangle+3|e_2\rangle)+2(-i|e_1\rangle+5|e_2\rangle) \\
&= (2-2i)|e_1\rangle+13|e_2\rangle \quad \checkmark
\end{align*}
\end{example}

\paragraph{算符的基展开公式}
利用完备性关系,可将算符 $A$ 表示为外积 $|e_i\rangle\langle e_j|$ 的线性组合:
\[
A = \sum_{i,j} A^{(e)}_{ij}\,|e_i\rangle\langle e_j|
\]
其中 $A^{(e)}_{ij}=\langle e_i|A|e_j\rangle$。

\begin{derivation}
\textbf{推导:}在 $A$ 左右两侧插入完备性关系:
\begin{align*}
A &= I\cdot A\cdot I
= \left(\sum_i |e_i\rangle\langle e_i|\right)A\left(\sum_j |e_j\rangle\langle e_j|\right) \\
&= \sum_{i,j} |e_i\rangle\underbrace{\langle e_i|A|e_j\rangle}_{=A^{(e)}_{ij}}\langle e_j|
= \sum_{i,j} A^{(e)}_{ij}\,|e_i\rangle\langle e_j|
\end{align*}
\end{derivation}

这个公式的物理意义:外积 $|e_i\rangle\langle e_j|$ 是"从 $|e_j\rangle$ 映射到 $|e_i\rangle$"的算符,$A^{(e)}_{ij}$ 是其系数。

\subsection{【背】算符的基变换}\label{sec:operator-basis-change}

\paragraph{算符在不同基下的矩阵元}
在 2.7 节我们定义了基变换矩阵 $U_{ij}=\langle e_i|f_j\rangle$,其中 $\{|e_j\rangle\}$ 是旧基,$\{|f_j\rangle\}$ 是新基。$U$ 是酉矩阵,满足 $U^\dagger U=I$。

算符 $A$ 在两组基下的矩阵元分别为
\[
A^{(e)}_{ij}=\langle e_i|A|e_j\rangle,\qquad
A^{(f)}_{mn}=\langle f_m|A|f_n\rangle
\]

\begin{derivation}
\textbf{由完备关系得到基变换公式:}

在 $A$ 的左右分别插入旧基的完备关系 $I=\sum_i |e_i\rangle\langle e_i|$:
\begin{align*}
A^{(f)}_{mn}
&=\langle f_m|A|f_n\rangle \\
&=\langle f_m|\left(\sum_i |e_i\rangle\langle e_i|\right)
A\left(\sum_j |e_j\rangle\langle e_j|\right)|f_n\rangle \\
&=\sum_{i,j}\langle f_m|e_i\rangle\langle e_i|A|e_j\rangle\langle e_j|f_n\rangle \\
&=\sum_{i,j}U_{im}^\ast\,A^{(e)}_{ij}\,U_{jn}
\end{align*}
因此
\[
\boxed{A^{(f)}=U^\dagger A^{(e)}U}
\]
这就是\textbf{算符在不同基下的相似变换公式}。
\end{derivation}

\begin{example}
\textbf{例:基变换}

取基 $\{|e_1\rangle,|e_2\rangle\}$,设算符 $A$ 满足
\[
A|e_1\rangle=2|e_1\rangle,\quad A|e_2\rangle=3|e_2\rangle
\]
则在该基下
\[
A^{(e)} = \begin{pmatrix}2&0\\0&3\end{pmatrix}
\]

现换新基 $\{|f_1\rangle,|f_2\rangle\}$,其中
\[
|f_1\rangle=\frac{|e_1\rangle+|e_2\rangle}{\sqrt{2}},\quad
|f_2\rangle=\frac{|e_1\rangle-|e_2\rangle}{\sqrt{2}}
\]
基变换矩阵
\[
U=\frac{1}{\sqrt{2}}\begin{pmatrix}1&1\\1&-1\end{pmatrix}
\]
在新基下
\[
A^{(f)} = U^\dagger A^{(e)}U
= \begin{pmatrix}\frac{5}{2}&-\frac{1}{2}\\-\frac{1}{2}&\frac{5}{2}\end{pmatrix}
\]
矩阵形式不同,但描述的是同一个算符 $A$。
\end{example}

\subsection{【背】期望值与演化的矩阵形式}

有了算符的矩阵表示,我们可以将量子力学中的许多重要公式转化为矩阵形式,这在实际计算中非常有用。

\paragraph{期望值的矩阵计算}
设态 $|\psi\rangle$ 在正交归一基 $\{|e_i\rangle\}$ 下的展开系数为列向量 $\vec{c}=(c_1,c_2,\ldots,c_n)^T$,算符 $A$ 在该基下的矩阵表示为 $A^{(e)}$,矩阵元为 $A^{(e)}_{ij}=\langle e_i|A|e_j\rangle$,则可观测量 $A$ 的期望值可写成矩阵形式:
\[
\langle A\rangle_\psi = \langle\psi|A|\psi\rangle = \vec{c}^{\,\dagger} A^{(e)}\vec{c}
\]

\begin{derivation}
\textbf{推导:}设 $|\psi\rangle=\sum_j c_j|e_j\rangle$,则
\begin{align*}
\langle A\rangle_\psi
&= \langle\psi|A|\psi\rangle
= \left(\sum_i c_i^\ast\langle e_i|\right)A\left(\sum_j c_j|e_j\rangle\right) \\
&= \sum_{i,j} c_i^\ast c_j\langle e_i|A|e_j\rangle
= \sum_{i,j} c_i^\ast A^{(e)}_{ij} c_j \\
&= \vec{c}^{\,\dagger} A^{(e)}\vec{c}
\end{align*}
其中最后一步就是矩阵乘法 $\vec{c}^{\,\dagger} A^{(e)}\vec{c}$ 的定义。
\end{derivation}

\begin{example}
\textbf{例:计算自旋期望值}

设自旋 $1/2$ 粒子处于态
\[
|\psi\rangle = \frac{1}{\sqrt{2}}(|0\rangle+|1\rangle)
\]
在标准基 $\{|0\rangle,|1\rangle\}$ 下,坐标为 $\vec{c}=\frac{1}{\sqrt{2}}\begin{pmatrix}1\\1\end{pmatrix}$。

Pauli 算符 $\sigma_x$ 在标准基下的矩阵表示为 $\sigma_x^{(0,1)}=\begin{pmatrix}0&1\\1&0\end{pmatrix}$,则期望值
\[
\langle\sigma_x\rangle_\psi = \vec{c}^{\,\dagger}\sigma_x^{(0,1)}\vec{c}
= \frac{1}{2}\begin{pmatrix}1&1\end{pmatrix}\begin{pmatrix}0&1\\1&0\end{pmatrix}\begin{pmatrix}1\\1\end{pmatrix}
= \frac{1}{2}\begin{pmatrix}1&1\end{pmatrix}\begin{pmatrix}1\\1\end{pmatrix}
= 1
\]
\end{example}

\paragraph{薛定谔方程的矩阵形式}
量子态的时间演化由薛定谔方程描述。在给定基下,薛定谔方程可转化为矩阵微分方程。

设 $|\psi(t)\rangle$ 在基 $\{|e_i\rangle\}$ 下的坐标为 $\vec{c}(t)=(c_1(t),\ldots,c_n(t))^T$,哈密顿量 $H$ 的矩阵表示为 $H^{(e)}$,则薛定谔方程
\[
i\hbar\frac{d}{dt}|\psi(t)\rangle = H|\psi(t)\rangle
\]
在该基下等价于矩阵微分方程:
\[
i\hbar\frac{d}{dt}\vec{c}(t) = H^{(e)}\vec{c}(t)
\]

\begin{derivation}
\textbf{推导:}将 $|\psi(t)\rangle=\sum_i c_i(t)|e_i\rangle$ 代入薛定谔方程:
\[
i\hbar\sum_i\frac{dc_i}{dt}|e_i\rangle = H\left(\sum_j c_j|e_j\rangle\right)
\]
在两侧左乘 $\langle e_k|$:
\begin{align*}
i\hbar\frac{dc_k}{dt}
&= \langle e_k|H\left(\sum_j c_j|e_j\rangle\right)
= \sum_j c_j\langle e_k|H|e_j\rangle \\
&= \sum_j H^{(e)}_{kj}c_j(t)
\end{align*}
这对每个 $k=1,2,\ldots,n$ 都成立,因此写成矩阵形式:
\[
i\hbar\,\dot{\vec{c}}(t)=H^{(e)}\vec{c}(t)
\]
\end{derivation}

\paragraph{时间演化算符的矩阵形式}
若哈密顿量 $H$ 不依赖时间,薛定谔方程的解为
\[
|\psi(t)\rangle = e^{-iHt/\hbar}|\psi(0)\rangle
\]
在给定基下,这对应坐标的矩阵变换:
\[
\vec{c}(t) = e^{-iH^{(e)}t/\hbar}\vec{c}(0)
\]
其中矩阵指数定义为
\[
e^{-iH^{(e)}t/\hbar} = \sum_{n=0}^\infty\frac{(-it/\hbar)^n}{n!}(H^{(e)})^n
\]

\begin{example}
\textbf{例:二能级系统的演化}

考虑哈密顿量 $H=\omega\sigma_z/2=\frac{\omega}{2}\begin{pmatrix}1&0\\0&-1\end{pmatrix}$,初态 $|\psi(0)\rangle=|0\rangle$,坐标 $\vec{c}(0)=\begin{pmatrix}1\\0\end{pmatrix}$。

演化算符
\[
e^{-iHt/\hbar} = e^{-i\omega t\sigma_z/(2\hbar)}
= \begin{pmatrix}e^{-i\omega t/2}&0\\0&e^{i\omega t/2}\end{pmatrix}
\]
因此
\[
\vec{c}(t) = \begin{pmatrix}e^{-i\omega t/2}&0\\0&e^{i\omega t/2}\end{pmatrix}\begin{pmatrix}1\\0\end{pmatrix}
= \begin{pmatrix}e^{-i\omega t/2}\\0\end{pmatrix}
\]
即 $|\psi(t)\rangle=e^{-i\omega t/2}|0\rangle$,除了整体相位,态保持不变(这是因为 $|0\rangle$ 是能量本征态)。
\end{example}

\subsection{【背】厄米算符的矩阵性质}

厄米算符在量子力学中扮演可观测量的角色,其矩阵表示具有特殊性质。

\paragraph{厄米矩阵的定义}
若算符 $A=A^\dagger$(厄米算符),则在任何正交归一基 $\{|e_i\rangle\}$ 下,其矩阵元满足
\[
A^{(e)}_{ij} = A^{(e)}_{ji}{}^\ast
\]
即矩阵 $A^{(e)}$ 是厄米矩阵($A^{(e)}=A^{(e)\dagger}$)。

\begin{derivation}
\textbf{推导矩阵元关系:}
\begin{align*}
A^{(e)}_{ij} &= \langle e_i|A|e_j\rangle
= \langle e_i|A^\dagger|e_j\rangle \quad \text{(因为 $A=A^\dagger$)} \\
&= \langle e_j|A|e_i\rangle^\ast \quad \text{(伴随的定义)}
= A^{(e)}_{ji}{}^\ast
\end{align*}
这说明矩阵关于对角线共轭对称。
\end{derivation}

\begin{example}
\textbf{例:验证 Pauli 矩阵是厄米的}

在标准基下,$\sigma_x=\begin{pmatrix}0&1\\1&0\end{pmatrix}$:显然 $\sigma_x^\dagger=\sigma_x$,满足矩阵元关系。

$\sigma_z=\begin{pmatrix}1&0\\0&-1\end{pmatrix}$:对角矩阵且元素为实数,显然厄米。
\end{example}

\paragraph{本征基下的对角化}
厄米算符的一个重要性质是:总可以选择一组正交归一的本征基 $\{|a_i\rangle\}$(满足 $A|a_i\rangle=a_i|a_i\rangle$),使得在该基下算符的矩阵表示为对角形式:
\[
A^{(\text{本征基})} = \begin{pmatrix}a_1&0&\cdots&0\\0&a_2&\cdots&0\\\vdots&\vdots&\ddots&\vdots\\0&0&\cdots&a_n\end{pmatrix}
\]
主对角元即为本征值 $a_i$(均为实数)。

\begin{derivation}
\textbf{验证:}在本征基 $\{|a_i\rangle\}$ 下,矩阵元为
\[
A^{(a)}_{ij} = \langle a_i|A|a_j\rangle = \langle a_i|(a_j|a_j\rangle) = a_j\langle a_i|a_j\rangle = a_j\delta_{ij}
\]
因此只有对角元 $A^{(a)}_{ii}=a_i$ 非零,矩阵为对角矩阵。
\end{derivation}

\subsection{【了解】连续基中的矩阵表示}

前面讨论的都是离散基(有限维或可数无限维),但量子力学中经常遇到连续基(如位置本征态、动量本征态)。矩阵元在连续基下变为函数(积分核)。

\paragraph{连续基的矩阵元}
对连续基 $\{|x\rangle\}$(如位置本征态,满足 $\langle x|x'\rangle=\delta(x-x')$),算符 $A$ 的矩阵元定义为
\[
A(x,x')=\langle x|A|x'\rangle
\]
这是一个二元函数,称为\textbf{积分核}(integral kernel)。

\paragraph{算符作用的积分形式}
算符 $A$ 作用在态 $|\psi\rangle$ 上,在位置表象下写为积分:
\[
(A\psi)(x) = \int_{-\infty}^\infty A(x,x')\psi(x')\,dx'
\]
其中 $\psi(x)=\langle x|\psi\rangle$ 是波函数。

\begin{derivation}
\textbf{推导:}插入连续完备关系 $I=\int|x'\rangle\langle x'|\,dx'$:
\begin{align*}
(A\psi)(x) &= \langle x|A|\psi\rangle
= \langle x|A\left(\int|x'\rangle\langle x'|\,dx'\right)|\psi\rangle \\
&= \int \langle x|A|x'\rangle\langle x'|\psi\rangle\,dx'
= \int A(x,x')\psi(x')\,dx'
\end{align*}
\end{derivation}

这完全类比于离散情形的矩阵乘法 $(A^{(e)}\vec{c})_i=\sum_j A^{(e)}_{ij}c_j$,只是求和变成了积分。

\begin{example}
\textbf{例:位置算符}

位置算符 $X$ 在位置表象下的矩阵元为
\[
X(x,x') = \langle x|X|x'\rangle = x\langle x|x'\rangle = x\delta(x-x')
\]
因此作用在波函数上:
\[
(X\psi)(x) = \int x\delta(x-x')\psi(x')\,dx' = x\psi(x)
\]
即位置算符在位置表象下就是"乘以 $x$"。
\end{example}

\begin{example}
\textbf{例:动量算符}

动量算符 $P$ 在位置表象下的矩阵元为
\[
P(x,x') = \langle x|P|x'\rangle = -i\hbar\frac{\partial}{\partial x}\delta(x-x')
\]
因此作用在波函数上:
\[
(P\psi)(x) = \int\left[-i\hbar\frac{\partial}{\partial x}\delta(x-x')\right]\psi(x')\,dx' = -i\hbar\frac{\partial\psi}{\partial x}(x)
\]
即动量算符在位置表象下是微分算符 $-i\hbar\partial/\partial x$。
\end{example}

\paragraph{不同表象之间的关系}
同一个算符在不同连续基下有不同的积分核。例如,在动量表象下(基为 $\{|p\rangle\}$),算符 $A$ 的矩阵元为
\[
\tilde{A}(p,p')=\langle p|A|p'\rangle
\]
两个表象通过傅里叶变换联系:
\[
\tilde{\psi}(p) = \int e^{-ipx/\hbar}\psi(x)\,dx
\]

\subsection{【背】Pauli矩阵}

Pauli 矩阵是三个 $2\times 2$ 厄米算符,在标准计算基 $\{|0\rangle,|1\rangle\}$ 下的矩阵表示为:
\[
\sigma_x = \begin{pmatrix}0&1\\1&0\end{pmatrix}, \quad
\sigma_y = \begin{pmatrix}0&-i\\i&0\end{pmatrix}, \quad
\sigma_z = \begin{pmatrix}1&0\\0&-1\end{pmatrix}
\]
在量子信息中,通常默认 $\sigma_x$, $\sigma_y$, $\sigma_z$ 指的就是这些矩阵。

\paragraph{记号约定}
有时用下标 $i,j\in\{x,y,z\}$ 表示三个 Pauli 矩阵中的任意一个:
\[
\sigma_i \in \{\sigma_x,\sigma_y,\sigma_z\}
\]
例如 $\{\sigma_i,\sigma_j\}=2\delta_{ij}I$ 表示"当 $i=j$ 时反对易子为 $2I$,当 $i\neq j$ 时为 $0$"。

\paragraph{性质}
\begin{itemize}
  \item 都是厄米的:$\sigma_i=\sigma_i^\dagger$
  \item 都是酉的:$\sigma_i^2=I$
  \item 无迹:$\text{tr}(\sigma_i)=0$
  \item 反对易关系:$\{\sigma_i,\sigma_j\}=2\delta_{ij}I$
  \item 对易关系:$[\sigma_x,\sigma_y]=2i\sigma_z$(及其循环)
  \item 乘法公式:$\sigma_i\sigma_j=\delta_{ij}I+i\sum_k\varepsilon_{ijk}\sigma_k$
\end{itemize}

\paragraph{2×2厄米矩阵展开}
任意 $2\times 2$ 厄米矩阵都可写成
\[
H=a_0 I + a_x\sigma_x + a_y\sigma_y + a_z\sigma_z,\quad a_0,a_x,a_y,a_z\in\mathbb{R}
\]

\paragraph{与自旋的关系(补充)}
对自旋 $1/2$ 粒子,$S_i=\frac{\hbar}{2}\sigma_i$,测量结果对应 $\pm\frac{\hbar}{2}$。

\section{【背】本征值与本征向量(Eigenvalues \& Eigenvectors)}\label{sec:eigen}

\subsection{【背】定义与几何意义}

\paragraph{本征值问题}
给定一个算符(矩阵)$A$,如果存在非零向量 $|\phi\rangle$ 和复数 $\lambda$ 使得
\[
\boxed{A|\phi\rangle = \lambda|\phi\rangle}
\]
则称:
\begin{itemize}
  \item $\lambda$ 为 $A$ 的\textbf{本征值}(eigenvalue,特征值)
  \item $|\phi\rangle$ 为 $\lambda$ 对应的\textbf{本征向量}(eigenvector,特征向量)
\end{itemize}

\paragraph{几何意义}
本征向量是那些被算符 $A$ 作用后\textbf{方向不变}(只是长度变为 $\lambda$ 倍)的特殊向量。

\paragraph{求解方法}
要找本征值,需要解方程:
\[
A|\phi\rangle = \lambda|\phi\rangle \quad \Rightarrow \quad (A-\lambda I)|\phi\rangle = 0
\]
这要求 $\det(A-\lambda I)=0$(特征方程,也称久期方程),解出的 $\lambda$ 就是本征值。

\paragraph{为什么要令 $\det(A-\lambda I)=0$?}
见\hyperref[para:det-invertibility]{行列式与可逆性(补充)}。

\paragraph{完整求解流程(必会,含细节与例子)}
\begin{enumerate}
  \item \textbf{写出特征方程}
  \[
  \det(A-\lambda I)=0
  \]
  \item \textbf{展开行列式并解 $\lambda$} 若 $A=\begin{pmatrix}a&b\\c&d\end{pmatrix}$,则
  \[
  \det(A-\lambda I)=\begin{vmatrix}a-\lambda&b\\c&d-\lambda\end{vmatrix}
  =(a-\lambda)(d-\lambda)-bc=0
  \]
  解得
  \[
  \lambda_{\pm}=\frac{a+d}{2}\pm\sqrt{\left(\frac{a-d}{2}\right)^2+bc}
  \]
  \item \textbf{对每个 $\lambda$ 解本征向量} 以 $\lambda=\lambda_+$ 为例,解
  \[
  (A-\lambda_+ I)\begin{pmatrix}x\\y\end{pmatrix}=0
  \]
  只需取其中一行(另一行线性相关),如
  \[
  (a-\lambda_+)x+by=0 \Rightarrow x=-\frac{b}{a-\lambda_+}y
  \]
  取 $y=1$ 即可得到一个本征向量,再做归一化。
  \item \textbf{归一化与正交化} 令 $|\phi\rangle=(x,y)^T$,归一化为
  \[
  |\tilde\phi\rangle=\frac{1}{\sqrt{|x|^2+|y|^2}}\begin{pmatrix}x\\y\end{pmatrix}
  \]
  若有简并,则对同一本征值的多个本征向量做 Gram--Schmidt 正交化。
\end{enumerate}

\begin{example}
\textbf{2×2 实矩阵求本征值/向量}

设
\[
A=\begin{pmatrix}2&1\\1&2\end{pmatrix}
\]
\textbf{特征方程:}
\[
\det\!\begin{pmatrix}2-\lambda&1\\1&2-\lambda\end{pmatrix}
=(2-\lambda)^2-1=0
\Rightarrow \lambda_1=3,\ \lambda_2=1
\]
\textbf{求本征向量:}
\begin{itemize}
  \item 对 $\lambda_1=3$:$(A-3I)\begin{pmatrix}x\\y\end{pmatrix}=0$
  得 $-x+y=0 \Rightarrow x=y$,取 $(1,1)^T$,归一化为 $\frac{1}{\sqrt{2}}(1,1)^T$。
  \item 对 $\lambda_2=1$:$(A-I)\begin{pmatrix}x\\y\end{pmatrix}=0$
  得 $x+y=0 \Rightarrow x=-y$,取 $(1,-1)^T$,归一化为 $\frac{1}{\sqrt{2}}(1,-1)^T$。
\end{itemize}
\end{example}

\begin{example}
\textbf{简并情况:本征子空间的正交化}

考虑 $3\times 3$ 矩阵
\[
B=\begin{pmatrix}1&1&0\\1&1&0\\0&0&2\end{pmatrix}
\]

\textbf{第1步:求本征值}

特征方程:
\[
\det(B-\lambda I)=\det\begin{pmatrix}1-\lambda&1&0\\1&1-\lambda&0\\0&0&2-\lambda\end{pmatrix}
=(2-\lambda)\det\begin{pmatrix}1-\lambda&1\\1&1-\lambda\end{pmatrix}
\]
\[
=(2-\lambda)[(1-\lambda)^2-1]=(2-\lambda)(1-\lambda-1)(1-\lambda+1)=(2-\lambda)(-\lambda)(2-\lambda)
\]
简化得:$(2-\lambda)^2\cdot(-\lambda)=0$

因此本征值为:$\lambda_1=0$(非简并),$\lambda_2=2$(二重简并)。

\textbf{第2步:求本征向量}

\textbf{对 $\lambda_1=0$(非简并):}
\[
(B-0\cdot I)\begin{pmatrix}x\\y\\z\end{pmatrix}=\begin{pmatrix}1&1&0\\1&1&0\\0&0&2\end{pmatrix}\begin{pmatrix}x\\y\\z\end{pmatrix}=0
\]
得到 $x+y=0$ 和 $2z=0$,即 $y=-x, z=0$。

取 $x=1$,得本征向量 $(1,-1,0)^T$,归一化:
\[
|v_1\rangle=\frac{1}{\sqrt{2}}\begin{pmatrix}1\\-1\\0\end{pmatrix}
\]

\textbf{对 $\lambda_2=2$(二重简并):}
\[
(B-2I)\begin{pmatrix}x\\y\\z\end{pmatrix}=\begin{pmatrix}-1&1&0\\1&-1&0\\0&0&0\end{pmatrix}\begin{pmatrix}x\\y\\z\end{pmatrix}=0
\]
得到 $-x+y=0$,即 $x=y$,而 $z$ 任意。

本征子空间的通解为:
\[
\begin{pmatrix}x\\y\\z\end{pmatrix}=\begin{pmatrix}x\\x\\z\end{pmatrix}
=x\begin{pmatrix}1\\1\\0\end{pmatrix}+z\begin{pmatrix}0\\0\\1\end{pmatrix}
\]
其中 $x,z$ 是任意常数。因此本征子空间由两个线性无关向量张成:
\[
|u_1\rangle=\begin{pmatrix}1\\1\\0\end{pmatrix}, \quad
|u_2\rangle=\begin{pmatrix}0\\0\\1\end{pmatrix}
\]
验证它们线性无关且是本征向量:
\[
B|u_1\rangle=\begin{pmatrix}2\\2\\0\end{pmatrix}=2|u_1\rangle, \quad
B|u_2\rangle=\begin{pmatrix}0\\0\\2\end{pmatrix}=2|u_2\rangle
\]

\textbf{第3步:对简并子空间正交化}

检查 $|u_1\rangle$ 和 $|u_2\rangle$ 是否正交:
\[
\langle u_1|u_2\rangle=\begin{pmatrix}1&1&0\end{pmatrix}\begin{pmatrix}0\\0\\1\end{pmatrix}=0
\]
已经正交!只需归一化:
\[
|v_2\rangle=\frac{1}{\sqrt{2}}\begin{pmatrix}1\\1\\0\end{pmatrix}, \quad
|v_3\rangle=\begin{pmatrix}0\\0\\1\end{pmatrix}
\]

\textbf{总结:}正交归一本征基为
\[
|v_1\rangle=\frac{1}{\sqrt{2}}\begin{pmatrix}1\\-1\\0\end{pmatrix}\ (\lambda=0), \quad
|v_2\rangle=\frac{1}{\sqrt{2}}\begin{pmatrix}1\\1\\0\end{pmatrix}\ (\lambda=2), \quad
|v_3\rangle=\begin{pmatrix}0\\0\\1\end{pmatrix}\ (\lambda=2)
\]
\end{example}

\begin{example}
\textbf{简并本征向量不正交时的 Gram--Schmidt 正交化}

考虑矩阵
\[
A=\begin{pmatrix}1&0&1\\0&2&0\\1&0&1\end{pmatrix}
\]

\textbf{第1步:求本征值}

特征方程:
\[
\det(A-\lambda I)=\det\begin{pmatrix}1-\lambda&0&1\\0&2-\lambda&0\\1&0&1-\lambda\end{pmatrix}
=(2-\lambda)\det\begin{pmatrix}1-\lambda&1\\1&1-\lambda\end{pmatrix}
\]
\[
=(2-\lambda)[(1-\lambda)^2-1]=(2-\lambda)(\lambda^2-2\lambda)=(2-\lambda)\lambda(\lambda-2)
\]
因此本征值为:$\lambda_1=0$(非简并),$\lambda_2=2$(二重简并)。

\textbf{第2步:求本征向量}

\textbf{对 $\lambda_1=0$(非简并):}
\[
(A-0\cdot I)\begin{pmatrix}x\\y\\z\end{pmatrix}=\begin{pmatrix}1&0&1\\0&2&0\\1&0&1\end{pmatrix}\begin{pmatrix}x\\y\\z\end{pmatrix}=0
\]
得到 $x+z=0$ 和 $2y=0$,即 $z=-x, y=0$。

本征向量 $(1,0,-1)^T$,归一化:
\[
|v_0\rangle=\frac{1}{\sqrt{2}}\begin{pmatrix}1\\0\\-1\end{pmatrix}
\]

\textbf{对 $\lambda_2=2$(二重简并):}
\[
(A-2I)\begin{pmatrix}x\\y\\z\end{pmatrix}=\begin{pmatrix}-1&0&1\\0&0&0\\1&0&-1\end{pmatrix}\begin{pmatrix}x\\y\\z\end{pmatrix}=0
\]
得到 $-x+z=0$,即 $z=x$,而 $y$ 任意。

本征子空间的通解为:
\[
\begin{pmatrix}x\\y\\z\end{pmatrix}=\begin{pmatrix}x\\y\\x\end{pmatrix}
=x\begin{pmatrix}1\\0\\1\end{pmatrix}+y\begin{pmatrix}0\\1\\0\end{pmatrix}
\]
其中 $x,y$ 是任意常数。

\textbf{第3步:选择不正交的基向量并正交化}

从通解中,我们可以任意选择两个线性无关的向量。如果选择:
\[
|u_1\rangle=\begin{pmatrix}1\\1\\1\end{pmatrix}, \quad
|u_2\rangle=\begin{pmatrix}1\\0\\1\end{pmatrix}
\]
(它们分别对应 $x=1,y=1$ 和 $x=1,y=0$)

验证它们是 $\lambda=2$ 的本征向量:
\[
A|u_1\rangle=\begin{pmatrix}2\\2\\2\end{pmatrix}=2|u_1\rangle, \quad
A|u_2\rangle=\begin{pmatrix}2\\0\\2\end{pmatrix}=2|u_2\rangle\ \checkmark
\]

检查内积:
\[
\langle u_1|u_2\rangle=1\cdot 1+1\cdot 0+1\cdot 1=2\neq 0
\]
它们不正交!需要 Gram--Schmidt 正交化。

\textbf{Gram--Schmidt 正交化:}
\begin{enumerate}
  \item 归一化 $|u_1\rangle$:
  \[
  |e_1\rangle=\frac{|u_1\rangle}{\||u_1\||}=\frac{1}{\sqrt{1^2+1^2+1^2}}\begin{pmatrix}1\\1\\1\end{pmatrix}
  =\frac{1}{\sqrt{3}}\begin{pmatrix}1\\1\\1\end{pmatrix}
  \]

  \item 从 $|u_2\rangle$ 中减去 $|e_1\rangle$ 方向的投影:

  计算内积:$\langle e_1|u_2\rangle=\frac{1}{\sqrt{3}}(1+0+1)=\frac{2}{\sqrt{3}}$
  \[
  |u_2'\rangle=|u_2\rangle-\langle e_1|u_2\rangle|e_1\rangle
  =\begin{pmatrix}1\\0\\1\end{pmatrix}-\frac{2}{\sqrt{3}}\cdot\frac{1}{\sqrt{3}}\begin{pmatrix}1\\1\\1\end{pmatrix}
  =\begin{pmatrix}1\\0\\1\end{pmatrix}-\frac{2}{3}\begin{pmatrix}1\\1\\1\end{pmatrix}
  \]
  \[
  =\begin{pmatrix}1-2/3\\0-2/3\\1-2/3\end{pmatrix}
  =\begin{pmatrix}1/3\\-2/3\\1/3\end{pmatrix}
  \]

  \item 归一化:
  \[
  \||u_2'\||=\sqrt{(1/3)^2+(-2/3)^2+(1/3)^2}=\sqrt{1/9+4/9+1/9}=\sqrt{6/9}=\sqrt{2/3}
  \]
  \[
  |e_2\rangle=\frac{|u_2'\rangle}{\||u_2'\||}=\sqrt{\frac{3}{2}}\begin{pmatrix}1/3\\-2/3\\1/3\end{pmatrix}
  =\frac{1}{\sqrt{6}}\begin{pmatrix}1\\-2\\1\end{pmatrix}
  \]
\end{enumerate}

验证正交性:
\[
\langle e_1|e_2\rangle=\frac{1}{\sqrt{3}}\cdot\frac{1}{\sqrt{6}}(1\cdot 1+1\cdot(-2)+1\cdot 1)
=\frac{1}{\sqrt{18}}(1-2+1)=0\ \checkmark
\]

\textbf{总结:}完整的正交归一本征基为
\[
|v_0\rangle=\frac{1}{\sqrt{2}}\begin{pmatrix}1\\0\\-1\end{pmatrix}\ (\lambda=0), \quad
|e_1\rangle=\frac{1}{\sqrt{3}}\begin{pmatrix}1\\1\\1\end{pmatrix}\ (\lambda=2), \quad
|e_2\rangle=\frac{1}{\sqrt{6}}\begin{pmatrix}1\\-2\\1\end{pmatrix}\ (\lambda=2)
\]
\end{example}

\paragraph{简并}
若存在 $\lambda_i=\lambda_j\ (i\neq j)$,则称为\textbf{简并}。

\textbf{为什么需要简并的概念?}
\begin{itemize}
  \item \textbf{物理意义}:简并表示系统存在多个线性无关的状态对应同一个测量结果(本征值)。例如,自由粒子在三维空间中,不同方向的动量可能对应相同的能量。
  \item \textbf{数学后果}:简并时,对应该本征值的本征向量不唯一——该本征值的所有本征向量张成一个子空间(称为\textbf{本征子空间}或\textbf{简并子空间}),该子空间内任意向量都是本征向量。
  \item \textbf{实际影响}:
  \begin{itemize}
    \item \textbf{非简并}:每个本征值对应唯一的本征向量方向(归一化后),不同本征值的本征向量天然正交。
    \item \textbf{简并}:同一本征值对应的多个本征向量可能不正交,需要在本征子空间内进行 Gram--Schmidt 正交化,才能得到正交归一本征基。
  \end{itemize}
\end{itemize}

\textbf{非简并与简并的区别总结:}
\begin{center}
\begin{tabular}{l|l|l}
\hline
& \textbf{非简并} & \textbf{简并} \\
\hline
本征值 & 互不相同 & 至少有两个相等 \\
\hline
本征向量 & 每个本征值对应唯一方向 & 每个简并本征值对应子空间 \\
\hline
正交性 & 不同本征值的本征向量天然正交 & 需在简并子空间内正交化 \\
\hline
测量后态 & 塌缩到唯一本征态 & 塌缩到简并子空间内某态 \\
\hline
\end{tabular}
\end{center}

\paragraph{常见特殊矩阵的本征值(速记)}
\begin{itemize}
  \item \textbf{对角矩阵}:本征值就是对角元;本征向量为标准基。
  \item \textbf{上/下三角矩阵}:本征值为对角元(证明由行列式展开)。
  \item \textbf{酉矩阵}:本征值模长为 1(位于单位圆)。
  \item \textbf{厄米矩阵}:本征值为实数,且不同本征值的本征向量正交。
\end{itemize}

\paragraph{本征值/本征向量有什么用?}
\begin{itemize}
  \item \textbf{对角化与简化计算}:若 $A=U\Lambda U^\dagger$,则 $A^n=U\Lambda^n U^\dagger$,指数/函数也可直接在对角线上计算。
  \item \textbf{量子测量}:厄米算符的本征值是测量可能结果,本征态是测后态。
  \item \textbf{时间演化}:若 $H|\phi_n\rangle=E_n|\phi_n\rangle$,则 $e^{-iHt/\hbar}|\phi_n\rangle=e^{-iE_n t/\hbar}|\phi_n\rangle$。
  \item \textbf{物理含义清晰}:本征态是“保持方向不变”的模式,很多物理量在其本征基下最直观。
  \item \textbf{谱分解}:可把算符写成本征态投影的加权和,便于理解与计算。
\end{itemize}

\subsection{【背】厄米算符的谱定理}

若 $H=H^\dagger$,则存在正交归一本征基 $\{|\phi_1\rangle,\ldots,|\phi_n\rangle\}$ 和实本征值 $\{\lambda_1,\ldots,\lambda_n\}$ 使得
\[
\boxed{H = \sum_{j=1}^n \lambda_j|\phi_j\rangle\langle\phi_j|}
\]
这称为 $H$ 的\textbf{谱分解(spectral decomposition)}。
\paragraph{与对角矩阵投影分解的关系(理解用)}
把 $H$ 在本征基下写成对角矩阵 $\Lambda$,再做相似变换回到原基:
\[
H=U\Lambda U^\dagger
\]
其中 $\Lambda$ 可由\hyperref[para:diag-projector]{对角矩阵的投影分解}写成
\[
\Lambda=\sum_j \lambda_j|e_j\rangle\langle e_j|
\]
于是
\[
H=\sum_j \lambda_j\,U|e_j\rangle\langle e_j|U^\dagger
=\sum_j \lambda_j|\phi_j\rangle\langle\phi_j|
\]
也就是说,谱定理就是“\textbf{对角矩阵的投影分解}在本征基下成立,
再用酉变换换回原基”的结果。
\begin{derivation}
\textbf{关键结论 1:厄米算符本征值为实数}

设 $H|\phi\rangle=\lambda|\phi\rangle$,则
\[
\langle\phi|H|\phi\rangle=\lambda\langle\phi|\phi\rangle
\]
对两边取共轭并用 $H^\dagger=H$,
\[
\langle\phi|H|\phi\rangle^\ast=\langle\phi|H^\dagger|\phi\rangle=\langle\phi|H|\phi\rangle
\]
因此 $\langle\phi|H|\phi\rangle$ 为实数,又因 $\langle\phi|\phi\rangle>0$,可得 $\lambda\in\mathbb{R}$。

\textbf{关键结论 2:不同本征值的本征向量正交}

若 $H|\phi\rangle=\lambda|\phi\rangle$、$H|\psi\rangle=\mu|\psi\rangle$ 且 $\lambda\neq\mu$,则
\[
\langle\phi|H|\psi\rangle=\mu\langle\phi|\psi\rangle
\]
另一方面
\[
\langle\phi|H|\psi\rangle=\langle H\phi|\psi\rangle=\lambda\langle\phi|\psi\rangle
\]
两式相减得 $(\lambda-\mu)\langle\phi|\psi\rangle=0$,故 $\langle\phi|\psi\rangle=0$。

\textbf{关键结论 3:存在正交归一本征基(有限维证明思路)}

设 $H$ 为 $n$ 维厄米算符。由特征多项式可知 $H$ 至少有一个本征值 $\lambda_1$ 及对应本征向量 $|\phi_1\rangle$。
将其归一化,使 $\langle\phi_1|\phi_1\rangle=1$。考虑正交补空间
\[
\mathcal{H}_1=\{|\psi\rangle:\langle\phi_1|\psi\rangle=0\}
\]
对任意 $|\psi\rangle\in\mathcal{H}_1$,
\[
\langle\phi_1|H|\psi\rangle=\langle H\phi_1|\psi\rangle=\lambda_1\langle\phi_1|\psi\rangle=0
\]
故 $H|\psi\rangle\in\mathcal{H}_1$,即 $\mathcal{H}_1$ 在 $H$ 作用下不变。
将 $H$ 限制到 $\mathcal{H}_1$ 上,仍是厄米算符。对维数 $n-1$ 的空间重复上述步骤,
用归纳法得到一组正交归一的本征向量 $\{|\phi_1\rangle,\ldots,|\phi_n\rangle\}$ 组成基。

\textbf{关键结论 4:谱分解的构造}

由前文\hyperref[para:orthonormal-unitary]{正交归一基与酉矩阵}可知,令
\[
U=\big(|\phi_1\rangle\ |\phi_2\rangle\ \cdots\ |\phi_n\rangle\big)
\]
则 $U$ 为酉矩阵。因为
\[
H|\phi_j\rangle=\lambda_j|\phi_j\rangle
\]
把各列“并排拼成矩阵”即可得到矩阵等式:矩阵乘法按列作用满足
\[
H\big(|\phi_1\rangle\ \cdots\ |\phi_n\rangle\big)
=\big(H|\phi_1\rangle\ \cdots\ H|\phi_n\rangle\big),
\]
而 $\Lambda=\mathrm{diag}(\lambda_1,\ldots,\lambda_n)$ 为对角矩阵,因此
\[
\big(|\phi_1\rangle\ \cdots\ |\phi_n\rangle\big)\Lambda
=\big(\lambda_1|\phi_1\rangle\ \cdots\ \lambda_n|\phi_n\rangle\big).
\]
逐列比较 $H|\phi_j\rangle=\lambda_j|\phi_j\rangle$ 即得
\[
HU = U\Lambda
\]
两边左乘 $U^\dagger$,得到
\[
U^\dagger H U=\Lambda=\mathrm{diag}(\lambda_1,\ldots,\lambda_n)
\]
其中 $\Lambda$ 表示由本征值组成的对角矩阵。
下面说明 $U\Lambda U^\dagger$ 为何等于投影和形式。由\hyperref[para:diag-projector]{对角矩阵的投影分解}可知
\[
\Lambda=\sum_j \lambda_j |e_j\rangle\langle e_j|
\]
其中 $\{|e_j\rangle\}$ 为标准基。
又因为 $U$ 的第 $j$ 列就是 $|\phi_j\rangle$,矩阵乘法“乘以 $|e_j\rangle$”会选出第 $j$ 列,
所以 $U|e_j\rangle=|\phi_j\rangle$。于是
\[
U\Lambda U^\dagger
=\sum_j \lambda_j\,U|e_j\rangle\langle e_j|U^\dagger
=\sum_j \lambda_j|\phi_j\rangle\langle\phi_j|
\]
因此
\[
H=U\Lambda U^\dagger=\sum_j \lambda_j|\phi_j\rangle\langle\phi_j|
\]
并且满足完备关系
\[
\sum_j |\phi_j\rangle\langle\phi_j|=I
\]
\textbf{也可直接由完备关系推出谱分解}

由于 $\{|\phi_j\rangle\}$ 是正交归一本征基,任意态 $|\psi\rangle$ 可展开为
\[
|\psi\rangle=\sum_j |\phi_j\rangle\langle\phi_j|\psi\rangle
\]
于是
\[
H|\psi\rangle=\sum_j H|\phi_j\rangle\langle\phi_j|\psi\rangle
=\sum_j \lambda_j|\phi_j\rangle\langle\phi_j|\psi\rangle
\]
该式对任意 $|\psi\rangle$ 成立,因此算符恒等成立:
\[
H=\sum_j \lambda_j|\phi_j\rangle\langle\phi_j|
\]
\end{derivation}

\paragraph{谱投影形式(简并情形)}
当本征值存在简并时,对应本征向量不唯一,但\textbf{简并子空间}是确定的。
设 $\lambda$ 的简并子空间为
\[
\mathcal{E}_\lambda=\mathrm{span}\{|\phi_k\rangle:\lambda_k=\lambda\}
\]
其中 $\mathrm{span}\{\cdots\}$ 的含义见\hyperref[para:span]{线性组合与张成};
这里 $\lambda$ 为固定本征值,$\lambda_k$ 为第 $k$ 个本征向量对应的本征值。
并定义索引集 $\mathcal{I}_\lambda$(索引集概念见\hyperref[para:index-set]{索引集})
\[
\mathcal{I}_\lambda=\{\,k:\lambda_k=\lambda\,\}
\]
在该子空间内任选一组正交归一基 $\{|\phi_k\rangle\}_{k\in\mathcal{I}_\lambda}$,
即可定义\textbf{投影算符}
\[
P_\lambda=\sum_{k\in \mathcal{I}_\lambda}|\phi_k\rangle\langle\phi_k|
\]
它把任意态的“\,$\lambda$ 子空间分量\,”投影出来:$P_\lambda|\psi\rangle\in\mathcal{E}_\lambda$。
由于在简并子空间内的基变化只是一个酉变换,$P_\lambda$ 与基的选择无关。

由正交归一性可直接验证
\[
P_\lambda^2=P_\lambda,\quad P_\lambda^\dagger=P_\lambda,\quad
P_\lambda P_{\lambda'}=0\ (\lambda\neq\lambda'),\quad \sum_\lambda P_\lambda=I
\]

再由谱分解按相同本征值分组:
\[
H=\sum_j \lambda_j|\phi_j\rangle\langle\phi_j|
  =\sum_{\lambda}\lambda\sum_{k\in\mathcal{I}_\lambda}|\phi_k\rangle\langle\phi_k|
  =\sum_{\lambda}\lambda\,P_\lambda
\]
因此谱分解等价写为
\[
H=\sum_{\lambda}\lambda\,P_\lambda,\qquad P_\lambda P_{\lambda'}=\delta_{\lambda\lambda'}P_\lambda,\quad \sum_\lambda P_\lambda=I
\]
这在量子测量中尤其常用:$P_\lambda$ 就是测量结果 $\lambda$ 的投影算符。

\begin{example}
\textbf{简并谱的谱分解与投影算符}

设
\[
H=\begin{pmatrix}
2&0&0\\
0&2&0\\
0&0&3
\end{pmatrix}
\]
\textbf{解:}
取标准正交基
\[
|e_1\rangle=\begin{pmatrix}1\\0\\0\end{pmatrix},\quad
|e_2\rangle=\begin{pmatrix}0\\1\\0\end{pmatrix},\quad
|e_3\rangle=\begin{pmatrix}0\\0\\1\end{pmatrix}
\]
$\lambda=2$ 为二重简并,本征子空间由 $|e_1\rangle,|e_2\rangle$ 张成;$\lambda=3$ 的本征向量为 $|e_3\rangle$。
因此
\[
P_{2}=|e_1\rangle\langle e_1|+|e_2\rangle\langle e_2|,\qquad
P_{3}=|e_3\rangle\langle e_3|
\]
对应的矩阵形式为
\[
P_2=\begin{pmatrix}1&0&0\\0&1&0\\0&0&0\end{pmatrix},\qquad
P_3=\begin{pmatrix}0&0&0\\0&0&0\\0&0&1\end{pmatrix}
\]
因此 $P_2+P_3=I$,且 $P_2P_3=0$。
谱分解为
\[
H=2P_{2}+3P_{3}
\]
为了更直观,取具体态
\[
|\psi\rangle=\frac{1}{\sqrt{14}}\begin{pmatrix}1\\2\\3\end{pmatrix}
\]
则
\[
P_2|\psi\rangle=\frac{1}{\sqrt{14}}\begin{pmatrix}1\\2\\0\end{pmatrix},\quad
P_3|\psi\rangle=\frac{1}{\sqrt{14}}\begin{pmatrix}0\\0\\3\end{pmatrix}
\]
并且
\[
H|\psi\rangle=2P_2|\psi\rangle+3P_3|\psi\rangle
=\frac{1}{\sqrt{14}}\begin{pmatrix}2\\4\\9\end{pmatrix}
\]
在 $P_2$ 的子空间内,可用任意 $2\times 2$ 的酉矩阵重新组合基向量,但 $P_2$ 不变。
\textbf{例:}取
\[
U=\frac{1}{\sqrt2}\begin{pmatrix}1&1\\-1&1\end{pmatrix}
\]
在子空间 $\mathrm{span}\{|e_1\rangle,|e_2\rangle\}$ 内做基变换,按
\[
|\tilde e_i\rangle=\sum_{j=1}^2 |e_j\rangle\,U_{ji}\quad (i=1,2)
\]
这等价于把基向量“并成列矩阵”,再右乘 $U$:
\[
\big(|\tilde e_1\rangle\ |\tilde e_2\rangle\big)
=\big(|e_1\rangle\ |e_2\rangle\big)U
\]
(见 2.7 节“基变换与酉矩阵”的列矩阵写法)。因此这是子空间内的酉变换,$P_2$ 保持不变。
\end{example}

\begin{example}
\textbf{由本征对写出谱分解}

已知 $H$ 的正交归一本征对为
$\lambda_1=1,|\phi_1\rangle=\frac{1}{\sqrt2}\begin{pmatrix}1\\1\end{pmatrix}$,
$\lambda_2=-1,|\phi_2\rangle=\frac{1}{\sqrt2}\begin{pmatrix}1\\-1\end{pmatrix}$。

\textbf{解:}
\[
H=1\,|\phi_1\rangle\langle\phi_1|+(-1)\,|\phi_2\rangle\langle\phi_2|
=\begin{pmatrix}0&1\\1&0\end{pmatrix}=\sigma_x
\]
\end{example}

\begin{example}
\textbf{完整本征分解}

求 $H=\begin{pmatrix}1&i\\-i&3\end{pmatrix}$ 的本征值与归一化本征向量。

\textbf{解:}
\begin{enumerate}
  \item 验证厄米性:$H^\dagger=\begin{pmatrix}1&-(-i)^\ast\\i^\ast&3\end{pmatrix}=\begin{pmatrix}1&i\\-i&3\end{pmatrix}=H$ \checkmark

  \item 计算本征值:解 $\det(H-\lambda I)=0$。
  \[
  \det\!\begin{pmatrix}1-\lambda&i\\-i&3-\lambda\end{pmatrix}
  =(1-\lambda)(3-\lambda)-1=\lambda^2-4\lambda+2=0
  \]
  因此
  \[
  \lambda_\pm=2\pm\sqrt{2}
  \]

  \item 对 $\lambda_-=2-\sqrt{2}$,解 $(H-\lambda_- I)|v\rangle=0$:
  \[
  \begin{pmatrix}
  \sqrt{2}-1 & i \\
  -i & 1+\sqrt{2}
  \end{pmatrix}\begin{pmatrix}x\\y\end{pmatrix}=0
  \]
  从第一行:$(\sqrt{2}-1)x+iy=0 \Rightarrow x=\frac{-iy}{\sqrt{2}-1}=\frac{-iy(\sqrt{2}+1)}{1}=-iy(\sqrt{2}+1)$

  取 $y=1$,得 $x=-i(\sqrt{2}+1)$,向量 $\begin{pmatrix}-i(\sqrt{2}+1)\\1\end{pmatrix}$

  范数:$|x|^2+|y|^2=(\sqrt{2}+1)^2+1=3+2\sqrt{2}+1=4+2\sqrt{2}=2(2+\sqrt{2})$

  归一化:$|\phi_-\rangle=\frac{1}{\sqrt{2(2+\sqrt{2})}}\begin{pmatrix}-i(\sqrt{2}+1)\\1\end{pmatrix}$

  \item 类似方法求 $|\phi_+\rangle$ 或利用正交性构造
\end{enumerate}
\end{example}

\section{【背】张量积(Tensor Product)}

\paragraph{为什么需要张量积?}
在量子信息中,我们经常需要处理\textbf{多个量子比特组成的复合系统}。例如:
\begin{itemize}
  \item 两个量子比特:Alice 有一个量子比特,Bob 有一个量子比特
  \item 量子纠缠态:如 Bell 态 $|\Phi^+\rangle=\frac{|00\rangle+|11\rangle}{\sqrt{2}}$
  \item 量子计算:多比特量子门(如 CNOT)作用于两个量子比特
  \item 量子通信:量子隐形传态需要共享纠缠对
\end{itemize}

\textbf{核心问题}:如何用数学描述"两个独立的量子系统组合在一起"?

\textbf{答案}:张量积 $\otimes$ 是描述复合量子系统的标准工具。

\paragraph{张量积的物理意义}
\begin{itemize}
  \item \textbf{独立系统的组合}:若 Alice 的量子比特处于态 $|\psi_A\rangle$,Bob 的处于态 $|\psi_B\rangle$,
  则整个系统处于态 $|\psi_A\rangle\otimes|\psi_B\rangle$
  \item \textbf{维度增长}:单比特是 2 维空间,两比特是 $2\times 2=4$ 维空间,$n$ 比特是 $2^n$ 维空间
  \item \textbf{纠缠的基础}:纠缠态(如 Bell 态)\textbf{无法}写成 $|a\rangle\otimes|b\rangle$ 的形式,
  这正是量子系统超越经典的核心!
  \item \textbf{多比特门的构造}:CNOT 门作用于两比特系统,其矩阵通过张量积构造
\end{itemize}

\paragraph{实际应用举例}
\begin{itemize}
  \item \textbf{计算基展开}:两比特态 $|\psi\rangle=\alpha|00\rangle+\beta|01\rangle+\gamma|10\rangle+\delta|11\rangle$,
  其中 $|00\rangle=|0\rangle\otimes|0\rangle$ 等
  \item \textbf{Bell 态制备}:从 $|00\rangle$ 出发,通过 $H\otimes I$ 和 CNOT 制备纠缠态
  \item \textbf{部分测量}:测量两比特系统中的第一个比特,需要用张量积结构计算约化密度矩阵
  \item \textbf{量子算法}:Shor 算法、Grover 算法等都在多比特空间中运行
\end{itemize}

\textbf{总结}:张量积是量子信息的"语言"——没有它,我们无法谈论多比特系统、纠缠、量子门、量子算法。

\subsection{【背】定义与维度}

对两个向量空间 $V_1,V_2$,其张量积空间 $V_1\otimes V_2$ 满足
\[
\dim(V_1\otimes V_2) = \dim(V_1)\times\dim(V_2)
\]

\paragraph{向量的张量积}
\[
|v\rangle\otimes|w\rangle = \begin{pmatrix}v_1\\v_2\end{pmatrix}\otimes\begin{pmatrix}w_1\\w_2\end{pmatrix}
= \begin{pmatrix}v_1w_1\\v_1w_2\\v_2w_1\\v_2w_2\end{pmatrix}
\]

\paragraph{矩阵的张量积(Kronecker 积)(补充)}
若 $A=\begin{pmatrix}a_{11}&a_{12}\\a_{21}&a_{22}\end{pmatrix}$,$B$ 为任意矩阵,则
\[
A\otimes B=
\begin{pmatrix}
a_{11}B & a_{12}B\\
a_{21}B & a_{22}B
\end{pmatrix}
\]
它与向量张量积的定义一致,可用于构造多比特算符。

\begin{example}
\textbf{计算矩阵的张量积}

计算 $X\otimes Z$,其中 $X=\begin{pmatrix}0&1\\1&0\end{pmatrix}$,$Z=\begin{pmatrix}1&0\\0&-1\end{pmatrix}$

\textbf{解:}根据定义
\begin{align*}
X\otimes Z &= \begin{pmatrix}0&1\\1&0\end{pmatrix}\otimes\begin{pmatrix}1&0\\0&-1\end{pmatrix} \\
&= \begin{pmatrix}
0\cdot\begin{pmatrix}1&0\\0&-1\end{pmatrix} & 1\cdot\begin{pmatrix}1&0\\0&-1\end{pmatrix} \\[0.5em]
1\cdot\begin{pmatrix}1&0\\0&-1\end{pmatrix} & 0\cdot\begin{pmatrix}1&0\\0&-1\end{pmatrix}
\end{pmatrix} \\
&= \begin{pmatrix}
0 & 0 & 1 & 0 \\
0 & 0 & 0 & -1 \\
1 & 0 & 0 & 0 \\
0 & -1 & 0 & 0
\end{pmatrix}
\end{align*}
\end{example}

\subsection{【背】运算规则}

\paragraph{双线性性}
\begin{align}
(c|v\rangle)\otimes|w\rangle &= c(|v\rangle\otimes|w\rangle) = |v\rangle\otimes(c|w\rangle) \\
(|v_1\rangle+|v_2\rangle)\otimes|w\rangle &= |v_1\rangle\otimes|w\rangle + |v_2\rangle\otimes|w\rangle \\
|v\rangle\otimes(|w_1\rangle+|w_2\rangle) &= |v\rangle\otimes|w_1\rangle + |v\rangle\otimes|w_2\rangle
\end{align}

\paragraph{不可交换性}
一般有 $|v\rangle\otimes|w\rangle \neq |w\rangle\otimes|v\rangle$。

\paragraph{标量相乘}
\[
c|v\rangle\otimes|w\rangle = |v\rangle\otimes(c|w\rangle)
\]

\paragraph{内积规则}
\[
\boxed{\langle v_1\otimes w_1 | v_2\otimes w_2\rangle = \langle v_1|v_2\rangle \cdot \langle w_1|w_2\rangle}
\]
\paragraph{张量积基(补充)}
若 $\{|e_i\rangle\}$ 与 $\{|f_j\rangle\}$ 分别为两空间的正交归一基,则
$\{|e_i\rangle\otimes|f_j\rangle\}$ 构成张量积空间的正交归一基。

\paragraph{算符作用规则}
\[
(A\otimes B)(|v\rangle\otimes|w\rangle) = (A|v\rangle)\otimes(B|w\rangle)
\]

\paragraph{算符复合}
\[
(A\otimes B)(C\otimes D) = (AC)\otimes(BD)
\]
特别地,
\[
(A\otimes I)(I\otimes B)=A\otimes B=(I\otimes B)(A\otimes I)
\]

\paragraph{共轭与迹(补充)}
\[
(A\otimes B)^\dagger=A^\dagger\otimes B^\dagger,\qquad
\mathrm{tr}(A\otimes B)=\mathrm{tr}(A)\,\mathrm{tr}(B)
\]
\[
\mathrm{tr}_B(A\otimes B)=\mathrm{tr}(B)\,A
\]

\paragraph{交换算符(Swap)(补充)}
交换算符 $S$ 定义为 $S(|v\rangle\otimes|w\rangle)=|w\rangle\otimes|v\rangle$,
满足 $S^2=I$。在多比特系统中,$S$ 用于改变张量积的次序。
在计算基下可写为
\[
S=\sum_{i,j} |ij\rangle\langle ji|
\]

\subsection{【背】量子比特系统}

\paragraph{单比特计算基} $|0\rangle=\begin{pmatrix}1\\0\end{pmatrix}$,$|1\rangle=\begin{pmatrix}0\\1\end{pmatrix}$

\paragraph{两比特计算基}
\begin{align*}
|00\rangle &= |0\rangle\otimes{}|0\rangle = \begin{pmatrix}1\\0\\0\\0\end{pmatrix} \\
|01\rangle &= |0\rangle\otimes{}|1\rangle = \begin{pmatrix}0\\1\\0\\0\end{pmatrix} \\
|10\rangle &= |1\rangle\otimes{}|0\rangle = \begin{pmatrix}0\\0\\1\\0\end{pmatrix} \\
|11\rangle &= |1\rangle\otimes{}|1\rangle = \begin{pmatrix}0\\0\\0\\1\end{pmatrix}
\end{align*}

\begin{example}
\textbf{展开张量积}

计算 $(|0\rangle+|1\rangle)\otimes(|0\rangle-|1\rangle)$

\textbf{解:}利用双线性性:
\begin{align*}
&(|0\rangle+|1\rangle)\otimes(|0\rangle-|1\rangle) \\
&= |0\rangle\otimes|0\rangle - |0\rangle\otimes|1\rangle + |1\rangle\otimes|0\rangle - |1\rangle\otimes|1\rangle \\
&= |00\rangle - |01\rangle + |10\rangle - |11\rangle \\
&= \begin{pmatrix}1\\-1\\1\\-1\end{pmatrix}
\end{align*}
\end{example}

\begin{example}
\textbf{算符作用}

计算 $(\sigma_x\otimes I)|01\rangle$

\textbf{解:}
\begin{align*}
(\sigma_x\otimes I)|01\rangle
&= (\sigma_x\otimes I)(|0\rangle\otimes|1\rangle) \\
&= (\sigma_x|0\rangle)\otimes(I|1\rangle) \\
&= |1\rangle\otimes|1\rangle = |11\rangle
\end{align*}

或用矩阵形式:
\[
\sigma_x\otimes I = \begin{pmatrix}0&1\\1&0\end{pmatrix}\otimes\begin{pmatrix}1&0\\0&1\end{pmatrix}
= \begin{pmatrix}
0&0&1&0\\
0&0&0&1\\
1&0&0&0\\
0&1&0&0
\end{pmatrix}
\]
\[
(\sigma_x\otimes I)\begin{pmatrix}0\\1\\0\\0\end{pmatrix} = \begin{pmatrix}0\\0\\0\\1\end{pmatrix} = |11\rangle
\]
\end{example}

\subsection{【了解】Bell态(纠缠态例子)}

\[
|\Phi^+\rangle = \frac{|00\rangle+|11\rangle}{\sqrt{2}}, \quad
|\Phi^-\rangle = \frac{|00\rangle-|11\rangle}{\sqrt{2}}
\]
\[
|\Psi^+\rangle = \frac{|01\rangle+|10\rangle}{\sqrt{2}}, \quad
|\Psi^-\rangle = \frac{|01\rangle-|10\rangle}{\sqrt{2}}
\]

注意:Bell态\textbf{不能}写成 $|a\rangle\otimes|b\rangle$ 的形式,这是纠缠的标志。

\section{【背】对易子(Commutators)}

\subsection{【背】定义}

\[
[A,B] := AB - BA
\]

若 $[A,B]=0$,称 $A,B$ \textbf{对易}。

\paragraph{反对易子(Anticommutator)}
\[
\{A,B\} := AB + BA
\]
若 $\{A,B\}=0$,称 $A,B$ \textbf{反对易}。

\subsection{【背】同时对角化定理}

若 $A,B$ 都是厄米算符,则
\[
[A,B]=0 \quad \Longleftrightarrow \quad A,B\text{ 存在共同的正交归一本征基}
\]

\textbf{物理意义:}对易的可观测量可以同时精确测量。
若 $[A,B]\neq 0$,则先测 $A$ 再测 $B$ 与先测 $B$ 再测 $A$ 的统计结果一般不同(测量顺序相关)。
这与测不准关系 $\Delta A\,\Delta B \ge \frac{1}{2}|\langle[A,B]\rangle|$ 相一致。

\subsection{【背】重要例子}

\paragraph{Pauli矩阵}
Pauli 矩阵 $\sigma_x, \sigma_y, \sigma_z$ 之间的对易与反对易关系(定义见 3.9 节):
\begin{itemize}
  \item $[\sigma_x,\sigma_y]=2i\sigma_z$(不对易)
  \item $[\sigma_y,\sigma_z]=2i\sigma_x$
  \item $[\sigma_z,\sigma_x]=2i\sigma_y$
  \item $\{\sigma_i,\sigma_j\}=2\delta_{ij}I$(反对易,其中 $i,j\in\{x,y,z\}$)
  \item $[\sigma_i,I]=0$(任何算符与单位矩阵对易)
\end{itemize}

\begin{example}
\textbf{验证 $[\sigma_x,\sigma_y]=2i\sigma_z$}

\textbf{解:}
\begin{align*}
\sigma_x\sigma_y &= \begin{pmatrix}0&1\\1&0\end{pmatrix}\begin{pmatrix}0&-i\\i&0\end{pmatrix}
= \begin{pmatrix}i&0\\0&-i\end{pmatrix} \\
\sigma_y\sigma_x &= \begin{pmatrix}0&-i\\i&0\end{pmatrix}\begin{pmatrix}0&1\\1&0\end{pmatrix}
= \begin{pmatrix}-i&0\\0&i\end{pmatrix} \\
[\sigma_x,\sigma_y] &= \begin{pmatrix}i&0\\0&-i\end{pmatrix} - \begin{pmatrix}-i&0\\0&i\end{pmatrix}
= \begin{pmatrix}2i&0\\0&-2i\end{pmatrix} = 2i\sigma_z
\end{align*}
\end{example}

\section{【背】重要不等式}

\subsection{【背】Cauchy-Schwarz不等式}

\[
|\langle\phi|\psi\rangle| \le \||\phi\rangle\| \cdot \||\psi\rangle\|
\]

等号成立当且仅当 $|\phi\rangle$ 与 $|\psi\rangle$ 线性相关。

\begin{derivation}
\textbf{证明:}考虑非负量(对任意 $c\in\mathbb{C}$)
\[
0 \le \langle\psi-c\phi|\psi-c\phi\rangle = \langle\psi|\psi\rangle - c\langle\phi|\psi\rangle - c^\ast\langle\psi|\phi\rangle + |c|^2\langle\phi|\phi\rangle
\]
取 $c=\frac{\langle\phi|\psi\rangle}{\langle\phi|\phi\rangle}$(设 $|\phi\rangle\neq 0$),代入整理得
\[
\langle\psi|\psi\rangle - \frac{|\langle\phi|\psi\rangle|^2}{\langle\phi|\phi\rangle} \ge 0
\]
即 $|\langle\phi|\psi\rangle|^2 \le \langle\phi|\phi\rangle\langle\psi|\psi\rangle$,开方即得。
\end{derivation}

\subsection{【了解】三角不等式}

\[
\||\phi\rangle+|\psi\rangle\| \le \||\phi\rangle\| + \||\psi\rangle\|
\]

\section{【背】量子力学基本假设(Postulates)}

\subsection{【背】公设一(假设一):状态空间与量子态}

\paragraph{陈述} 孤立量子系统的状态由复希尔伯特空间中的一个\textbf{归一化态矢} $|\psi\rangle$ 描述(或等价地,由相差全局相位的态矢等价类描述)。

\paragraph{为什么需要复数?为什么需要相位?}
量子态的系数是\textbf{复数}而非实数,这是量子力学的核心特征:
\begin{itemize}
  \item \textbf{复数表示}:任意复数可写成 $\alpha=|\alpha|e^{i\theta}$,其中 $|\alpha|$ 是\textbf{模}(幅度),$\theta$ 是\textbf{相位}(辐角)
  \item \textbf{相位的作用}:相位决定了量子态之间的\textbf{干涉行为}
  \item \textbf{为什么实数不够}:如果系数都是实数,量子态只能是"正叠加"或"负叠加",无法产生量子干涉现象(如双缝实验的干涉条纹)
\end{itemize}

\textbf{物理类比}:量子态的相位类似于波的相位——两个波相遇时,相位决定了是相长干涉还是相消干涉。

\paragraph{归一化与全局相位}
\[
\langle\psi|\psi\rangle = 1, \qquad |\psi\rangle \sim e^{i\gamma}|\psi\rangle
\]

\paragraph{全局相位 vs 相对相位}
\begin{itemize}
  \item \textbf{全局相位}:整个态乘以 $e^{i\gamma}$,即 $|\psi\rangle\to e^{i\gamma}|\psi\rangle$

  例如:$|\psi\rangle=\frac{1}{\sqrt{2}}(|0\rangle+|1\rangle)$ 与 $|\psi'\rangle=e^{i\pi/4}\frac{1}{\sqrt{2}}(|0\rangle+|1\rangle)$ 是\textbf{同一个量子态}

  \textbf{原因}:全局相位不影响任何测量概率,物理上不可观测

  \item \textbf{相对相位}:不同项之间的相位差,如 $\alpha|0\rangle+\beta|1\rangle$ 中 $\beta/\alpha$ 的辐角

  例如:$|\psi_1\rangle=\frac{1}{\sqrt{2}}(|0\rangle+|1\rangle)$ 与 $|\psi_2\rangle=\frac{1}{\sqrt{2}}(|0\rangle-|1\rangle)$ 是\textbf{不同的量子态}

  \textbf{原因}:相对相位影响干涉结果,物理上可观测(例如通过不同基的测量)
\end{itemize}

\textbf{关键结论}:只有\textbf{相对相位}才是物理上有意义的;全局相位可以任意选取(通常选为0)。

\begin{example}
\textbf{全局相位不影响测量概率}

设 $|\psi'\rangle=e^{i\gamma}|\psi\rangle$,则对任意测量态 $|\phi\rangle$,
\[
|\langle\phi|\psi'\rangle|^2 = |\langle\phi|e^{i\gamma}|\psi\rangle|^2
= |e^{i\gamma}|^2|\langle\phi|\psi\rangle|^2
= |\langle\phi|\psi\rangle|^2
\]
因此全局相位不可观测。
\end{example}

\paragraph{叠加原理}
若 $|\psi_1\rangle,|\psi_2\rangle$ 是允许的量子态,则
\[
c_1|\psi_1\rangle+c_2|\psi_2\rangle
\]
仍是允许的量子态(归一化后使用)。

\paragraph{量子比特}
单量子比特态可写为
\[
|\psi\rangle=\alpha|0\rangle+\beta|1\rangle,\quad \alpha,\beta\in\mathbb{C},\quad |\alpha|^2+|\beta|^2=1
\]
$|0\rangle,|1\rangle$ 为\textbf{计算基},两者正交归一。

\paragraph{概率解释(Born 规则预览)}
系数的模方给出测量概率(详见公设三):在计算基 $\{|0\rangle,|1\rangle\}$ 上测量时,
\[
P(0)=|\alpha|^2,\quad P(1)=|\beta|^2
\]
归一化条件 $|\alpha|^2+|\beta|^2=1$ 保证总概率为 1。

\textbf{不同基表示:}同一态可在任意正交基下展开,例如
\[
|+\rangle=\frac{|0\rangle+|1\rangle}{\sqrt{2}},\quad
|-\rangle=\frac{|0\rangle-|1\rangle}{\sqrt{2}}
\]

\begin{example}
\textbf{由测量概率写出态}

若在计算基测量时,得到 $|0\rangle$ 的概率为 $1/3$、$|1\rangle$ 的概率为 $2/3$,则态可写为
\[
|\psi\rangle=\sqrt{\frac{1}{3}}|0\rangle+e^{i\varphi}\sqrt{\frac{2}{3}}|1\rangle
\]
其中 $e^{i\varphi}$ 是相对相位(可取任意值)。通常简化取 $\varphi=0$。
\end{example}

\paragraph{施特恩--盖拉赫实验(量子比特的实验起源)}

施特恩-盖拉赫实验(1922年)是展示量子测量和二能级系统的经典实验。下面用量子信息的语言完整描述。

\textbf{量子系统}:电子自旋是一个\textbf{二能级系统},希尔伯特空间为 $\mathbb{C}^2$。

\textbf{计算基($z$ 基)}:
\[
|\uparrow\rangle = |0\rangle = \begin{pmatrix}1\\0\end{pmatrix},\quad
|\downarrow\rangle = |1\rangle = \begin{pmatrix}0\\1\end{pmatrix}
\]
这两个态是自旋沿 $z$ 方向的\textbf{本征态}。

\textbf{一般量子态}:在测量前,自旋可以处于任意叠加态:
\[
|\psi\rangle = \alpha|\uparrow\rangle + \beta|\downarrow\rangle,\quad |\alpha|^2+|\beta|^2=1
\]

\begin{figure}[htbp]
  \centering
  \includegraphics[width=0.7\linewidth]{fig_stern_gerlach.png}
  \caption{施特恩--盖拉赫实验装置示意:非均匀磁场导致自旋分束。}
\end{figure}

\textbf{测量过程}:S-G 装置实际上是对自旋进行\textbf{投影测量}。

沿 $z$ 方向测量对应的投影算符为:
\[
P_\uparrow = |\uparrow\rangle\langle\uparrow| = \begin{pmatrix}1&0\\0&0\end{pmatrix},\quad
P_\downarrow = |\downarrow\rangle\langle\downarrow| = \begin{pmatrix}0&0\\0&1\end{pmatrix}
\]
满足 $P_\uparrow + P_\downarrow = I$(完备性)。

\textbf{测量概率(Born 规则)}:对态 $|\psi\rangle$ 测量,得到结果的概率为:
\[
P(\uparrow) = \langle\psi|P_\uparrow|\psi\rangle = |\langle\uparrow|\psi\rangle|^2 = |\alpha|^2
\]
\[
P(\downarrow) = \langle\psi|P_\downarrow|\psi\rangle = |\langle\downarrow|\psi\rangle|^2 = |\beta|^2
\]

\textbf{实验结果}:银原子束\textbf{分裂为两束},对应两个可能的测量结果。这证明:
\begin{itemize}
  \item 测量结果是\textbf{离散的}(只有两个可能值),不是连续的
  \item 测量是\textbf{随机的}:相同制备的原子可能得到不同结果
  \item 分束比例由概率 $|\alpha|^2:|\beta|^2$ 决定
\end{itemize}

\textbf{态塌缩}:测量后,量子态塌缩到对应的本征态:
\[
|\psi\rangle = \alpha|\uparrow\rangle + \beta|\downarrow\rangle
\xrightarrow{\text{测得}\uparrow} |\uparrow\rangle
\quad\text{或}\quad
\xrightarrow{\text{测得}\downarrow} |\downarrow\rangle
\]

\textbf{连续测量:不同基的测量}

现在考虑沿 $x$ 方向测量。$x$ 基的本征态为:
\[
|\rightarrow\rangle = |+\rangle = \frac{1}{\sqrt{2}}(|\uparrow\rangle+|\downarrow\rangle),\quad
|\leftarrow\rangle = |-\rangle = \frac{1}{\sqrt{2}}(|\uparrow\rangle-|\downarrow\rangle)
\]

\textbf{关键观察}:$z$ 基的本征态在 $x$ 基下是\textbf{叠加态}!
\[
|\uparrow\rangle = \frac{1}{\sqrt{2}}|\rightarrow\rangle + \frac{1}{\sqrt{2}}|\leftarrow\rangle
\]

\textbf{连续测量实验}:
\begin{enumerate}
  \item 第一次沿 $z$ 方向测量,得到 $|\uparrow\rangle$(概率 $|\alpha|^2$)
  \item 态塌缩到 $|\uparrow\rangle$
  \item 第二次沿 $x$ 方向测量,得到 $|\rightarrow\rangle$ 或 $|\leftarrow\rangle$ 的概率各为:
  \[
  P(\rightarrow) = |\langle\rightarrow|\uparrow\rangle|^2 = \left|\frac{1}{\sqrt{2}}\right|^2 = \frac{1}{2}
  \]
  \[
  P(\leftarrow) = |\langle\leftarrow|\uparrow\rangle|^2 = \left|\frac{1}{\sqrt{2}}\right|^2 = \frac{1}{2}
  \]
  \item 结果:原本"确定"是 $|\uparrow\rangle$ 的原子束,在 $x$ 方向测量时\textbf{又分裂为两束}!
\end{enumerate}

\textbf{量子信息的核心概念}:
\begin{itemize}
  \item \textbf{测量基的选择}:同一个态在不同基下有不同的表示
  \item \textbf{不确定性}:在 $z$ 基确定的态($|\uparrow\rangle$),在 $x$ 基是完全不确定的(50\%-50\%)
  \item \textbf{不对易性}:$z$ 和 $x$ 方向的测量算符不对易,无法同时确定两个方向的自旋
\end{itemize}

\textbf{数学刻画}:测量算符(可观测量)为 Pauli 矩阵:
\[
\sigma_z = |\uparrow\rangle\langle\uparrow| - |\downarrow\rangle\langle\downarrow| = \begin{pmatrix}1&0\\0&-1\end{pmatrix}
\]
\[
\sigma_x = |\rightarrow\rangle\langle\rightarrow| - |\leftarrow\rangle\langle\leftarrow| = \begin{pmatrix}0&1\\1&0\end{pmatrix}
\]
它们满足 $[\sigma_x,\sigma_z]=2i\sigma_y\neq 0$(不对易)。

\textbf{总结}:施特恩-盖拉赫实验直接展示了量子信息的核心要素:
\begin{enumerate}
  \item 二能级系统(量子比特)
  \item Born 规则($P=|\langle\phi|\psi\rangle|^2$)
  \item 测量导致态塌缩
  \item 测量基的选择和不确定性原理
\end{enumerate}

\paragraph{量子比特的物理载体}
任意二能级系统均可作为 qubit,如光子偏振、电子自旋、原子两能级等。

\paragraph{Bloch球表示}
任意纯态等价表示为
\[
|\psi\rangle=\cos\frac{\theta}{2}|0\rangle+e^{i\varphi}\sin\frac{\theta}{2}|1\rangle,\quad \theta\in[0,\pi],\ \varphi\in[0,2\pi)
\]
$(\theta,\varphi)$ 对应 Bloch 球面上一点。
对应的 Bloch 向量为
\[
\vec{r}=(\sin\theta\cos\varphi,\ \sin\theta\sin\varphi,\ \cos\theta)
\]

\begin{figure}[htbp]
  \centering
  \reflectbox{\includegraphics[width=0.45\linewidth,angle=180]{fig_bloch_sphere.png}}
  \caption{Bloch 球:单比特纯态对应球面点,混态对应球内点。}
\end{figure}

\paragraph{纯态与混合态}
纯态可由单个态矢描述;混合态来自不完备知识或统计混合,需用\textbf{密度算子}描述。

\subsection{【背】公设二(假设二):演化(Unitary Evolution)}

\paragraph{陈述} 封闭系统的时间演化由酉算符描述:
\[
|\psi(t_2)\rangle = U(t_1,t_2)|\psi(t_1)\rangle,\quad U^\dagger U=I
\]

\paragraph{薛定谔方程}
\[
i\hbar\frac{\partial}{\partial t}|\psi(t)\rangle = H|\psi(t)\rangle
\]
其中 $H$ 为\textbf{哈密顿量}(哈密尔顿算符)。
若 $H$ 与时间无关,则
\[
U(t_1,t_2)=\exp\!\left[-\frac{i}{\hbar}H(t_2-t_1)\right]
\]

\paragraph{因果性(初值决定演化)}
给定初始态 $|\psi(t_0)\rangle$,薛定谔方程确定其在任意时刻的态。
因此量子力学的时间演化是确定性的(概率性来自测量)。

\paragraph{量子逻辑门(离散演化)}
常用单比特门(均为酉矩阵):
\[
X=\begin{pmatrix}0&1\\1&0\end{pmatrix},\quad
Z=\begin{pmatrix}1&0\\0&-1\end{pmatrix},\quad
H=\frac{1}{\sqrt{2}}\begin{pmatrix}1&1\\1&-1\end{pmatrix}
\]
其中 $X$ 为比特翻转,$Z$ 为相位翻转,$H$ 可在计算基与 Hadamard 基之间切换。

\paragraph{线性性与量子并行性}
由于酉算符线性,
\[
U\!\left(\sum_i a_i|i\rangle\right)=\sum_i a_i\,U|i\rangle
\]
量子门可在叠加态上“并行”作用。

\begin{example}
\textbf{Hadamard 门作用}
\[
H|0\rangle=\frac{|0\rangle+|1\rangle}{\sqrt{2}}=|+\rangle,\quad
H|1\rangle=\frac{|0\rangle-|1\rangle}{\sqrt{2}}=|-\rangle
\]
\end{example}

\subsection{【背】公设三(假设三):测量(Measurement)}\label{sec:measurement}

\paragraph{【背】一般测量(测量算子)}
给定测量算子集合 $\{M_m\}$,满足
\[
\sum_m M_m^\dagger M_m = I
\]
对态 $|\psi\rangle$,得到结果 $m$ 的概率与测后态为:
\[
p(m)=\langle\psi|M_m^\dagger M_m|\psi\rangle,\qquad
|\psi_m\rangle=\frac{M_m|\psi\rangle}{\sqrt{p(m)}}
\]
定义 POVM 元素 $E_m=M_m^\dagger M_m$。
\[
E_m\succeq 0,\qquad \sum_m E_m=I,\qquad p(m)=\langle\psi|E_m|\psi\rangle
\]
\textbf{POVM} 适用于非正交态判别、非破坏/非投影测量等情形。
\textbf{要点:}非正交量子态\textbf{不可}被完美区分,POVM 允许在\textbf{正确率}与\textbf{“不确定/失败”}之间权衡。

\paragraph{【了解】POVM 小例子(补充)}
任意满足 $0\preceq E \preceq I$ 的算符都可构成二元 POVM:$\{E,\ I-E\}$。
例如“带噪 $Z$ 测量”可取
\[
E_0=(1-\epsilon)|0\rangle\langle 0|+\epsilon|1\rangle\langle 1|,\quad
E_1=I-E_0
\]
其中 $\epsilon\in[0,1/2]$ 表示判错概率。$\epsilon=0$ 时退化为投影测量。

\paragraph{【了解】含“失败”结果的 POVM(补充)}
也可引入三元测量 $\{E_0,E_1,E_?\}$,其中
\[
E_0=(1-\eta)|0\rangle\langle 0|,\quad E_1=(1-\eta)|1\rangle\langle 1|,\quad
E_?=\eta I
\]
$E_?$ 表示“不确定/失败”结果($\eta\in[0,1]$)。此类设计常用于在\textbf{降低错误率}与\textbf{允许失败}之间权衡。

\begin{figure}[htbp]
  \centering
  \scalebox{1}[-1]{\includegraphics[width=0.85\linewidth]{fig_measurement_model.png}}
  \caption{测量的间接模型:系统与辅助体系经酉演化后进行读出。}
\end{figure}

若系统与辅助态 $|0\rangle_E$ 先经历酉演化 $U$,再在辅助系统上做投影测量 $\{|m\rangle_E\}$,则
\[
M_m=\langle m|_E\,U\,|0\rangle_E
\]
因此 $\{M_m\}$ 称为 Kraus(测量)算子集。
满足完备性条件 $\sum_m M_m^\dagger M_m=I$,保证非选择性测量保持 $\mathrm{tr}(\rho)$ 不变。
更一般的开放系统演化也可写成 Kraus 形式:$\rho'=\sum_k E_k\rho E_k^\dagger$(完全正、保迹映射)。

对密度算子 $\rho$,有
\[
p(m)=\mathrm{tr}(M_m\rho M_m^\dagger),\quad
\rho_m=\frac{M_m\rho M_m^\dagger}{p(m)}
\]
并可写成
\[
p(m)=\mathrm{tr}(\rho E_m),\quad E_m=M_m^\dagger M_m
\]
若不记录测量结果(\textbf{非选择性测量}),测后态为
\[
\rho'=\sum_m M_m\rho M_m^\dagger
\]

\paragraph{【背】投影测量(可观测量)}
若测量由厄米算符 $M$ 描述,其谱分解为
\[
M=\sum_m m P_m,\quad P_mP_n=\delta_{mn}P_m,\quad \sum_m P_m=I
\]
则 $p(m)=\langle\psi|P_m|\psi\rangle$,测后态为 $P_m|\psi\rangle/\sqrt{p(m)}$。
若忽略结果,则 $\rho'=\sum_m P_m\rho P_m$。
\paragraph{【背】非简并情形(补充)}
若本征值互不相同,则 $P_m=|m\rangle\langle m|$,
\[
p(m)=\mathrm{tr}(\rho P_m)=\langle m|\rho|m\rangle
\]
测后态为 $|m\rangle$(纯态情况)或 $\rho_m=P_m\rho P_m/p(m)$(密度算子形式)。

\paragraph{【了解】简并与非简并测量}
若本征值互不相同,测量后态塌缩为对应本征向量;
若存在简并,则塌缩到对应本征\textbf{子空间}中。

\paragraph{【了解】期望值与概率}
测量可观测量 $M$ 的期望值为
\[
\langle M\rangle=\sum_m m\,p(m)
\]

\paragraph{【了解】三类测量的关系}
当测量算子 $M_m$ 恰为一组\textbf{正交投影算符}时,一般测量退化为投影测量;
POVM 则是最一般的测量描述,可由投影测量在更大系统上实现。

\paragraph{【了解】可重复性与破坏性(补充)}
投影测量具有\textbf{可重复性}:若对同一可观测量立即重复测量,结果保持不变。
一般测量(POVM)可描述\textbf{破坏性测量}(如吸收型探测),测后态未必仍在原可观测量的本征态中。

\paragraph{【了解】测量基的意义}
选择不同测量基等价于在测量前做一个酉变换。比如对 $X$ 方向测量,
可以先施加 $H$ 将基底从 $\{|+\rangle,|-\rangle\}$ 旋回到计算基后再测 $Z$。

\begin{example}
\textbf{对 $|+\rangle$ 做 $Z$ 测量}

计算基为 $|0\rangle,|1\rangle$,$|+\rangle=\frac{|0\rangle+|1\rangle}{\sqrt{2}}$。
投影算子 $P_0=|0\rangle\langle0|$,$P_1=|1\rangle\langle1|$,于是
\[
p(0)=\langle+|P_0|+\rangle=\frac{1}{2},\quad p(1)=\frac{1}{2}
\]
测量后态分别塌缩为 $|0\rangle$ 或 $|1\rangle$。
\end{example}

\begin{example}
\textbf{对 $|0\rangle$ 做 $X$ 测量}

$X$ 基为 $|+\rangle,|-\rangle$,且 $|0\rangle=\frac{|+\rangle+|-\rangle}{\sqrt{2}}$,
因此测得 $|+\rangle$ 与 $|-\rangle$ 的概率均为 $1/2$。
\end{example}

\begin{example}
\textbf{一般态的测量概率计算}

设 $|\psi\rangle=\frac{1}{\sqrt{3}}|0\rangle+\sqrt{\frac{2}{3}}e^{i\pi/4}|1\rangle$。

\textbf{(1) 在计算基 $\{|0\rangle,|1\rangle\}$ 测量:}

投影算符:$P_0=|0\rangle\langle0|$,$P_1=|1\rangle\langle1|$

概率:
\[
p(0)=|\langle0|\psi\rangle|^2=\left|\frac{1}{\sqrt{3}}\right|^2=\frac{1}{3}
\]
\[
p(1)=|\langle1|\psi\rangle|^2=\left|\sqrt{\frac{2}{3}}e^{i\pi/4}\right|^2=\frac{2}{3}
\]

测后态:若测得0,态塌缩为 $|0\rangle$;若测得1,态塌缩为 $|1\rangle$。

\textbf{(2) 在 Hadamard 基 $\{|+\rangle,|-\rangle\}$ 测量:}

首先将 $|\psi\rangle$ 在 $\{|+\rangle,|-\rangle\}$ 基下展开。由于
\[
|0\rangle=\frac{|+\rangle+|-\rangle}{\sqrt{2}},\quad |1\rangle=\frac{|+\rangle-|-\rangle}{\sqrt{2}}
\]

代入得:
\begin{align*}
|\psi\rangle &= \frac{1}{\sqrt{3}}\cdot\frac{|+\rangle+|-\rangle}{\sqrt{2}}+\sqrt{\frac{2}{3}}e^{i\pi/4}\cdot\frac{|+\rangle-|-\rangle}{\sqrt{2}} \\
&= \left(\frac{1}{\sqrt{6}}+\sqrt{\frac{1}{3}}e^{i\pi/4}\right)|+\rangle + \left(\frac{1}{\sqrt{6}}-\sqrt{\frac{1}{3}}e^{i\pi/4}\right)|-\rangle
\end{align*}

概率:
\[
p(+)=\left|\frac{1}{\sqrt{6}}+\sqrt{\frac{1}{3}}e^{i\pi/4}\right|^2,\quad
p(-)=\left|\frac{1}{\sqrt{6}}-\sqrt{\frac{1}{3}}e^{i\pi/4}\right|^2
\]

\textbf{关键观察}:相位 $e^{i\pi/4}$ 在计算基测量中不影响概率(只有模方),但在 Hadamard 基测量中会影响干涉项!
\end{example}

\begin{example}
\textbf{连续测量与态塌缩}

初始态:$|\psi\rangle=\frac{1}{\sqrt{2}}(|0\rangle+|1\rangle)=|+\rangle$

\textbf{第一次测量}:在计算基测量
\begin{itemize}
  \item 概率:$p(0)=p(1)=1/2$
  \item 假设测得结果为 $0$,态塌缩为 $|0\rangle$
\end{itemize}

\textbf{第二次测量}:对塌缩后的态 $|0\rangle$ 在计算基再次测量
\begin{itemize}
  \item 概率:$p(0)=1$,$p(1)=0$
  \item 结果一定是 $0$(可重复性)
\end{itemize}

\textbf{第三次测量}:对 $|0\rangle$ 在 Hadamard 基测量
\begin{itemize}
  \item $|0\rangle=\frac{1}{\sqrt{2}}(|+\rangle+|-\rangle)$
  \item 概率:$p(+)=p(-)=1/2$
  \item 假设测得 $|+\rangle$,态塌缩为 $|+\rangle$
\end{itemize}

\textbf{第四次测量}:对 $|+\rangle$ 在计算基测量
\begin{itemize}
  \item $|+\rangle=\frac{1}{\sqrt{2}}(|0\rangle+|1\rangle)$
  \item 概率:$p(0)=p(1)=1/2$
  \item 原本确定的 $|0\rangle$ 信息丢失了!
\end{itemize}

\textbf{结论}:不同基的测量不对易——在一个基确定的信息,在另一个基测量时会被破坏。
\end{example}

\begin{example}
\textbf{用 POVM 区分非正交态}

考虑两个非正交态:$|0\rangle$ 和 $|+\rangle=\frac{1}{\sqrt{2}}(|0\rangle+|1\rangle)$,出现概率各为 $1/2$。

\textbf{问题}:能否完美区分它们?

\textbf{答案}:不能!因为 $\langle0|+\rangle=\frac{1}{\sqrt{2}}\neq 0$(非正交)。

\textbf{策略1}:投影测量到 $\{|0\rangle,|1\rangle\}$
\begin{itemize}
  \item 若测得 $|1\rangle$,确定是 $|+\rangle$(成功)
  \item 若测得 $|0\rangle$,可能是 $|0\rangle$(概率1)或 $|+\rangle$(概率1/2)——无法确定!
  \item 总错误率:$P_{\text{err}}=\frac{1}{2}\cdot\frac{1}{2}=\frac{1}{4}$
\end{itemize}

\textbf{策略2}:允许"不确定"结果的 POVM

设计三元 POVM $\{E_0,E_+,E_?\}$,其中 $E_?$ 表示"不确定":
\[
E_0=(1-\eta)|0\rangle\langle0|,\quad E_+=\frac{1-\eta}{2}(|-\rangle\langle-|),\quad E_?=\eta I
\]
选择 $\eta=\frac{1}{2}$,可使得在确定时永不出错,但有50\%概率得到"不确定"结果。

\textbf{权衡}:降低错误率 vs 增加失败率。
\end{example}

\paragraph{【了解】施特恩--盖拉赫测量模型(补充)}
沿 $z$ 方向的 SG 装置对应投影算符
\[
P_0=|0\rangle\langle 0|,\quad P_1=|1\rangle\langle 1|
\]
对自旋态测量会将其塌缩到 $|0\rangle$ 或 $|1\rangle$。
若改为沿 $x$ 方向测量,则投影算符为 $|+\rangle\langle +|$ 与 $|-\rangle\langle -|$,
可用 $H$ 门先把 $x$ 基旋回到 $z$ 基再测量。

\subsection{【背】公设四(假设四):复合系统(Composite Systems)}

\paragraph{陈述} 若系统 $A$ 与 $B$ 的状态空间分别为 $\mathcal{H}_A,\mathcal{H}_B$,
则复合系统的状态空间为 $\mathcal{H}_A\otimes\mathcal{H}_B$。

\paragraph{一般态展开}
两比特系统可写成
\[
|\psi\rangle=\sum_{i,j\in\{0,1\}} c_{ij}|i\rangle\otimes|j\rangle
\]
若存在 $|a\rangle,|b\rangle$ 使得 $|\psi\rangle=|a\rangle\otimes|b\rangle$,称为\textbf{可分态};
否则为\textbf{纠缠态}。

\paragraph{多量子比特计算基}
两比特基:$|00\rangle,|01\rangle,|10\rangle,|11\rangle$;
三比特基:$|000\rangle,\ldots,|111\rangle$(共 $2^3$ 个)。
一般 $n$ 比特系统维度为 $2^n$,基可用二进制索引表示。
因此多比特态是 $2^n$ 个基态的线性叠加,但测量只能读出有限经典结果。
\[
|\psi\rangle=\sum_{i=0}^{2^n-1} a_i |i\rangle,\qquad \sum_i |a_i|^2=1
\]

\paragraph{二进制索引与次序(补充)}
通常约定
\[
|i\rangle\otimes|j\rangle = |2i+j\rangle
\]
例如 $|1\rangle\otimes|0\rangle=|10\rangle=|2\rangle$。
具体次序应与电路图中的线路顺序保持一致。

\paragraph{【背】复合系统的部分测量}

在多比特系统中,我们经常需要\textbf{只测量部分比特},而不测量其他比特。这是量子信息处理的核心操作。

\textbf{问题设定}:假设有两个量子比特:
\begin{itemize}
  \item \textbf{系统 $A$}:第一个量子比特(也叫 Alice 的比特)
  \item \textbf{系统 $B$}:第二个量子比特(也叫 Bob 的比特)
  \item \textbf{复合系统 $AB$}:两个比特组成的整体系统,希尔伯特空间是 $\mathcal{H}_A\otimes\mathcal{H}_B$
\end{itemize}

整个系统的态可以写成:
\[
|\psi\rangle_{AB} = \sum_{i,j\in\{0,1\}} c_{ij}|i\rangle_A\otimes|j\rangle_B = \sum_{i,j} c_{ij}|ij\rangle
\]
其中 $|i\rangle_A$ 表示系统 $A$ 的态,$|j\rangle_B$ 表示系统 $B$ 的态。

\textbf{目标}:只测量系统 $A$(第一个比特),不测量系统 $B$(第二个比特)。

\textbf{数学描述}:只测量 $A$ 的算符形式为:
\[
M_m^A = M_m \otimes I_B
\]

\textbf{符号解释}:
\begin{itemize}
  \item $M_m$:\textbf{作用在单个量子比特 $A$ 上的测量算符}($2\times 2$ 矩阵)

  例如,在计算基测量 $A$ 时:
  \[
  M_0 = P_0 = |0\rangle\langle0| = \begin{pmatrix}1&0\\0&0\end{pmatrix},\quad
  M_1 = P_1 = |1\rangle\langle1| = \begin{pmatrix}0&0\\0&1\end{pmatrix}
  \]

  \item $I_B$:\textbf{作用在量子比特 $B$ 上的单位算符}($2\times 2$ 单位矩阵)
  \[
  I_B = \begin{pmatrix}1&0\\0&1\end{pmatrix}
  \]

  \item $M_m \otimes I_B$:\textbf{张量积},得到作用在复合系统 $AB$ 上的算符($4\times 4$ 矩阵)
\end{itemize}

\textbf{物理意义}:
\begin{itemize}
  \item $M_m$ 对系统 $A$ 进行测量
  \item $I_B$ 表示"对系统 $B$ 什么也不做"——让 $B$ 保持原样,不测量它
  \item 整个算符 $M_m\otimes I_B$ 的作用是:测量 $A$,同时保留 $B$ 的信息
\end{itemize}

\textbf{具体例子}:在计算基测量第一个比特

\textbf{测量算符 0}:$(P_0\otimes I_B)$ 对应"测得第一个比特为0"
\[
P_0^A = P_0 \otimes I_B = \begin{pmatrix}1&0\\0&0\end{pmatrix}\otimes\begin{pmatrix}1&0\\0&1\end{pmatrix}
\]

使用张量积的定义(克罗内克积):
\[
P_0^A = \begin{pmatrix}
1\cdot\begin{pmatrix}1&0\\0&1\end{pmatrix} & 0\cdot\begin{pmatrix}1&0\\0&1\end{pmatrix} \\[1em]
0\cdot\begin{pmatrix}1&0\\0&1\end{pmatrix} & 0\cdot\begin{pmatrix}1&0\\0&1\end{pmatrix}
\end{pmatrix}
= \begin{pmatrix}
1&0&0&0\\
0&1&0&0\\
0&0&0&0\\
0&0&0&0
\end{pmatrix}
\]

\textbf{矩阵结构解释}(基序 $|00\rangle,|01\rangle,|10\rangle,|11\rangle$):
\begin{itemize}
  \item 第1行第1列为1:$|00\rangle$ 保留(第一个比特是0)
  \item 第2行第2列为1:$|01\rangle$ 保留(第一个比特是0)
  \item 第3、4行为0:$|10\rangle,|11\rangle$ 被投影掉(第一个比特不是0)
\end{itemize}

\textbf{测量算符 1}:$(P_1\otimes I_B)$ 对应"测得第一个比特为1"
\[
P_1^A = P_1 \otimes I_B = \begin{pmatrix}0&0\\0&1\end{pmatrix}\otimes\begin{pmatrix}1&0\\0&1\end{pmatrix}
= \begin{pmatrix}
0&0&0&0\\
0&0&0&0\\
0&0&1&0\\
0&0&0&1
\end{pmatrix}
\]

\textbf{矩阵结构解释}:
\begin{itemize}
  \item 前两行为0:$|00\rangle,|01\rangle$ 被投影掉(第一个比特不是1)
  \item 第3行第3列为1:$|10\rangle$ 保留(第一个比特是1)
  \item 第4行第4列为1:$|11\rangle$ 保留(第一个比特是1)
\end{itemize}

\textbf{关键观察}:
\[
P_0^A + P_1^A = \begin{pmatrix}
1&0&0&0\\
0&1&0&0\\
0&0&1&0\\
0&0&0&1
\end{pmatrix} = I_{AB}
\]
满足完备性条件!

\textbf{测量概率}:对态 $|\psi\rangle_{AB}=\sum_{i,j}c_{ij}|ij\rangle$,测得第一个比特 $A=0$ 的概率为:
\[
p(A=0) = \langle\psi|(P_0\otimes I)|\psi\rangle = \sum_{j\in\{0,1\}}|c_{0j}|^2 = |c_{00}|^2+|c_{01}|^2
\]

\textbf{物理解释}:把所有"第一个比特为0"的项的概率加起来。

\textbf{测后态}:若测得 $A=0$,复合系统塌缩为:
\[
|\psi'\rangle = \frac{(P_0\otimes I)|\psi\rangle}{\sqrt{p(A=0)}}
= \frac{c_{00}|00\rangle+c_{01}|01\rangle}{\sqrt{|c_{00}|^2+|c_{01}|^2}}
= |0\rangle_A \otimes \frac{c_{00}|0\rangle_B+c_{01}|1\rangle_B}{\sqrt{|c_{00}|^2+|c_{01}|^2}}
\]

\textbf{物理解释}:
\begin{itemize}
  \item 第一个比特(系统 $A$)确定为 $|0\rangle$
  \item 第二个比特(系统 $B$)处于叠加态,由原态中"第一个比特为0"的部分归一化得到
  \item 原态的 $|10\rangle$ 和 $|11\rangle$ 项消失了(被测量"塌缩掉")
\end{itemize}

\textbf{一般规则}($n$ 个量子比特):
\begin{itemize}
  \item 只测量第1个比特:$M_m^{(1)} = M_m \otimes I \otimes I \otimes \cdots \otimes I$
  \item 只测量第2个比特:$M_m^{(2)} = I \otimes M_m \otimes I \otimes \cdots \otimes I$
  \item 只测量第 $k$ 个比特:$M_m^{(k)} = I\otimes\cdots\otimes I \otimes \underbrace{M_m}_{\text{第}k\text{位}} \otimes I \otimes\cdots\otimes I$
\end{itemize}

其中每个 $I$ 都是 $2\times 2$ 单位矩阵,$M_m$ 是 $2\times 2$ 测量算符。

\begin{example}
\textbf{部分测量的详细计算}

考虑态 $|\psi\rangle = \frac{1}{2}(|00\rangle + |01\rangle + |10\rangle + |11\rangle)$

\textbf{(1) 只测量第一个比特(在计算基):}

测量算符:$P_0\otimes I$,$P_1\otimes I$

计算 $p(A=0)$:
\begin{align*}
p(A=0) &= \langle\psi|(P_0\otimes I)|\psi\rangle \\
&= \frac{1}{4}\left[\langle00|+\langle01|+\langle10|+\langle11|\right](P_0\otimes I)(|00\rangle+|01\rangle+|10\rangle+|11\rangle) \\
&= \frac{1}{4}\left[\langle00|+\langle01|+\langle10|+\langle11|\right](|00\rangle+|01\rangle) \\
&= \frac{1}{4}[1+1+0+0] = \frac{1}{2}
\end{align*}

同理 $p(A=1)=\frac{1}{2}$。

测后态(若测得 $A=0$):
\[
|\psi'\rangle = \frac{(P_0\otimes I)|\psi\rangle}{\sqrt{1/2}}
= \frac{\frac{1}{2}(|00\rangle+|01\rangle)}{\sqrt{1/2}}
= \frac{1}{\sqrt{2}}(|00\rangle+|01\rangle)
= |0\rangle\otimes\frac{1}{\sqrt{2}}(|0\rangle+|1\rangle)
\]

\textbf{物理解释}:
\begin{itemize}
  \item 第一个比特确定为 $|0\rangle$
  \item 第二个比特处于 $|+\rangle$ 态
  \item 原态的 $|10\rangle$ 和 $|11\rangle$ 项被"塌缩掉"了
\end{itemize}

\textbf{(2) 只测量第二个比特(在计算基):}

测量算符:$I\otimes P_0$,$I\otimes P_1$

$p(B=0)$:
\begin{align*}
p(B=0) &= \langle\psi|(I\otimes P_0)|\psi\rangle \\
&= \frac{1}{4}\left[\langle00|+\langle01|+\langle10|+\langle11|\right](|00\rangle+|10\rangle) \\
&= \frac{1}{2}
\end{align*}

测后态(若测得 $B=0$):
\[
|\psi'\rangle = \frac{(I\otimes P_0)|\psi\rangle}{\sqrt{1/2}}
= \frac{1}{\sqrt{2}}(|00\rangle+|10\rangle)
= \frac{1}{\sqrt{2}}(|0\rangle+|1\rangle)\otimes|0\rangle
\]

第二个比特确定为 $|0\rangle$,第一个比特处于 $|+\rangle$ 态。

\textbf{(3) 同时测量两个比特:}

这就是通常的测量,测量算符:$P_{ij}=|i\rangle\langle i|\otimes|j\rangle\langle j|$

概率都是 $p(ij)=\frac{1}{4}$,测后态为 $|ij\rangle$(四个可能结果之一)。
\end{example}

\paragraph{部分迹的矩阵直观(补充)}
对两比特密度矩阵 $\rho_{AB}$,对 $B$ 做偏迹等价于对 $B$ 的矩阵块求迹,
可理解为“对 $B$ 的指标求和”,从而得到 $A$ 的 $2\times2$ 矩阵。

\begin{example}
\textbf{可分态 vs 纠缠态的判别}

\textbf{态1}:$|\psi_1\rangle=\frac{1}{\sqrt{2}}(|00\rangle+|10\rangle)$

尝试分解:
\[
|\psi_1\rangle=\frac{1}{\sqrt{2}}(|0\rangle+|1\rangle)\otimes|0\rangle = |+\rangle\otimes|0\rangle
\]
这是\textbf{可分态}(product state)。

\textbf{态2}:$|\psi_2\rangle=\frac{1}{\sqrt{2}}(|00\rangle+|11\rangle)$(Bell态 $|\Phi^+\rangle$)

假设 $|\psi_2\rangle=(\alpha|0\rangle+\beta|1\rangle)\otimes(\gamma|0\rangle+\delta|1\rangle)$

展开右边:$\alpha\gamma|00\rangle+\alpha\delta|01\rangle+\beta\gamma|10\rangle+\beta\delta|11\rangle$

要求:$\alpha\gamma=\frac{1}{\sqrt{2}}$,$\alpha\delta=0$,$\beta\gamma=0$,$\beta\delta=\frac{1}{\sqrt{2}}$

从 $\alpha\delta=0$ 得 $\alpha=0$ 或 $\delta=0$;从 $\beta\gamma=0$ 得 $\beta=0$ 或 $\gamma=0$

但若 $\alpha=0$,则 $\alpha\gamma=0\neq\frac{1}{\sqrt{2}}$,矛盾!其他情况类似。

因此 $|\psi_2\rangle$ \textbf{不可分解},是\textbf{纠缠态}。
\end{example}

\begin{example}
\textbf{两比特系统的测量}

考虑 Bell 态 $|\Phi^+\rangle=\frac{1}{\sqrt{2}}(|00\rangle+|11\rangle)$

\textbf{(1) 只测量第一个比特(在计算基):}

测量算符:$P_0\otimes I=|0\rangle\langle0|\otimes I$,$P_1\otimes I=|1\rangle\langle1|\otimes I$

概率:
\[
p(0)=\langle\Phi^+|(P_0\otimes I)|\Phi^+\rangle
=\frac{1}{2}\left[\langle00|+\langle11|\right](|00\rangle+|11\rangle)
=\frac{1}{2}
\]

同理 $p(1)=\frac{1}{2}$

测后态:若测得0,态塌缩为
\[
\frac{(P_0\otimes I)|\Phi^+\rangle}{\sqrt{p(0)}}
=\frac{(P_0\otimes I)\frac{1}{\sqrt{2}}(|00\rangle+|11\rangle)}{\sqrt{1/2}}
=\frac{\frac{1}{\sqrt{2}}|00\rangle}{\sqrt{1/2}}=|00\rangle
\]

\textbf{关键}:测量第一个比特为0后,第二个比特也确定为0!这是纠缠的体现。

\textbf{(2) 同时测量两个比特(在计算基):}

四个测量算符:$P_{00}=|00\rangle\langle00|$,$P_{01}=|01\rangle\langle01|$,$P_{10}=|10\rangle\langle10|$,$P_{11}=|11\rangle\langle11|$

概率:
\[
p(00)=|\langle00|\Phi^+\rangle|^2=\left|\frac{1}{\sqrt{2}}\right|^2=\frac{1}{2}
\]
\[
p(11)=|\langle11|\Phi^+\rangle|^2=\frac{1}{2}
\]
\[
p(01)=p(10)=0
\]

\textbf{结论}:只能测得 $00$ 或 $11$,永远不会得到 $01$ 或 $10$——两个比特完全关联!
\end{example}

\begin{example}
\textbf{部分迹(偏迹)的计算}

考虑纠缠态 $|\Phi^+\rangle=\frac{1}{\sqrt{2}}(|00\rangle+|11\rangle)$

密度矩阵:
\begin{align*}
\rho_{AB} &= |\Phi^+\rangle\langle\Phi^+| \\
&= \frac{1}{2}(|00\rangle+|11\rangle)(\langle00|+\langle11|) \\
&= \frac{1}{2}(|00\rangle\langle00|+|00\rangle\langle11|+|11\rangle\langle00|+|11\rangle\langle11|)
\end{align*}

矩阵形式(基序 $|00\rangle,|01\rangle,|10\rangle,|11\rangle$):
\[
\rho_{AB}=\frac{1}{2}\begin{pmatrix}
1&0&0&1\\
0&0&0&0\\
0&0&0&0\\
1&0&0&1
\end{pmatrix}
\]

\textbf{对第二个比特做偏迹}:
\[
\rho_A=\mathrm{tr}_B(\rho_{AB})=\sum_{j=0}^1 (I_A\otimes\langle j|_B)\rho_{AB}(I_A\otimes|j\rangle_B)
\]

计算:
\begin{align*}
\rho_A &= (I\otimes\langle0|)\rho_{AB}(I\otimes|0\rangle)+(I\otimes\langle1|)\rho_{AB}(I\otimes|1\rangle) \\
&= \frac{1}{2}[\langle0|00\rangle\langle00|0\rangle+\langle0|11\rangle\langle11|0\rangle]+\frac{1}{2}[\langle1|00\rangle\langle00|1\rangle+\langle1|11\rangle\langle11|1\rangle] \\
&= \frac{1}{2}|0\rangle\langle0|+\frac{1}{2}|1\rangle\langle1| \\
&= \frac{1}{2}I
\end{align*}

矩阵形式:
\[
\rho_A=\frac{1}{2}\begin{pmatrix}1&0\\0&1\end{pmatrix}
\]

\textbf{关键结论}:
\begin{itemize}
  \item $\rho_A$ 是\textbf{最大混合态}(完全随机)
  \item 虽然整体态 $|\Phi^+\rangle$ 是纯态,但子系统 $A$ 的约化态是混态!
  \item 这是\textbf{纠缠}的标志:单独看一个子系统,看不出任何确定信息
  \item 纯度:$\mathrm{tr}(\rho_A^2)=\frac{1}{2}<1$(混态)
\end{itemize}
\end{example}

\begin{example}
\textbf{双自旋纠缠单重态}

双自旋施特恩--盖拉赫实验中常出现单重态
\[
|\Psi^-\rangle=\frac{|01\rangle-|10\rangle}{\sqrt{2}}
\]

\textbf{测量关联}:
\begin{itemize}
  \item 若测量第一个自旋为 $|0\rangle$(向上),第二个必为 $|1\rangle$(向下)
  \item 若测量第一个自旋为 $|1\rangle$(向下),第二个必为 $|0\rangle$(向上)
  \item 两个自旋呈现\textbf{完全反相关}(总自旋为0)
\end{itemize}

\textbf{约化密度矩阵}:
\[
\rho_A=\mathrm{tr}_B(|\Psi^-\rangle\langle\Psi^-|)=\frac{1}{2}|0\rangle\langle0|+\frac{1}{2}|1\rangle\langle1|=\frac{1}{2}I
\]
与 $|\Phi^+\rangle$ 相同——所有 Bell 态的约化态都是最大混合态!
\end{example}

\begin{example}
\textbf{可分态的约化密度矩阵}

考虑可分态 $|\psi\rangle=|+\rangle\otimes|0\rangle=\frac{1}{\sqrt{2}}(|00\rangle+|10\rangle)$

密度矩阵:
\[
\rho_{AB}=|\psi\rangle\langle\psi|=(|+\rangle\langle+|)\otimes(|0\rangle\langle0|)
\]

约化密度矩阵:
\[
\rho_A=\mathrm{tr}_B(\rho_{AB})=\mathrm{tr}_B[(|+\rangle\langle+|)\otimes(|0\rangle\langle0|)]=|+\rangle\langle+|\cdot\mathrm{tr}(|0\rangle\langle0|)=|+\rangle\langle+|
\]

这是\textbf{纯态}!矩阵形式:
\[
\rho_A=\frac{1}{2}\begin{pmatrix}1&1\\1&1\end{pmatrix}
\]

纯度:$\mathrm{tr}(\rho_A^2)=1$(纯态)

\textbf{对比}:可分态的约化态仍是纯态;纠缠态的约化态是混态。这是判别纠缠的重要方法!
\end{example}

\section{【了解】波函数与连续表象}

\paragraph{位置表象与玻恩(Born)解释}
在位置基 $|x\rangle$ 下,波函数定义为
\[
\psi(x,t)=\langle x|\psi(t)\rangle
\]
其物理意义是概率幅度,概率密度为
\[
\rho(x,t)=|\psi(x,t)|^2
\]
归一化条件为
\[
\int_{-\infty}^{\infty} |\psi(x,t)|^2\,dx = 1
\]
因此粒子出现在区间 $[a,b]$ 的概率为
\[
P_{[a,b]}=\int_a^b |\psi(x,t)|^2\,dx
\]
从离散基到连续基的极限中,$\psi(x,t)$ 可视作态矢在位置本征态上的展开系数。

\paragraph{德布罗意物质波与衍射直觉}
自由粒子可用平面波 $\psi(x)\propto e^{ikx}$ 描述,其动量满足
\[
p=\hbar k,\qquad \lambda=\frac{2\pi}{k}=\frac{h}{p}
\]
这对应德布罗意\textbf{物质波}。双缝\textbf{干涉/衍射}实验展示了粒子波动性。

\paragraph{波函数的基本要求}
波函数应当\textbf{单值、有限且连续}(在势能不发散处还应连续可导)。

\paragraph{概率流密度与连续性方程}
对一维系统,概率流密度
\[
j(x,t)=\frac{\hbar}{2mi}\left(\psi^\ast\frac{\partial\psi}{\partial x}-\psi\frac{\partial\psi^\ast}{\partial x}\right)
\]
满足连续性方程
\[
\frac{\partial\rho}{\partial t}+\frac{\partial j}{\partial x}=0
\]
体现了\textbf{概率守恒}。
对自由粒子平面波,可得 $j=\frac{p}{m}\rho$,体现“流密度 = 速度 $\times$ 密度”的直觉。
对全空间积分可得 $\frac{d}{dt}\int |\psi(x,t)|^2 dx=0$,说明归一化随时间保持。
三维情况下
\[
\frac{\partial\rho}{\partial t}+\nabla\cdot\vec{j}=0,\qquad
\vec{j}=\frac{\hbar}{2mi}\left(\psi^\ast\nabla\psi-\psi\nabla\psi^\ast\right)
\]

\paragraph{位置与动量算符}
\[
\hat{x}\psi(x)=x\psi(x),\qquad \hat{p}\psi(x)=-i\hbar\frac{\partial}{\partial x}\psi(x)
\]
其中 $\hat{x}$ 也称\textbf{坐标算符}。
因此
\[
\langle x\rangle=\int \psi^\ast x\psi\,dx,\quad
\langle p\rangle=\int \psi^\ast\left(-i\hbar\frac{\partial}{\partial x}\right)\psi\,dx
\]

\paragraph{能量算符(补充)}
若哈密顿量为 $H=\frac{\hat{p}^2}{2m}+V(\hat{x})$,则
\[
i\hbar\frac{\partial}{\partial t}\psi(x,t)=H\psi(x,t)
\]
形式上可把 $i\hbar\frac{\partial}{\partial t}$ 视为能量算符在位置表象中的作用。

\paragraph{动量表象中的平均值}
若动量波函数为 $\phi(p)$,则
\[
\langle p\rangle=\int p\,|\phi(p)|^2\,dp
\]
位置与动量表象通过傅里叶变换联系。

\begin{example}
\textbf{平面波的动量期望}

设 $\psi(x)=Ae^{ikx}$,则
\[
\hat{p}\psi=\hbar k\psi \Rightarrow \langle p\rangle=\hbar k
\]
\end{example}

\section{【了解】定态薛定谔方程与典型模型}

\subsection{【了解】定态薛定谔方程}
当哈密顿量 $H$ 不显含时间时,可将含时薛定谔方程分离变量:
\[
|\psi(t)\rangle = e^{-iEt/\hbar}|\phi\rangle
\]
代入得\textbf{定态薛定谔方程}(本征值问题):
\[
H|\phi\rangle = E|\phi\rangle
\]

\paragraph{定态的物理意义}
若系统处于某一能量本征态 $|\phi\rangle$,概率密度
\[
\rho(x,t)=|\psi(x,t)|^2
\]
与时间无关。能量本征态称为\textbf{定态}。

\paragraph{本征态/本征解展开}
任意初态都可展开为本征态叠加(离散谱为求和,连续谱为积分):
\[
|\psi(t)\rangle=\sum_n c_n e^{-iE_n t/\hbar}|\phi_n\rangle
  \quad\text{或}\quad
|\psi(t)\rangle=\int c(E)\,e^{-iEt/\hbar}|\phi_E\rangle\,dE
\]
$|c_n|^2$ 表示测得能量为 $E_n$ 的概率。

\subsection{【了解】本征函数性质与边界条件}
\begin{itemize}
  \item \textbf{波函数条件}:单值、有限、连续(在势能不发散处还应连续可导)
  \item \textbf{正交与完备}:束缚态本征函数彼此正交并可构成完备基
  \item \textbf{非简并性}:一维束缚态能级通常非简并
  \item \textbf{量子化来源}:边界条件使能量谱离散化
\end{itemize}

\subsection{【了解】典型模型(速记)}

\paragraph{一维无限深势阱}
势阱 $V(x)=0\ (0<x<L)$,其余处 $V=\infty$。本征函数与能级:
\[
\phi_n(x)=\sqrt{\frac{2}{L}}\sin\frac{n\pi x}{L},\quad
E_n=\frac{n^2\pi^2\hbar^2}{2mL^2},\quad n=1,2,\ldots
\]

\paragraph{一维谐振子}
能级为
\[
E_n=\left(n+\frac{1}{2}\right)\hbar\omega,\quad n=0,1,2,\ldots
\]
本征函数含\textbf{厄密多项式}(Hermite)。
其方程可化为\textbf{厄密方程}(Hermite equation)。

\paragraph{氢原子(中心势问题)}
在球坐标中分离变量,能级
\[
E_n=-\frac{13.6\ \text{eV}}{n^2},\quad n=1,2,\ldots
\]
波函数由径向部分与角向部分构成,径向部分含\textbf{广义拉盖尔多项式},
量子数为 $(n,l,m)$。基态波函数常写为
\[
\psi_{100}(r)=\frac{1}{\sqrt{\pi a_0^3}}e^{-r/a_0}
\]
其中 $a_0$ 为\textbf{波尔半径}。

\section{【了解】算符、对易关系与期望值演化}

\paragraph{算符与厄米性}
算符是将函数/态映射到函数/态的运算。物理可观测量对应厄米算符:
\[
\langle\phi|A|\psi\rangle=\langle A\phi|\psi\rangle \iff A^\dagger=A
\]

\paragraph{基本对易关系}
位置与动量满足
\[
[\hat{x},\hat{p}]=i\hbar
\]

\paragraph{角动量算符}
\[
\vec{L}=\vec{r}\times\vec{p},\qquad
L_x= y p_z - z p_y,\ L_y= z p_x - x p_z,\ L_z= x p_y - y p_x
\]
其对易关系为
\[
[L_x,L_y]=i\hbar L_z \quad (\text{循环置换})
\]

\paragraph{算符函数(提示)}
对算符 $A$ 的函数可用泰勒级数定义,例如
\[
f(A)=\sum_{n=0}^\infty c_n A^n
\]
当算符不对易时需注意乘法顺序不可交换。

\paragraph{Ehrenfest/海森堡型公式}
若 $A$ 不显含时间,则
\[
\frac{d}{dt}\langle A\rangle_\psi=\frac{i}{\hbar}\langle[H,A]\rangle_\psi
\]
若 $A$ 显含时间,则
\[
\frac{d}{dt}\langle A\rangle_\psi=\frac{i}{\hbar}\langle[H,A]\rangle_\psi+\left\langle\frac{\partial A}{\partial t}\right\rangle_\psi
\]
若 $[H,A]=0$(且 $A$ 不显含时间),则 $\langle A\rangle_\psi$ 为常量,对应守恒量。
特别地,若 $A=H$ 且 $H$ 不显含时间,则 $\frac{d}{dt}\langle H\rangle_\psi=0$(能量守恒)。

\paragraph{位置与动量的经典极限(补充)}
对 $H=\frac{\hat{p}^2}{2m}+V(\hat{x})$,可得
\[
\frac{d}{dt}\langle x\rangle_\psi=\frac{\langle p\rangle_\psi}{m},\qquad
\frac{d}{dt}\langle p\rangle_\psi=-\left\langle \frac{\partial V}{\partial x}\right\rangle_\psi
\]
这体现了 Ehrenfest 定理与经典牛顿方程的对应关系。

\section{【了解】薛定谔/海森堡/相互作用表象}

\paragraph{术语说明}
此处“表象”(picture)指动力学描述方式,不同于前文基变换意义下的“表象/表示”(representation)。

\paragraph{薛定谔表象}
态矢随时间演化,算符(若无显含时间)不随时间变:
\[
i\hbar\frac{\partial}{\partial t}|\psi(t)\rangle=H|\psi(t)\rangle
\]

\paragraph{海森堡表象}
态矢固定,算符随时间演化:
\[
\frac{dA_H}{dt}=\frac{i}{\hbar}[H,A_H]+\left(\frac{\partial A}{\partial t}\right)_H
\]
若 $U(t)=e^{-iHt/\hbar}$,则
\[
A_H(t)=U^\dagger(t)\,A_S\,U(t),\qquad |\psi_H\rangle=|\psi(0)\rangle
\]

\paragraph{相互作用表象}
将哈密顿量拆分 $H=H_0+V$,部分时间依赖移到态矢,部分移到算符,
便于处理微扰或相互作用问题。
\[
|\psi_I(t)\rangle=U_0^\dagger(t)|\psi_S(t)\rangle,\quad
A_I(t)=U_0^\dagger(t)A_SU_0(t),\quad U_0(t)=e^{-iH_0 t/\hbar}
\]

\paragraph{三种表象的等价性(补充)}
无论采用哪种表象,\textbf{物理预测一致}:
\[
\langle A\rangle
=\langle\psi_S(t)|A_S|\psi_S(t)\rangle
=\langle\psi_H|A_H(t)|\psi_H\rangle
\]
只是“时间依赖”分配在态矢还是算符上不同。

\section{【背】态叠加与概率幅度规则}

\paragraph{叠加原理}
若 $\psi_1,\psi_2$ 是薛定谔方程的解,则任意线性组合 $c_1\psi_1+c_2\psi_2$ 仍是解。

\paragraph{概率幅度的四条规则(简化版)}
\begin{enumerate}
  \item \textbf{不可区分路径}:总幅度为各路径幅度之和。
  \item \textbf{可区分末态}:总概率为各末态概率之和。
  \item \textbf{连续跃迁}:总幅度为分段幅度之积。
  \item \textbf{独立体系}:总体幅度为各子系统幅度之积。
\end{enumerate}
这说明量子叠加是\textbf{幅度相加}而非概率相加。
典型例子是\textbf{电子双缝干涉}:两路径幅度相加产生干涉条纹。
\textbf{Born 规则:}概率 $P=|\text{幅度}|^2$。

\paragraph{双缝干涉的幅度叠加(补充)}
若通过两条不可区分路径到达同一末态的幅度分别为 $\psi_1,\psi_2$,则
\[
P=|\psi_1+\psi_2|^2=|\psi_1|^2+|\psi_2|^2+2\mathrm{Re}(\psi_1^\ast\psi_2)
\]
最后一项即干涉项。若引入\textbf{路径信息}使两路径可区分,则干涉项消失,概率退化为 $|\psi_1|^2+|\psi_2|^2$。

\section{【背】表象与酉变换}

\paragraph{基与表象}
设 $\{|a_i\rangle\}$、$\{|b_j\rangle\}$ 是两组正交归一基,定义
\[
U_{ji}=\langle b_j|a_i\rangle
\]
则 $U$ 为酉矩阵,系数在两基之间变换为
\[
c'_j=\sum_i U_{ji}c_i \quad \Longleftrightarrow \quad {\vec{c}\,}'=U\vec{c}
\]

\paragraph{算符的表象变换}
若算符在 $a$ 表象中矩阵为 $A^{(a)}$,在 $b$ 表象中矩阵为 $A^{(b)}$,则
\[
A^{(b)}=UA^{(a)}U^\dagger
\]
酉变换不改变算符的本征值。

\paragraph{例:计算基与 Hadamard 基变换(补充)}
设 $a$ 表象为计算基 $\{|0\rangle,|1\rangle\}$,$b$ 表象为 $|+\rangle,|-\rangle$,
则
\[
U=\frac{1}{\sqrt{2}}
\begin{pmatrix}
1&1\\
1&-1
\end{pmatrix},\qquad {\vec{c}\,}'=U\vec{c}
\]
例如 $|0\rangle$ 的系数 $\vec{c}=(1,0)^T$,在 $b$ 表象中为 ${\vec{c}\,}'=(\frac{1}{\sqrt{2}},\frac{1}{\sqrt{2}})^T$。
算符也满足 $A^{(b)}=UA^{(a)}U^\dagger$,因此 $Z$ 在 Hadamard 基下变为
\[
UZU^\dagger = X
\]

\paragraph{术语对照}
课件中常写“么正变换”,即\textbf{幺正/酉变换}的同义说法。

\paragraph{连续完备基}
位置表象满足完备性:
\[
\int |x\rangle\langle x|\,dx = I,\quad \psi(x)=\langle x|\psi\rangle
\]
动量表象同理:$\int |p\rangle\langle p|\,dp=I$。

\section{【了解】位置/动量表象与傅里叶变换}

\paragraph{傅里叶变换对}
位置与动量波函数互为傅里叶变换:
\[
\phi(p)=\frac{1}{\sqrt{2\pi\hbar}}\int_{-\infty}^{\infty} e^{-ipx/\hbar}\psi(x)\,dx
\]
\[
\psi(x)=\frac{1}{\sqrt{2\pi\hbar}}\int_{-\infty}^{\infty} e^{ipx/\hbar}\phi(p)\,dp
\]
且归一化保持:
\[
\int |\psi(x)|^2 dx=\int |\phi(p)|^2 dp=1
\]

\paragraph{位置/动量本征态(补充)}
位置与动量本征态满足
\[
\langle x|p\rangle=\frac{1}{\sqrt{2\pi\hbar}}e^{ipx/\hbar},\quad
\langle x|x'\rangle=\delta(x-x'),\quad
\langle p|p'\rangle=\delta(p-p')
\]
因此动量本征态在位置表象中是\textbf{平面波},位置本征态则对应 $\delta$ 函数。
\[
\int |p\rangle\langle p|\,dp = I,\qquad \int |x\rangle\langle x|\,dx = I
\]

\section{【背】密度算子与混合态}

\paragraph{定义}
若系统以概率 $p_k$ 处于纯态 $|\psi_k\rangle$,则其密度算子为
\[
\rho=\sum_k p_k|\psi_k\rangle\langle\psi_k|,\quad p_k\ge 0,\quad \sum_k p_k=1
\]
\textbf{纯态特例:}若系统确定处于 $|\psi\rangle$,则
\[
\rho=|\psi\rangle\langle\psi|
\]

\paragraph{性质}
\begin{itemize}
  \item $\rho^\dagger=\rho$(厄米)
  \item $\rho\succeq 0$(半正定)
  \item $\mathrm{tr}(\rho)=1$
  \item 纯态判据:$\rho^2=\rho$ 或 $\mathrm{tr}(\rho^2)=1$
\end{itemize}
混态满足 $\mathrm{tr}(\rho^2)<1$。
\paragraph{迹的基不变性(补充)}
对任意酉变换 $U$,有 $\mathrm{tr}(U\rho U^\dagger)=\mathrm{tr}(\rho)$,
因此 $\mathrm{tr}(\rho)=1$ 与基的选择无关。
\paragraph{纯度的不变性(补充)}
纯度 $\mathrm{tr}(\rho^2)$ 在酉演化下保持不变:$\mathrm{tr}((U\rho U^\dagger)^2)=\mathrm{tr}(\rho^2)$。

\paragraph{谱分解与概率(补充)}
密度算子可在其本征基中写成
\[
\rho=\sum_k p_k |k\rangle\langle k|
\]
其中 $p_k\in[0,1]$ 且 $\sum_k p_k=1$。这些 $p_k$ 可解释为处于本征态 $|k\rangle$ 的概率。

\paragraph{【了解】相干与经典混合(补充)}
密度矩阵的\textbf{非对角元}描述量子相干。以单比特为例,
\[
|+\rangle=\frac{|0\rangle+|1\rangle}{\sqrt{2}}
\Rightarrow \rho_{+}=\frac{1}{2}\begin{pmatrix}1&1\\1&1\end{pmatrix}
\]
而\textbf{经典混合} $\rho_{\text{mix}}=\frac{1}{2}(|0\rangle\langle0|+|1\rangle\langle1|)=\frac{1}{2}I$
在 $Z$ 基测量上概率相同,但在 $X$ 基上表现不同。混态分解一般\textbf{不唯一}。
混态常由与\textbf{环境耦合}、或对环境做偏迹产生,即“丢失”部分自由度导致相干衰减(退相干)。

\paragraph{【了解】单比特密度矩阵与Bloch向量}
任意单比特密度矩阵可写为
\[
\rho=\frac{1}{2}(I+\vec{r}\cdot\vec{\sigma}),\quad |\vec{r}|\le 1
\]
其中 $\vec{\sigma}=(\sigma_x,\sigma_y,\sigma_z)$。纯态对应 $|\vec{r}|=1$,完全混态对应 $\vec{r}=0$。
纯度与 Bloch 向量长度满足
\[
\mathrm{tr}(\rho^2)=\frac{1+|\vec{r}|^2}{2}
\]
Bloch 向量可由
\[
r_i=\mathrm{tr}(\rho\,\sigma_i)
\]
直接求得。

\paragraph{【了解】Bloch球中的混态}
单比特纯态位于 Bloch 球面,混态对应球内点;密度矩阵可写为
\[
\rho=\sum_n p_n|\psi_n\rangle\langle\psi_n|
\]
其中 $p_n$ 为系统处于纯态 $|\psi_n\rangle$ 的概率。
若仅混合两个纯态 $|\psi_A\rangle,|\psi_B\rangle$,其 Bloch 向量位于两点连线段上,
体现混态是纯态的\textbf{凸组合}。

\paragraph{期望值与演化}
\[
\langle A\rangle = \mathrm{tr}(\rho A),\quad
\rho' = U\rho U^\dagger
\]
若 $U=\exp(-\frac{i}{\hbar}Ht)$,则满足冯诺依曼方程
\[
i\hbar\,\frac{d\rho}{dt}=[H,\rho]
\]
\paragraph{测量概率(补充)}
在正交基 $\{|e_i\rangle\}$ 上投影测量时,结果 $i$ 的概率为
\[
p_i=\langle e_i|\rho|e_i\rangle
\]
即密度矩阵在该基下的对角元。

\paragraph{约化密度算子(偏迹)}
对复合系统 $AB$ 的密度算子 $\rho_{AB}$,子系统 $A$ 的约化态为
\[
\rho_A=\mathrm{tr}_B(\rho_{AB})
\]
在 $B$ 的正交归一基 $\{|b_k\rangle\}$ 下可写为
\[
\rho_A=\sum_k (I\otimes \langle b_k|)\,\rho_{AB}\,(I\otimes |b_k\rangle)
\]

\paragraph{局域可观测量(补充)}
对子系统 $A$ 的可观测量 $A$,其期望值可用约化密度算子计算:
\[
\langle A\rangle=\mathrm{tr}_{AB}(\rho_{AB}\,A\otimes I)=\mathrm{tr}_A(\rho_A A)
\]
说明偏迹保留了子系统的全部可观测统计信息。
\paragraph{张量积的迹公式(补充)}
对可分态 $\rho_{AB}=\rho_A\otimes\rho_B$,有
\[
\mathrm{tr}(\rho_{AB} A\otimes B)=\mathrm{tr}(\rho_A A)\,\mathrm{tr}(\rho_B B)
\]

\begin{example}
\textbf{Bell态的约化密度矩阵}

设 $|\Phi^+\rangle=\frac{|00\rangle+|11\rangle}{\sqrt{2}}$,
\[
\rho_{AB}=|\Phi^+\rangle\langle\Phi^+|,\quad
\rho_A=\mathrm{tr}_B(\rho_{AB})=\frac{1}{2}I
\]
说明纠缠态的子系统表现为混态。
\end{example}

\paragraph{【了解】纠缠判据(纯态)(补充)}
对纯态 $|\psi\rangle_{AB}$,若其约化态满足
\[
\mathrm{tr}(\rho_A^2)=1
\]
则 $|\psi\rangle_{AB}$ 为可分态;若 $\mathrm{tr}(\rho_A^2)<1$,则为纠缠态。
\paragraph{【了解】Schmidt 分解(补充)}
任意纯态 $|\psi\rangle_{AB}$ 都可写成
\[
|\psi\rangle_{AB}=\sum_k \sqrt{\lambda_k}\,|u_k\rangle_A\otimes|v_k\rangle_B,\quad \lambda_k\ge 0,\ \sum_k\lambda_k=1
\]
若且仅若只有一个非零 $\lambda_k$,则态为可分态;否则为纠缠态。
\paragraph{【了解】Schmidt 数(补充)}
非零 Schmidt 系数的个数称为 Schmidt 数。Schmidt 数为 1 当且仅当态可分。
\paragraph{【了解】纠缠熵(补充)}
纯态的纠缠可用冯诺依曼熵刻画:
\[
S(\rho_A)=-\mathrm{tr}(\rho_A\log\rho_A)
\]
对纯态 $|\psi\rangle_{AB}$,有 $S(\rho_A)=S(\rho_B)$,且 $S=0$ 当且仅当可分。
\paragraph{【了解】熵的最大值(补充)}
对 $d$ 维系统,$S(\rho)\le \log d$,最大值在 $\rho=I/d$(完全混态)时取得。

\section{【了解】测不准关系(补充)}

\paragraph{一般形式}
对任意厄米算符 $A,B$,
\[
\Delta A\,\Delta B \ge \frac{1}{2}\left|\langle[A,B]\rangle\right|
\]
其中 $\Delta A=\sqrt{\langle A^2\rangle-\langle A\rangle^2}$。

\paragraph{不确定度的含义(补充)}
$\Delta A$ 是测量结果分布的\textbf{标准差}(statistical spread),
并非仪器精度的主观限制。Kennard(1927)给出了位置-动量的精确不等式形式。
若系统处于 $A$ 的本征态,则 $\Delta A=0$;但对与 $A$ 不对易的算符 $B$,仍受不确定性下界约束。

\paragraph{位置-动量}
\[
\Delta x\,\Delta p \ge \frac{\hbar}{2}
\]
该结论可由 Cauchy--Schwarz 不等式推出,是量子统计规律的直接体现。

\paragraph{时间-能量}
时间与能量也满足类似的不等式(形式上)
\[
\Delta t\,\Delta E \gtrsim \frac{\hbar}{2}
\]
常用于估计能级寿命与谱线宽度的关系。
不确定性反映的是\textbf{量子本征统计规律},而非测量技术不足。
对应到频率亦可写为 $\Delta t\,\Delta \omega \gtrsim \frac{1}{2}$(形式上)。

\section{【了解】量子纠缠、EPR佯谬与贝尔不等式}

\subsection{【了解】纠缠态的定义}
\paragraph{可分态 vs 纠缠态}
若两比特态可写成 $|\psi\rangle=|a\rangle\otimes|b\rangle$,则为可分态;
否则称为\textbf{纠缠态}。例如
\[
|\Psi^+\rangle=\frac{|01\rangle+|10\rangle}{\sqrt{2}}
\]
无法写成单比特态的张量积,因此是纠缠态。

\begin{example}
\textbf{可分态示例}
\[
\frac{|00\rangle+|10\rangle}{\sqrt{2}}
=\left(\frac{|0\rangle+|1\rangle}{\sqrt{2}}\right)\otimes|0\rangle
\]
该态可以分解为张量积,因此不是纠缠态。
\end{example}

\paragraph{纠缠的核心特征}
\begin{itemize}
  \item \textbf{强关联}:测量一个子系统会瞬时确定另一个子系统的结果分布。
  \item \textbf{非定域关联}:相关性与距离无关,但\textbf{不能}用于超光速通信。
  \item \textbf{应用广泛}:量子隐形传态、量子密钥分发、量子计算与量子精密测量的关键资源。
  \item \textbf{易受退相干}:高质量纠缠的制备与保持在实验上具有挑战。
\end{itemize}

\paragraph{四个Bell态}
\[
|\Phi^\pm\rangle=\frac{|00\rangle\pm|11\rangle}{\sqrt{2}},\quad
|\Psi^\pm\rangle=\frac{|01\rangle\pm|10\rangle}{\sqrt{2}}
\]
Bell态是两比特最大纠缠态,可由 $H$ 门 + CNOT 电路制备。
四个 Bell 态构成两比特的\textbf{正交归一基}:
\[
\sum_{\mu\in\{\Phi^\pm,\Psi^\pm\}} |\mu\rangle\langle\mu|=I
\]
\paragraph{完美关联示例(补充)}
对 $|\Psi^-\rangle$,若两方沿同一方向测量自旋,总是得到相反结果(完美反相关)。

\begin{figure}[htbp]
  \centering
  \includegraphics[width=0.6\linewidth]{fig_bell_measurement.png}
  \caption{Bell 态测量的光学干涉装置示意(分束器 + 偏振分束器 + 探测器)。}
\end{figure}

\paragraph{不同基下的表示}
例如
\[
|\Phi^+\rangle=\frac{|00\rangle+|11\rangle}{\sqrt{2}}
=\frac{|++\rangle+|--\rangle}{\sqrt{2}}
\]
说明纠缠态在不同测量基下仍保持强关联。

\subsection{【了解】EPR佯谬}
1935 年 EPR 提出“定域实在论”挑战:
\begin{enumerate}
  \item \textbf{实在性}:若可在不扰动系统的前提下精确预测某物理量,则该物理量具有客观实在。
  \item \textbf{定域性}:不存在超光速影响。
  \item \textbf{推论}:纠缠测量可瞬时决定远处粒子状态,说明远处粒子早已具有确定值(隐变量),量子力学不完备。
\end{enumerate}

\subsection{【了解】贝尔不等式(CHSH形式)}
设 $E(a,b)$ 表示在测量基 $a,b$ 下的相关函数(例如 $E=P_{\text{同}}-P_{\text{异}}$),
则经典定域隐变量理论满足
\[
-2 \le E(a,b)-E(a,b')+E(a',b)-E(a',b') \le 2
\]
量子力学在合适纠缠态与测量设置下可违反该不等式。实验结果支持量子预言,否定定域隐变量模型。
20 世纪 80 年代以来(如 Aspect 等实验)对贝尔不等式的违反被反复验证。
\paragraph{量子预测示例(补充)}
对 singlet 态 $|\Psi^-\rangle$,量子力学给出
\[
E(a,b)=-\cos\theta_{ab}
\]
在合适测量角度下可达到 $2\sqrt{2}$ 的最大违背。

\paragraph{CHSH 算符定义(补充)}
取 $A,A'$ 为 Alice 的两种测量(本征值 $\pm1$),$B,B'$ 为 Bob 的两种测量,则
\[
\hat{B}=A\otimes(B+B')+A'\otimes(B-B')
\]

\paragraph{算符形式}
定义 Bell 算符 $\hat{B}$,则
\[
-2 \le \langle \hat{B}\rangle \le 2
\]
是经典定域隐变量的上界;量子力学可给出更大的 $\langle\hat{B}\rangle$。

\paragraph{Tsirelson 上界(补充)}
量子力学对 CHSH 算符的上界为
\[
\langle \hat{B}\rangle \le 2\sqrt{2}
\]
表明量子相关性强于经典,但仍受量子力学限制。

\section{【了解】量子超密编码(Superdense Coding)}

\paragraph{核心思想}
利用纠缠作为共享资源,实现“发送 1 个量子比特传递 2 个经典比特”。
\textbf{提出者:}Bennett 与 Wiesner(1992)。
\textbf{对比:}经典信道中发送 1 个物理比特只能传递 1 个经典比特。
又称\textbf{量子密集编码}。

\paragraph{协议步骤}
\begin{enumerate}
  \item 共享 $|\Phi^+\rangle$:Alice 和 Bob 各持一半纠缠对。
  \item Alice 编码两比特信息,对自己量子比特施加幺正操作。
  \item Alice 发送该量子比特给 Bob(只发送 1 个 qubit)。
  \item Bob 对两比特做 Bell 基测量(Bell State Measurement, BSM),得到 2 个经典比特。
\end{enumerate}
\paragraph{Bell 基测量实现(补充)}
对两比特先施加 CNOT(第 1 控制第 2),再对第 1 比特施加 $H$,
随后在计算基测量即可完成 BSM(理想情况下)。

\paragraph{【了解】编码映射(以 $|\Phi^+\rangle$ 为初态)}
\[
\begin{array}{c|c}
\text{经典比特} & \text{Alice 操作} \\
\hline
00 & I \\
01 & X \\
10 & Z \\
11 & iY=ZX
\end{array}
\quad\Rightarrow\quad
\text{得到四个 Bell 态之一}
\]
对应关系为:
\[
00\to|\Phi^+\rangle,\quad 01\to|\Psi^+\rangle,\quad 10\to|\Phi^-\rangle,\quad 11\to|\Psi^-\rangle
\]

\paragraph{要点}
纠缠是核心资源;单独截获一个粒子无法读取信息;测量会破坏纠缠,具有安全性潜力。

\section{【了解】量子线路与量子逻辑门}

\subsection{【了解】线路模型}
量子算法以\textbf{线路}表示:量子比特为线路,酉门为方块,测量为读出。线路保持\textbf{可逆性}。
\textbf{门的分类:}按作用比特数分为单比特门、双比特门、三比特门等。
量子门通常通过其对\textbf{计算基}的作用来定义,再由线性性扩展到任意叠加态。

\subsection{【背】常用单比特门}
\[
X=\begin{pmatrix}0&1\\1&0\end{pmatrix},\quad
Z=\begin{pmatrix}1&0\\0&-1\end{pmatrix},\quad
H=\frac{1}{\sqrt{2}}\begin{pmatrix}1&1\\1&-1\end{pmatrix}
\]
\[
Y=\begin{pmatrix}0&-i\\i&0\end{pmatrix},\quad
S=\begin{pmatrix}1&0\\0&i\end{pmatrix},\quad
R_\phi=\begin{pmatrix}1&0\\0&e^{i\phi}\end{pmatrix}
\]
\[
T=\begin{pmatrix}1&0\\0&e^{i\pi/4}\end{pmatrix}
\]
\paragraph{【了解】常用关系(补充)}
$H^\dagger=H=H^{-1}$,$X^2=Y^2=Z^2=I$,$S^2=Z$,$T^2=S$,
且 $HZH=X$、$HXH=Z$。
\paragraph{【了解】相位门的物理含义(补充)}
$R_\phi$ 仅改变 $|1\rangle$ 的\textbf{相对相位}:$|0\rangle\mapsto|0\rangle$,$|1\rangle\mapsto e^{i\phi}|1\rangle$。
全局相位对测量概率无影响,但\textbf{相对相位}会影响干涉结果。
\paragraph{【了解】S/T 门与旋转门关系(补充)}
$S=R_z(\pi/2)$,$T=R_z(\pi/4)$(均忽略全局相位)。
\paragraph{【了解】旋转门(补充)}
绕 Bloch 球坐标轴的旋转门定义为
\[
R_x(\theta)=e^{-i\theta X/2},\quad R_y(\theta)=e^{-i\theta Y/2},\quad R_z(\theta)=e^{-i\theta Z/2}
\]
它们在单比特空间中对应空间旋转,是构造任意单比特门的基础。
当 $\theta=\pi$ 时,$R_x(\pi)\sim X$、$R_y(\pi)\sim Y$、$R_z(\pi)\sim Z$(忽略全局相位)。
\paragraph{【了解】旋转门的矩阵形式(补充)}
\[
R_z(\theta)=\begin{pmatrix}e^{-i\theta/2}&0\\0&e^{i\theta/2}\end{pmatrix},\qquad
R_x(\theta)=\begin{pmatrix}\cos\frac{\theta}{2}&-i\sin\frac{\theta}{2}\\-i\sin\frac{\theta}{2}&\cos\frac{\theta}{2}\end{pmatrix}
\]

\subsection{【背】受控运算}
控制比特为 $|1\rangle$ 时对目标比特施加门 $U$:
\[
|c\rangle|t\rangle \mapsto |c\rangle U^c|t\rangle
\]
等价写成算符形式:
\[
U_c=|0\rangle\langle 0|\otimes I + |1\rangle\langle 1|\otimes U
\]
在矩阵上表现为\textbf{块对角}结构。
\begin{itemize}
  \item \textbf{CNOT}:控制为 1 时翻转目标比特。
  \item \textbf{Toffoli(CCNOT)}:两个控制比特同时为 1 时翻转目标比特。
  \item \textbf{Fredkin(CSWAP)}:控制为 1 时交换两个目标比特。
\end{itemize}
Toffoli/Fredkin 门可实现\textbf{可逆经典逻辑},是量子线路构建的基础组件。
当控制比特为 $|0\rangle$ 时,受控门对目标比特不产生作用。
Toffoli 门配合单比特门可构成通用量子计算门集。

\paragraph{【了解】Toffoli 与 Fredkin 的逻辑形式(补充)}
Toffoli(CCNOT):
\[
|a,b,c\rangle \mapsto |a,b,c\oplus (ab)\rangle
\]
Fredkin(CSWAP):
\[
|c,a,b\rangle \mapsto |c,\ a,\ b\rangle\ (c=0),\qquad |c,a,b\rangle \mapsto |c,\ b,\ a\rangle\ (c=1)
\]
\paragraph{【了解】Toffoli 真值表(补充)}
\[
|000\rangle\to|000\rangle,\ |001\rangle\to|001\rangle,\ |010\rangle\to|010\rangle,\ |011\rangle\to|011\rangle,
\]
\[
|100\rangle\to|100\rangle,\ |101\rangle\to|101\rangle,\ |110\rangle\to|111\rangle,\ |111\rangle\to|110\rangle
\]

\begin{figure}[htbp]
  \centering
  \includegraphics[width=0.9\linewidth]{fig_cnot_h_cnot.png}
  \caption{受控门示意:左为 CNOT 逻辑($a\oplus b$),右为 $H$+CNOT 产生纠缠的线路。}
\end{figure}

\paragraph{【背】CNOT 矩阵表示}
\[
[\mathrm{CNOT}]=
\begin{pmatrix}
1&0&0&0\\
0&1&0&0\\
0&0&0&1\\
0&0&1&0
\end{pmatrix}
\]
其作用:$|00\rangle\!\to\!|00\rangle,\ |01\rangle\!\to\!|01\rangle,\ |10\rangle\!\to\!|11\rangle,\ |11\rangle\!\to\!|10\rangle$。
\[
|a\rangle|b\rangle \mapsto |a\rangle|a\oplus b\rangle
\]
\[
\mathrm{CNOT}^2=I\quad (\text{自逆})
\]
若控制比特处于叠加态,CNOT 可产生纠缠(例如 $|+\rangle|0\rangle \mapsto |\Phi^+\rangle$)。

\paragraph{【了解】SWAP 门}
交换两个量子比特的状态:
\[
\mathrm{SWAP}=
\begin{pmatrix}
1&0&0&0\\
0&0&1&0\\
0&1&0&0\\
0&0&0&1
\end{pmatrix}
\]
常用分解为三次 CNOT:$\mathrm{SWAP}=(\mathrm{CNOT}_{12})(\mathrm{CNOT}_{21})(\mathrm{CNOT}_{12})$。
\[
\mathrm{SWAP}=(\mathrm{CNOT}_{21})(\mathrm{CNOT}_{12})(\mathrm{CNOT}_{21})
\]
\[
\mathrm{SWAP}^2=I
\]

\paragraph{【了解】SWAP 真值表(补充)}
\[
|00\rangle\to|00\rangle,\quad |01\rangle\to|10\rangle,\quad
|10\rangle\to|01\rangle,\quad |11\rangle\to|11\rangle
\]

\begin{figure}[htbp]
  \centering
  \includegraphics[width=0.9\linewidth]{fig_fredkin.png}
  \caption{Fredkin(受控 SWAP)门:控制为 1 时交换两条线,控制为 0 时保持不变。}
\end{figure}

\begin{example}
\textbf{制备Bell态电路}

对 $|00\rangle$,先对第1比特施加 $H$,再施加 CNOT(第1控制第2):
\[
|00\rangle \xrightarrow{H\otimes I} \frac{|00\rangle+|10\rangle}{\sqrt{2}}
\xrightarrow{\text{CNOT}} \frac{|00\rangle+|11\rangle}{\sqrt{2}}=|\Phi^+\rangle
\]
对 $|\Phi^+\rangle$ 在计算基测量,只会得到 $00$ 或 $11$,且概率各为 $1/2$。
\end{example}

\subsection{【了解】可逆性与辅助比特}
量子门必须可逆;经典不可逆运算(如擦除、AND)需要引入辅助比特与可逆门实现。

\begin{example}
\textbf{二进制个位数相加(思路)}

为避免不可逆映射,需引入辅助比特。一个常见流程(与课件思路一致):
\begin{enumerate}
  \item 初始化 $|0; x_2; x_1\rangle$(辅助比特置 0)。
  \item 以第 2 比特为控制、辅助比特为目标,施加 CNOT。
  \item 以辅助比特为控制,施加 Fredkin 门。
  \item 以第 2 比特为控制、第 1 比特为目标,施加 CNOT。
  \item 测量前两个比特得到加法结果(进位与和)。
\end{enumerate}
\end{example}

\subsection{【了解】通用门集}
任意单比特酉门 + CNOT 组成\textbf{通用门集};常见实现为 $\{H,S,T,\text{CNOT}\}$。
任意单比特酉算符可分解为欧拉角形式(例如 $U=e^{i\alpha}R_z(\beta)R_y(\gamma)R_z(\delta)$),
因此只需有限基本门即可逼近任意量子电路。

\section{【了解】量子计算概述与算法}

\subsection{【了解】经典计算瓶颈}
\begin{itemize}
  \item \textbf{摩尔定律放缓}:器件尺寸逼近纳米尺度,量子隧穿与热效应使性能提升受限。
  \item \textbf{量子系统难模拟}:经典资源随系统规模指数爆炸。
\end{itemize}

\paragraph{经典计算机发展简述(课件要点)}
\begin{itemize}
  \item 1946 年:第一台经典电子计算机 ENIAC。
  \item 1982 年:286 时代的集成电路计算机。
  \item 现代芯片已集成\textbf{数十亿}晶体管(例:智能手机 SoC)。
\end{itemize}

\subsection{【了解】动机与基本需求}
\begin{itemize}
  \item \textbf{动机}:经典计算资源随量子系统规模指数爆炸(Feynman 模拟观点)。
  \item \textbf{基本需求}:可控物理比特、初始化、长相干时间、通用门集、测量读出与可扩展性。
\end{itemize}
可概括为常见的 DiVincenzo 判据:可扩展量子比特、初始化、通用门集、长相干、可靠测量等。

\paragraph{量子计算概念的提出}
Richard Feynman 提出:用量子系统模拟量子系统可避免经典计算资源的指数爆炸。

\paragraph{经典比特 vs 量子比特}
经典比特只能处于 $0/1$;量子比特可处于叠加态 $\alpha|0\rangle+\beta|1\rangle$,
并能体现相干与干涉效应。

\begin{figure}[htbp]
  \centering
  \includegraphics[width=0.9\linewidth]{fig_qc_timeline.png}
  \caption{量子计算发展里程碑时间线(课件示意)。}
\end{figure}

\subsection{【了解】两类量子计算机}
\begin{itemize}
  \item \textbf{量子退火/绝热计算}:专注优化问题(如 QUBO/Ising)。
  \item \textbf{通用门电路量子计算}:可执行任意量子算法。
\end{itemize}

\subsection{【了解】典型量子算法}
\begin{itemize}
  \item \textbf{Shor 算法}:大整数分解,指数级加速(威胁 RSA)。
  \item \textbf{Grover 算法}:无结构搜索,平方加速。
  \item \textbf{量子模拟}:直接模拟量子系统的动力学与性质。
\end{itemize}

\paragraph{量子计算应用(概览)}
优化、量子化学与材料模拟、密码学与安全分析、机器学习加速等是常见应用方向。

\paragraph{Grover 算法核心流程}
\[
\text{初始化} \rightarrow \text{Hadamard 叠加} \rightarrow \text{Oracle 标记} \rightarrow \text{扩散算子放大} \rightarrow \text{测量}
\]

\paragraph{Oracle 的相位翻转(补充)}
常用相位 Oracle 的作用为
\[
O_f|x\rangle = (-1)^{f(x)}|x\rangle
\]
其中 $f(x)=1$ 的目标态获得相位 $-1$。

\paragraph{Grover 扩散算子(补充)}
扩散算子可写为
\[
D=2|s\rangle\langle s|-I,\qquad |s\rangle=\frac{1}{\sqrt{N}}\sum_{x=0}^{N-1}|x\rangle
\]
其作用可理解为“关于平均幅度的反射”,与 Oracle 一起形成幅度放大。
\[
G = D\,O_f
\]
Grover 迭代次数约为 $\left\lfloor\frac{\pi}{4}\sqrt{N}\right\rfloor$。
若设目标态幅度角 $\sin\theta=1/\sqrt{N}$,则第 $k$ 次迭代后的成功概率为
\[
P_k=\sin^2\big((2k+1)\theta\big)
\]
若存在 $M$ 个目标态,则近似迭代次数为 $\left\lfloor\frac{\pi}{4}\sqrt{N/M}\right\rfloor$。

\paragraph{量子算法通用套路}
初始化(多比特置 $|0\rangle$)$\rightarrow$ 量子门序列/线路 $\rightarrow$ 测量得到\textbf{概率分布};
测量输出为经典比特,需重复多次以统计结果。
Grover 搜索将无结构搜索的复杂度从 $O(N)$ 降至 $O(\sqrt{N})$。

\paragraph{Shor 算法要点(补充)}
Shor 算法核心是将整数分解转化为\textbf{周期查找}问题,
利用量子傅里叶变换(QFT)高效提取周期,从而得到因子。
QFT 在计算基上的作用为
\[
\mathrm{QFT}_N|x\rangle=\frac{1}{\sqrt{N}}\sum_{k=0}^{N-1} e^{2\pi i xk/N}|k\rangle
\]
在电路实现中,QFT 可用 $O(n^2)$ 个受控相位门构成($n=\log_2 N$)。
其逆变换为
\[
\mathrm{QFT}_N^\dagger|k\rangle=\frac{1}{\sqrt{N}}\sum_{x=0}^{N-1} e^{-2\pi i xk/N}|x\rangle
\]
因此 $\mathrm{QFT}_N$ 是酉算符,且 $\mathrm{QFT}_N^{-1}=\mathrm{QFT}_N^\dagger$。
实际实现中需构造模指数函数并在经典端用连分数法提取周期。

\subsection{【了解】量子计算平台概览(常见实现)}
\paragraph{量子计算体系}
现有量子计算体系主要包括核磁共振(NMR)、光子、离子阱、超导芯片、金刚石 NV 色心等。
以 NMR 为例,核自旋在外磁场中发生\textbf{塞曼能级劈裂},可用两能级作为量子比特。
\begin{itemize}
  \item 超导量子比特(约瑟夫森结)
  \item 离子阱(激光操控)
  \item 光子量子计算
  \item 核磁共振(NMR)
  \item NV 色心与半导体量子点
  \item 拓扑量子计算(准粒子编织)
\end{itemize}
\paragraph{关键性能指标(补充)}
量子比特的相干时间($T_1,T_2$)需显著长于门操作时间,且门/测量\textbf{误差率}需低于容错阈值。
连接度、读出保真度与可扩展性也是平台比较的核心指标。

\subsection{【了解】量子退火与应用示例(概览)}
量子退火将问题转化为 Ising/QUBO 优化,通过量子涨落寻找最优解;
在物流路径、组合优化、机器学习等领域被探索应用(如 D-Wave 平台示例)。

\section{【了解】量子通信}

\subsection{【了解】基本概念}
量子通信利用\textbf{叠加}与\textbf{纠缠}传递信息,其安全性来自\textbf{不确定性、测量塌缩与不可克隆定理}。
典型任务包括:\textbf{量子隐形传态、量子密钥分发(QKD)、量子超密编码、量子中继}。

\paragraph{【背】不可克隆定理(要点)}
不存在一个固定的酉变换能将任意未知态克隆:
\[
|\psi\rangle|0\rangle \nrightarrow |\psi\rangle|\psi\rangle \quad (\text{对所有 }|\psi\rangle)
\]
这是量子通信安全性的根基之一。

\paragraph{不可克隆定理的线性性证明(补充)}
设某酉算符满足 $U|0\rangle|0\rangle=|0\rangle|0\rangle$、$U|1\rangle|0\rangle=|1\rangle|1\rangle$,
则对叠加态 $|\psi\rangle=\alpha|0\rangle+\beta|1\rangle$,
\[
U|\psi\rangle|0\rangle=\alpha|0\rangle|0\rangle+\beta|1\rangle|1\rangle \neq
(\alpha|0\rangle+\beta|1\rangle)\otimes(\alpha|0\rangle+\beta|1\rangle)
\]
因此不存在能克隆任意未知态的线性(酉)操作。
\paragraph{内积证明(补充)}
若 $U|\psi\rangle|0\rangle=|\psi\rangle|\psi\rangle$ 且 $U|\phi\rangle|0\rangle=|\phi\rangle|\phi\rangle$,
则
\[
\langle\psi|\phi\rangle=\langle\psi|\phi\rangle^2
\]
除非 $|\langle\psi|\phi\rangle|=0$ 或 $1$,否则矛盾,故不能克隆任意非正交态。

\paragraph{量子密集编码}
量子密集编码与超密编码同义,本质是\textbf{纠缠 + 局域操作 + 联合测量}实现容量提升。
\textbf{要点:}单独截获一个量子比特无法读出信息,因为信息编码在\textbf{整体 Bell 态}上。

\paragraph{典型任务速记}
\begin{itemize}
  \item \textbf{隐形传态}:传输量子态本身(需纠缠 + 经典信道)。
  \item \textbf{QKD}:生成安全密钥而非直接传消息。
  \item \textbf{超密编码}:1 个量子比特传 2 个经典比特。
  \item \textbf{量子中继}:基于纠缠交换延伸距离。
\end{itemize}
\paragraph{不可克隆与安全性(补充)}
未知量子态无法被完美复制,任何窃听尝试都会改变状态并可被检测,这是量子通信安全的核心原理之一。

\paragraph{量子通信的几类技术}
\begin{itemize}
  \item \textbf{隐形传态}:传送量子态
  \item \textbf{QKD}:分发密钥
  \item \textbf{超密编码}:提升信道容量
  \item \textbf{量子中继}:延伸距离
\end{itemize}
\paragraph{信道损耗与距离(补充)}
光纤与自由空间信道存在指数衰减与噪声,直接传输距离受限,
因此需要中继、纠缠交换与量子存储器等技术扩展通信距离。
\paragraph{可信中继 vs 量子中继(补充)}
可信中继依赖中继节点的\textbf{可信任假设};量子中继基于纠缠交换与量子存储器,
目标是实现端到端的量子安全。

\subsection{【了解】量子隐形传态(Teleportation)}
\paragraph{提出}
1993 年 Bennett 等提出量子隐形传态方案。
\paragraph{协议步骤}
\begin{enumerate}
  \item Alice 与 Bob 共享 Bell 态 $|\Phi^+\rangle$。
  \item Alice 对待传态量子比特与自己的一半纠缠比特做 CNOT。
  \item Alice 对其第一个比特做 $H$,并在 Bell 基下测量两个比特(BSM),得到 2 个经典比特。
  \item Alice 通过经典信道发送测量结果;Bob 据此施加校正 $I,X,Z,XZ$ 之一。
\end{enumerate}

\begin{figure}[htbp]
  \centering
  \includegraphics[width=0.75\linewidth]{fig_teleportation.png}
  \caption{量子隐形传态示意:纠缠对 + 经典信道共同完成态的传输。}
\end{figure}

\paragraph{【了解】纠正映射}
\[
\begin{array}{c|c}
\text{Alice测量结果} & \text{Bob操作} \\
\hline
00 & I \\
01 & X \\
10 & Z \\
11 & XZ
\end{array}
\]

\paragraph{重要说明}
传输的是\textbf{量子态}而非物质本体;需要经典信道,\textbf{不能超光速};测量破坏原态,不违背不可克隆定理。
发送者与接收者对该未知量子态的具体参数可以始终\textbf{一无所知}。

\subsection{【了解】量子密钥分发(QKD)}
QKD 只生成共享密钥,不直接传输消息。任何窃听测量都会扰动量子态并引入可检测错误(如 BB84/E91 思想)。
其安全性来自测量不可避免地引入扰动与不可克隆原理。

\paragraph{典型协议}
\begin{itemize}
  \item \textbf{BB84}:发送方在两组互补基中随机制备单光子态,接收方随机测量后进行\textbf{基筛选}。
  \item \textbf{E91}:基于纠缠分发密钥,可结合 Bell 不等式进行安全性验证。
\end{itemize}

\paragraph{基本流程(要点)}
制备与测量 $\rightarrow$ 基筛选 $\rightarrow$ 误码率估计 $\rightarrow$ 纠错 $\rightarrow$ 隐私放大。
\paragraph{BB84 安全性直觉(补充)}
窃听者若在错误基测量,会\textbf{扰动}量子态并引入额外误码;
通过抽样估计误码率可检测窃听并决定是否丢弃密钥。
\paragraph{基筛选效率(补充)}
在 BB84 中双方随机选择基,只有\textbf{基一致}的测量结果被保留,平均保留比例约为 $1/2$。
\paragraph{拦截-重发攻击(补充)}
若窃听者随机选择基测量并重发(intercept-resend),会在筛选后的密钥中引入显著误码(典型约 $25\%$),
因此可被误码率检测到。
\paragraph{纠错与隐私放大(补充)}
纠错用于消除信道噪声导致的不一致;隐私放大通过哈希压缩密钥,降低窃听者可能掌握的信息。
\paragraph{经典信道认证(补充)}
QKD 仍需要\textbf{认证的经典信道}完成基筛选与纠错,否则可能遭遇“中间人”攻击。

\subsection{【了解】量子中继与网络}
\paragraph{量子中继}
基于\textbf{纠缠交换}与\textbf{纠缠纯化}延伸通信距离,需要\textbf{量子存储器}支持。
\paragraph{纠缠交换示意(补充)}
若 $A\!-\!B$ 与 $B\!-\!C$ 共享 Bell 态(如 $|\Phi^+\rangle_{AB}$ 与 $|\Phi^+\rangle_{BC}$),
在 $B$ 处做 Bell 基测量可使 $A$ 与 $C$ 纠缠,从而实现“远距离纠缠”。
\paragraph{纠缠纯化(补充)}
纠缠在传输中会退化,可通过 LOCC(局域操作 + 经典通信)对多对低保真纠缠态进行处理,
以获得较高保真度的纠缠对,但会牺牲成功率。

\begin{figure}[htbp]
  \centering
  \includegraphics[width=0.85\linewidth]{fig_trusted_relay.png}
  \caption{可信中继(Trusted Relay)量子网络的示意。}
\end{figure}

\paragraph{网络架构}
\begin{enumerate}
  \item 主动光交换的不可信网络(光开关)
  \item 被动光学器件网络(分束器、WDM)
  \item 可信节点中继网络
  \item 量子中继纯量子网络
\end{enumerate}
基于纠缠交换的\textbf{量子中继}难度较高,工程上常以\textbf{可信中继}作为过渡方案。
基于光开关/无源器件的多用户 QKD 复用方案可扩展性有限,仍受信道损耗约束。

\begin{figure}[htbp]
  \centering
  \includegraphics[width=0.75\linewidth]{fig_qkd_stack.png}
  \caption{量子通信协议栈与QKD设备位置(课件示意)。}
\end{figure}

\paragraph{常见拓扑}
星形、环形、总线型。

\begin{figure}[htbp]
  \centering
  \includegraphics[width=0.8\linewidth]{fig_tokyo_network.png}
  \caption{量子通信网络拓扑示例(东京QKD网络示意)。}
\end{figure}

\paragraph{典型网络示例}
美国 DARPA 量子网络、欧洲 SECOQC、日本东京量子通信网络、中国天地一体量子通信网络与 500 公里无中继光纤 QKD 示范。

\paragraph{关键节点设备(示意)}
纠缠源/单光子源、量子存储器、贝尔态测量模块、单光子探测器、时间同步与经典控制/通信模块等。

\subsection{【了解】量子隐形传态研究进展(课件提要)}
\begin{itemize}
  \item 2010:自由空间 16 km 量子隐形传态实验。
  \item 2015:光纤中超过 100 km 的传态实验(高效探测器)。
  \item 2017:“墨子号”卫星实现地到星超过 1400 km 传态。
\end{itemize}

\subsection{【了解】量子通信网络进展(课件提要)}
\begin{itemize}
  \item 2002--2007:美国 DARPA 量子保密通信网络(多节点、光纤/自由空间)。
  \item 2004/2008:欧洲 SECOQC 项目启动并建成 QKD 网络。
  \item 2010:日本东京 6 节点城域 QKD 网络。
  \item 2021:中国天地一体广域量子通信网络(京沪干线+“墨子号”)。
  \item 2021:中国 500 km 级无中继光纤 TF-QKD 现场示范。
\end{itemize}

\section{【了解】常见题型总结}

\begin{enumerate}
  \item \textbf{复数运算}:求模、共轭、逆、指数形式
  \item \textbf{向量归一化}:计算 $\langle\psi|\psi\rangle$ 并归一化
  \item \textbf{内积与正交性}:计算内积,判断是否正交
  \item \textbf{线性无关与基}:判断是否线性无关,写出基与维数
  \item \textbf{展开系数}:在给定正交归一基下展开向量
  \item \textbf{矩阵基本操作}:求转置、共轭、迹、行列式与逆
  \item \textbf{判断厄米/酉性}:验证 $H=H^\dagger$ 或 $U^\dagger U=I$
  \item \textbf{本征值问题}:求本征值、本征向量、谱分解
  \item \textbf{张量积展开}:用双线性性完全展开
  \item \textbf{算符作用}:计算 $(A\otimes B)|v\otimes w\rangle$
  \item \textbf{对易子/反对易子}:计算 $[A,B]$ 与 $\{A,B\}$
  \item \textbf{量子比特与Bloch球}:将态写成 $\cos\frac{\theta}{2}|0\rangle+e^{i\varphi}\sin\frac{\theta}{2}|1\rangle$
  \item \textbf{酉演化/量子门}:由 $U|\psi\rangle$ 计算演化后态
  \item \textbf{测量与塌缩}:用 $P_m$ 或 $M_m$ 求测量概率与测后态
  \item \textbf{密度矩阵}:判断纯/混态,计算期望值与偏迹
  \item \textbf{测不准关系}:计算 $\Delta A$ 并应用 $\Delta A\,\Delta B$ 下界
  \item \textbf{纠缠态判定}:判断可分/纠缠,写出 Bell 态
  \item \textbf{EPR/Bell}:写出 CHSH 不等式并解释物理含义
  \item \textbf{超密编码}:给出编码映射与解码步骤
  \item \textbf{量子线路}:分析门序列并计算输出态
  \item \textbf{隐形传态}:写出 CNOT + H + 测量 + 纠正的流程
  \item \textbf{量子通信}:解释 QKD/中继/网络架构的基本思路
\end{enumerate}

\section*{【了解】学习建议}

\begin{itemize}
  \item 掌握这些数学工具需要\textbf{大量练习},建议每个概念至少手算2-3个例子
  \item 重点关注\textbf{从定义出发的推导过程},而不是死记公式
  \item 遇到复杂计算时,\textbf{写出完整步骤},避免跳步导致错误
  \item 理解\textbf{物理意义}:厄米算符→可观测量,酉算符→演化,张量积→复合系统
  \item 将\textbf{四大公设}与计算习惯绑定:写清“态—演化—测量—复合系统”的链条
  \item 练习\textbf{密度矩阵与偏迹}:用 $|\Phi^+\rangle$ 等纠缠态练手
  \item 画\textbf{线路图}并手算:Bell 态制备、超密编码、隐形传态
  \item 建议将常用门($H,X,Y,Z,S,T$)与 Bloch 球旋转对应起来记忆
  \item 对 QKD/纠缠/测量等内容,先抓住\textbf{因果链条}:制备 → 演化 → 测量 → 统计
\end{itemize}

\section*{【背】必背公式清单}
\begin{itemize}
  \item \textbf{复数与欧拉公式}:$z=x+iy,\ z^\ast=x-iy,\ |z|^2=zz^\ast$,$z=|z|e^{i\theta}$,$e^{i\theta}=\cos\theta+i\sin\theta$。
  \item \textbf{内积与归一化}:$\langle\phi|\psi\rangle=\sum_k a_k^\ast b_k$,$\langle\psi|\psi\rangle=1$。
  \item \textbf{正交归一基与完备关系}:$\langle e_i|e_j\rangle=\delta_{ij}$,$I=\sum_j |e_j\rangle\langle e_j|$。
  \item \textbf{外积/投影}:$P_\psi=|\psi\rangle\langle\psi|$,$P_\psi^2=P_\psi$,$P_\psi^\dagger=P_\psi$。
  \item \textbf{厄米共轭}:$(A^\dagger)_{ij}=(A_{ji})^\ast$,$(AB)^\dagger=B^\dagger A^\dagger$。
  \item \textbf{厄米/酉算符}:$H=H^\dagger$,$U^\dagger U=I$,$U^{-1}=U^\dagger$。
  \item \textbf{本征值与谱分解}:$A|\phi\rangle=\lambda|\phi\rangle$,$H=\sum_j \lambda_j|\phi_j\rangle\langle\phi_j|$(厄米算符)。
  \item \textbf{Pauli 矩阵与对易}(在标准基 $\{|0\rangle,|1\rangle\}$ 下):
  \[
  \sigma_x=\begin{pmatrix}0&1\\1&0\end{pmatrix},\
  \sigma_y=\begin{pmatrix}0&-i\\i&0\end{pmatrix},\
  \sigma_z=\begin{pmatrix}1&0\\0&-1\end{pmatrix}
  \]
  $[\sigma_x,\sigma_y]=2i\sigma_z$,$\{\sigma_i,\sigma_j\}=2\delta_{ij}I$。
  \item \textbf{张量积运算}:$\langle v_1\!\otimes\! w_1|v_2\!\otimes\! w_2\rangle=\langle v_1|v_2\rangle\langle w_1|w_2\rangle$,
  $(A\!\otimes\!B)(C\!\otimes\!D)=(AC)\!\otimes\!(BD)$。
  \item \textbf{对易子}:$[A,B]=AB-BA$。
  \item \textbf{Cauchy--Schwarz}:$|\langle\phi|\psi\rangle|\le \||\phi\rangle\|\,\||\psi\rangle\|$。
  \item \textbf{量子比特态}:$|\psi\rangle=\alpha|0\rangle+\beta|1\rangle$,$|\alpha|^2+|\beta|^2=1$。
  \item \textbf{薛定谔方程与酉演化}:$i\hbar\,\partial_t|\psi\rangle=H|\psi\rangle$,$U=\exp(-\frac{i}{\hbar}H\Delta t)$。
  \item \textbf{测量公式}:$p(m)=\langle\psi|M_m^\dagger M_m|\psi\rangle$,$|\psi_m\rangle=\dfrac{M_m|\psi\rangle}{\sqrt{p(m)}}$;
  投影测量 $p(m)=\langle\psi|P_m|\psi\rangle$。
  \item \textbf{复合系统}:$\mathcal{H}_{AB}=\mathcal{H}_A\otimes\mathcal{H}_B$,$|\psi\rangle=\sum_{ij}c_{ij}|i\rangle|j\rangle$。
  \item \textbf{密度算子}:$\rho=\sum_k p_k|\psi_k\rangle\langle\psi_k|$,$\mathrm{tr}(\rho)=1$,
  $\langle A\rangle=\mathrm{tr}(\rho A)$,$\rho_A=\mathrm{tr}_B(\rho_{AB})$。
  \item \textbf{CNOT 矩阵}:
  \[
  [\mathrm{CNOT}]=
  \begin{pmatrix}
  1&0&0&0\\
  0&1&0&0\\
  0&0&0&1\\
  0&0&1&0
  \end{pmatrix}
  \]
\end{itemize}

\end{document}
