\documentclass[12pt]{article}

\usepackage[a4paper,margin=2.2cm]{geometry}
\usepackage{amsmath,amssymb}
\usepackage{booktabs}
\usepackage{enumitem}
\usepackage{xeCJK}
\usepackage{hyperref}

\setCJKmainfont{Source Han Serif CN}
\setCJKsansfont{Source Han Sans CN}
\setCJKmonofont{Source Han Sans CN}
\setlist[itemize]{leftmargin=*,nosep}
\setlist[enumerate]{leftmargin=*,nosep}

\title{量子信息知识点精要}
\author{}
\date{}

\begin{document}
\maketitle
\tableofcontents

\section{基础数学工具}

\subsection{Dirac符号与完备关系}
\begin{itemize}
\item \textbf{外积:}$|i\rangle\langle j|$ 表示算符
\item \textbf{完备关系:}$I=\sum_i |i\rangle\langle i|$
\item \textbf{应用:}展开算符 $A = \sum_{i,j} A|i\rangle\langle i|j\rangle\langle j| = \sum_{i,j} \langle i|A|j\rangle |i\rangle\langle j|$
\end{itemize}

\subsection{线性算符的矩阵表示}
\begin{itemize}
\item \textbf{矩阵元:}$A_{ij} = \langle i|A|j\rangle$
\item \textbf{基变换:}若基从 $\{|v_i\rangle\}$ 变为 $\{|w_i\rangle\}$,算符矩阵表示满足 $A'' = U^\dagger A' U$,其中 $U_{ij}=\langle v_i|w_j\rangle$
\end{itemize}

\subsection{特征分解}
\begin{itemize}
\item \textbf{特征方程:}$\det(A-\lambda I)=0$
\item \textbf{对角化:}Hermite算符可对角化,$A = \sum_i \lambda_i |u_i\rangle\langle u_i|$
\end{itemize}

\subsection{张量积}
\begin{itemize}
\item \textbf{态的张量积:}$|a\rangle\otimes|b\rangle$ 记为 $|ab\rangle$ 或 $|a,b\rangle$
\item \textbf{Kronecker积:}$|\psi\rangle\otimes|\phi\rangle = \begin{pmatrix}a_0\\a_1\end{pmatrix}\otimes\begin{pmatrix}b_0\\b_1\end{pmatrix} = \begin{pmatrix}a_0b_0\\a_0b_1\\a_1b_0\\a_1b_1\end{pmatrix}$
\item \textbf{算符张量积:}$(A\otimes B)(|\psi\rangle\otimes|\phi\rangle) = (A|\psi\rangle)\otimes(B|\phi\rangle)$
\item \textbf{分配律:}$(|a\rangle+|b\rangle)\otimes|c\rangle = |a\rangle\otimes|c\rangle + |b\rangle\otimes|c\rangle$
\end{itemize}

\section{量子态}

\subsection{单量子比特态}
\begin{itemize}
\item \textbf{计算基:}$|0\rangle=\begin{pmatrix}1\\0\end{pmatrix}$,$|1\rangle=\begin{pmatrix}0\\1\end{pmatrix}$
\item \textbf{一般纯态:}$|\psi\rangle = \alpha|0\rangle + \beta|1\rangle$,满足 $|\alpha|^2+|\beta|^2=1$
\item \textbf{Hadamard基:}
  \begin{align*}
  |+\rangle &= \frac{|0\rangle+|1\rangle}{\sqrt{2}} \\
  |-\rangle &= \frac{|0\rangle-|1\rangle}{\sqrt{2}}
  \end{align*}
\item \textbf{Y算符本征态:}
  \begin{align*}
  |y_+\rangle &= \frac{|0\rangle+i|1\rangle}{\sqrt{2}} \\
  |y_-\rangle &= \frac{|0\rangle-i|1\rangle}{\sqrt{2}}
  \end{align*}
\end{itemize}

\subsection{布洛赫球表示}
\begin{itemize}
\item \textbf{参数化形式:}$|\psi\rangle = \cos\frac{\theta}{2}|0\rangle + e^{i\phi}\sin\frac{\theta}{2}|1\rangle$
\item \textbf{球坐标:}$\theta\in[0,\pi]$ 为极角,$\phi\in[0,2\pi)$ 为方位角
\item \textbf{特殊点:}
  \begin{itemize}
  \item $|0\rangle$:北极 $(\theta=0)$
  \item $|1\rangle$:南极 $(\theta=\pi)$
  \item $|+\rangle$:赤道+x方向 $(\theta=\pi/2,\phi=0)$
  \item $|-\rangle$:赤道-x方向 $(\theta=\pi/2,\phi=\pi)$
  \item $|y_+\rangle$:赤道+y方向 $(\theta=\pi/2,\phi=\pi/2)$
  \item $|y_-\rangle$:赤道-y方向 $(\theta=\pi/2,\phi=3\pi/2)$
  \end{itemize}
\end{itemize}

\subsection{多量子比特态}
\begin{itemize}
\item \textbf{两比特计算基:}$|00\rangle, |01\rangle, |10\rangle, |11\rangle$(维数 $2^2=4$)
\item \textbf{n比特计算基:}$\{|x\rangle : x\in\{0,1\}^n\}$(维数 $2^n$)
\item \textbf{可分离态:}$|\psi\rangle = |\phi\rangle_A\otimes|\chi\rangle_B$
\item \textbf{纠缠态:}不能写成张量积形式的态
\end{itemize}

\subsection{Bell态(最大纠缠态)}
\begin{align*}
|\Phi^+\rangle &= \frac{1}{\sqrt{2}}(|00\rangle+|11\rangle) \\
|\Phi^-\rangle &= \frac{1}{\sqrt{2}}(|00\rangle-|11\rangle) \\
|\Psi^+\rangle &= \frac{1}{\sqrt{2}}(|01\rangle+|10\rangle) \\
|\Psi^-\rangle &= \frac{1}{\sqrt{2}}(|01\rangle-|10\rangle) \quad\text{(单态)}
\end{align*}

\section{量子算符与量子门}

\subsection{Pauli算符}
\begin{align*}
X &= \begin{pmatrix}0&1\\1&0\end{pmatrix} = |0\rangle\langle 1| + |1\rangle\langle 0| \\
Y &= \begin{pmatrix}0&-i\\i&0\end{pmatrix} = i|1\rangle\langle 0| - i|0\rangle\langle 1| \\
Z &= \begin{pmatrix}1&0\\0&-1\end{pmatrix} = |0\rangle\langle 0| - |1\rangle\langle 1|
\end{align*}
\textbf{性质:}
\begin{itemize}
\item Hermite:$X^\dagger=X$,$Y^\dagger=Y$,$Z^\dagger=Z$
\item 幺正:$X^2=Y^2=Z^2=I$
\item 特征值:均为 $\pm 1$
\item X特征态:$|+\rangle$ ($\lambda=+1$),$|-\rangle$ ($\lambda=-1$)
\item Y特征态:$|y_+\rangle$ ($\lambda=+1$),$|y_-\rangle$ ($\lambda=-1$)
\item Z特征态:$|0\rangle$ ($\lambda=+1$),$|1\rangle$ ($\lambda=-1$)
\end{itemize}

\subsection{Hadamard门}
\[
H = \frac{1}{\sqrt{2}}\begin{pmatrix}1&1\\1&-1\end{pmatrix}
\]
\textbf{性质:}
\begin{itemize}
\item 自伴:$H=H^\dagger$
\item 幺正:$H^2=I$
\item 基变换:$H|0\rangle=|+\rangle$,$H|1\rangle=|-\rangle$
\item 共轭关系:$HXH=Z$,$HZH=X$,$HYH=-Y$
\end{itemize}

\subsection{CNOT门(受控非门)}
\[
\text{CNOT}|a,b\rangle = |a, a\oplus b\rangle
\]
\[
\text{CNOT} = \begin{pmatrix}1&0&0&0\\0&1&0&0\\0&0&0&1\\0&0&1&0\end{pmatrix} = |0\rangle\langle 0|\otimes I + |1\rangle\langle 1|\otimes X
\]
\textbf{应用:}Bell态制备、Bell基测量、量子隐形传态

\subsection{其他重要门}
\begin{itemize}
\item \textbf{单位门:}$I=\begin{pmatrix}1&0\\0&1\end{pmatrix}$
\item \textbf{$i\sigma_y$门:}$\begin{pmatrix}0&-1\\1&0\end{pmatrix} = iY$
\item \textbf{Controlled-U:}$\text{C-}U|0,\psi\rangle = |0,\psi\rangle$,$\text{C-}U|1,\psi\rangle = |1, U\psi\rangle$
\end{itemize}

\section{量子测量}

\subsection{投影测量}
\begin{itemize}
\item \textbf{投影算符:}$P_i = |i\rangle\langle i|$,满足 $\sum_i P_i = I$,$P_i P_j = \delta_{ij}P_i$
\item \textbf{Born规则:}测量结果 $i$ 的概率 $P(i) = |\langle i|\psi\rangle|^2 = \langle\psi|P_i|\psi\rangle$
\item \textbf{测量后态:}$|\psi'\rangle = \frac{P_i|\psi\rangle}{\sqrt{P(i)}} = \frac{P_i|\psi\rangle}{\|P_i|\psi\rangle\|}$
\end{itemize}

\subsection{可观测量}
\begin{itemize}
\item \textbf{定义:}Hermite算符,谱分解 $A = \sum_i \lambda_i P_i$
\item \textbf{期望值:}$\langle A\rangle = \langle\psi|A|\psi\rangle = \sum_i \lambda_i P(i)$
\item \textbf{方差:}$(\Delta A)^2 = \langle A^2\rangle - \langle A\rangle^2$
\item \textbf{标准偏差(不确定度):}$\Delta A = \sqrt{(\Delta A)^2}$
\end{itemize}

\subsection{特殊测量}
\begin{itemize}
\item \textbf{计算基测量:}投影到 $\{|0\rangle, |1\rangle\}$ 基,对应Z算符
\item \textbf{本征态测量:}测量本征态 $|u_i\rangle$ 得到确定结果 $\lambda_i$,态不变
\item \textbf{Bell基测量:}通过 CNOT+H+计算基测量实现,区分四个Bell态
\end{itemize}

\section{密度矩阵}

\subsection{密度算符}
\begin{itemize}
\item \textbf{纯态:}$\rho = |\psi\rangle\langle\psi|$
\item \textbf{混合态:}$\rho = \sum_i p_i |\psi_i\rangle\langle\psi_i|$,其中 $\sum_i p_i=1$,$p_i\geq 0$
\item \textbf{性质:}$\rho^\dagger=\rho$,$\mathrm{Tr}(\rho)=1$,$\rho\geq 0$
\end{itemize}

\subsection{纯态判据}
\begin{itemize}
\item $\rho^2=\rho$ 等价于纯态
\item $\mathrm{Tr}(\rho^2)=1$ 等价于纯态
\item $\mathrm{Tr}(\rho^2)<1$ 则为混合态
\end{itemize}

\subsection{约化密度算符(偏迹)}
\begin{itemize}
\item \textbf{定义:}对复合系统 $AB$,子系统 $A$ 的约化密度算符为
  \[
  \rho_A = \mathrm{Tr}_B(\rho_{AB}) = \sum_j (\mathbb{I}_A\otimes\langle j|_B)\rho_{AB}(\mathbb{I}_A\otimes|j\rangle_B)
  \]
\item \textbf{物理意义:}子系统的统计描述
\end{itemize}

\subsection{最大混合态}
\begin{itemize}
\item \textbf{单比特:}$\rho = \frac{I}{2} = \frac{1}{2}\begin{pmatrix}1&0\\0&1\end{pmatrix}$
\item \textbf{性质:}完全退相干,$\mathrm{Tr}(\rho^2)=\frac{1}{2}$
\item \textbf{来源:}纠缠态的子系统是最大混合态
\end{itemize}

\section{量子纠缠}

\subsection{纠缠判据}
\begin{itemize}
\item \textbf{定义:}$|\psi\rangle_{AB}$ 是纠缠态当且仅当不存在 $|\phi\rangle_A$ 和 $|\chi\rangle_B$ 使得 $|\psi\rangle_{AB}=|\phi\rangle_A\otimes|\chi\rangle_B$
\item \textbf{检验方法:}尝试因式分解;计算约化密度矩阵的纯度
\item \textbf{最大纠缠态:}四个Bell态
\end{itemize}

\subsection{纠缠的性质}
\begin{itemize}
\item \textbf{子系统混合性:}纠缠态的约化密度算符是混合态
\item \textbf{例:}Bell态 $|\Phi^+\rangle$ 的约化态 $\rho_A = \rho_B = \frac{I}{2}$
\item \textbf{非局域关联:}测量一个子系统会瞬间影响另一个子系统的态(但不能传递信息)
\end{itemize}

\subsection{Bell态的变换}
Pauli算符作用在 $|\Phi^+\rangle$ 的第一个量子比特上:
\begin{align*}
I\otimes I|\Phi^+\rangle &= |\Phi^+\rangle \\
X\otimes I|\Phi^+\rangle &= |\Psi^+\rangle \\
Z\otimes I|\Phi^+\rangle &= |\Phi^-\rangle \\
iY\otimes I|\Phi^+\rangle &= |\Psi^-\rangle
\end{align*}

\section{量子协议}

\subsection{超密编码(Superdense Coding)}
\textbf{目标:}利用一个量子比特传输两个经典比特

\textbf{资源:}Alice和Bob共享Bell态 $|\Phi^+\rangle$

\textbf{编码:}Alice根据要发送的2比特信息施加门:
\begin{itemize}
\item "00":施加 $I$,态保持 $|\Phi^+\rangle$
\item "01":施加 $Z$,态变为 $|\Phi^-\rangle$
\item "10":施加 $X$,态变为 $|\Psi^+\rangle$
\item "11":施加 $iY$(或$XZ$),态变为 $|\Psi^-\rangle$
\end{itemize}

\textbf{解码:}Bob收到Alice的量子比特后,进行Bell基测量(CNOT+H+计算基测量)

\subsection{量子隐形传态(Quantum Teleportation)}
\textbf{目标:}Alice将未知量子态 $|\psi\rangle=\alpha|0\rangle+\beta|1\rangle$ 传送给Bob

\textbf{资源:}共享Bell态 $|\Phi^+\rangle_{23}$(Alice持有量子比特2,Bob持有量子比特3)

\textbf{步骤:}
\begin{enumerate}
\item \textbf{初态:}$|\psi_0\rangle = |\psi\rangle_1\otimes|\Phi^+\rangle_{23}$
\item \textbf{Alice进行Bell基测量:}对量子比特1和2进行CNOT+H+计算基测量,得到2比特经典结果 $(M_1, M_2)$
\item \textbf{Bob根据测量结果修正:}
  \begin{itemize}
  \item 若 $(M_1, M_2)=(0,0)$:施加 $I$
  \item 若 $(M_1, M_2)=(0,1)$:施加 $X$
  \item 若 $(M_1, M_2)=(1,0)$:施加 $Z$
  \item 若 $(M_1, M_2)=(1,1)$:施加 $XZ$
  \end{itemize}
\item \textbf{结果:}Bob的量子比特3处于态 $|\psi\rangle=\alpha|0\rangle+\beta|1\rangle$
\end{enumerate}

\textbf{关键性质:}
\begin{itemize}
\item 需要经典信道传输测量结果,不能超光速
\item Alice的原始态被破坏,不违反不可克隆定理
\item 传输的是量子态,而非经典信息
\end{itemize}

\section{量子线路}

\subsection{Bell态制备}
初态 $|00\rangle$ 经过 H(作用在第一个量子比特)和 CNOT:
\[
|00\rangle \xrightarrow{H\otimes I} |+0\rangle \xrightarrow{\text{CNOT}} |\Phi^+\rangle
\]

其他初态:
\begin{itemize}
\item $|01\rangle \to |\Psi^+\rangle$
\item $|10\rangle \to |\Phi^-\rangle$
\item $|11\rangle \to |\Psi^-\rangle$
\end{itemize}

\subsection{Bell基测量}
量子线路:CNOT(控制比特在前)+ H(作用在第一个量子比特)+ 计算基测量

测量结果对应:
\begin{itemize}
\item $(0,0) \to |\Phi^+\rangle$
\item $(0,1) \to |\Psi^+\rangle$
\item $(1,0) \to |\Phi^-\rangle$
\item $(1,1) \to |\Psi^-\rangle$
\end{itemize}

\subsection{线路恒等式}
\begin{itemize}
\item $HXH = Z$
\item $HZH = X$
\item $HYH = -Y$
\item $H^2 = I$
\end{itemize}

\subsection{多量子比特门的张量积}
\textbf{例1:}$I\otimes H$ 作用在两量子比特态 $|ab\rangle$ 上,只对第二个量子比特施加Hadamard门

\textbf{例2:}$H\otimes I$ 只对第一个量子比特施加Hadamard门

\textbf{矩阵形式:}
\[
I\otimes H = \begin{pmatrix}H & 0 \\ 0 & H\end{pmatrix}, \quad
H\otimes I = \frac{1}{\sqrt{2}}\begin{pmatrix}I & I \\ I & -I\end{pmatrix}
\]

\subsection{辅助量子比特与间接测量}
\textbf{原理:}通过测量辅助量子比特实现对系统的测量,避免直接破坏系统态

\textbf{线路:}
\begin{enumerate}
\item 辅助比特初始化为 $|0\rangle$
\item 施加Hadamard门到辅助比特
\item 施加Controlled-U门(辅助比特为控制比特)
\item 再次对辅助比特施加Hadamard门
\item 测量辅助比特
\end{enumerate}

\textbf{结果:}测量辅助比特得到的期望值对应算符U的本征值

\section{物理实现}

\subsection{施特恩-格拉赫实验}
\begin{itemize}
\item \textbf{装置:}不均匀磁场使自旋向上/向下的粒子空间分离
\item \textbf{量子对应:}自旋向上 $\leftrightarrow |0\rangle$,自旋向下 $\leftrightarrow |1\rangle$
\item \textbf{连续测量:}不同方向测量对应不同基(Z方向、X方向等)
\end{itemize}

\subsection{量子计算物理平台}
\begin{itemize}
\item \textbf{超导量子计算:}基于约瑟夫森结,微波控制,低温运行
\item \textbf{离子阱量子计算:}电磁场囚禁离子,激光操控,声学模式耦合
\item 两种平台各有优劣,都能实现通用量子计算
\end{itemize}

\section{量子信息基本原理}

\subsection{不可克隆定理}
不存在幺正操作能够复制任意未知量子态:
\[
U(|\psi\rangle\otimes|0\rangle) = |\psi\rangle\otimes|\psi\rangle \quad\text{(不可能对所有$|\psi\rangle$成立)}
\]

\subsection{相位}
\begin{itemize}
\item \textbf{全局相位:}$e^{i\theta}|\psi\rangle$ 与 $|\psi\rangle$ 物理等价
\item \textbf{相对相位:}$\alpha|0\rangle+\beta|1\rangle$ 中 $\beta/\alpha$ 的相位有物理意义,影响干涉和测量结果
\end{itemize}

\subsection{量子演化的可逆性}
\begin{itemize}
\item \textbf{幺正演化:}$|\psi(t)\rangle = U(t)|\psi(0)\rangle$,$U$ 是幺正算符,可逆
\item \textbf{测量不可逆:}测量后态坍缩,信息损失,不可逆
\end{itemize}

\end{document}
