\documentclass[12pt]{article}

\usepackage[a4paper,margin=2.2cm]{geometry}
\usepackage{amsmath,amssymb}
\usepackage{booktabs}
\usepackage{enumitem}
\usepackage{xeCJK}
\usepackage{hyperref}
\usepackage{graphicx}
\usepackage{physics}
\usepackage{xcolor}
\usepackage{tcolorbox}

\setCJKmainfont{Source Han Serif CN}
\setCJKsansfont{Source Han Sans CN}
\setCJKmonofont{Source Han Sans CN}
\setlist[itemize]{leftmargin=*,nosep}
\setlist[enumerate]{leftmargin=*,nosep}
\hypersetup{
  colorlinks=true,
  linkcolor=blue!55!black,
  urlcolor=blue!55!black,
  citecolor=blue!55!black
}

\tcbuselibrary{breakable}
\newtcolorbox{example}[1][]{
  colback=blue!5!white,
  colframe=blue!75!black,
  title=\textbf{例题},
  breakable,
  #1
}

\title{量子力学导论\\基于量子信息知识的深入学习}
\author{}
\date{}

\begin{document}
\maketitle
\tableofcontents
\newpage

\section{前言与学习目标}

本讲义以已掌握《量子信息知识点》的读者为理想起点,同时提供零基础预备内容,便于新读者快速上手。
我们将用\textbf{连续变量体系}的语言把量子信息中的抽象公理落地为波函数、微分方程与可观测量的计算,
目标是让你能够\textbf{独立完成《量子力学作业》中的全部题目}。

\textbf{核心学习目标:}
\begin{itemize}
\item 从离散态过渡到连续态,掌握位置表象与动量表象的精确定义。
\item 熟练使用薛定谔方程与连续性方程,理解概率守恒与概率流。
\item 会处理球坐标与氢原子基态的期望值计算。
\item 掌握角动量算符、对易关系与表象变换。
\item 能系统推导不确定关系、Ehrenfest定理与干涉现象。
\item 能在能量表象、位置表象、动量表象之间自由切换,并计算矩阵元。
\end{itemize}

\textbf{符号约定:}
\begin{itemize}
\item $|\psi\rangle$ 表示态矢(向量),$\langle\phi|$ 为对偶向量。
\item 几何向量用 $\vec{r}$、$\vec{p}$ 等箭头记号。
\item $A$ 表示算符,$\langle A \rangle$ 表示期望值;位置与动量算符仍写作 $\hat{x},\hat{p}$。
\item $\psi(x,t)=\langle x|\psi(t)\rangle$ 为位置表象波函数;
      $\phi(p,t)=\langle p|\psi(t)\rangle$ 为动量表象波函数。
\item 归一化:$\int |\psi|^2\,dx=1$;连续谱正交:$\langle x|x'\rangle=\delta(x-x')$。
\item $\hbar$ 为约化普朗克常数;$m$ 为粒子质量;$V(x)$ 为势能函数。
\item $\nabla$ 为梯度算符,$\nabla^2$ 为拉普拉斯算符(空间二阶导数的总和)。
\item 位置变量用 $x$(一维)或 $\vec{r}$(三维),动量变量用 $p$ 或 $\vec{p}$。
\end{itemize}
波函数不是“概率”,而是\textbf{概率幅}。只有取模平方
$|\psi(x,t)|^2$ 才是“在位置 $x$ 附近被测到的概率密度”。同理,$|\phi(p,t)|^2$
是测到动量 $p$ 的概率密度。
算符是“做动作的机器”,期望值就是测量很多次的平均结果。
若写成 $\psi(\vec{r},t)$,$\vec{r}=(x,y,z)$;
若写成 $\psi(r,\theta,\varphi)$,$r$ 是到原点的距离,$\theta$ 为极角,
$\varphi$ 为方位角。
除非特别说明,$\psi$ 默认满足可归一化(平方可积)的物理态条件。

\section{零基础预备:狄拉克 $\delta$ 函数}
狄拉克 $\delta$ 函数不是普通函数,而是\textbf{分布}。
它的核心作用是“从积分中抽取某点的值”。最重要的性质是
\[
\int_{-\infty}^{\infty} f(x)\,\delta(x-x_0)\,dx=f(x_0).
\]
因此有归一化
\[
\int_{-\infty}^{\infty}\delta(x-x_0)\,dx=1,
\]
并且 $\delta(x-x')$ 表示“只有当 $x=x'$ 时才有贡献”,这就是写成“差”的原因。

\textbf{缩放与换元:}若 $a\neq 0$,则
\[
\delta(ax)=\frac{1}{|a|}\delta(x).
\]
若 $g(x)$ 在 $x_i$ 处有简单零点($g(x_i)=0,\ g'(x_i)\neq 0$),则
\[
\delta(g(x))=\sum_i \frac{\delta(x-x_i)}{|g'(x_i)|}.
\]

\textbf{单位与三维形式:}$\delta(x)$ 的量纲是 $1/\text{length}$。
三维形式为
\[
\delta^{(3)}(\vec{r}-\vec{r}')=\delta(x-x')\delta(y-y')\delta(z-z'),
\]
并满足 $\int \delta^{(3)}(\vec{r}-\vec{r}')\,d^3r=1$。

\begin{example}
\textbf{$\delta$ 的抽样示例:}计算 $\int f(x)\delta(2x-1)\,dx$。

\textbf{解:}零点为 $x_0=1/2$,且 $g'(x)=2$,所以
\[
\delta(2x-1)=\frac{1}{2}\delta\left(x-\frac{1}{2}\right),\quad
\int f(x)\delta(2x-1)\,dx=\frac{1}{2}f\left(\frac{1}{2}\right).
\]
\end{example}

\section{零基础预备:波函数与概率幅}
量子态在位置表象下用\textbf{波函数}表示:
\[
\psi(x,t)=\langle x|\psi(t)\rangle.
\]
它是\textbf{概率幅}(可能为复数),概率密度由
\[
\rho(x,t)=|\psi(x,t)|^2
\]
给出,因此粒子出现在区间 $[a,b]$ 的概率为
\[
P_{[a,b]}=\int_a^b |\psi(x,t)|^2\,dx.
\]
\textbf{归一化条件:}
\[
\int_{-\infty}^{\infty}|\psi(x,t)|^2\,dx=1.
\]
在三维情形,$\psi(\vec{r},t)$ 满足
\[
\int |\psi(\vec{r},t)|^2\,d^3r=1.
\]

\begin{example}
\textbf{平面波(动量本征态):}一维自由粒子可用
\[
\psi(x,t)=A\,e^{i(kx-\omega t)}
\]
表示。它的概率密度为 $|\psi|^2=|A|^2$,在空间中均匀分布。
由于全空间均匀,严格意义下不可归一化,只是理想化的动量本征态。
\end{example}

\begin{example}
\textbf{高斯波包(可归一化态):}设
\[
\psi(x,0)=\left(\frac{1}{\pi a^2}\right)^{1/4}e^{-x^2/(2a^2)}.
\]
则
\[
|\psi(x,0)|^2=\frac{1}{\sqrt{\pi}a}e^{-x^2/a^2},
\]
满足 $\int_{-\infty}^{\infty}|\psi(x,0)|^2dx=1$。
这里概率密度集中在 $x=0$ 附近,$a$ 表示波包宽度。
\textbf{计算说明:}
\[
|\psi|^2=\psi^\ast\psi
\quad\text{其中}\quad
\psi^\ast=\left(\frac{1}{\pi a^2}\right)^{1/4}e^{-x^2/(2a^2)}
\quad(\text{本例指数为实数,不含 }i\text{,故共轭不变}),
\]
\[
|\psi|^2=\left(\frac{1}{\pi a^2}\right)^{1/2}
\exp\!\left(-\frac{x^2}{2a^2}\right)^2
=\frac{1}{\sqrt{\pi}a}e^{-x^2/a^2}.
\]
\end{example}

\begin{example}
\textbf{简易“盒子态”:}在区间 $[-L,L]$ 上取常数波函数
\[
\psi(x)=\begin{cases}
\frac{1}{\sqrt{2L}}, & |x|\le L,\\
0, & \text{其他}.
\end{cases}
\]
则 $|\psi(x)|^2=\frac{1}{2L}$,在 $[-L,L]$ 内均匀分布,且归一化成立。
\end{example}

\section{零基础预备:位置与动量算符}

量子力学中“可观测量”由算符表示。最基本的是\textbf{位置算符} $\hat{x}$ 和
\textbf{动量算符} $\hat{p}$。它们的定义可以通过在位置表象中的作用来理解:
\begin{itemize}
\item $\psi(x)$ 是\textbf{位置表象波函数},表示粒子在位置 $x$ 的概率幅度。
\item $\hbar$ 是\textbf{约化普朗克常数}($\hbar=h/2\pi$),是量子力学的基本常数;
      其中 $h$ 是\textbf{普朗克常数},数值约为
      $h\approx 6.626\times10^{-34}\,\mathrm{J\cdot s}$,
      $\hbar\approx 1.055\times10^{-34}\,\mathrm{J\cdot s}$。
\item $\partial/\partial x$ 表示对 $x$ 的偏导数,$i$ 为虚数单位($i^2=-1$)。
\end{itemize}
\[
\hat{x}\psi(x)=x\,\psi(x),\qquad
\hat{p}\psi(x)=-i\hbar\frac{\partial}{\partial x}\psi(x).
\]
也就是说,$\hat{x}$ 只是把函数乘以 $x$,而 $\hat{p}$ 通过求导来作用在波函数上。
在三维情形,$\hat{\vec{p}}=-i\hbar\nabla$。

\begin{example}
\textbf{连续基下的“矩阵元”表示:}在位置基 $\{|x\rangle\}$ 下,
\textbf{注意}:$\delta(x-x')$ 是狄拉克 $\delta$ 函数,
表示“当 $x=x'$ 时取值无穷大、其余为 0,但积分为 1”的理想化函数。
写成 $x-x'$ 的形式是为了强调“只有两者相等时才有贡献”,并保证平移不变性。
\[
\langle x|\hat{x}|x'\rangle=x\,\delta(x-x'),\qquad
\langle x|\hat{p}|x'\rangle=-i\hbar\,\frac{\partial}{\partial x}\delta(x-x').
\]
这表示 $\hat{x}$ 在位置表象中是“对角”的,而 $\hat{p}$ 通过导数作用。
在动量基 $\{|p\rangle\}$ 下则相反:
\[
\langle p|\hat{p}|p'\rangle=p\,\delta(p-p'),\qquad
\langle p|\hat{x}|p'\rangle=i\hbar\,\frac{\partial}{\partial p}\delta(p-p').
\]
\end{example}

\begin{example}
\textbf{离散化直观(示意):}若只取三个位置点 $x_1,x_2,x_3$ 作基,
位置算符在该基下近似为对角矩阵
\[
\hat{x}\approx
\begin{pmatrix}
x_1&0&0\\
0&x_2&0\\
0&0&x_3
\end{pmatrix}.
\]
这说明“在位置基下,$\hat{x}$ 只给出对应位置的数值”。
动量算符在离散基下可用有限差分近似,其矩阵不再对角(体现“含导数”)。
\end{example}

\textbf{本征态与本征值:}若
\[
\hat{A}|\psi\rangle=a|\psi\rangle,
\]
则称 $|\psi\rangle$ 为算符 $\hat{A}$ 的本征态,$a$ 为本征值。
例如
\[
\hat{x}|x\rangle=x|x\rangle
\]
表示 $|x\rangle$ 是“位置确定为 $x$”的理想态。

\textbf{期望值:}态 $|\psi\rangle$ 中位置与动量的平均值为
\[
\langle x\rangle=\int \psi^\ast(x)\,x\,\psi(x)\,dx,\qquad
\langle p\rangle=\int \psi^\ast(x)\left(-i\hbar\frac{\partial}{\partial x}\right)\psi(x)\,dx.
\]

\textbf{基本对易关系:}
\[
[\hat{x},\hat{p}]=i\hbar,
\]
这表示位置与动量不能同时被无限精确地确定,是不确定关系的根源。

\section{零基础预备:微积分与向量微分}

本讲义会频繁出现导数、积分与向量微分算符。
下面给出最少但够用的定义与直觉,方便没有基础的读者直接进入量子力学计算。

\subsection{函数、导数与偏导}
一元函数 $f(x)$ 的导数表示变化率:
\[
\frac{df}{dx}=\lim_{\Delta x\to 0}\frac{f(x+\Delta x)-f(x)}{\Delta x}.
\]
二阶导数 $d^2f/dx^2$ 描述“弯曲程度”。

多元函数 $f(x,y,z)$ 的偏导定义为
\[
\frac{\partial f}{\partial x}=\lim_{\Delta x\to 0}\frac{f(x+\Delta x,y,z)-f(x,y,z)}{\Delta x},
\]
其余变量保持不变。
二阶偏导 $\partial^2 f/\partial x^2$ 在量子力学里经常出现。

\begin{example}
\textbf{一阶导数数值:}设 $f(x)=x^2-3x+1$,求 $f'(x)$ 及 $f'(2)$。

\textbf{解:}$f'(x)=2x-3$,因此 $f'(2)=1$。
\end{example}

\begin{example}
\textbf{偏导数数值:}设 $f(x,y)=x^2y+y^2$,求 $\partial f/\partial x$,
并计算 $\left.\partial f/\partial x\right|_{(1,2)}$。

\textbf{解:}$\partial f/\partial x=2xy$,代入 $(1,2)$ 得 $4$。
\end{example}

\subsection{乘积法则与复共轭求导}
若 $u(t),v(t)$ 可导,则
\[
\frac{d}{dt}[u(t)v(t)]=\frac{du}{dt}v+u\frac{dv}{dt}.
\]
对复函数 $\psi(t)$,共轭与求导可交换:
\[
\frac{d}{dt}\psi^\ast=\left(\frac{d\psi}{dt}\right)^\ast.
\]
因此
\[
\frac{d}{dt}|\psi|^2=\frac{d}{dt}(\psi^\ast\psi)
=\psi^\ast\frac{d\psi}{dt}+\frac{d\psi^\ast}{dt}\psi.
\]

\subsection{向量、点积与叉积}
三维向量写作 $\vec{a}=(a_x,a_y,a_z)$,模长
$|\vec{a}|=\sqrt{a_x^2+a_y^2+a_z^2}$。点积定义为
\[
\vec{a}\cdot\vec{b}=a_x b_x+a_y b_y+a_z b_z=|\vec{a}||\vec{b}|\cos\theta,
\]
用于衡量方向相似程度。叉积
\[
\vec{a}\times\vec{b}=
\begin{pmatrix}
a_y b_z-a_z b_y\\
a_z b_x-a_x b_z\\
a_x b_y-a_y b_x
\end{pmatrix}
\]
给出垂直于两向量平面的向量,大小为 $|\vec{a}||\vec{b}|\sin\theta$,
角动量定义正是基于叉积。
方向由右手定则给出:右手四指从 $\vec{a}$ 指向 $\vec{b}$ 弯曲,
拇指方向就是 $\vec{a}\times\vec{b}$。
公式可理解为对基向量的规则 $\vec{e}_x\times\vec{e}_y=\vec{e}_z$ 等做线性展开,
也可用行列式记忆
\[
\vec{a}\times\vec{b}=
\begin{vmatrix}
\vec{e}_x&\vec{e}_y&\vec{e}_z\\
a_x&a_y&a_z\\
b_x&b_y&b_z
\end{vmatrix}.
\]
更详细的计算过程如下。令
\[
\vec{a}=a_x\vec{e}_x+a_y\vec{e}_y+a_z\vec{e}_z,\quad
\vec{b}=b_x\vec{e}_x+b_y\vec{e}_y+b_z\vec{e}_z.
\]
利用双线性与基向量叉乘规则($\vec{e}_x\times\vec{e}_x=0$,
$\vec{e}_x\times\vec{e}_y=\vec{e}_z$,$\vec{e}_x\times\vec{e}_z=-\vec{e}_y$ 等),
\begin{align*}
\vec{a}\times\vec{b}
&=(a_x\vec{e}_x+a_y\vec{e}_y+a_z\vec{e}_z)\times(b_x\vec{e}_x+b_y\vec{e}_y+b_z\vec{e}_z)\\
&=a_x\vec{e}_x\times(b_y\vec{e}_y+b_z\vec{e}_z)
 +a_y\vec{e}_y\times(b_x\vec{e}_x+b_z\vec{e}_z)
 +a_z\vec{e}_z\times(b_x\vec{e}_x+b_y\vec{e}_y)\\
&=a_x(b_y\vec{e}_z-b_z\vec{e}_y)
 +a_y(-b_x\vec{e}_z+b_z\vec{e}_x)
 +a_z(b_x\vec{e}_y-b_y\vec{e}_x)\\
&=(a_y b_z-a_z b_y)\vec{e}_x+(a_z b_x-a_x b_z)\vec{e}_y+(a_x b_y-a_y b_x)\vec{e}_z,
\end{align*}
从而得到前面的分量公式。

\subsection{标量场与向量场}
场(field)是“空间中每个点都对应一个量”的函数。

\begin{itemize}
\item 若每个点对应一个数,称为\textbf{标量场}。
\item 若每个点对应一个向量,称为\textbf{向量场}。
\end{itemize}

典型标量场如温度分布 $T(\vec{r})$、势能 $V(\vec{r})$、概率密度 $\rho(\vec{r})$。
典型向量场如速度场 $\vec{v}(\vec{r})$、电场 $\vec{E}(\vec{r})$、概率流 $\vec{j}(\vec{r})$。

在量子力学中,波函数 $\psi(\vec{r},t)$ 是复\textbf{标量场},
其模平方给出标量场 $\rho(\vec{r},t)=|\psi(\vec{r},t)|^2$,
而概率流 $\vec{j}(\vec{r},t)$ 是向量场。

举一个具体数值例子:令二维位置 $\vec{r}=(x,y)$。
标量场 $T(x,y)=x^2+y^2$ 在点 $(1,2)$ 处取值为 $T(1,2)=1^2+2^2=5$。
向量场 $\vec{v}(x,y)=(y,-x)$ 在同一点取值为 $\vec{v}(1,2)=(2,-1)$,
是一个带方向的量。

\subsection{梯度、散度、旋度与拉普拉斯算符}
定义向量微分算符(nabla)
\[
\nabla=\left(\frac{\partial}{\partial x},\frac{\partial}{\partial y},\frac{\partial}{\partial z}\right).
\]
对标量场 $f(\vec{r})$,梯度为
\[
\nabla f=\left(\frac{\partial f}{\partial x},\frac{\partial f}{\partial y},\frac{\partial f}{\partial z}\right),
\]
方向指向函数增长最快的方向。

对向量场 $\vec{A}=(A_x,A_y,A_z)$,散度为
\[
\nabla\cdot\vec{A}=\frac{\partial A_x}{\partial x}
+\frac{\partial A_y}{\partial y}
+\frac{\partial A_z}{\partial z},
\]
衡量“源/汇”强度;旋度为
其中 $\partial_x A_z$ 表示对 $x$ 的偏导:$\partial_x A_z=\frac{\partial A_z}{\partial x}$,
其余分量同理。
\[
\nabla\times\vec{A}=
\begin{pmatrix}
\partial_y A_z-\partial_z A_y\\
\partial_z A_x-\partial_x A_z\\
\partial_x A_y-\partial_y A_x
\end{pmatrix},
\]
衡量“旋转”强度。拉普拉斯算符定义为
\[
\nabla^2 f=\nabla\cdot(\nabla f)
=\frac{\partial^2 f}{\partial x^2}
+\frac{\partial^2 f}{\partial y^2}
+\frac{\partial^2 f}{\partial z^2}.
\]
一维情形下 $\nabla^2 f=d^2f/dx^2$。

\begin{example}
\textbf{梯度数值:}设 $f(x,y,z)=x^2+y^2+z^2$,求 $\nabla f$,
并计算在 $(1,-1,2)$ 处的梯度。

\textbf{解:}$\nabla f=(2x,2y,2z)$,代入得 $\nabla f(1,-1,2)=(2,-2,4)$。
\end{example}

\begin{example}
\textbf{散度数值:}设 $\vec{A}(x,y,z)=(xy,\ y^2,\ z)$,求 $\nabla\cdot\vec{A}$,
并计算在 $(1,2,3)$ 处的散度。

\textbf{解:}$\nabla\cdot\vec{A}=y+2y+1=3y+1$,
代入 $(1,2,3)$ 得 $7$。
\end{example}

\begin{example}
\textbf{拉普拉斯数值:}设 $f(x,y,z)=x^2+y^2+z^2$,求 $\nabla^2 f$。

\textbf{解:}$\nabla^2 f=2+2+2=6$。
\end{example}

\subsection{积分与归一化}
定积分 $\int_a^b f(x)\,dx$ 表示在区间上的“累积量”。

在概率论中,概率密度 $\rho(x)$ 的含义是“单位长度上的概率”,因此必须非负,
并满足归一化
\[
\int_{-\infty}^{\infty}\rho(x)\,dx=1.
\]
在量子力学里,概率密度由波函数给出:一维情形
\[
\rho(x,t)=|\psi(x,t)|^2,
\]
三维情形写作
\[
\int |\psi(\vec{r},t)|^2\,d^3r=1,\quad d^3r=dx\,dy\,dz.
\]
后续在球坐标中会用到 $d^3r=r^2\sin\theta\,dr\,d\theta\,d\varphi$ 的体积元。

\begin{example}
\textbf{归一化与概率:}设
\[
\rho(x)=C(1-x^2),\quad -1\le x\le 1,
\]
其余处为 0。
求常数 $C$,并计算 $x\in[0,1/2]$ 的概率。

\textbf{解:}归一化给出
$1=\int_{-1}^1 C(1-x^2)\,dx=C\left[x-\frac{x^3}{3}\right]_{-1}^1=C\cdot\frac{4}{3}$,
故 $C=3/4$。
所求概率为
$\int_0^{1/2} \frac{3}{4}(1-x^2)\,dx=\frac{3}{4}\left[x-\frac{x^3}{3}\right]_0^{1/2}
=\frac{11}{32}$。
\end{example}

\subsection{复数与相位(最少够用)}
复数写作 $z=a+ib$,其中 $i^2=-1$,共轭为 $z^\ast=a-ib$,模长 $|z|=\sqrt{a^2+b^2}$。
欧拉公式
\[
e^{i\theta}=\cos\theta+i\sin\theta
\]
说明复指数只改变相位,不改变模长($|e^{i\theta}|=1$)。波函数是复函数,
概率密度由 $|\psi|^2=\psi^\ast\psi$ 给出。

\section{快速预备知识回顾(从量子信息到量子力学)}

你已掌握:Dirac符号、完备关系、Hermite算符、期望与方差、幺正变换等。
下面只强调连续谱的关键补充。

\subsection{连续谱与狄拉克 \boldmath$\delta$\unboldmath}
\begin{itemize}
\item 位置算符 $\hat{x}$ 的本征态:$\hat{x}|x\rangle=x|x\rangle$($x$ 为本征值/标签)。
\item 完备性:$\int_{-\infty}^{\infty} |x\rangle\langle x|\,dx=\mathbb{I}$。
\item 正交性:$\langle x'|x\rangle=\delta(x-x')$($\delta$ 的性质见前置小节)。
\end{itemize}
连续谱的本征态是\textbf{广义函数}意义下的态,$\delta(x-x')$ 必须在积分中理解。
完备性也可写作
\[
|\psi\rangle=\int_{-\infty}^{\infty}|x\rangle\langle x|\psi\rangle\,dx.
\]
又如,换元型 $\delta$ 给出 $\int f(x)\delta(2x-1)\,dx$ 的结果:零点为
$x_0=\frac{1}{2}$,且 $g'(x)=2$。
因此
\[
\delta(2x-1)=\frac{1}{2}\delta\left(x-\frac{1}{2}\right),
\quad
\int f(x)\delta(2x-1)\,dx=\frac{1}{2}f\left(\frac{1}{2}\right).
\]
同理,$\psi(x)=\int_{-\infty}^{\infty}\psi(x')\delta(x-x')\,dx'$。
由 $\delta$ 的抽取性质,积分结果等于被积函数在 $x'=x$ 处的取值,
因此 $\psi(x)$ 被“原样取出”。
这就是完备性在坐标表象中的具体体现。

\subsection{期望值与方差}
\[\langle A\rangle=\langle\psi|A|\psi\rangle,
\qquad (\Delta A)^2=\langle A^2\rangle-\langle A\rangle^2\]
期望值就是“做很多次实验后的平均读数”。方差衡量分布的“宽度”,
$\Delta A$ 越大说明测量结果越分散。
在坐标表象中,若算符 $A$ 在位置表象中的作用为 $A(x,-i\hbar\partial_x)$,
则
\[
\langle A\rangle=\int_{-\infty}^{\infty}\psi^*(x)\,A(x,-i\hbar\partial_x)\,\psi(x)\,dx
\]
同理可得 $\langle A^2\rangle$。
这一步常用于“把 Dirac 记号翻成积分”。

\subsection{表象变换(离散形式回顾)}
若两组完备正交基 $\{|a_i\rangle\}$ 与 $\{|b_j\rangle\}$,
则变换矩阵 $S_{ij}=\langle a_i|b_j\rangle$:
\[A_B=S^\dagger A_A S,\qquad |\psi\rangle_B=S^\dagger|\psi\rangle_A\]
同一个物理态在不同“坐标系”(基)下有不同的分量表达。
矩阵 $S$ 就是把“旧坐标”旋转到“新坐标”的变换。
算符在不同表象下的矩阵也会相应改变,但\textbf{物理可观测量不变}。

\section{数学工具箱(必须熟练)}

\subsection{高斯积分与常用积分}
\begin{itemize}
\item $\displaystyle \int_{-\infty}^{\infty}e^{-ax^2}dx=\sqrt{\frac{\pi}{a}}\quad(a>0)$
\item $\displaystyle \int_0^{\infty}x^n e^{-ax}dx=\frac{n!}{a^{n+1}}\quad(a>0)$
\item $\displaystyle \int_0^{\infty}r^2 e^{-\alpha r}dr=\frac{2}{\alpha^3}$,
      $\displaystyle \int_0^{\infty}r^3 e^{-\alpha r}dr=\frac{6}{\alpha^4}$
\end{itemize}

\begin{example}
\textbf{高斯积分与归一化的完整推导:}
设
\[
I=\int_{-\infty}^{\infty} e^{-a x^2}\,dx\quad (a>0).
\]
这里是\textbf{定义} $I$,不是推导等式。
则
\[
I^2=\left(\int_{-\infty}^{\infty} e^{-a x^2}\,dx\right)
\left(\int_{-\infty}^{\infty} e^{-a y^2}\,dy\right)
=\iint_{\mathbb{R}^2} e^{-a(x^2+y^2)}\,dx\,dy.
\]
改用极坐标 $x=r\cos\theta,\ y=r\sin\theta$,且
\[
dx\,dy=\left|\frac{\partial(x,y)}{\partial(r,\theta)}\right|dr\,d\theta
=\begin{vmatrix}
\cos\theta & -r\sin\theta\\
\sin\theta & r\cos\theta
\end{vmatrix}dr\,d\theta
=r\,dr\,d\theta,
\]
其中
\[
\frac{\partial(x,y)}{\partial(r,\theta)}
=\begin{pmatrix}
\frac{\partial x}{\partial r} & \frac{\partial x}{\partial \theta}\\
\frac{\partial y}{\partial r} & \frac{\partial y}{\partial \theta}
\end{pmatrix}
\]
称为\textbf{雅可比矩阵},它把 $(r,\theta)$ 的微小变化映射到 $(x,y)$ 的微小变化。
其行列式
\[
\left|\frac{\partial(x,y)}{\partial(r,\theta)}\right|
=\det\begin{pmatrix}
\frac{\partial x}{\partial r} & \frac{\partial x}{\partial \theta}\\
\frac{\partial y}{\partial r} & \frac{\partial y}{\partial \theta}
\end{pmatrix}
\]
称为\textbf{雅可比行列式},给出面积元的缩放因子。
因此 $(r,\theta)$ 的小矩形 $dr\,d\theta$ 在 $(x,y)$ 平面中变为面积
$\left|\frac{\partial(x,y)}{\partial(r,\theta)}\right|dr\,d\theta$ 的平行四边形。
因此得
\[
I^2=\int_0^{2\pi}\!\!\int_0^\infty e^{-a r^2}\,r\,dr\,d\theta
=\left(\int_0^{2\pi} d\theta\right)\left(\int_0^\infty r e^{-a r^2}\,dr\right).
\]
令 $u=a r^2$,则 $du=2a r\,dr$,因此
\[
\int_0^\infty r e^{-a r^2}\,dr=\frac{1}{2a}\int_0^\infty e^{-u}\,du=\frac{1}{2a}.
\]
所以
\[
I^2=2\pi\cdot\frac{1}{2a}=\frac{\pi}{a},
\quad\Rightarrow\quad
I=\sqrt{\frac{\pi}{a}}.
\]
于是对于
\[
\psi(x,t)=A\,e^{-\frac{1}{2}a x^2-i\omega t},
\]
归一化条件给出
\[
1=\int_{-\infty}^{\infty}|\psi|^2dx
=A^2\int_{-\infty}^{\infty} e^{-a x^2}\,dx
=A^2\sqrt{\frac{\pi}{a}},
\]
从而
\[
A=\left(\frac{a}{\pi}\right)^{1/4}.
\]
\end{example}

\begin{example}
\textbf{指数积分的一般推导:}设
\[
I_n(a)=\int_0^{\infty} x^n e^{-a x}\,dx,\quad a>0.
\]
对 $x$ 做变量替换 $u=ax$,则 $x=u/a$,$dx=du/a$:
\[
I_n(a)=\int_0^{\infty}\left(\frac{u}{a}\right)^n e^{-u}\frac{du}{a}
=\frac{1}{a^{n+1}}\int_0^{\infty}u^n e^{-u}\,du.
\]
记
\[
\Gamma(n+1)=\int_0^{\infty}u^n e^{-u}\,du.
\]
对整数 $n$,做分部积分。取
$u_1=u^n$,$dv_1=e^{-u}du$,
则 $du_1=nu^{\,n-1}du$,且
\[
v_1=\int e^{-u}du=-e^{-u}.
\]
因为 $\frac{d}{du}(e^{-u})=-e^{-u}$,所以原函数是 $-e^{-u}$(常数略去)。
因此
\[
\Gamma(n+1)=\left[-u^n e^{-u}\right]_{0}^{\infty}
 +n\int_0^{\infty}u^{n-1}e^{-u}\,du.
\]
边界项为 0($u^n e^{-u}\to 0$),于是得到递推
\[
\Gamma(n+1)=n\,\Gamma(n).
\]
又有
\[
\Gamma(1)=\int_0^{\infty}e^{-u}du=1,
\]
所以
\[
\Gamma(n+1)=n(n-1)\cdots 1=n!.
\]
因此
\[
I_n(a)=\frac{n!}{a^{n+1}}.
\]
\end{example}

\begin{example}
\textbf{常用径向积分:}由上式取 $n=2,3$ 并记 $a=\alpha$,得
\[
\int_0^{\infty} r^2 e^{-\alpha r}\,dr=\frac{2}{\alpha^3},\qquad
\int_0^{\infty} r^3 e^{-\alpha r}\,dr=\frac{6}{\alpha^4}.
\]
若要直接分部积分,可令 $u=r^n$,$dv=e^{-\alpha r}dr$,反复应用即可得到同样结果。
\end{example}

\subsection{分部积分与边界条件}
量子力学中常用:
\[\int u\,dv = uv-\int v\,du\]
典型边界条件:$\psi(\pm\infty)=0$ 或在势阱壁处 $\psi=0$。
分部积分常用于把导数从一个函数“移”到另一个函数上。
边界项为零的前提是波函数在无穷远或势垒处足够快地衰减/消失,
这正是“可归一化”的物理要求。
因此类似
\[
\int_{-\infty}^{\infty}\frac{\partial}{\partial x}F(x)\,dx
=F(\infty)-F(-\infty)=0
\]
的结果来自边界条件($F$ 在无穷远消失或取相同值),
常用于证明概率守恒。

\subsection{傅里叶变换}
\[\phi(p)=\frac{1}{\sqrt{2\pi\hbar}}\int e^{-ipx/\hbar}\psi(x)\,dx\]
\[\psi(x)=\frac{1}{\sqrt{2\pi\hbar}}\int e^{ipx/\hbar}\phi(p)\,dp\]
Parseval定理:$\int|\psi|^2 dx=\int|\phi|^2 dp$。
位置表象与动量表象互为傅里叶变换。
空间里“越窄”的波包,对应动量空间“越宽”的分布,这正是测不准关系的数学根源。
指数中的 $px/\hbar$ 必须无量纲;
因此 $\hbar$ 保证单位一致。系数 $1/\sqrt{2\pi\hbar}$ 的选择保证
傅里叶变换是幺正的,从而概率守恒(Parseval 定理)。
例如,对高斯波包
$\psi(x)=\left(\frac{1}{\pi a^2}\right)^{1/4}e^{-x^2/(2a^2)}$,
则 $\phi(p)$ 仍是高斯,宽度与 $a$ 成反比,
并满足 $\Delta x\,\Delta p=\hbar/2$。

\subsection{球坐标}
\[x=r\sin\theta\cos\varphi,\quad y=r\sin\theta\sin\varphi,\quad z=r\cos\theta\]
\[d^3r=r^2\sin\theta\,dr\,d\theta\,d\varphi\]
$r\ge 0$,$\theta\in[0,\pi]$(从 $+z$ 轴往下量),
$\varphi\in[0,2\pi)$(在 $x$-$y$ 平面内的方位角)。
体积元中的 $r^2\sin\theta$ 是坐标变换的雅可比因子。

\subsection{球坐标下拉普拉斯算符}
\[\nabla^2=\frac{1}{r^2}\frac{\partial}{\partial r}\left(r^2\frac{\partial}{\partial r}\right)
+\frac{1}{r^2\sin\theta}\frac{\partial}{\partial\theta}\left(\sin\theta\frac{\partial}{\partial\theta}\right)
+\frac{1}{r^2\sin^2\theta}\frac{\partial^2}{\partial\varphi^2}\]
若函数只依赖 $r$(球对称),角度导数项为零,
于是 $\nabla^2 f(r)=\frac{1}{r^2}\frac{d}{dr}\left(r^2\frac{df}{dr}\right)$,
这在氢原子基态计算中非常常见。

\section{连续表象与波函数}

\subsection{位置表象}
\textbf{定义:} $\psi(x,t)=\langle x|\psi(t)\rangle$

\textbf{归一化:}
\[\int_{-\infty}^{\infty}|\psi(x,t)|^2\,dx=1\]
$|\psi(x,t)|^2dx$ 是粒子在 $x$ 到 $x+dx$ 内被测到的概率。
如果把波函数看成“热度图”,归一化就是“总概率为 1”。

\begin{example}
\textbf{位置表象概率:}当 $\psi(x)=\sqrt{\frac{2}{a}}\sin\left(\frac{\pi x}{a}\right)$($0<x<a$)时,
粒子落在区间 $(0,a/2)$ 的概率为
\[
P=\int_0^{a/2}\frac{2}{a}\sin^2\left(\frac{\pi x}{a}\right)dx
=\frac{1}{2}.
\]
这也体现了基态在势阱两侧对称分布。
\end{example}

\subsection{动量表象}
动量本征态满足 $\hat{p}|p\rangle=p|p\rangle$,并取连续谱归一化
\[\langle p|p'\rangle=\delta(p-p')\]

由平移对称性可得位置表象中的动量本征函数:
\[\langle x|p\rangle=\frac{1}{\sqrt{2\pi\hbar}}e^{ipx/\hbar}\]
$|p\rangle$ 是“动量确定”的态,在位置表象中是平面波。
平面波在全空间均匀分布,因此不能被归一化,它是理想化的“纯动量”极限。
动量表象波函数 $\phi(p,t)=\langle p|\psi(t)\rangle$
满足归一化 $\int_{-\infty}^{\infty}|\phi(p,t)|^2\,dp=1$,
与位置表象的归一化完全等价(由 Parseval 定理保证)。

\begin{example}
\textbf{动量表象概率:}若 $\phi(p)$ 在区间 $(p_0-\Delta,p_0+\Delta)$ 内近似为常数,
则测得动量位于该区间的概率约为
\[
\int_{p_0-\Delta}^{p_0+\Delta}|\phi(p)|^2dp\approx 2\Delta\,|\phi(p_0)|^2.
\]
这体现了“概率密度 $\times$ 区间长度”的含义。
\end{example}

\subsection{算符在位置/动量表象中的表示}
\begin{itemize}
\item 位置表象:$\hat{x}\to x$,$\hat{p}\to -i\hbar\frac{\partial}{\partial x}$
\item 动量表象:$\hat{p}\to p$,$\hat{x}\to i\hbar\frac{\partial}{\partial p}$
\end{itemize}
动量是“平移的生成元”。当波函数整体向右平移一点点时,
变化率正比于空间导数,因此 $\hat{p}$ 在位置表象中变成微分算符。
这也是 $[\hat{x},\hat{p}]=i\hbar$ 的根源。
用表象表示计算对易子:在位置表象中
$[\hat{x},\hat{p}]\psi=x(-i\hbar\partial_x\psi)-(-i\hbar\partial_x)(x\psi)
=i\hbar\psi$,
因此 $[\hat{x},\hat{p}]=i\hbar$。

\section{薛定谔方程与概率守恒}

\subsection{含时薛定谔方程}
\[i\hbar\frac{\partial}{\partial t}|\psi\rangle=H|\psi\rangle\]
位置表象:
\[i\hbar\frac{\partial\psi}{\partial t}=-\frac{\hbar^2}{2m}\nabla^2\psi+V\psi\]
薛定谔方程告诉我们“量子态如何随时间演化”。
哈密顿量 $H$ 就是\textbf{总能量算符}(动能 + 势能)。
对时间无关势能,能量本征态只会积累相位。
$-\frac{\hbar^2}{2m}\nabla^2$ 是动能算符,
来源于经典关系 $p^2/2m$ 与量子替换 $\hat{p}=-i\hbar\nabla$。
$V(\vec{r},t)$ 表示外加势能,若 $V$ 与时间无关,可采用分离变量求定态。

\subsection{连续性方程与概率流}
定义概率密度 $\rho=|\psi|^2$,概率流密度\textbf{定义}为
\[
\vec{j}=\frac{\hbar}{2mi}(\psi^*\nabla\psi-\psi\nabla\psi^*)
\]
并满足连续性方程(概率守恒)
\[\frac{\partial \rho}{\partial t}+\nabla\cdot\vec{j}=0\]
这是“概率守恒”的数学表达:概率不会凭空产生或消失,
只能在空间中流动。
$\vec{j}$ 就是“概率流密度”。
$\vec{j}$ 的单位为“概率/时间/面积”,
$\nabla\cdot\vec{j}$ 描述流出某点附近体积的“净流量”。
一维情形下,方程变为 $\partial_t\rho+\partial_x j=0$。
一维平面波 $\psi=Ae^{ikx-i\omega t}$ 的概率流为
$j=\frac{\hbar k}{m}|A|^2$,为常数。
这表示粒子以恒定平均速度向右传播。

\subsection{例:球面波概率流}
若 $\psi=\frac{1}{r}e^{ikr}$,则
\[\vec{j}=\frac{\hbar k}{mr^2}\hat{r}=\frac{\hbar k}{mr^3}\vec{r}\]
径向流随 $r^{-2}$ 衰减,与球面积 $4\pi r^2$ 相乘给出恒定总流量。
计算要点:$r=\sqrt{x^2+y^2+z^2}$,因此
\[
\nabla r=\left(\frac{x}{r},\frac{y}{r},\frac{z}{r}\right)=\hat{r}.
\]
对 $\psi=\frac{1}{r}e^{ikr}$ 有
\[
\nabla\psi=\nabla\!\left(\frac{1}{r}\right)e^{ikr}+\frac{1}{r}\nabla\!\left(e^{ikr}\right).
\]
其中
\[
\nabla\!\left(\frac{1}{r}\right)
=-\frac{1}{r^2}\nabla r
=-\frac{1}{r^2}\hat{r},
\]
以及由链式法则
\[
\nabla\!\left(e^{ikr}\right)=e^{ikr}\nabla(ikr)
=ik\,e^{ikr}\nabla r
=ik\,e^{ikr}\hat{r}.
\]
代回得
\[
\nabla\psi
=-\frac{\hat{r}}{r^2}e^{ikr}+\frac{1}{r}\left(ik\,e^{ikr}\hat{r}\right)
=\left(\frac{ik}{r}-\frac{1}{r^2}\right)e^{ikr}\hat{r}.
\]
同理
\[
\nabla\psi^\ast=\left(-\frac{ik}{r}-\frac{1}{r^2}\right)e^{-ikr}\hat{r}.
\]
代入
\[
\vec{j}=\frac{\hbar}{2mi}\left(\psi^\ast\nabla\psi-\psi\nabla\psi^\ast\right)
\]
即可得到 $\vec{j}=\frac{\hbar k}{mr^2}\hat{r}$。

\section{定态理论与能量展开}

\subsection{分离变量}
若 $V$ 与时间无关:
\[\psi(\vec{r},t)=u(\vec{r})e^{-iEt/\hbar}\]
代入薛定谔方程得定态方程:
\[Hu=Eu\]
定态的“时间演化”只是一个整体相位。
因此 $|\psi|^2$ 与所有与时间无关算符的期望值都不随时间变化。

\subsection{能量本征态展开}
本征态完备:$\sum_n |n\rangle\langle n|=\mathbb{I}$,
\[|\psi(t)\rangle=\sum_n c_n e^{-iE_n t/\hbar}|n\rangle\]
期望值:
\[\langle H\rangle=\sum_n |c_n|^2 E_n\]
这是作业中“能量期望等于系数模平方加权”的核心来源。
系数由投影得到 $c_n=\langle n|\psi(0)\rangle$。
若能量谱连续,则求和换成积分 $\int c(E)e^{-iEt/\hbar}|E\rangle\,dE$。
若初态 $|\psi(0)\rangle=\frac{1}{\sqrt{2}}(|1\rangle+|2\rangle)$,
则测量能量得到 $E_1$ 或 $E_2$ 的概率各为 $1/2$。
随时间演化:
$|\psi(t)\rangle=\frac{1}{\sqrt{2}}(e^{-iE_1t/\hbar}|1\rangle+e^{-iE_2t/\hbar}|2\rangle)$,
概率分布保持不变,但相位差会影响可观测量的干涉项。

\section{一维无限深势阱}

\subsection{定态解}
势能:$V(x)=0$($0\le x\le a$),其他位置 $V=\infty$。
定态方程:
\[-\frac{\hbar^2}{2m}\frac{d^2u}{dx^2}=Eu\]
边界条件 $u(0)=u(a)=0$,得
\[u_n(x)=\sqrt{\frac{2}{a}}\sin\left(\frac{n\pi x}{a}\right),\quad
E_n=\frac{n^2\pi^2\hbar^2}{2ma^2}\]
无限深势阱把粒子“关”在 $[0,a]$。
边界处波函数必须为零,像一根固定两端的琴弦,因此能量离散、波形为正弦。
$n=1,2,3,\dots$ 为正整数($n=0$ 会使波函数为零)。
归一化系数 $\sqrt{2/a}$ 来自 $\int_0^a |u_n|^2 dx=1$。
能级间隔随 $1/a^2$ 变化,阱越窄能级越稀疏(更“高能”)。

\subsection{任意初态展开}
给定 $\psi(x,0)$,展开为
\[\psi(x,0)=\sum_n c_n u_n(x),\quad c_n=\int_0^a u_n(x)\psi(x,0)\,dx\]
时间演化:
\[\psi(x,t)=\sum_n c_n u_n(x)e^{-iE_n t/\hbar}\]

\textbf{提示:}若初态关于 $x=\frac{a}{2}$ 对称或反对称,可快速判断奇偶 $n$ 的系数为零。
若初态满足 $\psi(a/2+x)=\psi(a/2-x)$(对称),则只有\textbf{奇数} $n$ 的系数非零;
若满足 $\psi(a/2+x)=-\psi(a/2-x)$(反对称),则只有\textbf{偶数} $n$ 非零。
这能极大简化系数计算。

\section{球坐标与氢原子基态}

\subsection{基态波函数}
\[\psi_{100}(r)=\frac{1}{\sqrt{\pi a_0^3}}e^{-r/a_0}\]
球对称的基态只依赖 $r$,没有角度信息。
真正的“半径分布”不是 $|\psi|^2$,而是
$P(r)=4\pi r^2|\psi(r)|^2$,因为球壳体积随 $r^2$ 增大。
例如,氢原子基态的最可几半径由最大化
$P(r)=4\pi r^2|\psi|^2\propto r^2 e^{-2r/a_0}$ 得出,
令 $\frac{d}{dr}(r^2 e^{-2r/a_0})=0$ 可得 $r=a_0$。

\subsection{期望值计算要点}
\begin{enumerate}
\item $\langle r\rangle=\int r|\psi|^2 dV$,使用体积元 $dV=r^2\sin\theta\,dr\,d\theta\,d\varphi$。
\item $\langle V\rangle=\left\langle -\frac{e^2}{r}\right\rangle$,利用 $r$ 积分公式。
\item $\langle T\rangle=\left\langle -\frac{\hbar^2}{2m}\nabla^2\right\rangle$,
      球对称时 $\nabla^2\psi=\frac{1}{r^2}\frac{d}{dr}\left(r^2\frac{d\psi}{dr}\right)$。
\end{enumerate}
最终结果:
\[\langle r\rangle=\frac{3a_0}{2},\quad
\langle V\rangle=-\frac{e^2}{a_0},\quad
\langle T\rangle=\frac{\hbar^2}{2ma_0^2}\]
总能量 $E=-\frac{e^2}{2a_0}=-13.6\,\mathrm{eV}$。

\section{角动量算符与对易关系}

\subsection{定义与分量}
\[\vec{L}=\vec{r}\times\vec{p}\]
位置表象中:
\[L_x=-i\hbar\left(y\frac{\partial}{\partial z}-z\frac{\partial}{\partial y}\right),\quad
L_y=-i\hbar\left(z\frac{\partial}{\partial x}-x\frac{\partial}{\partial z}\right),\quad
L_z=-i\hbar\left(x\frac{\partial}{\partial y}-y\frac{\partial}{\partial x}\right)
\]
角动量描述“绕原点旋转”的量。
它的分量不能同时精确测量,但 $L^2$ 可以与任意一个分量同时测量。
在中心势中,能量本征态同时是 $L^2,L_z$ 的本征态。

\subsection{对易关系}
\[ [L_x,L_y]=i\hbar L_z\]
循环置换得到其余两条。
结论:不同分量不能同时测准,但 $L^2$ 与任一分量对易。
对易子 $[A,B]=AB-BA$。
若对易子为零,表示两可观测量可同时精确测量(共享本征态)。
若 $|l,m\rangle$ 是角动量本征态,则
$L^2|l,m\rangle=\hbar^2 l(l+1)|l,m\rangle$,
$L_z|l,m\rangle=\hbar m|l,m\rangle$。
因此测量 $L_z$ 的可能结果是离散的整数(或半整数)倍 $\hbar$。

\section{Ehrenfest定理(量子-经典对应)}

\subsection{一般形式}
\[
\frac{d\langle A\rangle}{dt}
=\frac{1}{i\hbar}\langle[A,H]\rangle+\left\langle\frac{\partial A}{\partial t}\right\rangle
\]
Ehrenfest 定理把量子平均值的演化和经典方程联系起来。
当波包足够窄、势能变化缓慢时,$\langle x\rangle$ 和 $\langle p\rangle$
几乎满足经典牛顿方程。

\subsection{对动量的应用}
令 $A=\hat{p}$,$H=\frac{\hat{p}^2}{2m}+V(x)$,可得
\[\frac{d\langle p\rangle}{dt}=\left\langle-\frac{\partial V}{\partial x}\right\rangle\]
这是量子版本的牛顿第二定律。
自由粒子 $V=0$ 时,$\frac{d\langle p\rangle}{dt}=0$,
因此 $\langle p\rangle$ 为常数,$\langle x\rangle$ 随时间线性变化。

\section{不确定关系}

\subsection{一般不确定关系(Robertson)}
\[\Delta A\,\Delta B\ge\frac{1}{2}\left|\langle[A,B]\rangle\right|\]
测不准不是“测量仪器不够好”,而是波函数的\textbf{内在涨落}。
当两个算符不对易时,不可能同时让它们的分布都很窄。

\subsection{位置-动量}
\[\Delta x\,\Delta p\ge\frac{\hbar}{2}\]
高斯波包满足等号。
若 $\psi(x)\propto e^{-x^2/(4\sigma^2)}$,
则 $\Delta x=\sigma$,$\Delta p=\hbar/(2\sigma)$,
因此 $\Delta x\,\Delta p=\hbar/2$,达到最小不确定。

\begin{example}
\textbf{高斯波包数值:}取 $\sigma=0.5$,则
$\Delta x=0.5$,$\Delta p=\hbar/(2\sigma)=\hbar$,
因此 $\Delta x\,\Delta p=0.5\,\hbar=\hbar/2$。
\end{example}

\subsection{时间-频率}
对归一化信号 $f(t)$ 及其傅里叶变换 $F(\omega)$:
\[\Delta t\,\Delta\omega\ge\frac{1}{2}\]
方法:用 Parseval 定理与 Cauchy-Schwarz 不等式,配合分部积分。
$\omega$ 是角频率(单位 rad/s),
$\Delta t$ 与 $\Delta\omega$ 描述信号在时间与频率上的“扩展宽度”。
这与量子中的 $\Delta x\,\Delta p$ 属于同一数学结构。

\section{双缝干涉与相干性}

\subsection{概率幅叠加}
若通过缝1、缝2到达屏幕点 $x$ 的概率幅为 $\alpha_1,\alpha_2$,则
\[P(x)=|\alpha_1+\alpha_2|^2=P_1+P_2+2\sqrt{P_1P_2}\cos(\varphi_1-\varphi_2)\]
干涉条纹来自相位差。
概率\textbf{不是}直接相加,而是\textbf{概率幅}相加后再取模平方。
这就是为什么会出现“亮条纹更亮、暗条纹变暗”的干涉现象。
若两缝强度相同,$P_1=P_2$,则
当 $\Delta\varphi=2\pi m$ 时出现亮纹(最大值),
当 $\Delta\varphi=(2m+1)\pi$ 时出现暗纹(最小值)。

\subsection{Which-way 信息的影响}
当路径信息与探测器纠缠,概率幅变为 $\alpha_i\beta_j$,
若能完全区分路径(如 $\beta_2\to 0$),干涉项消失。
一旦“知道走哪条缝”,两条路径就不再相干,
干涉项被探测器“记录”掉。
可见性与可区分性此消彼长。

\section{表象理论与矩阵元}

\subsection{表象变换通式}
\[S_{ij}=\langle a_i|b_j\rangle,\quad A_B=S^\dagger A_A S,\quad |\psi\rangle_B=S^\dagger|\psi\rangle_A\]
若 $F$ 在 $Q$ 表象为 $F_Q$,对角化 $F_Q=U\Lambda U^\dagger$,
则在 $F$ 表象下 $|\psi\rangle_F=U^\dagger|\psi\rangle_Q$,
其分量模平方即测得本征值的概率。
把算符“对角化”就是找到它的本征表象。
在该表象中,测量结果直接对应向量分量,概率就是分量模平方(Born 规则)。

\subsection{一维势阱中的矩阵元(坐标与动量)}
能量本征态 $u_n(x)=\sqrt{\frac{2}{a}}\sin\frac{n\pi x}{a}$。
\begin{itemize}
\item $x_{nn}=\frac{a}{2}$。
\item $x_{mn}=\frac{4mna}{\pi^2(m^2-n^2)^2}\left[(-1)^{m-n}-1\right]$($m-n$ 奇数时非零)。
\item $p_{nn}=0$,
      $p_{mn}=\frac{2imn\hbar}{a(m^2-n^2)}\left[(-1)^{m-n}-1\right]$。
\end{itemize}
由 $x_{mn}$ 的表达式可见:当 $m-n$ 为偶数时,$x_{mn}=0$。
这体现了势阱中“奇偶选择定则”,可快速判断哪些跃迁矩阵元为零。
矩阵元 $A_{mn}=\langle m|A|n\rangle$
反映“从 $|n\rangle$ 到 $|m\rangle$ 的跃迁强度”。
对角元 $A_{nn}$ 是在能量本征态中的期望值。

\subsection{动量表象中的角动量算符}
利用 $\vec{r}\leftrightarrow i\hbar\nabla_p$:
\[
(L_x)_{p'p}=-i\hbar\left(p_z\frac{\partial}{\partial p_y}-p_y\frac{\partial}{\partial p_z}\right)
\delta(\vec{p}-\vec{p}')
\]
在动量表象中,位置算符变成对动量的微分,
因此角动量依然表现为“绕动量空间的旋转生成元”。

\section{线性代数速记:对角化与本征表象}

\subsection{一般步骤}
给定矩阵 $A$:
\begin{enumerate}
\item 解特征方程 $\det(A-\lambda I)=0$ 得特征值。
\item 对每个 $\lambda$ 求特征向量并正交归一,组成酉矩阵 $U$。
\item $A_{\text{diag}}=U^\dagger A U$。
\end{enumerate}
对角化就是“找到让算符看起来最简单的坐标系”。
在本征基下,算符只剩下对角元,测量结果一目了然。
$A=\begin{pmatrix}0&1\\1&0\end{pmatrix}$ 的本征值为 $\pm1$,
本征向量为 $\frac{1}{\sqrt{2}}(1,\pm1)^T$。
这对应把 $A$ 变成对角矩阵 $\mathrm{diag}(1,-1)$。

\subsection{本征表象中的概率}
若 $|\psi\rangle$ 在原表象为列向量 $\psi$,则
\[\psi_{\text{eig}}=U^\dagger\psi\]
其分量模平方即测得对应本征值的概率。
$U$ 是由本征向量组成的酉矩阵,满足 $U^\dagger U=I$。
向量分量的模平方之和为 1,体现归一化。

\section{相干态与投影算符}

相干态定义:
\[|\alpha\rangle=\sum_{n=0}^\infty e^{-\frac{1}{2}|\alpha|^2}\frac{\alpha^n}{\sqrt{n!}}|n\rangle\]
投影期望:
\[\langle\alpha|k\rangle\langle k|\alpha\rangle=e^{-|\alpha|^2}\frac{|\alpha|^{2k}}{k!}\]
即 Poisson 分布,平均光子数为 $|\alpha|^2$。
相干态是“最像经典光场”的量子态,
既是简谐振子湮灭算符的本征态,也是最小不确定态。
当 $|\alpha|^2=4$ 时,光子数的平均值为 4,方差也为 4(Poisson 分布的性质)。

\section{Dirac符号翻译清单}
\begin{itemize}
\item $F(x,i\hbar\partial_x)\psi(x)=\Phi(x)\quad\Leftrightarrow\quad F|\psi\rangle=|\Phi\rangle$
\item $i\hbar\partial_t\psi=H(x,-i\hbar\partial_x)\psi\quad\Leftrightarrow$
      $i\hbar\partial_t|\psi\rangle=H|\psi\rangle$
\item $H|n\rangle=E_n|n\rangle$
\item $\int u_m^*(x)u_n(x)dx=\delta_{mn}\quad\Leftrightarrow\quad\langle m|n\rangle=\delta_{mn}$
\item $\psi(x,t)=\sum_n a_n(t)u_n(x)\quad\Leftrightarrow\quad|\psi\rangle=\sum_n a_n|n\rangle$
\end{itemize}
把 Dirac 记号翻译成微分方程或积分表达,
只是“同一个物理内容的两种语言”。
熟练互译能大幅降低做题成本。
$F(x,i\hbar\partial_x)$ 表示把算符写成“坐标 + 导数”的形式;
$u_n(x)=\langle x|n\rangle$ 是能量本征态在位置表象的波函数。
离散谱满足 $\delta_{mn}$,连续谱对应 $\delta(x-x')$。

\section{作业题对应索引(建议路线)}
\begin{center}
\begin{tabular}{ll}
\toprule
作业题 & 对应章节 \\
\midrule
1 & 数学工具箱:高斯积分 \\
2--3 & 薛定谔方程与概率守恒 \\
4 & 球坐标与氢原子基态 \\
5 & 角动量算符与对易关系 \\
6 & Ehrenfest定理 \\
7--8 & 不确定关系 \\
9 & 双缝干涉与相干性 \\
10 & 定态理论与能量展开 \\
11 & 一维无限深势阱 \\
12--13 & 表象理论与矩阵元 \\
14--15 & 线性代数速记:对角化与表象变换 \\
16 & 相干态与投影算符 \\
17--18 & Dirac符号翻译 + 表象变换 \\
\bottomrule
\end{tabular}
\end{center}

\section{总结}
本讲义把量子信息中的抽象公理具体化为连续变量的计算框架,
涵盖波函数、薛定谔方程、角动量、定态理论、表象变换与测不准等内容。
按照“作业题对应索引”逐章学习与练习,
即可独立完成《量子力学作业》中所有题目。

\end{document}
