\documentclass[12pt]{article}

\usepackage[a4paper,margin=2.2cm]{geometry}
\usepackage{amsmath,amssymb}
\usepackage{enumitem}
\usepackage{xcolor}
\usepackage{hyperref}
\usepackage{xeCJK}

\setCJKmainfont{Source Han Serif CN}
\setCJKsansfont{Source Han Sans CN}
\setCJKmonofont{Source Han Sans CN}
\setlist[itemize]{leftmargin=*,nosep}
\setlist[enumerate]{leftmargin=*,nosep}
\hypersetup{
  colorlinks=true,
  linkcolor=blue!55!black,
  urlcolor=blue!55!black,
  citecolor=blue!55!black
}

\title{量子信息手写笔记(整理版)}
\author{}
\date{}

\begin{document}
\maketitle
\tableofcontents
\newpage

\section{基、基变换与酉矩阵}

\subsection{基向量与基变换矩阵}
两组基向量分别为 $\{|e_1\rangle,\ldots,|e_n\rangle\}$ 与 $\{|f_1\rangle,\ldots,|f_n\rangle\}$。
把基向量按列排成矩阵:
\[
  E=\big(|e_1\rangle\ |e_2\rangle\ \cdots\ |e_n\rangle\big),\quad
  F=\big(|f_1\rangle\ |f_2\rangle\ \cdots\ |f_n\rangle\big).
\]
定义基变换矩阵
\[
  U_{ij}=\langle e_i|f_j\rangle .
\]
新基向量在旧基下展开为
\[
  |f_j\rangle=\sum_i U_{ij}|e_i\rangle.
\]
矩阵形式写作
\[
  F=E\,U.
\]
对任意态 $|\psi\rangle$,在旧基与新基下分别展开:
\[
  |\psi\rangle=\sum_i c_i |e_i\rangle=\sum_j c'_j |f_j\rangle,
\]
其中
\[
  \vec c=\begin{pmatrix}c_1\\ \vdots\\ c_n\end{pmatrix},\qquad
  \vec c\,'=\begin{pmatrix}c'_1\\ \vdots\\ c'_n\end{pmatrix}
\]
为展开系数组成的列向量。两者满足
\[
  \vec c\,'=U^\dagger \vec c,\qquad \vec c=U\vec c\,'.
\]
若 $E,F$ 为正交归一基,则 $U$ 为酉矩阵($U^\dagger U=I$)。

\subsection{算符的基变换}
矩阵元定义(以旧基 $E$ 为例):
\[
  A^{(E)}_{ij}=\langle e_i|A|e_j\rangle.
\]
算符在新基下的矩阵表示为
\[
  A^{(F)}=U^\dagger A^{(E)}U.
\]

\section{本征方程与对角化}
\[
  A|\phi\rangle=\lambda|\phi\rangle,\qquad
  (A-\lambda I)|\phi\rangle=0.
\]
非平凡解要求
\[
  \det(A-\lambda I)=0
\]
从而求得本征值 $\lambda$,再代回求本征向量。

\paragraph{厄米矩阵要点}
若 $A=A^\dagger$,则本征值全为实数,本征向量可取正交归一基;
因此 $A$ 可被酉对角化:$A=U\Lambda U^\dagger$,
其中 $U$ 的列向量就是按同一顺序排列的本征向量,
$\Lambda=\mathrm{diag}(\lambda_1,\lambda_2,\ldots)$ 为对应本征值的对角矩阵。

\section{单比特态与 Bloch 球}
任意纯态可写成
\[
  |\psi\rangle=\cos\frac{\theta}{2}\,|0\rangle
  +e^{i\varphi}\sin\frac{\theta}{2}\,|1\rangle,
\]
其中 $\theta\in[0,\pi]$ 为极角,$\varphi\in[0,2\pi]$ 为方位角。
纯态对应 Bloch 球面上的点。

\section{密度矩阵}
\[
  \rho=|\psi\rangle\langle\psi|.
\]
基本性质:$\rho^\dagger=\rho$,$\rho\ge 0$,$\mathrm{tr}(\rho)=1$。

\paragraph{纯态与混态}
纯态满足 $\rho^2=\rho$,$\mathrm{tr}(\rho^2)=1$;
混态满足 $\mathrm{tr}(\rho^2)<1$。
一般混态可写成
\[
  \rho=\sum_k p_k|\psi_k\rangle\langle\psi_k|,\quad p_k\ge 0,\ \sum_k p_k=1,
\]
并可酉对角化为
\[
  \rho=U\,\mathrm{diag}(p_1,p_2,\ldots)\,U^\dagger.
\]

\section{Bell 态与最大纠缠}
四个 Bell 态:
\[
  |\Phi^\pm\rangle=\frac{1}{\sqrt{2}}(|00\rangle\pm|11\rangle),\qquad
  |\Psi^\pm\rangle=\frac{1}{\sqrt{2}}(|01\rangle\pm|10\rangle).
\]
它们都是最大纠缠态。

\subsection{对 $\Phi^+$ 施加单比特门}
在第一比特上作用(超密编码常用):
\[
\begin{array}{c|c}
\text{操作} & \text{结果} \\ \hline
I & |\Phi^+\rangle \\
X & |\Psi^+\rangle \\
Z & |\Phi^-\rangle \\
iY\ (XZ) & |\Psi^-\rangle
\end{array}
\]

\section{量子超密编码}
预共享 Bell 态 $|\Phi^+\rangle=\frac{1}{\sqrt{2}}(|00\rangle+|11\rangle)$。按时间顺序:
\begin{enumerate}
  \item $t_0$:Alice 与 Bob 共享 $|\Phi^+\rangle$(Alice 持第 1 比特,Bob 持第 2 比特)。
  \item $t_1$:Alice 将 2 比特信息编码为本地操作
  \[
  00\to I,\quad 01\to X,\quad 10\to Z,\quad 11\to XZ\ (=iY),
  \]
  并把她的量子比特发送给 Bob。
  \item $t_2$:Bob 进行 Bell 测量:先对第 1 比特施 CNOT(第1控第2),再对第 1 比特施 $H$,
  最后在计算基测量两比特。
  \item $t_3$:测量结果 $\{00,01,10,11\}$ 分别对应
  $\{|\Phi^+\rangle,|\Psi^+\rangle,|\Phi^-\rangle,|\Psi^-\rangle\}$,解码得到 Alice 的 2 比特信息。
\end{enumerate}
结论:1 个量子比特 + 1 对纠缠可传 2 比特经典信息。

\section{量子隐形传态}
预共享 Bell 对,待传态为 $|\psi\rangle=\alpha|0\rangle+\beta|1\rangle$。按时间顺序:
\begin{enumerate}
  \item $t_0$:Alice 与 Bob 共享 Bell 对(Alice 持第 2 比特,Bob 持第 3 比特),
  Alice 还有待传态第 1 比特 $|\psi\rangle$。
  \item $t_1$:Alice 对第 1、2 比特执行 CNOT(第1控第2),再对第 1 比特施 $H$。
  \item $t_2$:Alice 在计算基测量第 1、2 比特,得到 $(M_1,M_2)\in\{0,1\}^2$,
  并通过经典信道告知 Bob。
  \item $t_3$:Bob 对第 3 比特施加纠正
  \[
  X^{M_2}Z^{M_1},
  \]
  得到原态 $|\psi\rangle$。
\end{enumerate}
该过程不违反不可克隆与超光速通信(经典信道不可省)。

\section{发展史与里程碑(背诵)}
\begin{itemize}
  \item 1992:Bennett 与 Wiesner 提出量子超密编码(量子密集编码)。
  \item 1993:Bennett 等提出量子隐形传态方案。
  \item 1996:首个超密编码实验(光子偏振编码)。
  \item 2004:原子系综中的超密编码实验。
  \item 2008:连续变量超密编码实验。
  \item 2012:超导量子比特超密编码实验。
  \item 2017:高维纠缠的超密编码(利用高维系统传更多信息)。
  \item 2010:自由空间 16 km 量子隐形传态实验。
\end{itemize}

\section{实验装置与关键技术(背诵)}
\begin{itemize}
  \item \textbf{纠缠源:}需要高保真度 Bell 态;常见平台有光子偏振、原子系综、超导量子比特等。
  \item \textbf{Bell 态制备:}从 $|00\rangle$ 出发,$H\otimes I$ + CNOT 可制备 $|\Phi^+\rangle$。
  \item \textbf{Bell 基测量:}标准实现为 CNOT(第1控第2)+ $H$(第1比特)+ 计算基测量;
        完全区分 4 个 Bell 态在实验上有技术难度。
  \item \textbf{经典信道:}超密编码需要传送 1 个量子比特;隐形传态需要传 2 个经典比特;
        经典信道限制整体速度,不可超光速。
  \item \textbf{退相干与损耗:}纠缠在传输/存储中易退相干,探测效率与信道损耗会降低保真度。
\end{itemize}

\section{计算类背诵要点(导论/作业常考)}
\begin{itemize}
  \item \textbf{CNOT 作用:}$|ab\rangle\mapsto|a,\,b\oplus a\rangle$。
  \item \textbf{Bell 测量映射:}
  $|\Phi^+\rangle\to|00\rangle,\ |\Psi^+\rangle\to|01\rangle,\ |\Phi^-\rangle\to|10\rangle,\ |\Psi^-\rangle\to|11\rangle$(经 CNOT+H 后)。
  \item \textbf{超密编码映射:}
  $I\to|\Phi^+\rangle,\ X\to|\Psi^+\rangle,\ Z\to|\Phi^-\rangle,\ XZ(=iY)\to|\Psi^-\rangle$。
  \item \textbf{隐形传态纠正:}Bob 按 $(M_1,M_2)$ 施加 $X^{M_2}Z^{M_1}$。
  \item \textbf{资源消耗:}超密编码与隐形传态均消耗 1 对纠缠;隐形传态还需 2 个经典比特。
  \item \textbf{互为对偶:}超密编码“量子→经典”,隐形传态“经典→量子”。
\end{itemize}

\section{常见量子门(简表)}
单比特门:
\[
X=\begin{pmatrix}0&1\\1&0\end{pmatrix},\quad
Z=\begin{pmatrix}1&0\\0&-1\end{pmatrix},\quad
H=\frac{1}{\sqrt{2}}\begin{pmatrix}1&1\\1&-1\end{pmatrix},
\]
\[
S=\begin{pmatrix}1&0\\0&i\end{pmatrix},\quad
T=\begin{pmatrix}1&0\\0&e^{i\pi/4}\end{pmatrix},
\quad
R_\alpha(\theta)=e^{-i\theta\sigma_\alpha/2}\ (\alpha=x,y,z).
\]
双比特门:$\mathrm{CNOT}$ 定义为
\[
|ab\rangle\mapsto|a,\,b\oplus a\rangle,
\]
矩阵表示为 $\mathrm{CNOT}=\begin{pmatrix}1&0&0&0\\0&1&0&0\\0&0&0&1\\0&0&1&0\end{pmatrix}$。
常用通用门集:$\{H,S,T,\mathrm{CNOT}\}$。

\end{document}
