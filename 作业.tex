\documentclass[12pt]{article}

\usepackage[a4paper,margin=2.2cm]{geometry}
\usepackage{amsmath,amssymb}
\usepackage{booktabs}
\usepackage{enumitem}
\usepackage{xeCJK}
\usepackage{hyperref}
\usepackage{graphicx}

\setCJKmainfont{Source Han Serif CN}
\setCJKsansfont{Source Han Sans CN}
\setCJKmonofont{Source Han Sans CN}
\setlist[itemize]{leftmargin=*,nosep}
\setlist[enumerate]{leftmargin=*,nosep}

\title{量子信息导论:作业解答}
\author{}
\date{}

\begin{document}
\maketitle
\tableofcontents

\section{量子信息}
\subsection{1}

\subsubsection*{练习2.2(矩阵表示例子)}
\textbf{题目:}设 $V$ 是以 $|0\rangle$ 和 $|1\rangle$ 为基向量的向量空间,$A$ 是从 $V$ 到 $V$ 的线性算子,使 $A|0\rangle=|1\rangle$,$A|1\rangle=|0\rangle$。给出 $A$ 相对于输入基 $|0\rangle,|1\rangle$ 和输出基 $|0\rangle,|1\rangle$ 的矩阵表示。找出使 $A$ 具有不同矩阵表示的输入输出基。

\textbf{解:}
先写出算符本身(与基无关)的表达式。由
\[
A|0\rangle=|1\rangle,\qquad A|1\rangle=|0\rangle
\]
并利用完备关系 $I=|0\rangle\langle 0|+|1\rangle\langle 1|$,
\[
A=A I=A|0\rangle\langle 0|+A|1\rangle\langle 1|
=|1\rangle\langle 0|+|0\rangle\langle 1|
\]
这就是 Pauli-$X$ 算符。

\textbf{(1) 输入基与输出基同为 $\{|0\rangle,|1\rangle\}$:}
按 3.4 节的记号,矩阵元
\[
A^{(0,1)}_{ij}=\langle i|A|j\rangle
\]
且第 $j$ 列是 $A|j\rangle$ 在基 $\{|0\rangle,|1\rangle\}$ 下的展开系数。由
\[
A|0\rangle=|1\rangle,\quad A|1\rangle=|0\rangle
\]
得到
\[
A^{(0,1)}=\begin{pmatrix}0&1\\1&0\end{pmatrix}.
\]

\textbf{(2) 取不同基得到不同矩阵表示:}
令新基为 Hadamard 基
\[
|+\rangle=\frac{|0\rangle+|1\rangle}{\sqrt{2}},\quad
|-\rangle=\frac{|0\rangle-|1\rangle}{\sqrt{2}}.
\]
基变换矩阵(按 2.7 节定义)为
\[
U_{ij}=\langle e_i|f_j\rangle,\quad
U=\frac{1}{\sqrt{2}}\begin{pmatrix}1&1\\1&-1\end{pmatrix}.
\]
于是算符在新基下的矩阵表示为
\[
A^{(+,-)}=U^\dagger A^{(0,1)}U
=\begin{pmatrix}1&0\\0&-1\end{pmatrix},
\]
与 $A^{(0,1)}$ 不同。

\subsubsection*{练习2.11(Pauli 矩阵的特征分解)}
\textbf{题目:}找出 Pauli 矩阵 $X,Y,Z$ 的特征向量、特征值和对角表示。

\textbf{解:}
\[ X=\begin{pmatrix}0&1\\1&0\end{pmatrix},\quad
Y=\begin{pmatrix}0&-i\\ i&0\end{pmatrix},\quad
Z=\begin{pmatrix}1&0\\0&-1\end{pmatrix}
\]

\paragraph{(1) $X$ 的特征分解}
解特征方程 $\det(X-\lambda I)=0$:
\[
\det\!\begin{pmatrix}-\lambda&1\\1&-\lambda\end{pmatrix}
=\lambda^2-1=0
\Rightarrow \lambda=\pm 1.
\]
当 $\lambda=+1$ 时,解 $(X-I)\vec v=0$:
\[
\begin{pmatrix}-1&1\\1&-1\end{pmatrix}
\begin{pmatrix}a\\b\end{pmatrix}=0
\Rightarrow a=b.
\]
取归一化向量
\[
|+\rangle=\frac{|0\rangle+|1\rangle}{\sqrt{2}} \quad(\lambda=+1).
\]
当 $\lambda=-1$ 时,解 $(X+I)\vec v=0$:
\[
\begin{pmatrix}1&1\\1&1\end{pmatrix}
\begin{pmatrix}a\\b\end{pmatrix}=0
\Rightarrow a=-b.
\]
取归一化向量
\[
|-\rangle=\frac{|0\rangle-|1\rangle}{\sqrt{2}} \quad(\lambda=-1).
\]
因此在基 $\{|+\rangle,|-\rangle\}$ 下,
\[
X \sim \mathrm{diag}(1,-1).
\]
具体计算如下:令
\[
U=\big(|+\rangle\ |-\rangle\big)
=\frac{1}{\sqrt{2}}\begin{pmatrix}1&1\\1&-1\end{pmatrix},
\quad U^\dagger=U.
\]
则
\[
U^\dagger X U
=\frac{1}{2}\begin{pmatrix}1&1\\1&-1\end{pmatrix}
\begin{pmatrix}0&1\\1&0\end{pmatrix}
\begin{pmatrix}1&1\\1&-1\end{pmatrix}
=\begin{pmatrix}1&0\\0&-1\end{pmatrix}.
\]

\paragraph{(2) $Y$ 的特征分解}
解特征方程 $\det(Y-\lambda I)=0$:
\[
\det\!\begin{pmatrix}-\lambda&-i\\ i&-\lambda\end{pmatrix}
=\lambda^2-1=0
\Rightarrow \lambda=\pm 1.
\]
当 $\lambda=+1$ 时,解 $(Y-I)\vec v=0$:
\[
\begin{pmatrix}-1&-i\\ i&-1\end{pmatrix}
\begin{pmatrix}a\\b\end{pmatrix}=0
\Rightarrow -a-ib=0 \Rightarrow a=-ib.
\]
取 $b=1$,归一化为
\[
|y_+\rangle=\frac{|0\rangle+i|1\rangle}{\sqrt{2}} \quad(\lambda=+1).
\]
当 $\lambda=-1$ 时,解 $(Y+I)\vec v=0$:
\[
\begin{pmatrix}1&-i\\ i&1\end{pmatrix}
\begin{pmatrix}a\\b\end{pmatrix}=0
\Rightarrow a=ib.
\]
取 $b=1$,归一化为
\[
|y_-\rangle=\frac{|0\rangle-i|1\rangle}{\sqrt{2}} \quad(\lambda=-1).
\]
因此在基 $\{|y_+\rangle,|y_-\rangle\}$ 下,
\[
Y \sim \mathrm{diag}(1,-1).
\]
具体计算如下:令
\[
U=\big(|y_+\rangle\ |y_-\rangle\big)
=\frac{1}{\sqrt{2}}\begin{pmatrix}1&1\\ i&-i\end{pmatrix},
\quad U^\dagger=\frac{1}{\sqrt{2}}\begin{pmatrix}1&-i\\ 1&i\end{pmatrix}.
\]
则
\[
U^\dagger Y U
=\frac{1}{2}\begin{pmatrix}1&-i\\ 1&i\end{pmatrix}
\begin{pmatrix}0&-i\\ i&0\end{pmatrix}
\begin{pmatrix}1&1\\ i&-i\end{pmatrix}
=\begin{pmatrix}1&0\\0&-1\end{pmatrix}.
\]

\paragraph{(3) $Z$ 的特征分解}
解特征方程 $\det(Z-\lambda I)=0$:
\[
\det\!\begin{pmatrix}1-\lambda&0\\0&-1-\lambda\end{pmatrix}
=(1-\lambda)(-1-\lambda)=\lambda^2-1=0
\Rightarrow \lambda=\pm 1.
\]
当 $\lambda=+1$ 时,解 $(Z-I)\vec v=0$:
\[
\begin{pmatrix}0&0\\0&-2\end{pmatrix}
\begin{pmatrix}a\\b\end{pmatrix}=0
\Rightarrow b=0,
\]
取归一化向量 $|0\rangle$。
当 $\lambda=-1$ 时,解 $(Z+I)\vec v=0$:
\[
\begin{pmatrix}2&0\\0&0\end{pmatrix}
\begin{pmatrix}a\\b\end{pmatrix}=0
\Rightarrow a=0,
\]
取归一化向量 $|1\rangle$。
因此在基 $\{|0\rangle,|1\rangle\}$ 下,
\[
Z \sim \mathrm{diag}(1,-1).
\]
具体计算如下:令
\[
U=\big(|0\rangle\ |1\rangle\big)=\begin{pmatrix}1&0\\0&1\end{pmatrix}=I,\quad U^\dagger=I.
\]
则
\[
U^\dagger Z U = Z = \begin{pmatrix}1&0\\0&-1\end{pmatrix}.
\]

\subsubsection*{练习2.20(更换基)}
\textbf{题目:}设 $A'$ 和 $A''$ 是向量空间 $V$ 上一个算子 $A$ 对两个不同的标准正交基 $\{|v_i\rangle\}$ 和 $\{|w_i\rangle\}$ 的矩阵表示,则
\[
A'_{ij}=\langle v_i|A|v_j\rangle,\quad
A''_{ij}=\langle w_i|A|w_j\rangle
\]
刻画 $A'$ 和 $A''$ 之间的关系。

\textbf{解:}
设基变换矩阵 $U$ 的元素为
\[
U_{ij}=\langle v_i|w_j\rangle
\]
则 $U$ 为酉矩阵,并有
\[
|w_j\rangle=\sum_i |v_i\rangle U_{ij}
\]
代入可得
\[
A'' = U^{\dagger} A' U
\]
因此两者相似(酉相似),对应同一线性算子的不同基表示。

\subsubsection*{练习2.26}
\textbf{题目:}令 $|\psi\rangle=(|0\rangle+|1\rangle)/\sqrt{2}$,以 $|0\rangle,|1\rangle$ 的张量积形式,并采用 Kronecker 积,具体写出 $|\psi\rangle^{\otimes 2}$ 和 $|\psi\rangle^{\otimes 3}$。

\textbf{解:}
\[
|\psi\rangle=\frac{1}{\sqrt{2}}\begin{pmatrix}1\\1\end{pmatrix}
\]
因此
\[
|\psi\rangle^{\otimes 2}=|\psi\rangle\otimes|\psi\rangle
=\frac{1}{2}\begin{pmatrix}1\\1\\1\\1\end{pmatrix}
=\frac{1}{2}(|00\rangle+|01\rangle+|10\rangle+|11\rangle)
\]
其中列向量顺序为 $|00\rangle,|01\rangle,|10\rangle,|11\rangle$。

同理
\[
|\psi\rangle^{\otimes 3}=|\psi\rangle\otimes|\psi\rangle\otimes|\psi\rangle
=\frac{1}{2\sqrt{2}}\begin{pmatrix}1\\1\\1\\1\\1\\1\\1\\1\end{pmatrix}
\]
即
\[
|\psi\rangle^{\otimes 3}=\frac{1}{2\sqrt{2}}\sum_{b\in\{0,1\}^3}|b\rangle
\]
列向量顺序为 $|000\rangle,|001\rangle,|010\rangle,|011\rangle,|100\rangle,|101\rangle,|110\rangle,|111\rangle$。

\subsection{2}

\subsubsection*{请利用布洛赫球表示以下量子态}

布洛赫球表示中,单量子比特纯态可写为
\[
|\psi\rangle=\cos\frac{\theta}{2}|0\rangle+e^{i\phi}\sin\frac{\theta}{2}|1\rangle
\]
其中 $\theta\in[0,\pi]$ 为极角,$\phi\in[0,2\pi)$ 为方位角。

\paragraph{(1)} \textbf{题目:}利用布洛赫球表示量子态 $|\psi\rangle=\frac{|0\rangle+|1\rangle}{\sqrt{2}}$。

\textbf{解:}$|\psi\rangle=\frac{|0\rangle+|1\rangle}{\sqrt{2}}=|+\rangle$

比较系数:
\[
\cos\frac{\theta}{2}=\frac{1}{\sqrt{2}},\quad
e^{i\phi}\sin\frac{\theta}{2}=\frac{1}{\sqrt{2}}
\]
得到 $\theta=\pi/2$,$\phi=0$。

布洛赫球坐标为 $(\theta,\phi)=(\pi/2,0)$,即赤道上 $+x$ 方向。

\paragraph{(2)} \textbf{题目:}利用布洛赫球表示量子态 $|\psi\rangle=\frac{|0\rangle-|1\rangle}{\sqrt{2}}$。

\textbf{解:}$|\psi\rangle=\frac{|0\rangle-|1\rangle}{\sqrt{2}}=|-\rangle$

比较系数:
\[
\cos\frac{\theta}{2}=\frac{1}{\sqrt{2}},\quad
e^{i\phi}\sin\frac{\theta}{2}=-\frac{1}{\sqrt{2}}=e^{i\pi}\frac{1}{\sqrt{2}}
\]
得到 $\theta=\pi/2$,$\phi=\pi$。

布洛赫球坐标为 $(\theta,\phi)=(\pi/2,\pi)$,即赤道上 $-x$ 方向。

\paragraph{(3)} \textbf{题目:}利用布洛赫球表示量子态 $|\psi\rangle=\frac{|0\rangle+i|1\rangle}{\sqrt{2}}$。

\textbf{解:}比较系数:
\[
\cos\frac{\theta}{2}=\frac{1}{\sqrt{2}},\quad
e^{i\phi}\sin\frac{\theta}{2}=\frac{i}{\sqrt{2}}=e^{i\pi/2}\frac{1}{\sqrt{2}}
\]
得到 $\theta=\pi/2$,$\phi=\pi/2$。

布洛赫球坐标为 $(\theta,\phi)=(\pi/2,\pi/2)$,即赤道上 $+y$ 方向。

\paragraph{(4)} \textbf{题目:}利用布洛赫球表示量子态 $|\psi\rangle=\frac{|0\rangle-i|1\rangle}{\sqrt{2}}$。

\textbf{解:}比较系数:
\[
\cos\frac{\theta}{2}=\frac{1}{\sqrt{2}},\quad
e^{i\phi}\sin\frac{\theta}{2}=-\frac{i}{\sqrt{2}}=e^{i3\pi/2}\frac{1}{\sqrt{2}}
\]
得到 $\theta=\pi/2$,$\phi=3\pi/2$。

布洛赫球坐标为 $(\theta,\phi)=(\pi/2,3\pi/2)$,即赤道上 $-y$ 方向。

\subsection{3}

\subsubsection*{施特恩-格拉赫实验与量子测量}

\paragraph{题目1} 用 $\{|0\rangle,|1\rangle\}$ 基测量量子态 $|+\rangle$,会得到什么结果?其中 $|0\rangle$ 发生的概率是多少?

\textbf{解:}$|+\rangle=\frac{|0\rangle+|1\rangle}{\sqrt{2}}$

在 $\{|0\rangle,|1\rangle\}$ 基下测量,测量结果为 $|0\rangle$ 或 $|1\rangle$。
\[
P(|0\rangle)=|\langle 0|+\rangle|^2=\left|\frac{1}{\sqrt{2}}\right|^2=\frac{1}{2}
\]
\[
P(|1\rangle)=|\langle 1|+\rangle|^2=\left|\frac{1}{\sqrt{2}}\right|^2=\frac{1}{2}
\]
因此测量结果为 $|0\rangle$ 或 $|1\rangle$,各以 $1/2$ 的概率出现。

\paragraph{题目2} 在施特恩-格拉赫实验中,对由高温炉产生的银原子施加Z方向的不均匀磁场,对这个银原子进行测量,测量结果是自旋向上的概率是 $p_1=a$,结果是自旋向下的概率是 $p_2=b$。

\begin{enumerate}
\item[(1)] 请写出高温炉射出的银原子的自旋状态 $|\varphi\rangle$。
\item[(2)] 在施特恩-格拉赫实验中,沿 Z 方向施加磁场,对应的可观测算子是泡利矩阵 Z,请计算自旋状态为 $|\varphi\rangle$ 系统的可观测算子 Z 的测量不确定度(标准偏差)。
\end{enumerate}

\textbf{解:}

\textbf{(1)} 由题意,测量 Z 方向自旋,向上($|0\rangle$)的概率为 $a$,向下($|1\rangle$)的概率为 $b$,且 $a+b=1$。

因此初态为
\[
|\varphi\rangle=\sqrt{a}|0\rangle+\sqrt{b}|1\rangle
\]
(取实数系数,全局相位任意)。

\textbf{(2)} 泡利矩阵 $Z=|0\rangle\langle 0|-|1\rangle\langle 1|$,特征值为 $\pm 1$。

期望值:
\[
\langle Z\rangle=\langle\varphi|Z|\varphi\rangle
=a\cdot(+1)+b\cdot(-1)=a-b
\]
期望的平方:
\[
\langle Z^2\rangle=\langle\varphi|Z^2|\varphi\rangle
=\langle\varphi|I|\varphi\rangle=1
\]
方差:
\[
\Delta Z^2=\langle Z^2\rangle-\langle Z\rangle^2=1-(a-b)^2
\]
标准偏差(测量不确定度):
\[
\Delta Z=\sqrt{1-(a-b)^2}=\sqrt{1-(a-b)^2}=\sqrt{4ab}=2\sqrt{ab}
\]
其中用到了 $a+b=1$。

\paragraph{题目3} 如图所示施特恩-格拉赫实验,请填出每次测量坍缩到的量子态,并给出详细计算过程。图示为:初态 $|0\rangle$ 依次经过 Z 测量、X 测量、Z 测量、Z 测量的量子线路。

\begin{center}
\includegraphics[width=0.95\textwidth]{assets/作业/量子信息/3/3_pic.png}
\end{center}

\textbf{解:}量子线路的演化过程:

第一次 Z 测量后得到 $|0\rangle$(上方输出)。从 $|0\rangle$ 开始后续测量。

\textbf{步骤1:}X 测量

$X$ 的本征态为 $|+\rangle=\frac{|0\rangle+|1\rangle}{\sqrt{2}}$(本征值 $+1$)和 $|-\rangle=\frac{|0\rangle-|1\rangle}{\sqrt{2}}$(本征值 $-1$)。

将 $|0\rangle$ 投影到 X 基:
\[
|0\rangle=\frac{1}{\sqrt{2}}|+\rangle+\frac{1}{\sqrt{2}}|-\rangle
\]
测量后,以各 $1/2$ 概率坍缩到 $|+\rangle$(上方输出)或 $|-\rangle$(下方输出)。

\textbf{步骤2:}第一次 Z 测量

\textbf{(a) 若 X 测量得到 $|+\rangle$:}

将 $|+\rangle$ 投影到 Z 基:
\[
|+\rangle=\frac{|0\rangle+|1\rangle}{\sqrt{2}}
\]
测量后,以各 $1/2$ 概率坍缩到 $|0\rangle$(上方输出)或 $|1\rangle$(下方输出)。

\textbf{(b) 若 X 测量得到 $|-\rangle$:}

将 $|-\rangle$ 投影到 Z 基:
\[
|-\rangle=\frac{|0\rangle-|1\rangle}{\sqrt{2}}
\]
测量后,以各 $1/2$ 概率坍缩到 $|0\rangle$(上方输出)或 $|1\rangle$(下方输出)。

\textbf{步骤3:}第二次 Z 测量

无论前面路径如何,经过第一次 Z 测量后得到的是 $Z$ 的本征态 $|0\rangle$ 或 $|1\rangle$。
对 $Z$ 的本征态再次进行 Z 测量,结果确定不变。

从图中可以看出,只有从 $|0\rangle$ 路径继续进行第二次 Z 测量,输出确定为 $|0\rangle$。

\textbf{图中6个空白处应填写的量子态:}

图中标记的空白 $|\quad\rangle$ 应填写为:
\begin{enumerate}
\item X 测量上方输出:$|+\rangle$
\item X 测量下方输出:$|-\rangle$
\item 第一次 Z 测量上方输出(从 $|+\rangle$ 或 $|-\rangle$ 路径):$|0\rangle$
\item 第一次 Z 测量下方输出(从 $|+\rangle$ 或 $|-\rangle$ 路径):$|1\rangle$
\item 第二次 Z 测量输出(从 $|0\rangle$ 路径):$|0\rangle$
\item (图中此处若有标记)从 $|1\rangle$ 路径的输出:$|1\rangle$(或不填写,因为测量本征态结果确定)
\end{enumerate}

\textbf{说明:}
\begin{itemize}
\item 对 $Z$ 的本征态进行 Z 测量,结果确定,不产生新的分支
\item $|0\rangle$ 经 Z 测量仍为 $|0\rangle$(本征值 $+1$)
\item $|1\rangle$ 经 Z 测量仍为 $|1\rangle$(本征值 $-1$)
\item 图中最后的测量主要追踪从 $|0\rangle$ 路径的演化
\end{itemize}

\subsection{4}

\subsubsection*{双自旋纠缠态}

\paragraph{题目} 如图为比特施特恩-格拉赫实验,蒸发炉能够产生一对一对的自旋,两个自旋具有相反的动量,图中示意地描绘了仅有的两种可能观测结果,即如果自旋1处于向上的状态那么自旋2处于向下的状态;如果自旋1处于向下的状态那么自旋2处于向上的状态。

\begin{center}
\includegraphics[width=0.7\textwidth]{assets/作业/量子信息/4/1_pic.png}
\end{center}

\begin{enumerate}
\item[(1)] 请写出该双自旋量子态 $|\varphi\rangle$ 的表达式。
\item[(2)] 测量的是沿 Z 方向的自旋,对应的可观测算子是泡利矩阵 $Z_1Z_2$,请计算状态 $|\varphi\rangle$ 系统可观测量 $Z_1Z_2$ 的平均值。
\item[(3)] 计算 $|\varphi\rangle$ 的密度算子,并判断它是否为纯态。
\item[(4)] 计算对第一个量子比特的约化密度算子(对第二量子比特取迹),并判断第一个量子比特是否为纯态。
\end{enumerate}

\textbf{解:}

\textbf{(1)} 根据题意,两个自旋反关联:自旋1向上则自旋2向下,反之亦然。

这意味着量子态只在 $|01\rangle$ 和 $|10\rangle$ 两个基态上有振幅,可以写为:
\[
|\varphi\rangle=\alpha|01\rangle+\beta|10\rangle
\]
其中 $|\alpha|^2+|\beta|^2=1$(归一化条件)。

从图中可以看出,两种测量结果(自旋1上/自旋2下 或 自旋1下/自旋2上)出现的概率相等,因此 $|\alpha|^2=|\beta|^2=\frac{1}{2}$,即 $|\alpha|=|\beta|=\frac{1}{\sqrt{2}}$。

最一般的形式为:
\[
|\varphi\rangle=\frac{1}{\sqrt{2}}(|01\rangle+e^{i\phi}|10\rangle)
\]
其中 $\phi$ 是相对相位。

\textbf{特殊情况:}如果该态是\textbf{单态}(singlet state,总自旋为0的态),则相对相位 $\phi=\pi$,得到:
\[
|\varphi\rangle_{\text{singlet}}=\frac{1}{\sqrt{2}}(|01\rangle-|10\rangle)
=\frac{1}{\sqrt{2}}(|0\rangle_1\otimes|1\rangle_2-|1\rangle_1\otimes|0\rangle_2)
\]

\textbf{注:}仅从"反关联"这一条件无法唯一确定相对相位。物理上,单态是最常见的自旋反关联态,因此通常默认为单态。后续计算我们采用单态形式。

\textbf{(2)} 可观测算子 $Z_1Z_2=Z\otimes Z$,其中
\[
Z=|0\rangle\langle 0|-|1\rangle\langle 1|
\]
因此
\[
Z_1Z_2=(|0\rangle\langle 0|-|1\rangle\langle 1|)\otimes(|0\rangle\langle 0|-|1\rangle\langle 1|)
\]
计算期望值:
\[
\langle Z_1Z_2\rangle=\langle\varphi|Z_1Z_2|\varphi\rangle
\]
注意到
\[
Z_1Z_2|01\rangle=(Z|0\rangle)\otimes(Z|1\rangle)=(+|0\rangle)\otimes(-|1\rangle)=-|01\rangle
\]
\[
Z_1Z_2|10\rangle=(Z|1\rangle)\otimes(Z|0\rangle)=(-|1\rangle)\otimes(+|0\rangle)=-|10\rangle
\]
因此
\[
Z_1Z_2|\varphi\rangle=\frac{-|01\rangle-(-|10\rangle)}{\sqrt{2}}
=\frac{-|01\rangle+|10\rangle}{\sqrt{2}}=-|\varphi\rangle
\]
所以
\[
\langle Z_1Z_2\rangle=\langle\varphi|(-|\varphi\rangle)=-1
\]

\textbf{(3)} 密度算子为
\[
\rho=|\varphi\rangle\langle\varphi|
=\frac{1}{2}(|01\rangle-|10\rangle)(\langle 01|-\langle 10|)
\]
\[
=\frac{1}{2}(|01\rangle\langle 01|-|01\rangle\langle 10|-|10\rangle\langle 01|+|10\rangle\langle 10|)
\]
计算 $\rho^2$:
\[
\rho^2=|\varphi\rangle\langle\varphi|\varphi\rangle\langle\varphi|
=|\varphi\rangle\langle\varphi|=\rho
\]
因为 $\mathrm{Tr}(\rho^2)=\mathrm{Tr}(\rho)=1$,且 $\rho^2=\rho$,所以 $|\varphi\rangle$ 是纯态。

\textbf{(4)} 对第二个量子比特取迹,得到第一个量子比特的约化密度算子:
\[
\rho_1=\mathrm{Tr}_2(\rho)=\sum_{i\in\{0,1\}}\langle i|_2\rho|i\rangle_2
\]
计算:
\[
\langle 0|_2\rho|0\rangle_2
=\frac{1}{2}\langle 0|_2(|01\rangle\langle 01|-|01\rangle\langle 10|-|10\rangle\langle 01|+|10\rangle\langle 10|)|0\rangle_2
\]
注意到 $\langle 0|_2|01\rangle=|0\rangle_1$,$\langle 0|_2|10\rangle=0$,所以
\[
\langle 0|_2\rho|0\rangle_2=\frac{1}{2}|0\rangle_1\langle 0|_1
\]
类似地
\[
\langle 1|_2\rho|1\rangle_2=\frac{1}{2}|1\rangle_1\langle 1|_1
\]
因此
\[
\rho_1=\frac{1}{2}(|0\rangle\langle 0|+|1\rangle\langle 1|)=\frac{I}{2}
\]
这是最大混合态。计算 $\mathrm{Tr}(\rho_1^2)$:
\[
\rho_1^2=\frac{I}{4},\quad
\mathrm{Tr}(\rho_1^2)=\frac{1}{2}<1
\]
因此第一个量子比特不是纯态,而是混合态。这表明纠缠态的子系统总是混合态。

\subsection{5}

\subsubsection*{Bell态与纠缠判断}

量子态 $|\psi\rangle_{AB}$ 是纠缠态当且仅当它不能写成张量积形式 $|\phi\rangle_A\otimes|\chi\rangle_B$。

\paragraph{题目1} 判断 $|\psi_1\rangle=\frac{1}{\sqrt{2}}(|00\rangle+|11\rangle)$ 是否为纠缠态,并说明理由。

\textbf{解:}假设 $|\psi_1\rangle=(a|0\rangle+b|1\rangle)\otimes(c|0\rangle+d|1\rangle)$,则
\[
|\psi_1\rangle=ac|00\rangle+ad|01\rangle+bc|10\rangle+bd|11\rangle
\]
比较系数:$ac=\frac{1}{\sqrt{2}}$,$ad=0$,$bc=0$,$bd=\frac{1}{\sqrt{2}}$。

由 $ad=0$ 得 $a=0$ 或 $d=0$;由 $bc=0$ 得 $b=0$ 或 $c=0$。
若 $a=0$,则 $ac=0\neq\frac{1}{\sqrt{2}}$,矛盾。
若 $d=0$,则 $bd=0\neq\frac{1}{\sqrt{2}}$,矛盾。

因此 $|\psi_1\rangle$ 不能分解为张量积,是纠缠态(Bell态 $|\Phi^+\rangle$)。

\paragraph{题目2} 判断 $|\psi_2\rangle=\frac{1}{2}(|00\rangle+|01\rangle+|10\rangle+|11\rangle)$ 是否为纠缠态,并说明理由。

\textbf{解:}注意到
\[
|\psi_2\rangle=\frac{1}{2}(|0\rangle+|1\rangle)\otimes(|0\rangle+|1\rangle)
=|+\rangle\otimes|+\rangle
\]
其中 $|+\rangle=\frac{|0\rangle+|1\rangle}{\sqrt{2}}$。

因此 $|\psi_2\rangle$ 可以分解为张量积,不是纠缠态(是可分离态)。

\paragraph{题目3} 判断 $|\psi_3\rangle=\frac{1}{\sqrt{3}}|00\rangle+\sqrt{\frac{2}{3}}|11\rangle$ 是否为纠缠态,并说明理由。

\textbf{解:}假设 $|\psi_3\rangle=(a|0\rangle+b|1\rangle)\otimes(c|0\rangle+d|1\rangle)$,则
\[
|\psi_3\rangle=ac|00\rangle+ad|01\rangle+bc|10\rangle+bd|11\rangle
\]
比较系数:$ac=\frac{1}{\sqrt{3}}$,$ad=0$,$bc=0$,$bd=\sqrt{\frac{2}{3}}$。

由 $ad=0$ 得 $a=0$ 或 $d=0$;由 $bc=0$ 得 $b=0$ 或 $c=0$。
若 $a=0$,则 $ac=0\neq\frac{1}{\sqrt{3}}$,矛盾。
若 $d=0$,则 $bd=0\neq\sqrt{\frac{2}{3}}$,矛盾。

因此 $|\psi_3\rangle$ 不能分解为张量积,是纠缠态。

\subsubsection*{Bell态测量}

已知四个贝尔态(Bell states)为:
\[
|\Phi^+\rangle=\frac{1}{\sqrt{2}}(|00\rangle+|11\rangle),\quad
|\Phi^-\rangle=\frac{1}{\sqrt{2}}(|00\rangle-|11\rangle)
\]
\[
|\Psi^+\rangle=\frac{1}{\sqrt{2}}(|01\rangle+|10\rangle),\quad
|\Psi^-\rangle=\frac{1}{\sqrt{2}}(|01\rangle-|10\rangle)
\]

若 Alice 和 Bob 共享一对处于 $|\Phi^+\rangle$ 的纠缠粒子,Alice 对她手中的量子比特在计算基 $\{|0\rangle,|1\rangle\}$ 下进行测量。

\paragraph{题目1} Alice 测量得到结果 $|0\rangle$,Bob 手中的量子比特立即处于什么状态?

\textbf{解:}初态为 $|\Phi^+\rangle=\frac{1}{\sqrt{2}}(|00\rangle+|11\rangle)$

Alice 在计算基下测量,投影算子为 $P_0=|0\rangle\langle 0|\otimes I$,$P_1=|1\rangle\langle 1|\otimes I$。

测量后态为
\[
P_0|\Phi^+\rangle=\frac{1}{\sqrt{2}}(|0\rangle\langle 0||0\rangle\otimes|0\rangle+|0\rangle\langle 0||1\rangle\otimes|1\rangle)
=\frac{1}{\sqrt{2}}|00\rangle
\]
归一化后
\[
|\psi'\rangle=\frac{|00\rangle}{\||00\rangle\|}=|00\rangle=|0\rangle_A\otimes|0\rangle_B
\]
因此 Bob 手中的量子比特处于 $|0\rangle$ 状态。

\paragraph{题目2} 若 Alice 测量得到 $|1\rangle$,Bob 的量子比特状态如何?

\textbf{解:}
\[
P_1|\Phi^+\rangle=\frac{1}{\sqrt{2}}|11\rangle
\]
归一化后
\[
|\psi'\rangle=|11\rangle=|1\rangle_A\otimes|1\rangle_B
\]
因此 Bob 手中的量子比特处于 $|1\rangle$ 状态。

\paragraph{题目3} 若 Alice 改为在基 $\{|+\rangle,|-\rangle\}$ 下测量,得到 $|+\rangle$,问 Bob 的量子比特状态是什么?

\textbf{解:}先将 $|\Phi^+\rangle$ 用 Hadamard 基表示。注意到
\[
|0\rangle=\frac{|+\rangle+|-\rangle}{\sqrt{2}},\quad
|1\rangle=\frac{|+\rangle-|-\rangle}{\sqrt{2}}
\]
代入:
\[
|\Phi^+\rangle=\frac{1}{\sqrt{2}}(|00\rangle+|11\rangle)
=\frac{1}{\sqrt{2}}\left(\frac{|+\rangle+|-\rangle}{\sqrt{2}}\otimes\frac{|+\rangle+|-\rangle}{\sqrt{2}}+\frac{|+\rangle-|-\rangle}{\sqrt{2}}\otimes\frac{|+\rangle-|-\rangle}{\sqrt{2}}\right)
\]
\[
=\frac{1}{2\sqrt{2}}\Big[(|+\rangle+|-\rangle)\otimes(|+\rangle+|-\rangle)+(|+\rangle-|-\rangle)\otimes(|+\rangle-|-\rangle)\Big]
\]
\[
=\frac{1}{2\sqrt{2}}\Big[|++\rangle+|+-\rangle+|-+\rangle+|--\rangle+|++\rangle-|+-\rangle-|-+\rangle+|--\rangle\Big]
\]
\[
=\frac{1}{2\sqrt{2}}\cdot 2(|++\rangle+|--\rangle)=\frac{1}{\sqrt{2}}(|++\rangle+|--\rangle)
\]

若 Alice 测量得到 $|+\rangle$,投影后态为
\[
(|+\rangle\langle +|\otimes I)|\Phi^+\rangle\propto|+\rangle\otimes|+\rangle
\]
因此 Bob 的量子比特处于 $|+\rangle=\frac{|0\rangle+|1\rangle}{\sqrt{2}}$ 状态。

\subsubsection*{超密编码}

\paragraph{题目1} 写出超密编码的详细步骤,包括:初始共享的纠缠态、Alice 根据她要发送的两位经典信息(00, 01, 10, 11)如何操作她手中的量子比特(列出对应的酉变换)、Alice 将操作后的量子比特发送给 Bob 后,Bob 如何进行联合测量以解码信息。

\textbf{解:超密编码协议详细步骤:}

\textbf{初始态:}Alice 和 Bob 共享 Bell 态
\[
|\Phi^+\rangle=\frac{1}{\sqrt{2}}(|00\rangle+|11\rangle)
\]
其中 Alice 持有第一个量子比特,Bob 持有第二个。

\textbf{编码:}Alice 根据要发送的两位经典信息,对她的量子比特进行如下酉操作:

\begin{itemize}
\item 发送"00":不做操作($I$),态保持为 $|\Phi^+\rangle=\frac{1}{\sqrt{2}}(|00\rangle+|11\rangle)$

\item 发送"01":施加 $Z$ 门,态变为 $|\Phi^-\rangle=\frac{1}{\sqrt{2}}(|00\rangle-|11\rangle)$

\textbf{详细计算:}
\begin{align*}
(Z\otimes I)|\Phi^+\rangle &= (Z\otimes I)\frac{1}{\sqrt{2}}(|00\rangle+|11\rangle) \\
&= \frac{1}{\sqrt{2}}[(Z\otimes I)|00\rangle + (Z\otimes I)|11\rangle] \\
&= \frac{1}{\sqrt{2}}[Z|0\rangle\otimes I|0\rangle + Z|1\rangle\otimes I|1\rangle] \\
&= \frac{1}{\sqrt{2}}[(+1)|0\rangle\otimes|0\rangle + (-1)|1\rangle\otimes|1\rangle] \\
&= \frac{1}{\sqrt{2}}(|00\rangle - |11\rangle) = |\Phi^-\rangle
\end{align*}
其中用到了 $Z|0\rangle=|0\rangle$(本征值$+1$),$Z|1\rangle=-|1\rangle$(本征值$-1$)。

\item 发送"10":施加 $X$ 门,态变为 $|\Psi^+\rangle=\frac{1}{\sqrt{2}}(|01\rangle+|10\rangle)$

\textbf{详细计算:}
\begin{align*}
(X\otimes I)|\Phi^+\rangle &= (X\otimes I)\frac{1}{\sqrt{2}}(|00\rangle+|11\rangle) \\
&= \frac{1}{\sqrt{2}}[(X\otimes I)|00\rangle + (X\otimes I)|11\rangle] \\
&= \frac{1}{\sqrt{2}}[X|0\rangle\otimes I|0\rangle + X|1\rangle\otimes I|1\rangle] \\
&= \frac{1}{\sqrt{2}}[|1\rangle\otimes|0\rangle + |0\rangle\otimes|1\rangle] \\
&= \frac{1}{\sqrt{2}}(|10\rangle + |01\rangle) = |\Psi^+\rangle
\end{align*}
其中用到了 $X|0\rangle=|1\rangle$,$X|1\rangle=|0\rangle$。

\item 发送"11":施加 $XZ=iY$ 门,态变为 $|\Psi^-\rangle=\frac{1}{\sqrt{2}}(|01\rangle-|10\rangle)$

\textbf{详细计算:}先施加 $Z$ 再施加 $X$:
\begin{align*}
(X\otimes I)(Z\otimes I)|\Phi^+\rangle &= (X\otimes I)|\Phi^-\rangle \\
&= (X\otimes I)\frac{1}{\sqrt{2}}(|00\rangle-|11\rangle) \\
&= \frac{1}{\sqrt{2}}[X|0\rangle\otimes|0\rangle - X|1\rangle\otimes|1\rangle] \\
&= \frac{1}{\sqrt{2}}[|1\rangle\otimes|0\rangle - |0\rangle\otimes|1\rangle] \\
&= \frac{1}{\sqrt{2}}(|10\rangle - |01\rangle) = -|\Psi^-\rangle
\end{align*}
相差全局相位$-1$,物理上等价于 $|\Psi^-\rangle$。
\end{itemize}

\textbf{传输:}Alice 将她的量子比特发送给 Bob。

\textbf{解码:}Bob 持有两个量子比特后,在 Bell 基 $\{|\Phi^+\rangle,|\Phi^-\rangle,|\Psi^+\rangle,|\Psi^-\rangle\}$ 下进行联合测量(Bell 测量)。测量结果直接对应 Alice 发送的经典信息:
\begin{itemize}
\item 测得 $|\Phi^+\rangle$ $\Rightarrow$ 信息为"00"
\item 测得 $|\Phi^-\rangle$ $\Rightarrow$ 信息为"01"
\item 测得 $|\Psi^+\rangle$ $\Rightarrow$ 信息为"10"
\item 测得 $|\Psi^-\rangle$ $\Rightarrow$ 信息为"11"
\end{itemize}

\paragraph{题目2} 验证当 Alice 要发送"11"时,她对手中的量子比特施加 $i\sigma_y$(即 $\begin{pmatrix}0&-1\\1&0\end{pmatrix}$)后,两量子比特的整体状态变为 $|\Psi^-\rangle$。

\textbf{解:}初态为 $|\Phi^+\rangle=\frac{1}{\sqrt{2}}(|00\rangle+|11\rangle)$

Alice 对第一个量子比特施加 $i\sigma_y=\begin{pmatrix}0&-1\\1&0\end{pmatrix}$。注意到
\[
i\sigma_y|0\rangle=\begin{pmatrix}0&-1\\1&0\end{pmatrix}\begin{pmatrix}1\\0\end{pmatrix}
=\begin{pmatrix}0\\1\end{pmatrix}=|1\rangle
\]
\[
i\sigma_y|1\rangle=\begin{pmatrix}0&-1\\1&0\end{pmatrix}\begin{pmatrix}0\\1\end{pmatrix}
=\begin{pmatrix}-1\\0\end{pmatrix}=-|0\rangle
\]
因此
\[
(i\sigma_y\otimes I)|\Phi^+\rangle
=\frac{1}{\sqrt{2}}(i\sigma_y|0\rangle\otimes|0\rangle+i\sigma_y|1\rangle\otimes|1\rangle)
\]
\[
=\frac{1}{\sqrt{2}}(|1\rangle\otimes|0\rangle-|0\rangle\otimes|1\rangle)
=\frac{1}{\sqrt{2}}(|10\rangle-|01\rangle)
\]
\[
=-\frac{1}{\sqrt{2}}(|01\rangle-|10\rangle)=-|\Psi^-\rangle
\]
相差一个全局相位 $-1$,物理上等价于 $|\Psi^-\rangle$。验证完毕。

\subsection{6}

\paragraph{题目1} (线路恒等式)能以熟知的恒等式来简化量子线路并常有用,证明如下三个恒等式:
\begin{itemize}
\item[(1)] $HXH = Z$
\item[(2)] $HYH = -Y$
\item[(3)] $HZH = X$
\end{itemize}
其中 H 为 Hadamard 门,X,Y,Z 分别为 Pauli X,Y,Z 门。

\textbf{解:}

我们先回顾各个门的矩阵表示:
\[
H = \frac{1}{\sqrt{2}}\begin{pmatrix}1&1\\1&-1\end{pmatrix},\quad
X = \begin{pmatrix}0&1\\1&0\end{pmatrix},\quad
Y = \begin{pmatrix}0&-i\\i&0\end{pmatrix},\quad
Z = \begin{pmatrix}1&0\\0&-1\end{pmatrix}
\]

\textbf{(1) 证明 $HXH = Z$:}

首先计算 $XH$:
\[
XH = \begin{pmatrix}0&1\\1&0\end{pmatrix}
\frac{1}{\sqrt{2}}\begin{pmatrix}1&1\\1&-1\end{pmatrix}
= \frac{1}{\sqrt{2}}\begin{pmatrix}1&-1\\1&1\end{pmatrix}
\]

再计算 $HXH$:
\[
HXH = \frac{1}{\sqrt{2}}\begin{pmatrix}1&1\\1&-1\end{pmatrix}
\frac{1}{\sqrt{2}}\begin{pmatrix}1&-1\\1&1\end{pmatrix}
= \frac{1}{2}\begin{pmatrix}2&0\\0&-2\end{pmatrix}
= \begin{pmatrix}1&0\\0&-1\end{pmatrix} = Z
\]

\textbf{(2) 证明 $HYH = -Y$:}

首先计算 $YH$:
\[
YH = \begin{pmatrix}0&-i\\i&0\end{pmatrix}
\frac{1}{\sqrt{2}}\begin{pmatrix}1&1\\1&-1\end{pmatrix}
= \frac{1}{\sqrt{2}}\begin{pmatrix}-i&i\\i&i\end{pmatrix}
\]

再计算 $HYH$:
\[
HYH = \frac{1}{\sqrt{2}}\begin{pmatrix}1&1\\1&-1\end{pmatrix}
\frac{1}{\sqrt{2}}\begin{pmatrix}-i&i\\i&i\end{pmatrix}
= \frac{1}{2}\begin{pmatrix}0&2i\\-2i&0\end{pmatrix}
= \begin{pmatrix}0&i\\-i&0\end{pmatrix} = -Y
\]

\textbf{(3) 证明 $HZH = X$:}

首先计算 $ZH$:
\[
ZH = \begin{pmatrix}1&0\\0&-1\end{pmatrix}
\frac{1}{\sqrt{2}}\begin{pmatrix}1&1\\1&-1\end{pmatrix}
= \frac{1}{\sqrt{2}}\begin{pmatrix}1&1\\-1&1\end{pmatrix}
\]

再计算 $HZH$:
\[
HZH = \frac{1}{\sqrt{2}}\begin{pmatrix}1&1\\1&-1\end{pmatrix}
\frac{1}{\sqrt{2}}\begin{pmatrix}1&1\\-1&1\end{pmatrix}
= \frac{1}{2}\begin{pmatrix}0&2\\2&0\end{pmatrix}
= \begin{pmatrix}0&1\\1&0\end{pmatrix} = X
\]

三个恒等式均得证。

\paragraph{题目2} (多量子比特门的矩阵表示)如下图线路(a)的 4×4 酉矩阵是什么?线路(b)的 4×4 酉矩阵是什么?

\begin{figure}[h]
\centering
\includegraphics[width=0.4\textwidth]{assets/作业/量子信息/6/2_pic.png}
\end{figure}

\textbf{解:}

\textbf{线路(a):}线路中在 $x_2$ 上施加 Hadamard 门,$x_1$ 不变。对应的 4×4 矩阵为:
\[
U_a = I \otimes H = \begin{pmatrix}1&0\\0&1\end{pmatrix} \otimes
\frac{1}{\sqrt{2}}\begin{pmatrix}1&1\\1&-1\end{pmatrix}
= \frac{1}{\sqrt{2}}\begin{pmatrix}
1&1&0&0\\
1&-1&0&0\\
0&0&1&1\\
0&0&1&-1
\end{pmatrix}
\]

\textbf{线路(b):}线路中在 $x_1$ 上施加 Hadamard 门,$x_2$ 不变。对应的 4×4 矩阵为:
\[
U_b = H \otimes I = \frac{1}{\sqrt{2}}\begin{pmatrix}1&1\\1&-1\end{pmatrix}
\otimes \begin{pmatrix}1&0\\0&1\end{pmatrix}
= \frac{1}{\sqrt{2}}\begin{pmatrix}
1&0&1&0\\
0&1&0&1\\
1&0&-1&0\\
0&1&0&-1
\end{pmatrix}
\]

\paragraph{题目3} 分析以下两个量子线路各实现什么功能,并给出分析验证过程。

\textbf{解:}

\textbf{基本门介绍:}

\textbf{1. CNOT 门:}

CNOT(Controlled-NOT)门是一个双量子比特门,也称为受控非门。它有一个控制位(control qubit)和一个目标位(target qubit):
\begin{itemize}
\item 当控制位为 $|0\rangle$ 时,目标位保持不变
\item 当控制位为 $|1\rangle$ 时,对目标位施加 X 门(比特翻转)
\end{itemize}

\textbf{线路图表示:}在量子线路图中,CNOT 门用以下符号表示:
\begin{itemize}
\item \textbf{实心圆点($\bullet$):}标记控制位所在的量子比特线
\item \textbf{带圈加号($\oplus$):}标记目标位所在的量子比特线
\item \textbf{竖线:}连接控制位和目标位
\end{itemize}

CNOT 门的作用规则:
\[
\text{CNOT}|00\rangle = |00\rangle, \quad
\text{CNOT}|01\rangle = |01\rangle, \quad
\text{CNOT}|10\rangle = |11\rangle, \quad
\text{CNOT}|11\rangle = |10\rangle
\]

\textbf{2. SWAP 门:}

SWAP 门用于交换两个量子比特的状态,在线路图中用×符号表示:
\[
\text{SWAP}|ab\rangle = |ba\rangle
\]
例如:$\text{SWAP}|01\rangle = |10\rangle$,$\text{SWAP}|10\rangle = |01\rangle$

\textbf{3. 测量:}

仪表盘符号表示测量操作,对量子比特进行投影测量,得到经典比特结果(0或1),并使量子态塌缩到相应的本征态。

---

\textbf{线路(a):}该线路实现 Bell 态制备功能。

\begin{figure}[h]
\centering
\includegraphics[width=0.3\textwidth]{assets/作业/量子信息/6/3a_pic.png}
\end{figure}

\textbf{分析过程:}

这个线路由 Hadamard 门和 CNOT 门组成,可以从不同的输入态制备出不同的 Bell 态。我们分析所有可能的输入:

\textbf{情况1:输入 $|00\rangle$}
\begin{enumerate}
\item 第一个量子比特经过 Hadamard 门:
\[
(H\otimes I)|00\rangle = |+\rangle\otimes|0\rangle = \frac{1}{\sqrt{2}}(|0\rangle+|1\rangle)\otimes|0\rangle
= \frac{1}{\sqrt{2}}(|00\rangle+|10\rangle)
\]
\item 经过 CNOT 门(控制位为第一个量子比特,目标位为第二个):
\[
\text{CNOT}\cdot\frac{1}{\sqrt{2}}(|00\rangle+|10\rangle)
= \frac{1}{\sqrt{2}}(|00\rangle+|11\rangle) = |\Phi^+\rangle
\]
\end{enumerate}

\textbf{情况2:输入 $|01\rangle$}
\begin{enumerate}
\item 经过 Hadamard 门:
\[
(H\otimes I)|01\rangle = \frac{1}{\sqrt{2}}(|0\rangle+|1\rangle)\otimes|1\rangle
= \frac{1}{\sqrt{2}}(|01\rangle+|11\rangle)
\]
\item 经过 CNOT 门:
\[
\text{CNOT}\cdot\frac{1}{\sqrt{2}}(|01\rangle+|11\rangle)
= \frac{1}{\sqrt{2}}(|01\rangle+|10\rangle) = |\Psi^+\rangle
\]
\end{enumerate}

\textbf{情况3:输入 $|10\rangle$}
\begin{enumerate}
\item 经过 Hadamard 门:
\[
(H\otimes I)|10\rangle = \frac{1}{\sqrt{2}}(|0\rangle-|1\rangle)\otimes|0\rangle
= \frac{1}{\sqrt{2}}(|00\rangle-|10\rangle)
\]
\item 经过 CNOT 门:
\[
\text{CNOT}\cdot\frac{1}{\sqrt{2}}(|00\rangle-|10\rangle)
= \frac{1}{\sqrt{2}}(|00\rangle-|11\rangle) = |\Phi^-\rangle
\]
\end{enumerate}

\textbf{情况4:输入 $|11\rangle$}
\begin{enumerate}
\item 经过 Hadamard 门:
\[
(H\otimes I)|11\rangle = \frac{1}{\sqrt{2}}(|0\rangle-|1\rangle)\otimes|1\rangle
= \frac{1}{\sqrt{2}}(|01\rangle-|11\rangle)
\]
\item 经过 CNOT 门:
\[
\text{CNOT}\cdot\frac{1}{\sqrt{2}}(|01\rangle-|11\rangle)
= \frac{1}{\sqrt{2}}(|01\rangle-|10\rangle) = |\Psi^-\rangle
\]
\end{enumerate}

\textbf{总结:}线路(a)从不同的计算基态输入可以制备出所有四个 Bell 态:
\begin{center}
\begin{tabular}{c|c}
\hline
输入态 & 输出 Bell 态 \\
\hline
$|00\rangle$ & $|\Phi^+\rangle = \frac{1}{\sqrt{2}}(|00\rangle+|11\rangle)$ \\
$|01\rangle$ & $|\Psi^+\rangle = \frac{1}{\sqrt{2}}(|01\rangle+|10\rangle)$ \\
$|10\rangle$ & $|\Phi^-\rangle = \frac{1}{\sqrt{2}}(|00\rangle-|11\rangle)$ \\
$|11\rangle$ & $|\Psi^-\rangle = \frac{1}{\sqrt{2}}(|01\rangle-|10\rangle)$ \\
\hline
\end{tabular}
\end{center}

因此,线路(a)是一个通用的 Bell 态制备线路,通过选择不同的输入计算基态,可以制备出任意一个 Bell 态。

---

\textbf{线路(b):}该线路实现受控Bell态制备和测量功能。

\begin{figure}[h]
\centering
\includegraphics[width=0.5\textwidth]{assets/作业/量子信息/6/3b_pic.png}
\end{figure}

\textbf{分析过程:}

仔细观察线路(b),从左到右依次有三个操作:

\textbf{步骤1:}$x_2$ 和第三个量子比特 $|0\rangle$ 之间的 CNOT 门,$x_2$ 为控制位,第三个量子比特为目标位

\textbf{步骤2:}$x_1$ 和 $x_2$ 之间的受控 SWAP 门(Controlled-SWAP,也称 Fredkin 门),第三个量子比特 $|0\rangle$ 为控制位

受控 SWAP 门的作用是:
\begin{itemize}
\item 当控制位为 $|0\rangle$ 时,$x_1$ 和 $x_2$ 保持不变
\item 当控制位为 $|1\rangle$ 时,交换 $x_1$ 和 $x_2$ 的状态
\end{itemize}

\textbf{步骤3:}$x_1$ 和 $x_2$ 之间的 CNOT 门,$x_2$ 为控制位,$x_1$ 为目标位

\textbf{完整态演化分析:}

假设初态为:$|\psi\rangle = |x_1, x_2, 0\rangle$,其中 $x_1, x_2 \in \{0, 1\}$

\textbf{步骤1后:}$x_2$ 和 $0$ 的 CNOT 操作
\[
|x_1, x_2, 0\rangle \xrightarrow{\text{CNOT}_{x_2 \to 0}} |x_1, x_2, x_2\rangle
\]

\textbf{步骤2后:}受控 SWAP 操作(控制位是第三个量子比特)
\begin{itemize}
\item 如果 $x_2 = 0$,第三个量子比特为 $|0\rangle$,不交换:
\[
|x_1, 0, 0\rangle \xrightarrow{\text{C-SWAP}} |x_1, 0, 0\rangle
\]
\item 如果 $x_2 = 1$,第三个量子比特为 $|1\rangle$,交换 $x_1$ 和 $x_2$:
\[
|x_1, 1, 1\rangle \xrightarrow{\text{C-SWAP}} |1, x_1, 1\rangle
\]
\end{itemize}

\textbf{步骤3后:}$x_1$ 和 $x_2$ 的 CNOT 操作($x_2$ 为控制位)

\textbf{具体计算四种输入情况:}

\textbf{输入 $|00\rangle$($x_1=0, x_2=0$):}
\begin{align*}
|0,0,0\rangle &\xrightarrow{\text{步骤1}} |0,0,0\rangle \\
&\xrightarrow{\text{步骤2}} |0,0,0\rangle \quad (\text{控制位为0,不交换})\\
&\xrightarrow{\text{步骤3}} |0,0,0\rangle \quad (\text{$x_2=0$,CNOT不作用})
\end{align*}
测量 $x_1, x_2$ 得到 $(0,0)$

\textbf{输入 $|01\rangle$($x_1=0, x_2=1$):}
\begin{align*}
|0,1,0\rangle &\xrightarrow{\text{步骤1}} |0,1,1\rangle \\
&\xrightarrow{\text{步骤2}} |1,0,1\rangle \quad (\text{控制位为1,交换})\\
&\xrightarrow{\text{步骤3}} |1,0,1\rangle \quad (\text{$x_2=0$,CNOT不作用})
\end{align*}
测量 $x_1, x_2$ 得到 $(1,0)$

\textbf{输入 $|10\rangle$($x_1=1, x_2=0$):}
\begin{align*}
|1,0,0\rangle &\xrightarrow{\text{步骤1}} |1,0,0\rangle \\
&\xrightarrow{\text{步骤2}} |1,0,0\rangle \quad (\text{控制位为0,不交换})\\
&\xrightarrow{\text{步骤3}} |1,0,0\rangle \quad (\text{$x_2=0$,CNOT不作用})
\end{align*}
测量 $x_1, x_2$ 得到 $(1,0)$

\textbf{输入 $|11\rangle$($x_1=1, x_2=1$):}
\begin{align*}
|1,1,0\rangle &\xrightarrow{\text{步骤1}} |1,1,1\rangle \\
&\xrightarrow{\text{步骤2}} |1,1,1\rangle \quad (\text{控制位为1,交换后还是$|1,1,1\rangle$})\\
&\xrightarrow{\text{步骤3}} |0,1,1\rangle \quad (\text{$x_2=1$,CNOT翻转$x_1$})
\end{align*}
测量 $x_1, x_2$ 得到 $(0,1)$

\textbf{功能总结:}

线路(b)的映射关系为:
\begin{center}
\begin{tabular}{c|c}
\hline
输入 $(x_1, x_2)$ & 输出测量结果 $(m_1, m_2)$ \\
\hline
$(0, 0)$ & $(0, 0)$ \\
$(0, 1)$ & $(1, 0)$ \\
$(1, 0)$ & $(1, 0)$ \\
$(1, 1)$ & $(0, 1)$ \\
\hline
\end{tabular}
\end{center}

注意到 $(0,1)$ 和 $(1,0)$ 都映射到 $(1,0)$。

\textbf{扩展分析:Bell 态输入的情况}

如果考虑 Bell 态输入(线路(a)的输出作为线路(b)的输入),这个线路可以部分区分 Bell 态。实际上,这个线路实现的是一个更复杂的量子态转换和测量功能,常用于量子通信协议中。下面详细分析:

\textbf{Bell 态回顾:}

线路(a)可以产生四个 Bell 态:
\begin{align*}
|\Phi^+\rangle &= \frac{1}{\sqrt{2}}(|00\rangle+|11\rangle) \\
|\Psi^+\rangle &= \frac{1}{\sqrt{2}}(|01\rangle+|10\rangle) \\
|\Phi^-\rangle &= \frac{1}{\sqrt{2}}(|00\rangle-|11\rangle) \\
|\Psi^-\rangle &= \frac{1}{\sqrt{2}}(|01\rangle-|10\rangle)
\end{align*}

现在将这些 Bell 态作为线路(b)的前两个量子比特输入(第三个量子比特仍为辅助 $|0\rangle$)。

\textbf{情况1:输入 $|\Phi^+\rangle = \frac{1}{\sqrt{2}}(|00\rangle+|11\rangle)$}

初始态:$|\Phi^+\rangle \otimes |0\rangle = \frac{1}{\sqrt{2}}(|000\rangle+|110\rangle)$

\textbf{步骤1:}$x_2$ 和辅助位的 CNOT($x_2$ 为控制位)
\begin{align*}
\frac{1}{\sqrt{2}}(|000\rangle+|110\rangle) &\xrightarrow{\text{CNOT}_{x_2 \to \text{aux}}} \frac{1}{\sqrt{2}}(|000\rangle+|111\rangle)
\end{align*}
说明:$|000\rangle$ 中 $x_2=0$ 不翻转辅助位;$|110\rangle$ 中 $x_2=1$ 翻转辅助位 $0 \to 1$

\textbf{步骤2:}受控 SWAP(辅助位控制,交换 $x_1$ 和 $x_2$)
\begin{align*}
\frac{1}{\sqrt{2}}(|000\rangle+|111\rangle) &\xrightarrow{\text{C-SWAP}} \frac{1}{\sqrt{2}}(|000\rangle+|111\rangle)
\end{align*}
说明:$|000\rangle$ 辅助位为 0 不交换;$|111\rangle$ 辅助位为 1 交换 $x_1$ 和 $x_2$,但 $|11\rangle \to |11\rangle$ 不变

\textbf{步骤3:}$x_2$ 和 $x_1$ 的 CNOT($x_2$ 为控制位,$x_1$ 为目标位)
\begin{align*}
\frac{1}{\sqrt{2}}(|000\rangle+|111\rangle) &\xrightarrow{\text{CNOT}_{x_2 \to x_1}} \frac{1}{\sqrt{2}}(|000\rangle+|011\rangle)
\end{align*}
说明:$|000\rangle$ 中 $x_2=0$ 不作用;$|111\rangle$ 中 $x_2=1$ 翻转 $x_1$: $1 \oplus 1 = 0$

\textbf{测量结果:}测量 $x_1, x_2$ 得到 $(0,0)$ 或 $(0,1)$,各占 50\% 概率

---

\textbf{情况2:输入 $|\Psi^+\rangle = \frac{1}{\sqrt{2}}(|01\rangle+|10\rangle)$}

初始态:$|\Psi^+\rangle \otimes |0\rangle = \frac{1}{\sqrt{2}}(|010\rangle+|100\rangle)$

\textbf{步骤1:}CNOT($x_2 \to$ 辅助)
\begin{align*}
\frac{1}{\sqrt{2}}(|010\rangle+|100\rangle) &\xrightarrow{\text{CNOT}} \frac{1}{\sqrt{2}}(|011\rangle+|100\rangle)
\end{align*}
说明:$|010\rangle$ 中 $x_2=1$ 翻转辅助位;$|100\rangle$ 中 $x_2=0$ 不翻转

\textbf{步骤2:}C-SWAP
\begin{align*}
\frac{1}{\sqrt{2}}(|011\rangle+|100\rangle) &\xrightarrow{\text{C-SWAP}} \frac{1}{\sqrt{2}}(|101\rangle+|100\rangle)
\end{align*}
说明:$|011\rangle$ 辅助位为 1,交换 $x_1$ 和 $x_2$:$|01\rangle \to |10\rangle$;$|100\rangle$ 辅助位为 0 不交换

\textbf{步骤3:}CNOT($x_2 \to x_1$)
\begin{align*}
\frac{1}{\sqrt{2}}(|101\rangle+|100\rangle) &\xrightarrow{\text{CNOT}} \frac{1}{\sqrt{2}}(|101\rangle+|100\rangle)
\end{align*}
说明:$|101\rangle$ 中 $x_2=0$(不是1!),不翻转 $x_1$;$|100\rangle$ 中 $x_2=0$,不翻转 $x_1$

\textbf{测量结果:}测量 $x_1, x_2$ \textbf{确定性地得到 $(1,0)$}(两项都给出相同结果)

---

\textbf{情况3:输入 $|\Phi^-\rangle = \frac{1}{\sqrt{2}}(|00\rangle-|11\rangle)$}

初始态:$|\Phi^-\rangle \otimes |0\rangle = \frac{1}{\sqrt{2}}(|000\rangle-|110\rangle)$

类似地逐步计算:

\textbf{步骤1:}
\begin{align*}
\frac{1}{\sqrt{2}}(|000\rangle-|110\rangle) &\xrightarrow{\text{CNOT}} \frac{1}{\sqrt{2}}(|000\rangle-|111\rangle)
\end{align*}

\textbf{步骤2:}
\begin{align*}
\frac{1}{\sqrt{2}}(|000\rangle-|111\rangle) &\xrightarrow{\text{C-SWAP}} \frac{1}{\sqrt{2}}(|000\rangle-|111\rangle)
\end{align*}

\textbf{步骤3:}
\begin{align*}
\frac{1}{\sqrt{2}}(|000\rangle-|111\rangle) &\xrightarrow{\text{CNOT}} \frac{1}{\sqrt{2}}(|000\rangle-|011\rangle)
\end{align*}

\textbf{测量结果:}测量 $x_1, x_2$ 得到 $(0,0)$ 或 $(0,1)$,各占 50\% 概率(注意负号不影响测量概率)

---

\textbf{情况4:输入 $|\Psi^-\rangle = \frac{1}{\sqrt{2}}(|01\rangle-|10\rangle)$}

初始态:$|\Psi^-\rangle \otimes |0\rangle = \frac{1}{\sqrt{2}}(|010\rangle-|100\rangle)$

\textbf{步骤1:}
\begin{align*}
\frac{1}{\sqrt{2}}(|010\rangle-|100\rangle) &\xrightarrow{\text{CNOT}} \frac{1}{\sqrt{2}}(|011\rangle-|100\rangle)
\end{align*}

\textbf{步骤2:}
\begin{align*}
\frac{1}{\sqrt{2}}(|011\rangle-|100\rangle) &\xrightarrow{\text{C-SWAP}} \frac{1}{\sqrt{2}}(|101\rangle-|100\rangle)
\end{align*}

\textbf{步骤3:}
\begin{align*}
\frac{1}{\sqrt{2}}(|101\rangle-|100\rangle) &\xrightarrow{\text{CNOT}} \frac{1}{\sqrt{2}}(|101\rangle-|100\rangle)
\end{align*}
说明:$|101\rangle$ 中 $x_2=0$,不翻转 $x_1$;$|100\rangle$ 中 $x_2=0$,不翻转 $x_1$

\textbf{测量结果:}测量 $x_1, x_2$ \textbf{确定性地得到 $(1,0)$}(两项都给出相同结果,负号不影响测量结果)

---

\textbf{总结:Bell 态的部分区分}

\begin{center}
\begin{tabular}{c|c}
\hline
\textbf{输入 Bell 态} & \textbf{可能的测量结果} \\
\hline
$|\Phi^+\rangle = \frac{1}{\sqrt{2}}(|00\rangle+|11\rangle)$ & $(0,0)$ 或 $(0,1)$(各50\%) \\
$|\Phi^-\rangle = \frac{1}{\sqrt{2}}(|00\rangle-|11\rangle)$ & $(0,0)$ 或 $(0,1)$(各50\%) \\
$|\Psi^+\rangle = \frac{1}{\sqrt{2}}(|01\rangle+|10\rangle)$ & \textbf{确定性地} $(1,0)$ \\
$|\Psi^-\rangle = \frac{1}{\sqrt{2}}(|01\rangle-|10\rangle)$ & \textbf{确定性地} $(1,0)$ \\
\hline
\end{tabular}
\end{center}

\textbf{关键观察:}

\begin{enumerate}
\item \textbf{部分区分能力}:线路(b)可以区分两类 Bell 态:
\begin{itemize}
  \item $\Phi$ 类($|\Phi^+\rangle$ 和 $|\Phi^-\rangle$):测量结果可能是 $(0,0)$ 或 $(0,1)$,各占50\%
  \item $\Psi$ 类($|\Psi^+\rangle$ 和 $|\Psi^-\rangle$):测量结果\textbf{确定性地}是 $(1,0)$
\end{itemize}

\item \textbf{无法完全区分}:
\begin{itemize}
  \item 无法区分 $|\Phi^+\rangle$ 和 $|\Phi^-\rangle$(相同的测量结果分布)
  \item 无法区分 $|\Psi^+\rangle$ 和 $|\Psi^-\rangle$(相同的测量结果,都是确定性的(1,0))
  \item 对于 $\Phi$ 类,这是因为测量只能获得概率信息,而 $|\Phi^+\rangle$ 和 $|\Phi^-\rangle$ 只相差一个整体相位
  \item 对于 $\Psi$ 类,两个态经过线路后都演化到相同测量结果的态(可能相差整体相位)
\end{itemize}

\item \textbf{测量结果的区分能力}:
\begin{itemize}
  \item 如果测量得到 $(0,0)$,可以确定输入是 $\Phi$ 类($|\Phi^+\rangle$ 或 $|\Phi^-\rangle$)
  \item 如果测量得到 $(0,1)$,可以确定输入是 $\Phi$ 类($|\Phi^+\rangle$ 或 $|\Phi^-\rangle$)
  \item 如果测量得到 $(1,0)$,可以确定输入是 $\Psi$ 类($|\Psi^+\rangle$ 或 $|\Psi^-\rangle$)
  \item 测量结果 $(1,1)$ 不会出现
\end{itemize}

\item \textbf{量子通信应用}:
\begin{itemize}
  \item 这种部分区分能力在量子通信协议中有重要应用
  \item 例如在量子密钥分发(QKD)中,可以用来检测窃听
  \item 在超密编码中,类似的线路用于解码经典信息
  \item 在纠缠交换中,部分 Bell 态测量可以建立远距离纠缠
\end{itemize}

\item \textbf{完全 Bell 态测量的困难性}:
\begin{itemize}
  \item 使用仅线性光学元件和单光子探测器,无法实现完全的 Bell 态测量
  \item 理论上只能确定性地区分 4 个 Bell 态中的 2 个
  \item 这是量子通信实验中的一个基本限制
  \item 需要使用非线性相互作用或辅助纠缠才能实现完全 Bell 态测量
\end{itemize}
\end{enumerate}

\textbf{物理意义:}

这个例子展示了量子测量的一个深刻性质:即使是最优的测量策略,也可能无法完全区分所有的量子态。Bell 态是最大纠缠态,它们构成了两量子比特系统的一组正交基,但用局部操作和经典通信(LOCC)无法完全区分它们。线路(b)实现的是一种特殊的 LOCC 测量,只能部分区分 Bell 态。

\paragraph{题目4} 设有一个具有特征值 $\pm1$ 的单量子比特门上的算子 U,U 既是 Hermite 的又是西的,故可以作此门是一个可观测量,又是一个幺正门。假设我们希望测量量子观测量 U,即我们希望获得指示两个特征值之一的测量结果,并将测量后的状态带到相应的特征向量。证明下面的线路可实现 U 的一个测量。

\begin{figure}[h]
\centering
\includegraphics[width=0.6\textwidth]{assets/作业/量子信息/6/4_pic.png}
\end{figure}

\textbf{解:}

我们逐步分析这个线路在各个阶段的量子态演化:

\textbf{记号说明:}
\begin{itemize}
\item 初态:第一个量子比特为 $|0\rangle$,第二个量子比特为待测态 $|\psi_{\text{in}}\rangle$
\item U 是 Hermite 算符且幺正,因此可以写为 $U = |u_+\rangle\langle u_+| - |u_-\rangle\langle u_-|$,其中 $|u_\pm\rangle$ 是特征值 $\pm1$ 对应的本征态
\end{itemize}

\textbf{态演化分析:}

\textbf{阶段 $|\psi_1\rangle$:}初态
\[
|\psi_1\rangle = |0\rangle\otimes|\psi_{\text{in}}\rangle
\]

\textbf{阶段 $|\psi_2\rangle$:}第一个量子比特经过 Hadamard 门
\[
|\psi_2\rangle = (H\otimes I)(|0\rangle\otimes|\psi_{\text{in}}\rangle)
= \frac{1}{\sqrt{2}}(|0\rangle+|1\rangle)\otimes|\psi_{\text{in}}\rangle
= \frac{1}{\sqrt{2}}(|0\rangle|\psi_{\text{in}}\rangle+|1\rangle|\psi_{\text{in}}\rangle)
\]

\textbf{阶段 $|\psi_3\rangle$:}经过 Controlled-U 门

Controlled-U 门的作用是:当控制位(第一个量子比特)为 $|1\rangle$ 时,对目标位施加 U 门;控制位为 $|0\rangle$ 时不作用。因此:
\[
|\psi_3\rangle = \frac{1}{\sqrt{2}}(|0\rangle|\psi_{\text{in}}\rangle+|1\rangle U|\psi_{\text{in}}\rangle)
\]

将 $|\psi_{\text{in}}\rangle$ 按照 U 的本征态展开:$|\psi_{\text{in}}\rangle = c_+|u_+\rangle + c_-|u_-\rangle$,其中 $|c_+|^2+|c_-|^2=1$。

\textbf{关键步骤:利用特征值 $\pm 1$}

由于 $|u_+\rangle$ 和 $|u_-\rangle$ 是 $U$ 的本征态,满足:
\[
U|u_+\rangle = (+1)|u_+\rangle = |u_+\rangle, \quad U|u_-\rangle = (-1)|u_-\rangle = -|u_-\rangle
\]

因此:
\[
U|\psi_{\text{in}}\rangle = U(c_+|u_+\rangle + c_-|u_-\rangle) = c_+ U|u_+\rangle + c_- U|u_-\rangle = c_+|u_+\rangle - c_-|u_-\rangle
\]

注意这里出现了负号!这正是特征值 $-1$ 的作用。

代入得:
\begin{align*}
|\psi_3\rangle &= \frac{1}{\sqrt{2}}[|0\rangle(c_+|u_+\rangle+c_-|u_-\rangle)+|1\rangle(c_+|u_+\rangle-c_-|u_-\rangle)]\\
&= \frac{c_+}{\sqrt{2}}(|0\rangle+|1\rangle)|u_+\rangle + \frac{c_-}{\sqrt{2}}(|0\rangle-|1\rangle)|u_-\rangle\\
&= c_+|+\rangle|u_+\rangle + c_-|-\rangle|u_-\rangle
\end{align*}

\textbf{阶段 $|\psi_4\rangle$:}第一个量子比特经过第二个 Hadamard 门

应用 $H\otimes I$:
\begin{align*}
|\psi_4\rangle &= c_+(H|+\rangle)|u_+\rangle + c_-(H|-\rangle)|u_-\rangle\\
&= c_+|0\rangle|u_+\rangle + c_-|1\rangle|u_-\rangle
\end{align*}

其中利用了 $H|+\rangle=|0\rangle$,$H|-\rangle=|1\rangle$。

\textbf{测量结果:}

对第一个量子比特(标记为 $a$)进行测量:
\begin{itemize}
\item 测得 $a=0$,概率为 $|c_+|^2$,第二个量子比特塌缩到 $|u_+\rangle$(对应本征值 $+1$)
\item 测得 $a=1$,概率为 $|c_-|^2$,第二个量子比特塌缩到 $|u_-\rangle$(对应本征值 $-1$)
\end{itemize}

因此,该线路通过测量辅助量子比特 $a$ 的值(0或1),间接实现了对 U 的本征值测量($+1$或$-1$),并将第二个量子比特投影到相应的本征态上。证毕。

\subsection{7}

\paragraph{题目1} 列举两种主流量子计算体系(如超导忽片、离子阱等),分别简述其核心工作原理及主要优缺点。

\textbf{解:}

\textbf{1. 超导量子计算(Superconducting Quantum Computing)}

\textbf{核心工作原理:}
\begin{itemize}
\item 利用超导电路中的约瑟夫森结(Josephson Junction)构造人造原子(如超导量子比特)
\item 在极低温($\sim$10-20 mK)下工作,使电路进入超导态,消除电阻损耗
\item 通过微波脉冲控制量子比特的状态和相互作用
\item 常见的超导量子比特类型包括 Transmon、Flux qubit 等
\end{itemize}

\textbf{主要优点:}
\begin{itemize}
\item 可利用成熟的半导体制造工艺进行芯片制造,具有良好的可扩展性
\item 量子门操作速度快(纳秒量级)
\item 可通过微波控制实现高保真度的单比特和双比特门
\item 目前已有商业化的超导量子计算机(如 IBM、Google)
\end{itemize}

\textbf{主要缺点:}
\begin{itemize}
\item 相干时间较短(微秒到毫秒量级),需要复杂的量子纠错
\item 需要极低温环境(稀释制冷机),系统复杂且成本高
\item 量子比特之间的串扰(crosstalk)问题需要精细调控
\item 对电磁噪声敏感
\end{itemize}

\textbf{2. 离子阱量子计算(Ion Trap Quantum Computing)}

\textbf{核心工作原理:}
\begin{itemize}
\item 利用电磁场将单个离子(如 $^{171}\text{Yb}^+$、$^{40}\text{Ca}^+$ 等)囚禁在真空中
\item 量子信息编码在离子的内部电子能级或运动状态上
\item 通过激光脉冲实现量子比特的初始化、操控和读出
\item 离子之间通过共同的声学振动模式实现耦合,从而实现双比特门
\end{itemize}

\textbf{主要优点:}
\begin{itemize}
\item 相干时间长(秒到分钟量级),量子比特质量高
\item 量子门保真度高(单比特门 $>99.9\%$,双比特门 $>99\%$)
\item 所有离子都是相同的(天然的量子比特一致性)
\item 可实现全连接拓扑(任意两个离子都可以相互作用)
\end{itemize}

\textbf{主要缺点:}
\begin{itemize}
\item 量子门操作速度慢(微秒到毫秒量级)
\item 可扩展性受限:单个离子阱中离子数增加时,控制复杂度急剧上升
\item 需要复杂的激光系统和真空系统,集成度较低
\item 量子比特数量的扩展面临技术挑战
\end{itemize}

\paragraph{题目2} 下图是量子隐形传态的线路图,假设待传输量子态为 $|\psi\rangle=\alpha|0\rangle+\beta|1\rangle$,纠缠态为 $|\beta_{00}\rangle=\frac{1}{\sqrt{2}}(|01\rangle+|10\rangle)$,请给出此程中 $|\psi_0\rangle, |\psi_1\rangle, |\psi_2\rangle, |\psi_3\rangle, |\psi_4\rangle$ 结果,并简述量子隐形传态的过程。

\begin{figure}[h]
\centering
\includegraphics[width=0.7\textwidth]{assets/作业/量子信息/7/2_pic.png}
\end{figure}

\textbf{解:}

\textbf{态演化分析:}

\textbf{$|\psi_0\rangle$:}初态

整体初态为待传输态 $|\psi\rangle$ 与纠缠态 $|\beta_{00}\rangle$ 的张量积:
\[
|\psi_0\rangle = |\psi\rangle \otimes |\beta_{00}\rangle
= (\alpha|0\rangle+\beta|1\rangle) \otimes \frac{1}{\sqrt{2}}(|01\rangle+|10\rangle)
\]
\[
= \frac{1}{\sqrt{2}}[\alpha|0\rangle(|01\rangle+|10\rangle) + \beta|1\rangle(|01\rangle+|10\rangle)]
\]
\[
= \frac{1}{\sqrt{2}}(\alpha|001\rangle+\alpha|010\rangle+\beta|101\rangle+\beta|110\rangle)
\]

\textbf{$|\psi_1\rangle$:}第一个和第二个量子比特经过 CNOT 门

CNOT 门作用在前两个量子比特上(第一个为控制位,第二个为目标位):
\[
|\psi_1\rangle = \frac{1}{\sqrt{2}}(\alpha|001\rangle+\alpha|011\rangle+\beta|111\rangle+\beta|100\rangle)
\]

\textbf{$|\psi_2\rangle$:}第一个量子比特经过 Hadamard 门
\[
|\psi_2\rangle = \frac{1}{2}[\alpha(|0\rangle+|1\rangle)|01\rangle+\alpha(|0\rangle+|1\rangle)|11\rangle
+\beta(|0\rangle-|1\rangle)|11\rangle+\beta(|0\rangle-|1\rangle)|00\rangle]
\]

整理得:
\[
|\psi_2\rangle = \frac{1}{2}[|00\rangle(\alpha|1\rangle+\beta|0\rangle)+|01\rangle(\alpha|1\rangle-\beta|0\rangle)
+|10\rangle(\alpha|1\rangle+\beta|0\rangle)+|11\rangle(\alpha|1\rangle-\beta|0\rangle)]
\]

更简洁的形式:
\[
|\psi_2\rangle = \frac{1}{2}[|00\rangle(\beta|0\rangle+\alpha|1\rangle)+|01\rangle(-\beta|0\rangle+\alpha|1\rangle)
+|10\rangle(\beta|0\rangle+\alpha|1\rangle)+|11\rangle(-\beta|0\rangle+\alpha|1\rangle)]
\]

\textbf{$|\psi_3\rangle$:}测量前两个量子比特($M_1, M_2$)

测量后,根据测量结果 $(M_1, M_2)$,第三个量子比特塌缩到以下四种状态之一:
\begin{itemize}
\item $M_1M_2=00$:第三个量子比特为 $\beta|0\rangle+\alpha|1\rangle$
\item $M_1M_2=01$:第三个量子比特为 $-\beta|0\rangle+\alpha|1\rangle$
\item $M_1M_2=10$:第三个量子比特为 $\beta|0\rangle+\alpha|1\rangle$
\item $M_1M_2=11$:第三个量子比特为 $-\beta|0\rangle+\alpha|1\rangle$
\end{itemize}

注意:这里需要根据测量结果确定具体哪种情况,因此 $|\psi_3\rangle$ 是上述四种之一(概率各为 1/4)。

\textbf{$|\psi_4\rangle$:}根据测量结果施加修正操作

根据测量结果 $(M_1, M_2)$,对第三个量子比特施加修正操作:
\begin{itemize}
\item 若 $M_2=1$:施加 $X$ 门(翻转 $|0\rangle \leftrightarrow |1\rangle$)
\item 若 $M_1=1$:施加 $Z$ 门(相位翻转 $|1\rangle \to -|1\rangle$)
\end{itemize}

修正后,无论测量结果如何,最终态都恢复为:
\[
|\psi_4\rangle = \alpha|0\rangle+\beta|1\rangle = |\psi\rangle
\]

\textbf{量子隐形传态过程总结:}

\begin{enumerate}
\item \textbf{准备阶段:}Alice 持有待传输态 $|\psi\rangle$,Alice 和 Bob 共享纠缠态 $|\beta_{00}\rangle$(Alice 持有第二个量子比特,Bob 持有第三个)

\item \textbf{Bell 基测量:}Alice 对手中的两个量子比特(待传输态和纠缠对的一半)进行 Bell 基测量,得到两个经典比特 $(M_1, M_2)$

\item \textbf{经典通信:}Alice 将测量结果 $(M_1, M_2)$ 通过经典信道发送给 Bob

\item \textbf{修正操作:}Bob 根据收到的测量结果,对自己手中的量子比特施加相应的幺正操作($X$ 和 $Z$ 门的组合)

\item \textbf{完成传输:}Bob 手中的量子比特最终恢复为原始态 $|\psi\rangle$,而 Alice 手中的量子态已被测量破坏
\end{enumerate}

\textbf{关键点:}
\begin{itemize}
\item 量子隐形传态不违反不可克隆定理:原始态被测量破坏
\item 需要经典信道传输测量结果,因此不能超光速通信
\item 利用量子纠缠和经典通信实现量子态的传输
\end{itemize}

\end{document}
